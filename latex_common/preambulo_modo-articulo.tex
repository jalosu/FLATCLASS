%% ================== PREÁMBULO CORPORATIVO FLATCLASS ==================
%% Mantiene compatibilidad con JupyterBook y añade formato de artículo científico

\def\sphinxdocclass{jupyterBook}
\documentclass[a4paper,10pt]{jupyterBook}

% ===============================================================
%   Paquetes base y compatibilidad Sphinx
% ===============================================================
\ifdefined\pdfpxdimen
   \let\sphinxpxdimen\pdfpxdimen\else\newdimen\sphinxpxdimen
\fi \sphinxpxdimen=.75bp\relax
\ifdefined\pdfimageresolution
    \pdfimageresolution= \numexpr \dimexpr1in\relax/\sphinxpxdimen\relax
\fi

\PassOptionsToPackage{bookmarksdepth=5}{hyperref}
\PassOptionsToPackage{hyperindex=false}{hyperref}
\makeatletter\@ifclassloaded{memoir}
{\ifdefined\memhyperindexfalse\memhyperindexfalse\fi}{}\makeatother
\PassOptionsToPackage{booktabs}{sphinx}
\PassOptionsToPackage{colorrows}{sphinx}
\PassOptionsToPackage{warn}{textcomp}

\catcode`^^^^00a0\active\protected\def^^^^00a0{\leavevmode\nobreak\ }
\usepackage{cmap}
\usepackage{fontspec}
\defaultfontfeatures[\rmfamily,\sffamily,\ttfamily]{}
\usepackage{amsmath,amssymb,amstext}

% ===============================================================
%   Idioma y fuentes
% ===============================================================
\usepackage{polyglossia}
\setmainlanguage{spanish}

\setmainfont{FreeSerif}[
  Extension      = .otf,
  UprightFont    = *,
  ItalicFont     = *Italic,
  BoldFont       = *Bold,
  BoldItalicFont = *BoldItalic
]
\setsansfont{FreeSans}[
  Extension      = .otf,
  UprightFont    = *,
  ItalicFont     = *Oblique,
  BoldFont       = *Bold,
  BoldItalicFont = *BoldOblique,
]
\setmonofont{FreeMono}[
  Extension      = .otf,
  UprightFont    = *,
  ItalicFont     = *Oblique,
  BoldFont       = *Bold,
  BoldItalicFont = *BoldOblique,
]

\usepackage[Bjarne]{fncychap}
\usepackage[,numfigreset=1,mathnumfig]{sphinx}

% ===============================================================
%   Ajustes visuales y corporativos
% ===============================================================
\fvset{fontsize=\footnotesize}
\usepackage[a4paper,margin=25mm]{geometry}
\usepackage{xcolor}
\usepackage{graphicx}
\usepackage{eso-pic}
\AddToShipoutPictureBG*{%
  \AtPageLowerLeft{\includegraphics[width=\paperwidth,height=\paperheight]{Fondo.png}}%
}

% ===============================================================
%   Artículo científico: Abstract + 2 columnas
% ===============================================================
\usepackage{cuted}      % 'strip' para bloques de una columna
\usepackage{abstract}   % entorno abstract
\usepackage{balance}    % balancea columnas al final
\renewcommand{\abstractname}{Resumen}
\setlength{\columnsep}{6mm}

% Texto del abstract (puedes sustituirlo por \input{abstract.tex})
\newcommand{\PaperAbstract}{%
Este trabajo presenta el sistema FLATCLASS, desarrollado para la clasificación morfológica de peces planos mediante visión artificial y aprendizaje automático.
Se describe la arquitectura hardware–software, el modelo predictivo y la estrategia de validación experimental implementada.
Los resultados muestran una precisión superior al 95 % en la detección de anomalías morfológicas y una reducción significativa del tiempo de clasificación.
}

% Activa abstract a ancho completo + resto en dos columnas
\makeatletter
\AtBeginDocument{%
  \onecolumn
  \begin{strip}
    \centering
    \ifdefined\sphinxmaketitle
      \sphinxmaketitle
    \else
      \maketitle
    \fi
    \vspace{-0.5em}
    \begin{abstract}
      \PaperAbstract
    \end{abstract}
    \vspace{1em}
  \end{strip}
  \twocolumn
}
\makeatother

% ===============================================================
%   Configuración hiperenlaces y mensajes Sphinx
% ===============================================================
\usepackage{hyperref}
\usepackage{hypcap}
\urlstyle{same}
\addto\captionsenglish{\renewcommand{\contentsname}{PT1 - Visión Artificial}}
\usepackage{sphinxmessages}

% ===============================================================
%   Bloque añadido por sphinx-jupyterbook-latex
% ===============================================================
\usepackage[Latin,Greek]{ucharclasses}
\usepackage{unicode-math}
\setmathfont{Latin Modern Math}
\addto\captionsenglish{\renewcommand{\contentsname}{Contents}}
\hypersetup{
    pdfencoding=auto,
    psdextra
}

% ===============================================================
%   Idioma, numeraciones y formato arábigo
% ===============================================================
\AtBeginDocument{%
  \addto\captionsenglish{%
    \renewcommand{\partname}{Paquete de Trabajo}%
    \renewcommand{\chaptername}{Capítulo}%
    \renewcommand{\figurename}{Figura}%
    \renewcommand{\tablename}{Tabla}%
  }%
  \makeatletter
  \providecommand{\sphinxnumstr}[1]{#1}%
  \renewcommand{\sphinxnumstr}[1]{#1}%
  \renewcommand{\thepart}{\arabic{part}}%
  \renewcommand{\thechapter}{\arabic{chapter}}%
  \renewcommand{\theHchapter}{\thechapter}%
  \makeatother
}
%% ================== FIN PREÁMBULO FLATCLASS ==================
