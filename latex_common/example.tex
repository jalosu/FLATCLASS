%% Generated by Sphinx.
\def\sphinxdocclass{jupyterBook}
\documentclass[a4paper,10pt,spanish]{jupyterBook}
\ifdefined\pdfpxdimen
   \let\sphinxpxdimen\pdfpxdimen\else\newdimen\sphinxpxdimen
\fi \sphinxpxdimen=.75bp\relax
\ifdefined\pdfimageresolution
    \pdfimageresolution= \numexpr \dimexpr1in\relax/\sphinxpxdimen\relax
\fi
%% let collapsible pdf bookmarks panel have high depth per default
\PassOptionsToPackage{bookmarksdepth=5}{hyperref}
%% turn off hyperref patch of \index as sphinx.xdy xindy module takes care of
%% suitable \hyperpage mark-up, working around hyperref-xindy incompatibility
\PassOptionsToPackage{hyperindex=false}{hyperref}
%% memoir class requires extra handling
\makeatletter\@ifclassloaded{memoir}
{\ifdefined\memhyperindexfalse\memhyperindexfalse\fi}{}\makeatother

\PassOptionsToPackage{booktabs}{sphinx}
\PassOptionsToPackage{colorrows}{sphinx}

\PassOptionsToPackage{warn}{textcomp}

\catcode`^^^^00a0\active\protected\def^^^^00a0{\leavevmode\nobreak\ }
\usepackage{cmap}
\usepackage{fontspec}
\defaultfontfeatures[\rmfamily,\sffamily,\ttfamily]{}
\usepackage{amsmath,amssymb,amstext}
\usepackage{polyglossia}
\setmainlanguage{spanish}
\renewcommand{\partname}{Paquete de Trabajo}
\renewcommand{\thepart}{\arabic{part}}
\renewcommand{\thechapter}{\arabic{chapter}}



\setmainfont{FreeSerif}[
  Extension      = .otf,
  UprightFont    = *,
  ItalicFont     = *Italic,
  BoldFont       = *Bold,
  BoldItalicFont = *BoldItalic
]
\setsansfont{FreeSans}[
  Extension      = .otf,
  UprightFont    = *,
  ItalicFont     = *Oblique,
  BoldFont       = *Bold,
  BoldItalicFont = *BoldOblique,
]
\setmonofont{FreeMono}[
  Extension      = .otf,
  UprightFont    = *,
  ItalicFont     = *Oblique,
  BoldFont       = *Bold,
  BoldItalicFont = *BoldOblique,
]



\usepackage[Bjarne]{fncychap}
\usepackage[,numfigreset=1,mathnumfig]{sphinx}

\fvset{fontsize=\scriptsize}
\usepackage{geometry}


% Include hyperref last.
\usepackage{hyperref}
% Fix anchor placement for figures with captions.
\usepackage{hypcap}% it must be loaded after hyperref.
% Set up styles of URL: it should be placed after hyperref.
\urlstyle{same}

\usepackage{sphinxmessages}



% Start of preamble defined in sphinx-jupyterbook-latex %
\usepackage[Latin,Greek]{ucharclasses}
\usepackage{unicode-math}
% fixing title of the toc
\addto\captionsenglish{\renewcommand{\contentsname}{Contents}}
\hypersetup{
    pdfencoding=auto,
    psdextra
}
% End of preamble defined in sphinx-jupyterbook-latex %



% ===================== PARCHE DEFINITIVO =====================
% Ejecutar las redefiniciones *al iniciar el documento* para que sean las últimas.
\AtBeginDocument{%
  % Forzar rótulos en español aunque la clase esté en 'english'
  \addto\captionsenglish{%
    \renewcommand{\partname}{Paquete de Trabajo}%
    \renewcommand{\chaptername}{Capítulo}%
    % Si quieres forzar "Índice" globalmente, descomenta:
    %\renewcommand{\contentsname}{Índice}%
    \renewcommand{\figurename}{Figura}%
    \renewcommand{\tablename}{Tabla}%
  }%
  % Evitar que Sphinx convierta 1,2,3 -> "One, Two, Three"
  \makeatletter
  \providecommand{\sphinxnumstr}[1]{#1}%
  \renewcommand{\sphinxnumstr}[1]{#1}% devuelve el número tal cual (arábigo)
  % Asegurar numeración arábiga en contadores principales
  \renewcommand{\thepart}{\arabic{part}}%
  \renewcommand{\thechapter}{\arabic{chapter}}%
  % (opcional) numeración 1.1, 1.2…:
  %\renewcommand{\thesection}{\thechapter.\arabic{section}}%
  %\renewcommand{\thesubsection}{\thesection.\arabic{subsection}}%
  \makeatother
}
% ================== FIN PARCHE DEFINITIVO ====================

\title{FLATCLASS - FLATfish Length and Thickness CLASification System by Artificial Intelligence}
\date{}
\release{}
\author{Javier Álvarez Osuna}
\newcommand{\sphinxlogo}{\vbox{}}
\renewcommand{\releasename}{}
\makeindex
\usepackage{eso-pic,graphicx}
\AddToShipoutPictureBG*{%
  \AtPageLowerLeft{\includegraphics[width=\paperwidth,height=\paperheight]{Fondo.png}}}
\begin{document}

\pagestyle{empty}
\sphinxmaketitle
\pagestyle{plain}
\sphinxtableofcontents
\pagestyle{normal}
\phantomsection\label{\detokenize{content/intro::doc}}


\sphinxAtStartPar
La acuicultura es uno de los sectores de producción de alimentos de más rápido crecimiento a nivel mundial, con un incremento anual del 5.3\% en la última década {[}{[}FAO, 2024{]}{]} (https://doi.org/10.4060/cd0683en). Dentro de este sector, la cría de peces planos, como el lenguado (\sphinxstyleemphasis{Solea solea}) y el rodaballo (\sphinxstyleemphasis{Scophthalmus maximus}), representa un
nicho importante debido a su alto valor comercial y demanda en mercados gourmet. Sin embargo, la producción de estos peces enfrenta desafíos significativos en la fase de alevinaje, donde la clasificación
de alevines por tamaño y la estimación de su peso son procesos críticos para garantizar un crecimiento uniforme y maximizar la supervivencia \sphinxhref{https://doi.org/10.1038/s41586-021-03308-6}{{[}Naylor et al., 2021{]}}.

\sphinxAtStartPar
Una clasificación de alevines precisa permite la agrupación por tamaños homogéneos, lo cual es crucial para optimizar la alimentación, minimizar la competencia intraespecífica y reducir el riesgo de mortalidad por agresión o estés. Éste es un proceso laborioso propenso a errores que también consume tiempo y recursos, lo que reduce la eficiencia operativa y aumenta los costos de producción \sphinxhref{https://doi.org/10.1016/S0144-8609(01)00091-7}{{[}Papandroulakis et al., 2002{]}}. En la actualidad la industria está usando sistemas semi\sphinxhyphen{}automáticos para la clasificación de alevines, que si bien suponen una clara mejora, aún presentan limitaciones significativas. Uno de los principales problemas es que estos sistemas suelen basarse en un único criterio de clasificación: el peso o la altura del pez. Esta falta de multifactorialidad en el proceso de clasificación (\sphinxstyleemphasis{gradding} en térmico anglosajón) limita su sensibilidad y precisión. Por ejemplo, un sistema que solo mide el peso puede pasar por alto variaciones importantes en la morfología del pez, como la longitud o el ancho, que también son indicadores clave de su desarrollo y salud. De manera similar, un sistema que solo considera la altura puede no capturar adecuadamente las diferencias en el volumen o la condición corporal de los alevines, lo que puede llevar a una clasificación subóptima.

\sphinxAtStartPar
Esta dependencia de un único parámetro no solo reduce la eficacia del proceso de clasificación, sino que también puede generar resultados inconsistentes. Por ejemplo, dos peces con el mismo peso podrían tener formas corporales muy diferentes, lo que afectaría su crecimiento futuro y su adaptación a los tanques de engorde. Además, estos sistemas semi\sphinxhyphen{}automáticos suelen ser extremadamente invasivos, ya que requieren manipulación física para medir el peso o la altura, lo que aumenta el estrés en los peces y puede elevar las tasas de mortalidad hasta un 15\% en algunos casos \sphinxhref{https://doi.org/10.1016/j.biosystemseng.2017.10.014}{{[}Føre et al., 2018{]}}. Esta falta de sensibilidad en el proceso de gradding limitan la capacidad de los productores para optimizar el manejo de los alevines, lo que puede resultar en una alta variabilidad en el crecimiento y una reducción en la calidad del producto final. no solo afecta la calidad del producto final, sino que también limita la capacidad de los productores para optimizar el manejo de los alevines y garantizar un crecimiento uniforme.

\sphinxAtStartPar
La limitación de los sistemas semi\sphinxhyphen{}automáticos a un único criterio de clasificación subraya la necesidad de adoptar tecnologías más avanzadas que puedan integrar múltiples parámetros morfométricos en tiempo real. Aquí es donde entra en juego la inteligencia artificial (IA). Tecnologías avanzadas de inteligencia artificial (IA) como el reconocimiento de imágenes (visión por computador) y aprendizaje profundo ofrece una solución prometedora. Las redes neuronales convolucionales (CNN) han demostrado ser herramientas poderosas para tareas de clasificación y segmentación de imágenes en múltiples dominios. Por ejemplo, trabajos recientes han utilizado CNN para la identificación de enfermedades en peces y la estimación de biomasa en sistemas de cultivo {[}\sphinxhref{https://doi.org/10.22075/ijnaa.2022.5839}{Hasan et al., 2022}, \sphinxhref{https://www.researchgate.net/publication/384679490\_A\_Convolutional\_Neural\_Network\_Approach\_for\_Precision\_Fish\_Disease\_Detection}{Haddad et al., 2024}{]}. Estas tecnologías no solo ofrecen altos niveles de precisión, sino que también permiten el procesamiento en tiempo real, un requisito indispensable en aplicaciones industriales. En el caso específico de la clasificación de alevines, la IA permite combinar precisión en la identificación de tamaños con la estimación de parámetros morfométricos, como longitud, anchura y superficie visible del pez, para estimar su peso sin necesidad de contacto físico.

\sphinxAtStartPar
La capacidad de estos sistemas para proporcionar los datos en tiempo real sobre las poblaciones de alevines proporciona a los operadores información valiosa para la toma de decisiones estratégicas, como el ajuste de tasas de alimentación o el diseño de planes de transferencia a estanques de engorde, lo que mejora significativamente la eficiencia operativa, reduciendo los costos y aumentando la rentabilidad de las operaciones acuícolas {[}\sphinxhref{https://doi.org/10.3389/fmars.2021.823173}{Kandimalia et al., 2022}{]}, \sphinxhref{https://doi.org/10.1038/s41586-021-03308-6}{{[}Naylor et al., 2021{]}}. Adicionalmente, la implantación de estos sistemas inteligentes de clasificación también contribuye a la sostenibilidad ambiental y económica de las piscifactorías. Al clasificar de forma más precisa y eficiente, se optimizan los recursos alimenticios y se reduce el desperdicio de alimento, que es uno de los costos más significativos en la acuicultura. Mientras que al reducir el estrés y la manipulación de los peces, se minimiza el riesgo de enfermedades y se promueve un crecimiento más saludable y uniforme {[}\sphinxhref{https://doi.org/10.1016/j.applanim.2006.09.001}{Ashley., 2007}{]}.

\sphinxAtStartPar
\sphinxstylestrong{FLATCLASS} es la propuesta de Feeding Systems S.L. para desarrollar un sistema automatizado de gradding de peces planos (lenguado y rodaballo), basado en visión artificial y algoritmos de aprendizaje automático, que permita no solo la clasificación precisa y en tiempo real de los individuos según su tamaño, sino también el contaje unitario de ejemplares procesados y la estimación individual de su peso. Todo ello garantizando una integración fluida en las líneas de producción existentes y proporcionando datos clave para una gestión eficiente del cultivo acuícola. La implementación de este sistema busca mejorar la productividad mediante la digitalización y automatización de procesos, reducir costos operativos y proporcionar datos estadísticos para la optimización del manejo y alimentación de los peces.

\sphinxAtStartPar
Este libro recopila los resultados científicos y técnicos obtenidos en el marco del proyecto \sphinxstylestrong{FLATCLASS}, organizados en capítulos correspondientes a cada paquete de trabajo. A lo largo de sus páginas, se presentan de manera estructurada las investigaciones realizadas, los avances alcanzados y las soluciones desarrolladas, sustentadas en metodologías rigurosas y análisis crítico. Cada apartado puede ser considerado un entregable y ofrece una visión detallada de los hallazgos, contribuciones y lecciones aprendidas, con el fin de servir como referencia para la comunidad académica y profesional. El documento no solo sintetiza el conocimiento generado, sino que también destaca su aplicabilidad y potencial impacto en el ámbito de la IA aplicada al sector de la acuicultura.

\sphinxstepscope


\part{PT1 \sphinxhyphen{} Visión Artificial}

\sphinxstepscope


\chapter{Especificación de requisitos}
\label{\detokenize{content/01/Requisitos:especificacion-de-requisitos}}\label{\detokenize{content/01/Requisitos::doc}}
\begin{sphinxadmonition}{note}{Resumen}

\sphinxAtStartPar
El presente documento recoge de forma estructurada los requisitos funcionales y técnicos necesarios para el desarrollo del sistema FLATCLASS, una solución automatizada basada en visión artificial e inteligencia artificial para la clasificación en tiempo real de alevines de lenguado. En él se definen los criterios operativos para la captura de imágenes, calibración de tamaños, clasificación morfométrica, estimación de peso, y separación automatizada de los ejemplares, así como los mecanismos de trazabilidad, visualización y seguridad en el acceso a los datos.

\sphinxAtStartPar
\sphinxstylestrong{Entregable}: E1.1\\
\sphinxstylestrong{Versión}: 1.0\\
\sphinxstylestrong{Autor}: Javier Álvarez Osuna\\
\sphinxstylestrong{Email}: javier.osuna@fishfarmfeeder.com\\
\sphinxstylestrong{ORCID}: \sphinxhref{https://orcid.org/0000-0001-7063-1279}{0000\sphinxhyphen{}0001\sphinxhyphen{}7063\sphinxhyphen{}1279}\\
\sphinxstylestrong{Licencia}: CC\sphinxhyphen{}BY\sphinxhyphen{}4.0\\
\sphinxstylestrong{Código proyecto}: IG408M.2025.000.000072

\begin{figure}[H]
\centering

\noindent\sphinxincludegraphics[width=1.000\linewidth]{{FLATCLASS_logo_publicidad}.png}
\end{figure}
\end{sphinxadmonition}


\section{Introducción}
\label{\detokenize{content/01/Requisitos:introduccion}}
\sphinxAtStartPar
El presente documento define los requisitos funcionales y técnicos del sistema \sphinxstylestrong{FLATCLASS}, un sistema de clasificación automatizada de alevines de lenguado especialmenteá orientado a mejorar la eficiencia y trazabilidad del proceso de \sphinxstyleemphasis{gradding} dentro de entornos acuícolas.

\sphinxAtStartPar
Este sistema se basa en una arquitectura de procesamiento distribuido que integra visión artificial, algoritmos de procesamiento de imágenes y modelos de aprendizaje automático para la clasificación y estimación del peso de los alevines en tiempo real. El flujo operativo se divide en dos fases principales: una fase de muestreo y parametrización, en la que se establecen los criterios de clasificación a partir del análisis estadístico de una muestra representativa, y una fase de operación continua, en la que el sistema ejecuta la clasificación y la estimación de peso de cada pez mientras este atraviesa la línea de producción.

\begin{figure}[htbp]
\centering

\noindent\sphinxincludegraphics[width=1.000\linewidth]{{Modelo_funcional}.png}
\end{figure}

\sphinxAtStartPar
Para cada uno de los procesos se han identificado las restricciones necesarias para garantizar que el sistema cumple con las funcionalidades esperadas. Dichas restricciones se han clasificado en uno de los siguientes grupos de requisitos: funcionales, técnicos y deseables.


\section{Requisitos Funcionales}
\label{\detokenize{content/01/Requisitos:requisitos-funcionales}}

\begin{savenotes}\sphinxattablestart
\sphinxthistablewithglobalstyle
\centering
\begin{tabulary}{\linewidth}[t]{TTT}
\sphinxtoprule
\sphinxstyletheadfamily
\sphinxAtStartPar
\sphinxstylestrong{Código}
&\sphinxstyletheadfamily
\sphinxAtStartPar
\sphinxstylestrong{Proceso}
&\sphinxstyletheadfamily
\sphinxAtStartPar
\sphinxstylestrong{Requisito Funcional}
\\
\sphinxmidrule
\sphinxtableatstartofbodyhook
\sphinxAtStartPar
\sphinxstylestrong{RF01}
&
\sphinxAtStartPar
Captura y análisis de imagen
&
\sphinxAtStartPar
El sistema debe procesar un alevín cada 5 segundos por canal, con un total de 4 canales funcionando en paralelo.
\\
\sphinxhline
\sphinxAtStartPar
\sphinxstylestrong{RF02}
&
\sphinxAtStartPar
Captura y análisis de imagen
&
\sphinxAtStartPar
Resolución de imagen: las imágenes deben capturarse a resolución 1920x1080 px usando cámaras para asegurar precisión morfométrica.
\\
\sphinxhline
\sphinxAtStartPar
\sphinxstylestrong{RF03}
&
\sphinxAtStartPar
Clasificación y separación
&
\sphinxAtStartPar
Cada sesión empieza una calibración previa con N ≤ 200 imágenes para definir rangos de tamaño establecidos por el usuario (pequeño, mediano, grande).
\\
\sphinxhline
\sphinxAtStartPar
\sphinxstylestrong{RF04}
&
\sphinxAtStartPar
Clasificación y separación
&
\sphinxAtStartPar
Separación automática en 4 categorías: el sistema debe clasificar en pequeño, mediano, grande o error.
\\
\sphinxhline
\sphinxAtStartPar
\sphinxstylestrong{RF05}
&
\sphinxAtStartPar
Calsificación y separación
&
\sphinxAtStartPar
Inferencia de peso por IA: el sistema debe estimar el peso con un error máximo del 5\% utilizando regresión multiparamétrica en log\sphinxhyphen{}transformada.
\\
\sphinxhline
\sphinxAtStartPar
\sphinxstylestrong{RF06}
&
\sphinxAtStartPar
Clasificación y separación
&
\sphinxAtStartPar
Separación física automatizada: los peces deben desviarse mediante compuertas automáticas según su categoría.
\\
\sphinxhline
\sphinxAtStartPar
\sphinxstylestrong{RF07}
&
\sphinxAtStartPar
Visualización
&
\sphinxAtStartPar
Visualización y control en interfaz web: el sistema deberá permitir parametrización, visualización en tiempo real y acceso a históricos.
\\
\sphinxhline
\sphinxAtStartPar
\sphinxstylestrong{RF08}
&
\sphinxAtStartPar
Registro de datos
&
\sphinxAtStartPar
Gestión de datos y trazabilidad: todos los datos se almacenan en MongoDB con UID, timestamp, medidas, peso y clasificación.
\\
\sphinxhline
\sphinxAtStartPar
\sphinxstylestrong{RF09}
&
\sphinxAtStartPar
Visualización
&
\sphinxAtStartPar
Validación supervisada: debe permitir revisar manualmente clasificaciones dudosas para retroalimentación del sistema.
\\
\sphinxbottomrule
\end{tabulary}
\sphinxtableafterendhook\par
\sphinxattableend\end{savenotes}


\section{Requisitos Técnicos}
\label{\detokenize{content/01/Requisitos:requisitos-tecnicos}}

\begin{savenotes}\sphinxattablestart
\sphinxthistablewithglobalstyle
\centering
\begin{tabulary}{\linewidth}[t]{TT}
\sphinxtoprule
\sphinxstyletheadfamily
\sphinxAtStartPar
\sphinxstylestrong{Código}
&\sphinxstyletheadfamily
\sphinxAtStartPar
\sphinxstylestrong{Requisito Técnico}
\\
\sphinxmidrule
\sphinxtableatstartofbodyhook
\sphinxAtStartPar
\sphinxstylestrong{RT01}
&
\sphinxAtStartPar
Arquitectura híbrida: procesamiento Edge con GPU avanzada (Jetson u otra), backend en Python y control en Node\sphinxhyphen{}RED.
\\
\sphinxhline
\sphinxAtStartPar
\sphinxstylestrong{RT02}
&
\sphinxAtStartPar
Modelo de datos: MongoDB almacenará UID, timestamp, dimensiones, clasificación y predicción de peso.
\\
\sphinxhline
\sphinxAtStartPar
\sphinxstylestrong{RT03}
&
\sphinxAtStartPar
Interfaz multiusuario: mínimo 2 perfiles (operador y supervisor) con distintos permisos.
\\
\sphinxhline
\sphinxAtStartPar
\sphinxstylestrong{RT04}
&
\sphinxAtStartPar
API REST: el sistema debe implementar al menos los siguientes endpoints:• \sphinxcode{\sphinxupquote{GET /records?from=YYYY\sphinxhyphen{}MM\sphinxhyphen{}DD\&to=YYYY\sphinxhyphen{}MM\sphinxhyphen{}DD}}: Recupera registros procesados en el intervalo de fechas indicado.• \sphinxcode{\sphinxupquote{GET /record/\{id\}}}: Devuelve los datos completos del pez individual con ID específico.• \sphinxcode{\sphinxupquote{GET /summary}}: Proporciona un resumen estadístico por lote (número por categoría, medias de peso y superficie).• \sphinxcode{\sphinxupquote{POST /calibration}}: Envía un nuevo conjunto de umbrales de clasificación definidos por el usuario.• \sphinxcode{\sphinxupquote{GET /image/\{id\}}}: Accede a la imagen almacenada correspondiente al pez con ID único.
\\
\sphinxhline
\sphinxAtStartPar
\sphinxstylestrong{RT05}
&
\sphinxAtStartPar
Seguridad de acceso a API:• Autenticación mediante token (JWT o equivalente).• Cifrado obligatorio vía HTTPS.• Registro de logs con IP, usuario y timestamp.• Control de acceso por roles (lectura, escritura, administración).• Almacenamiento seguro de credenciales con rotación periódica.
\\
\sphinxbottomrule
\end{tabulary}
\sphinxtableafterendhook\par
\sphinxattableend\end{savenotes}


\section{Requisitos Deseables}
\label{\detokenize{content/01/Requisitos:requisitos-deseables}}

\begin{savenotes}\sphinxattablestart
\sphinxthistablewithglobalstyle
\centering
\begin{tabulary}{\linewidth}[t]{TT}
\sphinxtoprule
\sphinxstyletheadfamily
\sphinxAtStartPar
\sphinxstylestrong{Código}
&\sphinxstyletheadfamily
\sphinxAtStartPar
\sphinxstylestrong{Requisito Deseable}
\\
\sphinxmidrule
\sphinxtableatstartofbodyhook
\sphinxAtStartPar
\sphinxstylestrong{RD01}
&
\sphinxAtStartPar
Minimización del estrés animal mediante separación suave.
\\
\sphinxhline
\sphinxAtStartPar
\sphinxstylestrong{RD02}
&
\sphinxAtStartPar
Cifrado opcional de datos y versionado de modelos IA.
\\
\sphinxhline
\sphinxAtStartPar
\sphinxstylestrong{RD03}
&
\sphinxAtStartPar
Sistema de alerta por error recurrente o desviación de modelo.
\\
\sphinxbottomrule
\end{tabulary}
\sphinxtableafterendhook\par
\sphinxattableend\end{savenotes}

\sphinxstepscope


\chapter{Sistema de visión artificial}
\label{\detokenize{content/01/Imagen:sistema-de-vision-artificial}}\label{\detokenize{content/01/Imagen::doc}}
\begin{sphinxadmonition}{note}{Resumen}

\sphinxAtStartPar
El presente documento recoge los trabajos relativos al módulo de visión artificial de \sphinxstylestrong{FLATCLASS}. Este módulo es el responsable de la adquisición, normalizado, segmentación y extracción de las características mormométricas de los alevines de lenguado.

\sphinxAtStartPar
\sphinxstylestrong{Entregable}: E1.2\\
\sphinxstylestrong{Versión}: 1.0\\
\sphinxstylestrong{Autor}: Javier Álvarez Osuna\\
\sphinxstylestrong{Email}: javier.osuna@fishfarmfeeder.com\\
\sphinxstylestrong{ORCID}: \sphinxhref{https://orcid.org/0000-0001-7063-1279}{0000\sphinxhyphen{}0001\sphinxhyphen{}7063\sphinxhyphen{}1279}\\
\sphinxstylestrong{Licencia}: CC\sphinxhyphen{}BY\sphinxhyphen{}4.0\\
\sphinxstylestrong{Código proyecto}: IG408M.2025.000.000072

\begin{figure}[H]
\centering

\noindent\sphinxincludegraphics[width=1.000\linewidth]{{FLATCLASS_logo_publicidad}.png}
\end{figure}
\end{sphinxadmonition}


\section{Introducción}
\label{\detokenize{content/01/Imagen:introduccion}}
\sphinxAtStartPar
Uno de los objetivos técnicos plantedos en \sphinxstylestrong{FLATCLASS} se centra en el Desarrollo e integración de un sistema de visión artificial de alta resolución que permita la captura precisa de imágenes de los alevines en tiempo real, asegurando una segmentación eficaz y el cálculo automático del área de superficie, longitud y anchura de cada individuo. El sistema automático desarrollado permite la medición sin contacto de las características morfométricas de lenguados (\sphinxstyleemphasis{Solea solea}) tales como la longitud total (L), la anchura corporal (A) y la superficie real (S) a partir de las imágenes fotográficas, obtenidas con una cámara Datalogic P22C 600\sphinxhyphen{}000 ML, mediante un algoritmo de procesado. Este algoritmo permite realizar la medición incluso cuando la cola del pez plano está deformada o cuando el pez se encuentra orientado de manera aleatoria, informando del nivel de fiabilidad de las mediciones cuando los resultados puedan persentar dudas. Este último aspecto es especialmente importante si tenemos en cuenta que el proceso de clasificación de juveniles se suele producir en ambientes de baja luminosidad y alta concentración de humedad que introduce importantes restricciones en la calidad de las imágenes obtenidas.

\sphinxAtStartPar
El núcleo del sistema se sustenta en \sphinxstylestrong{GrabCut}, un método de segmentación de imágenes interactivo basado en teoría de grafos y optimización de cortes mínimos en campos aleatorios de Markov (MRF) definido por primera vez por \sphinxhref{https://doi.org/10.1145/1015706.1015720}{Rother et al.,} en 2004. El usuario define inicialmente un recuadro aproximado que contiene el pez, permitiendo al algoritmo estimar la distribución de color del primer plano y el fondo mediante modelos de mezcla gaussiana (Gaussian Mixture Models, GMM). Posteriormente, se plantea una función de energía que combina un término regional (basado en las probabilidades del GMM) y un término de continuidad espacial (prefiriendo regiones conectadas con etiquetas uniformes), la cual se minimiza mediante un corte en el grafo (graph cut). Este proceso se itera hasta converger en una segmentación refinada del pez frente al fondo {[}\sphinxhref{https://doi.org/10.1016/j.jksuci.2020.07.005}{Alsmadi et al., 2022}{]}.

\sphinxAtStartPar
GrabCut ha sido objeto de mejoras continuas en los últimos años, incluyendo variantes que integran mapas de saliencia, superpíxeles o funciones de energía modificadas que permiten segmentaciones más precisas y automáticas {[}\sphinxhref{https://doi.org/10.3390/math11081965}{Wang et al., 2023}{]}. En el contexto acuícola, diferentes trabajos han aplicado algoritmos derivados de GrabCut (como versiones adaptadas o iterativas) para refinar la segmentación en imágenes de peces. Por ejemplo, \sphinxhref{https://doi.org/10.11975/j.issn.1002-6819.2020.21.026}{Huang et al., 2023} han desarrollado un sistema de visión binocular en acuicultura utilizó GrabCut adaptativo por contraste para segmentar peces bajo el agua, logrando un error porcentual medio en la medida de longitud de solo 0,9\%. Asimismo, en el ámbito de la fenotipificación de peces de tilapia, GrabCut se ha empleado para afinar muestras en los bordes de segmentación, mejorando el IoU hasta un 81 \% cuando se combina con redes profundas {[}\sphinxhref{https://doi.org/10.3390/app13179635}{Feng et al., 2023}{]}.

\sphinxAtStartPar
Para la extracción de los parámetros morfométricos de alevines de lenguado a partir de imágenes digitales en vista dorsal, GrabCut es especialmente adecuado por varias razones clave. Primero, requiere mínima interacción (únicamente un recuadro inicial o ROI \sphinxhyphen{} \sphinxstyleemphasis{Region Of Interest}), lo cual es práctico en entornos productivos con alto flujo de muestras. Segundo, al basarse en modelos estadísticos de color y continuidad espacial, puede separar con precisión el pez del fondo uniforme de la cinta que lo transporta, incluso con variaciones de iluminación. Esto resulta esencial para cálculos fiables de longitud, anchura y superficie del pez. Finalmente, es computacionalmente eficiente y ya está implementado en bibliotecas comunes como OpenCV, lo que facilita su integración en un sistema automatizado de procesamiento de imágenes en planta.

\sphinxAtStartPar
El flujograma del \sphinxstylestrong{algoritmo de extracción de características morfométricas} desarrollado se recoge en la siguiente figura:

\begin{figure}[htbp]
\centering
\capstart

\noindent\sphinxincludegraphics[width=0.250\linewidth]{{Process_Img_Master_final}.png}
\caption{Pipeline del sistema de extracción de características}\label{\detokenize{content/01/Imagen:figura-wp1-imagen-1}}\end{figure}

\sphinxAtStartPar
Como se puede apreciar, el sistema está formado por cuatro módulos funcionales cada uno de ellos responsable de una función específica. En las siguientes secciones se estudia con mayor detalle cada uno de ellos y las tecnologías hardware\sphinxhyphen{}software usadas.

\sphinxstepscope


\section{M1 \sphinxhyphen{} Captura}
\label{\detokenize{content/01/Modulo-1:m1-captura}}\label{\detokenize{content/01/Modulo-1::doc}}
\sphinxAtStartPar
Este módulo es el responsable de la adquisición de las imágenes. Funcionalmente este módulo se sustenta sobre una barrera láser \sphinxhyphen{} a modo de trigger \sphinxhyphen{} a la entrada del canal (cinta transportadora) que cuando es interrumpida por el paso de un alevín activa la captura de la fotografía. La cámara envía la foto capturada a través del protocolo TCP al sistema de visión en dónde se lleva a cabo el flujo de acciones recogido en el siguiente diagrama.

\begin{figure}[htbp]
\centering
\capstart

\noindent\sphinxincludegraphics[width=0.500\linewidth]{{Modulo-1}.png}
\caption{Flujograma del módulo de captura}\label{\detokenize{content/01/Modulo-1:figura-wp1-imagen-2}}\end{figure}

\sphinxAtStartPar
El motor que controla el flujo de acciones se ha desarrollado sobre la base de Node\sphinxhyphen{}RED. Node\sphinxhyphen{}RED es una herramienta de programación visual basada en flujos desarrollada inicialmente por IBM, que permite integrar dispositivos, APIs y servicios mediante nodos configurables. Su arquitectura está construida sobre Node.js, lo que le proporciona un alto rendimiento y la capacidad de ejecutar procesos en tiempo real sobre hardware ligero (desde servidores industriales hasta dispositivos embebidos como Raspberry Pi). Los flujos en Node\sphinxhyphen{}RED se representan gráficamente, facilitando la implementación de sistemas complejos en entornos de IoT e Industria 4.0. Además, su ecosistema de nodos permite la integración directa con protocolos industriales (p. ej., MQTT, OPC\sphinxhyphen{}UA, Modbus, TCP\sphinxhyphen{}UDP), bases de datos, sistemas de control (PLC) y librerías externas en Python o C++, lo que lo convierte en una plataforma idónea para la adquisición y procesamiento de datos heterogéneos.

\sphinxAtStartPar
En el contexto del módulo de captura de imágenes descrito, Node\sphinxhyphen{}RED ofrece ventajas críticas: permite orquestar la señal de disparo proveniente de una barrera, gestiona la adquisición de la imagen junto con sus metadatos (timestamp, ID, parámetros de cámara…) y aplicar rutinas de validación para garantizar la integridad del frame capturado. El hecho adicional de poder programar funciones personalizadas en JavaScript o integrar scripts externos en Python, permite ejecutar código necesario para validar si la imagen es utilizable (\sphinxcode{\sphinxupquote{¿Frame válido?}}) antes de enviarla a etapas posteriores de procesado, minimizando errores y pérdidas de información. El flujo responsable de la adquisición y los nodos implicados se refleja en la siguiente figura.

\begin{figure}[htbp]
\centering
\capstart

\noindent\sphinxincludegraphics[width=1.000\linewidth]{{Modulo-1_nodered}.png}
\caption{Flujo de acciones responsable del módulo de captura de imagen}\label{\detokenize{content/01/Modulo-1:figura-wp1-imagen-3}}\end{figure}

\sphinxstepscope


\section{M2 \sphinxhyphen{} Calibración}
\label{\detokenize{content/01/Modulo-2:m2-calibracion}}\label{\detokenize{content/01/Modulo-2::doc}}
\sphinxAtStartPar
Para poder extraer las dimensiones reales (mm) de la longitud, anchura y altura es necesario que el sistema disponga de algún tipo de calibración. Para ello la forma más rápida es disponer de un damero (Checkerboard) de dimensiones exactas y conocidas como el de la figura, en el que cada una de las inserciones internas mide exactamente 10 mm de lado.

\begin{figure}[htbp]
\centering
\capstart

\noindent\sphinxincludegraphics[width=0.750\linewidth]{{checkerboard_10x10}.png}
\caption{Checkerdoard (damero) para calibración}\label{\detokenize{content/01/Modulo-2:figura-wp1-imagen-4}}\end{figure}

\sphinxAtStartPar
Si \(𝐿_{px}\) es la longitud medida en píxeles de un objeto cuya longitud real \(L_{mm}\) es conocida (10 mm en nuestro damero), la escala se define como:
\begin{equation*}
\begin{split}
𝑠[mm/px]=\dfrac{L_{mm}}{𝐿_{px}}
\end{split}
\end{equation*}
\sphinxAtStartPar
Para robustez, se usa la mediana de múltiples aristas horizontales y verticales {[}\sphinxhref{https://doi.org/10.1109/34.888718}{Zang, Z., 2020}{]}. La incertidumbre relativa de primer orden es:
\begin{equation}\label{equation:content/01/Modulo-2:incertidumbre_calibracion}
\begin{split}\left(\frac{\sigma_s}{s}\right)^2 \;\approx\;
\left(\frac{\sigma_{L_{\text{mm}}}}{L_{\text{mm}}}\right)^2 \;+\;
\left(\frac{\sigma_{L_{\text{px}}}}{L_{\text{px}}}\right)^2\end{split}
\end{equation}
\sphinxAtStartPar
El flujo UML que resuelve la calibración es el siguiente:

\begin{figure}[htbp]
\centering
\capstart

\noindent\sphinxincludegraphics[width=0.750\linewidth]{{Modulo-2}.png}
\caption{UML proceso de calibración}\label{\detokenize{content/01/Modulo-2:figura-wp1-imagen-5}}\end{figure}

\begin{sphinxuseclass}{cell}\begin{sphinxVerbatimInput}

\begin{sphinxuseclass}{cell_input}
\begin{sphinxVerbatim}[commandchars=\\\{\}]
\PYG{c+c1}{\PYGZsh{}\PYGZsh{} INSTALACION DE LIBRERIAS}
\PYG{c+c1}{\PYGZsh{} Ejecutar la siguiente línea si no se ha previmente}

\PYG{c+c1}{\PYGZsh{}!pip install opencv\PYGZhy{}python}
\end{sphinxVerbatim}

\end{sphinxuseclass}\end{sphinxVerbatimInput}

\end{sphinxuseclass}
\begin{sphinxuseclass}{cell}\begin{sphinxVerbatimInput}

\begin{sphinxuseclass}{cell_input}
\begin{sphinxVerbatim}[commandchars=\\\{\}]
\PYG{c+c1}{\PYGZsh{}\PYGZsh{} IMPORT Y UTILIDAES DE E/S}

\PYG{c+c1}{\PYGZsh{} Importa librerías y define utilidades para leer/escribir JSON con seguridad.}

\PYG{k+kn}{import}\PYG{+w}{ }\PYG{n+nn}{json}
\PYG{k+kn}{from}\PYG{+w}{ }\PYG{n+nn}{dataclasses}\PYG{+w}{ }\PYG{k+kn}{import} \PYG{n}{dataclass}\PYG{p}{,} \PYG{n}{asdict}
\PYG{k+kn}{from}\PYG{+w}{ }\PYG{n+nn}{pathlib}\PYG{+w}{ }\PYG{k+kn}{import} \PYG{n}{Path}
\PYG{k+kn}{from}\PYG{+w}{ }\PYG{n+nn}{typing}\PYG{+w}{ }\PYG{k+kn}{import} \PYG{n}{Optional}\PYG{p}{,} \PYG{n}{Dict}\PYG{p}{,} \PYG{n}{Any}\PYG{p}{,} \PYG{n}{Tuple}

\PYG{k+kn}{import}\PYG{+w}{ }\PYG{n+nn}{numpy}\PYG{+w}{ }\PYG{k}{as}\PYG{+w}{ }\PYG{n+nn}{np}
\PYG{k+kn}{import}\PYG{+w}{ }\PYG{n+nn}{cv2}

\PYG{k}{def}\PYG{+w}{ }\PYG{n+nf}{load\PYGZus{}json}\PYG{p}{(}\PYG{n}{path}\PYG{p}{:} \PYG{n}{Path}\PYG{p}{)} \PYG{o}{\PYGZhy{}}\PYG{o}{\PYGZgt{}} \PYG{n}{Optional}\PYG{p}{[}\PYG{n+nb}{dict}\PYG{p}{]}\PYG{p}{:}
\PYG{+w}{    }\PYG{l+s+sd}{\PYGZdq{}\PYGZdq{}\PYGZdq{}Carga un JSON si existe; devuelve None si no existe.\PYGZdq{}\PYGZdq{}\PYGZdq{}}
    \PYG{k}{if} \PYG{o+ow}{not} \PYG{n}{path}\PYG{o}{.}\PYG{n}{exists}\PYG{p}{(}\PYG{p}{)}\PYG{p}{:}
        \PYG{k}{return} \PYG{k+kc}{None}
    \PYG{k}{with} \PYG{n}{path}\PYG{o}{.}\PYG{n}{open}\PYG{p}{(}\PYG{l+s+s2}{\PYGZdq{}}\PYG{l+s+s2}{r}\PYG{l+s+s2}{\PYGZdq{}}\PYG{p}{,} \PYG{n}{encoding}\PYG{o}{=}\PYG{l+s+s2}{\PYGZdq{}}\PYG{l+s+s2}{utf\PYGZhy{}8}\PYG{l+s+s2}{\PYGZdq{}}\PYG{p}{)} \PYG{k}{as} \PYG{n}{f}\PYG{p}{:}
        \PYG{k}{return} \PYG{n}{json}\PYG{o}{.}\PYG{n}{load}\PYG{p}{(}\PYG{n}{f}\PYG{p}{)}

\PYG{k}{def}\PYG{+w}{ }\PYG{n+nf}{save\PYGZus{}json}\PYG{p}{(}\PYG{n}{path}\PYG{p}{:} \PYG{n}{Path}\PYG{p}{,} \PYG{n}{data}\PYG{p}{:} \PYG{n+nb}{dict}\PYG{p}{)} \PYG{o}{\PYGZhy{}}\PYG{o}{\PYGZgt{}} \PYG{k+kc}{None}\PYG{p}{:}
\PYG{+w}{    }\PYG{l+s+sd}{\PYGZdq{}\PYGZdq{}\PYGZdq{}Guarda un dict como JSON (UTF\PYGZhy{}8, con indentado) y crea carpetas si no existen.\PYGZdq{}\PYGZdq{}\PYGZdq{}}
    \PYG{n}{path}\PYG{o}{.}\PYG{n}{parent}\PYG{o}{.}\PYG{n}{mkdir}\PYG{p}{(}\PYG{n}{parents}\PYG{o}{=}\PYG{k+kc}{True}\PYG{p}{,} \PYG{n}{exist\PYGZus{}ok}\PYG{o}{=}\PYG{k+kc}{True}\PYG{p}{)}
    \PYG{k}{with} \PYG{n}{path}\PYG{o}{.}\PYG{n}{open}\PYG{p}{(}\PYG{l+s+s2}{\PYGZdq{}}\PYG{l+s+s2}{w}\PYG{l+s+s2}{\PYGZdq{}}\PYG{p}{,} \PYG{n}{encoding}\PYG{o}{=}\PYG{l+s+s2}{\PYGZdq{}}\PYG{l+s+s2}{utf\PYGZhy{}8}\PYG{l+s+s2}{\PYGZdq{}}\PYG{p}{)} \PYG{k}{as} \PYG{n}{f}\PYG{p}{:}
        \PYG{n}{json}\PYG{o}{.}\PYG{n}{dump}\PYG{p}{(}\PYG{n}{data}\PYG{p}{,} \PYG{n}{f}\PYG{p}{,} \PYG{n}{indent}\PYG{o}{=}\PYG{l+m+mi}{2}\PYG{p}{,} \PYG{n}{ensure\PYGZus{}ascii}\PYG{o}{=}\PYG{k+kc}{False}\PYG{p}{)}
\end{sphinxVerbatim}

\end{sphinxuseclass}\end{sphinxVerbatimInput}

\end{sphinxuseclass}
\begin{sphinxuseclass}{cell}\begin{sphinxVerbatimInput}

\begin{sphinxuseclass}{cell_input}
\begin{sphinxVerbatim}[commandchars=\\\{\}]
\PYG{c+c1}{\PYGZsh{}\PYGZsh{} CONFIGURACION }

\PYG{c+c1}{\PYGZsh{} fija rutas y parámetros del patrón para el damero}

\PYG{c+c1}{\PYGZsh{} === CONFIG (ajustada al damero de calibrado) ===================================}
\PYG{n}{CALIB\PYGZus{}FILE} \PYG{o}{=} \PYG{n}{Path}\PYG{p}{(}\PYG{l+s+s2}{\PYGZdq{}}\PYG{l+s+s2}{./calibracion\PYGZus{}px\PYGZus{}mm.json}\PYG{l+s+s2}{\PYGZdq{}}\PYG{p}{)}

\PYG{c+c1}{\PYGZsh{} Patrón de referencia: DAMERO 10x10 cuadrados =\PYGZgt{} 9x9 intersecciones internas}
\PYG{n}{CHECKERBOARD\PYGZus{}SQUARE\PYGZus{}MM} \PYG{o}{=} \PYG{l+m+mf}{10.0}
\PYG{n}{CHECKERBOARD\PYGZus{}SQUARES\PYGZus{}COLS} \PYG{o}{=} \PYG{l+m+mi}{10}   \PYG{c+c1}{\PYGZsh{} número de cuadrados horizontales}
\PYG{n}{CHECKERBOARD\PYGZus{}SQUARES\PYGZus{}ROWS} \PYG{o}{=} \PYG{l+m+mi}{10}   \PYG{c+c1}{\PYGZsh{} número de cuadrados verticales}
\PYG{n}{CHECKERBOARD\PYGZus{}INTERNAL\PYGZus{}COLS} \PYG{o}{=} \PYG{n}{CHECKERBOARD\PYGZus{}SQUARES\PYGZus{}COLS} \PYG{o}{\PYGZhy{}} \PYG{l+m+mi}{1}  \PYG{c+c1}{\PYGZsh{} 9 intersecciones internas}
\PYG{n}{CHECKERBOARD\PYGZus{}INTERNAL\PYGZus{}ROWS} \PYG{o}{=} \PYG{n}{CHECKERBOARD\PYGZus{}SQUARES\PYGZus{}ROWS} \PYG{o}{\PYGZhy{}} \PYG{l+m+mi}{1}  \PYG{c+c1}{\PYGZsh{} 9 intersecciones internas}

\PYG{c+c1}{\PYGZsh{} Agregación robusta de distancias entre esquinas: \PYGZdq{}median\PYGZdq{} o \PYGZdq{}mean\PYGZdq{}}
\PYG{n}{AGGREGATION} \PYG{o}{=} \PYG{l+s+s2}{\PYGZdq{}}\PYG{l+s+s2}{median}\PYG{l+s+s2}{\PYGZdq{}}

\PYG{c+c1}{\PYGZsh{} Usa la imagen del  damero de referencia.}
\PYG{n}{IMG\PYGZus{}PATH} \PYG{o}{=} \PYG{n}{Path}\PYG{p}{(}\PYG{l+s+s2}{\PYGZdq{}}\PYG{l+s+s2}{.././assets/checkerboard\PYGZus{}10mm\PYGZus{}200pxmm.png}\PYG{l+s+s2}{\PYGZdq{}}\PYG{p}{)}
\end{sphinxVerbatim}

\end{sphinxuseclass}\end{sphinxVerbatimInput}

\end{sphinxuseclass}
\begin{sphinxuseclass}{cell}\begin{sphinxVerbatimInput}

\begin{sphinxuseclass}{cell_input}
\begin{sphinxVerbatim}[commandchars=\\\{\}]
\PYG{c+c1}{\PYGZsh{}\PYGZsh{} ESTRUCTURA DE CALIBRACION Y PERSISTENCIA}

\PYG{c+c1}{\PYGZsh{} Define la estructura (dataclass) que se guardará en JSON y las funciones de lectura/escritura}
\PYG{c+c1}{\PYGZsh{} del fichero de calibración}

\PYG{n+nd}{@dataclass}
\PYG{k}{class}\PYG{+w}{ }\PYG{n+nc}{Calibration}\PYG{p}{:}
\PYG{+w}{    }\PYG{l+s+sd}{\PYGZdq{}\PYGZdq{}\PYGZdq{}}
\PYG{l+s+sd}{    Estructura persistente de calibración.}
\PYG{l+s+sd}{    \PYGZhy{} mm\PYGZus{}per\PYGZus{}px: escala (milímetros por píxel).}
\PYG{l+s+sd}{    \PYGZhy{} method: método usado (\PYGZdq{}checkerboard\PYGZdq{}).}
\PYG{l+s+sd}{    \PYGZhy{} meta: metadatos/diagnósticos (estadísticos de distancias en píxeles, etc.).}
\PYG{l+s+sd}{    \PYGZdq{}\PYGZdq{}\PYGZdq{}}
    \PYG{n}{mm\PYGZus{}per\PYGZus{}px}\PYG{p}{:} \PYG{n+nb}{float}
    \PYG{n}{method}\PYG{p}{:} \PYG{n+nb}{str}
    \PYG{n}{meta}\PYG{p}{:} \PYG{n}{Dict}\PYG{p}{[}\PYG{n+nb}{str}\PYG{p}{,} \PYG{n}{Any}\PYG{p}{]}

\PYG{k}{def}\PYG{+w}{ }\PYG{n+nf}{load\PYGZus{}calibration}\PYG{p}{(}\PYG{n}{path}\PYG{p}{:} \PYG{n}{Path}\PYG{p}{)} \PYG{o}{\PYGZhy{}}\PYG{o}{\PYGZgt{}} \PYG{n}{Optional}\PYG{p}{[}\PYG{n}{Calibration}\PYG{p}{]}\PYG{p}{:}
\PYG{+w}{    }\PYG{l+s+sd}{\PYGZdq{}\PYGZdq{}\PYGZdq{}Lee la calibración desde JSON; devuelve None si no existe o si hay formato inválido.\PYGZdq{}\PYGZdq{}\PYGZdq{}}
    \PYG{n}{data} \PYG{o}{=} \PYG{n}{load\PYGZus{}json}\PYG{p}{(}\PYG{n}{path}\PYG{p}{)}
    \PYG{k}{if} \PYG{n}{data} \PYG{o+ow}{is} \PYG{k+kc}{None}\PYG{p}{:}
        \PYG{k}{return} \PYG{k+kc}{None}
    \PYG{k}{try}\PYG{p}{:}
        \PYG{k}{return} \PYG{n}{Calibration}\PYG{p}{(}
            \PYG{n}{mm\PYGZus{}per\PYGZus{}px}\PYG{o}{=}\PYG{n+nb}{float}\PYG{p}{(}\PYG{n}{data}\PYG{p}{[}\PYG{l+s+s2}{\PYGZdq{}}\PYG{l+s+s2}{mm\PYGZus{}per\PYGZus{}px}\PYG{l+s+s2}{\PYGZdq{}}\PYG{p}{]}\PYG{p}{)}\PYG{p}{,}
            \PYG{n}{method}\PYG{o}{=}\PYG{n+nb}{str}\PYG{p}{(}\PYG{n}{data}\PYG{p}{[}\PYG{l+s+s2}{\PYGZdq{}}\PYG{l+s+s2}{method}\PYG{l+s+s2}{\PYGZdq{}}\PYG{p}{]}\PYG{p}{)}\PYG{p}{,}
            \PYG{n}{meta}\PYG{o}{=}\PYG{n+nb}{dict}\PYG{p}{(}\PYG{n}{data}\PYG{p}{[}\PYG{l+s+s2}{\PYGZdq{}}\PYG{l+s+s2}{meta}\PYG{l+s+s2}{\PYGZdq{}}\PYG{p}{]}\PYG{p}{)}\PYG{p}{,}
        \PYG{p}{)}
    \PYG{k}{except} \PYG{n+ne}{Exception}\PYG{p}{:}
        \PYG{k}{return} \PYG{k+kc}{None}

\PYG{k}{def}\PYG{+w}{ }\PYG{n+nf}{save\PYGZus{}calibration}\PYG{p}{(}\PYG{n}{path}\PYG{p}{:} \PYG{n}{Path}\PYG{p}{,} \PYG{n}{cal}\PYG{p}{:} \PYG{n}{Calibration}\PYG{p}{)} \PYG{o}{\PYGZhy{}}\PYG{o}{\PYGZgt{}} \PYG{k+kc}{None}\PYG{p}{:}
\PYG{+w}{    }\PYG{l+s+sd}{\PYGZdq{}\PYGZdq{}\PYGZdq{}Persistencia de calibración a JSON.\PYGZdq{}\PYGZdq{}\PYGZdq{}}
    \PYG{n}{save\PYGZus{}json}\PYG{p}{(}\PYG{n}{path}\PYG{p}{,} \PYG{n}{asdict}\PYG{p}{(}\PYG{n}{cal}\PYG{p}{)}\PYG{p}{)}
\end{sphinxVerbatim}

\end{sphinxuseclass}\end{sphinxVerbatimInput}

\end{sphinxuseclass}
\begin{sphinxuseclass}{cell}\begin{sphinxVerbatimInput}

\begin{sphinxuseclass}{cell_input}
\begin{sphinxVerbatim}[commandchars=\\\{\}]
\PYG{c+c1}{\PYGZsh{}\PYGZsh{} DETECCIÓN DEL DAMERO Y CÁLCULO mm/px}

\PYG{k}{def}\PYG{+w}{ }\PYG{n+nf}{detect\PYGZus{}checkerboard\PYGZus{}scale}\PYG{p}{(}
    \PYG{n}{img\PYGZus{}bgr}\PYG{p}{:} \PYG{n}{np}\PYG{o}{.}\PYG{n}{ndarray}\PYG{p}{,}
    \PYG{n}{cols\PYGZus{}internal}\PYG{p}{:} \PYG{n+nb}{int}\PYG{p}{,}
    \PYG{n}{rows\PYGZus{}internal}\PYG{p}{:} \PYG{n+nb}{int}\PYG{p}{,}
    \PYG{n}{square\PYGZus{}mm}\PYG{p}{:} \PYG{n+nb}{float}\PYG{p}{,}
    \PYG{n}{aggregation}\PYG{p}{:} \PYG{n+nb}{str} \PYG{o}{=} \PYG{l+s+s2}{\PYGZdq{}}\PYG{l+s+s2}{median}\PYG{l+s+s2}{\PYGZdq{}}
\PYG{p}{)} \PYG{o}{\PYGZhy{}}\PYG{o}{\PYGZgt{}} \PYG{n}{Optional}\PYG{p}{[}\PYG{n}{Tuple}\PYG{p}{[}\PYG{n+nb}{float}\PYG{p}{,} \PYG{n}{Dict}\PYG{p}{[}\PYG{n+nb}{str}\PYG{p}{,} \PYG{n}{Any}\PYG{p}{]}\PYG{p}{]}\PYG{p}{]}\PYG{p}{:}
\PYG{+w}{    }\PYG{l+s+sd}{\PYGZdq{}\PYGZdq{}\PYGZdq{}}
\PYG{l+s+sd}{    Detecta un damero de (cols\PYGZus{}internal x rows\PYGZus{}internal) intersecciones internas.}
\PYG{l+s+sd}{    Calcula mm/px usando la mediana (por defecto) de distancias entre esquinas adyacentes.}

\PYG{l+s+sd}{    Devuelve: (mm\PYGZus{}per\PYGZus{}px, diagnostics) si detecta; None si no detecta.}
\PYG{l+s+sd}{    \PYGZdq{}\PYGZdq{}\PYGZdq{}}
    \PYG{n}{gray} \PYG{o}{=} \PYG{n}{cv2}\PYG{o}{.}\PYG{n}{cvtColor}\PYG{p}{(}\PYG{n}{img\PYGZus{}bgr}\PYG{p}{,} \PYG{n}{cv2}\PYG{o}{.}\PYG{n}{COLOR\PYGZus{}BGR2GRAY}\PYG{p}{)}

    \PYG{n}{flags} \PYG{o}{=} \PYG{n}{cv2}\PYG{o}{.}\PYG{n}{CALIB\PYGZus{}CB\PYGZus{}ADAPTIVE\PYGZus{}THRESH} \PYG{o}{|} \PYG{n}{cv2}\PYG{o}{.}\PYG{n}{CALIB\PYGZus{}CB\PYGZus{}NORMALIZE\PYGZus{}IMAGE}
    \PYG{n}{found}\PYG{p}{,} \PYG{n}{corners} \PYG{o}{=} \PYG{n}{cv2}\PYG{o}{.}\PYG{n}{findChessboardCorners}\PYG{p}{(}\PYG{n}{gray}\PYG{p}{,} \PYG{p}{(}\PYG{n}{cols\PYGZus{}internal}\PYG{p}{,} \PYG{n}{rows\PYGZus{}internal}\PYG{p}{)}\PYG{p}{,} \PYG{n}{flags}\PYG{p}{)}

    \PYG{k}{if} \PYG{o+ow}{not} \PYG{n}{found} \PYG{o+ow}{or} \PYG{n}{corners} \PYG{o+ow}{is} \PYG{k+kc}{None}\PYG{p}{:}
        \PYG{k}{return} \PYG{k+kc}{None}

    \PYG{c+c1}{\PYGZsh{} Refinamiento subpíxel}
    \PYG{n}{criteria} \PYG{o}{=} \PYG{p}{(}\PYG{n}{cv2}\PYG{o}{.}\PYG{n}{TERM\PYGZus{}CRITERIA\PYGZus{}EPS} \PYG{o}{+} \PYG{n}{cv2}\PYG{o}{.}\PYG{n}{TERM\PYGZus{}CRITERIA\PYGZus{}MAX\PYGZus{}ITER}\PYG{p}{,} \PYG{l+m+mi}{50}\PYG{p}{,} \PYG{l+m+mf}{1e\PYGZhy{}4}\PYG{p}{)}
    \PYG{n}{corners} \PYG{o}{=} \PYG{n}{cv2}\PYG{o}{.}\PYG{n}{cornerSubPix}\PYG{p}{(}\PYG{n}{gray}\PYG{p}{,} \PYG{n}{corners}\PYG{p}{,} \PYG{p}{(}\PYG{l+m+mi}{11}\PYG{p}{,} \PYG{l+m+mi}{11}\PYG{p}{)}\PYG{p}{,} \PYG{p}{(}\PYG{o}{\PYGZhy{}}\PYG{l+m+mi}{1}\PYG{p}{,} \PYG{o}{\PYGZhy{}}\PYG{l+m+mi}{1}\PYG{p}{)}\PYG{p}{,} \PYG{n}{criteria}\PYG{p}{)}
    \PYG{n}{corners} \PYG{o}{=} \PYG{n}{corners}\PYG{o}{.}\PYG{n}{reshape}\PYG{p}{(}\PYG{o}{\PYGZhy{}}\PYG{l+m+mi}{1}\PYG{p}{,} \PYG{l+m+mi}{2}\PYG{p}{)}  \PYG{c+c1}{\PYGZsh{} N x 2}

    \PYG{c+c1}{\PYGZsh{} Distancias horizontales}
    \PYG{n}{horiz} \PYG{o}{=} \PYG{p}{[}\PYG{p}{]}
    \PYG{k}{for} \PYG{n}{r} \PYG{o+ow}{in} \PYG{n+nb}{range}\PYG{p}{(}\PYG{n}{rows\PYGZus{}internal}\PYG{p}{)}\PYG{p}{:}
        \PYG{k}{for} \PYG{n}{c} \PYG{o+ow}{in} \PYG{n+nb}{range}\PYG{p}{(}\PYG{n}{cols\PYGZus{}internal} \PYG{o}{\PYGZhy{}} \PYG{l+m+mi}{1}\PYG{p}{)}\PYG{p}{:}
            \PYG{n}{i} \PYG{o}{=} \PYG{n}{r} \PYG{o}{*} \PYG{n}{cols\PYGZus{}internal} \PYG{o}{+} \PYG{n}{c}
            \PYG{n}{j} \PYG{o}{=} \PYG{n}{r} \PYG{o}{*} \PYG{n}{cols\PYGZus{}internal} \PYG{o}{+} \PYG{p}{(}\PYG{n}{c} \PYG{o}{+} \PYG{l+m+mi}{1}\PYG{p}{)}
            \PYG{n}{d} \PYG{o}{=} \PYG{n}{np}\PYG{o}{.}\PYG{n}{linalg}\PYG{o}{.}\PYG{n}{norm}\PYG{p}{(}\PYG{n}{corners}\PYG{p}{[}\PYG{n}{i}\PYG{p}{]} \PYG{o}{\PYGZhy{}} \PYG{n}{corners}\PYG{p}{[}\PYG{n}{j}\PYG{p}{]}\PYG{p}{)}
            \PYG{k}{if} \PYG{n}{d} \PYG{o}{\PYGZgt{}} \PYG{l+m+mi}{0}\PYG{p}{:}
                \PYG{n}{horiz}\PYG{o}{.}\PYG{n}{append}\PYG{p}{(}\PYG{n}{d}\PYG{p}{)}

    \PYG{c+c1}{\PYGZsh{} Distancias verticales}
    \PYG{n}{vert} \PYG{o}{=} \PYG{p}{[}\PYG{p}{]}
    \PYG{k}{for} \PYG{n}{r} \PYG{o+ow}{in} \PYG{n+nb}{range}\PYG{p}{(}\PYG{n}{rows\PYGZus{}internal} \PYG{o}{\PYGZhy{}} \PYG{l+m+mi}{1}\PYG{p}{)}\PYG{p}{:}
        \PYG{k}{for} \PYG{n}{c} \PYG{o+ow}{in} \PYG{n+nb}{range}\PYG{p}{(}\PYG{n}{cols\PYGZus{}internal}\PYG{p}{)}\PYG{p}{:}
            \PYG{n}{i} \PYG{o}{=} \PYG{n}{r} \PYG{o}{*} \PYG{n}{cols\PYGZus{}internal} \PYG{o}{+} \PYG{n}{c}
            \PYG{n}{j} \PYG{o}{=} \PYG{p}{(}\PYG{n}{r} \PYG{o}{+} \PYG{l+m+mi}{1}\PYG{p}{)} \PYG{o}{*} \PYG{n}{cols\PYGZus{}internal} \PYG{o}{+} \PYG{n}{c}
            \PYG{n}{d} \PYG{o}{=} \PYG{n}{np}\PYG{o}{.}\PYG{n}{linalg}\PYG{o}{.}\PYG{n}{norm}\PYG{p}{(}\PYG{n}{corners}\PYG{p}{[}\PYG{n}{i}\PYG{p}{]} \PYG{o}{\PYGZhy{}} \PYG{n}{corners}\PYG{p}{[}\PYG{n}{j}\PYG{p}{]}\PYG{p}{)}
            \PYG{k}{if} \PYG{n}{d} \PYG{o}{\PYGZgt{}} \PYG{l+m+mi}{0}\PYG{p}{:}
                \PYG{n}{vert}\PYG{o}{.}\PYG{n}{append}\PYG{p}{(}\PYG{n}{d}\PYG{p}{)}

    \PYG{n}{dists} \PYG{o}{=} \PYG{n}{np}\PYG{o}{.}\PYG{n}{array}\PYG{p}{(}\PYG{n}{horiz} \PYG{o}{+} \PYG{n}{vert}\PYG{p}{,} \PYG{n}{dtype}\PYG{o}{=}\PYG{n+nb}{float}\PYG{p}{)}
    \PYG{k}{if} \PYG{n}{dists}\PYG{o}{.}\PYG{n}{size} \PYG{o}{==} \PYG{l+m+mi}{0}\PYG{p}{:}
        \PYG{k}{return} \PYG{k+kc}{None}

    \PYG{c+c1}{\PYGZsh{} px por cuadrado (mediana por robustez ante outliers)}
    \PYG{n}{px\PYGZus{}per\PYGZus{}square} \PYG{o}{=} \PYG{n}{np}\PYG{o}{.}\PYG{n}{median}\PYG{p}{(}\PYG{n}{dists}\PYG{p}{)} \PYG{k}{if} \PYG{n}{aggregation} \PYG{o}{==} \PYG{l+s+s2}{\PYGZdq{}}\PYG{l+s+s2}{median}\PYG{l+s+s2}{\PYGZdq{}} \PYG{k}{else} \PYG{n}{np}\PYG{o}{.}\PYG{n}{mean}\PYG{p}{(}\PYG{n}{dists}\PYG{p}{)}
    \PYG{n}{mm\PYGZus{}per\PYGZus{}px}     \PYG{o}{=} \PYG{n}{square\PYGZus{}mm} \PYG{o}{/} \PYG{n}{px\PYGZus{}per\PYGZus{}square}

    \PYG{n}{diagnostics} \PYG{o}{=} \PYG{p}{\PYGZob{}}
        \PYG{l+s+s2}{\PYGZdq{}}\PYG{l+s+s2}{pattern}\PYG{l+s+s2}{\PYGZdq{}}\PYG{p}{:} \PYG{l+s+s2}{\PYGZdq{}}\PYG{l+s+s2}{checkerboard}\PYG{l+s+s2}{\PYGZdq{}}\PYG{p}{,}
        \PYG{l+s+s2}{\PYGZdq{}}\PYG{l+s+s2}{internal\PYGZus{}cols}\PYG{l+s+s2}{\PYGZdq{}}\PYG{p}{:} \PYG{n+nb}{int}\PYG{p}{(}\PYG{n}{cols\PYGZus{}internal}\PYG{p}{)}\PYG{p}{,}
        \PYG{l+s+s2}{\PYGZdq{}}\PYG{l+s+s2}{internal\PYGZus{}rows}\PYG{l+s+s2}{\PYGZdq{}}\PYG{p}{:} \PYG{n+nb}{int}\PYG{p}{(}\PYG{n}{rows\PYGZus{}internal}\PYG{p}{)}\PYG{p}{,}
        \PYG{l+s+s2}{\PYGZdq{}}\PYG{l+s+s2}{square\PYGZus{}mm}\PYG{l+s+s2}{\PYGZdq{}}\PYG{p}{:} \PYG{n+nb}{float}\PYG{p}{(}\PYG{n}{square\PYGZus{}mm}\PYG{p}{)}\PYG{p}{,}
        \PYG{l+s+s2}{\PYGZdq{}}\PYG{l+s+s2}{edges\PYGZus{}count}\PYG{l+s+s2}{\PYGZdq{}}\PYG{p}{:} \PYG{n+nb}{int}\PYG{p}{(}\PYG{n}{dists}\PYG{o}{.}\PYG{n}{size}\PYG{p}{)}\PYG{p}{,}
        \PYG{l+s+s2}{\PYGZdq{}}\PYG{l+s+s2}{px\PYGZus{}per\PYGZus{}square\PYGZus{}summary}\PYG{l+s+s2}{\PYGZdq{}}\PYG{p}{:} \PYG{p}{\PYGZob{}}
            \PYG{l+s+s2}{\PYGZdq{}}\PYG{l+s+s2}{mean}\PYG{l+s+s2}{\PYGZdq{}}\PYG{p}{:} \PYG{n+nb}{float}\PYG{p}{(}\PYG{n}{np}\PYG{o}{.}\PYG{n}{mean}\PYG{p}{(}\PYG{n}{dists}\PYG{p}{)}\PYG{p}{)}\PYG{p}{,}
            \PYG{l+s+s2}{\PYGZdq{}}\PYG{l+s+s2}{median}\PYG{l+s+s2}{\PYGZdq{}}\PYG{p}{:} \PYG{n+nb}{float}\PYG{p}{(}\PYG{n}{np}\PYG{o}{.}\PYG{n}{median}\PYG{p}{(}\PYG{n}{dists}\PYG{p}{)}\PYG{p}{)}\PYG{p}{,}
            \PYG{l+s+s2}{\PYGZdq{}}\PYG{l+s+s2}{std}\PYG{l+s+s2}{\PYGZdq{}}\PYG{p}{:} \PYG{n+nb}{float}\PYG{p}{(}\PYG{n}{np}\PYG{o}{.}\PYG{n}{std}\PYG{p}{(}\PYG{n}{dists}\PYG{p}{,} \PYG{n}{ddof}\PYG{o}{=}\PYG{l+m+mi}{1}\PYG{p}{)}\PYG{p}{)} \PYG{k}{if} \PYG{n}{dists}\PYG{o}{.}\PYG{n}{size} \PYG{o}{\PYGZgt{}} \PYG{l+m+mi}{1} \PYG{k}{else} \PYG{l+m+mf}{0.0}
        \PYG{p}{\PYGZcb{}}\PYG{p}{,}
        \PYG{l+s+s2}{\PYGZdq{}}\PYG{l+s+s2}{aggregation}\PYG{l+s+s2}{\PYGZdq{}}\PYG{p}{:} \PYG{n}{aggregation}
    \PYG{p}{\PYGZcb{}}
    \PYG{k}{return} \PYG{n}{mm\PYGZus{}per\PYGZus{}px}\PYG{p}{,} \PYG{n}{diagnostics}
\end{sphinxVerbatim}

\end{sphinxuseclass}\end{sphinxVerbatimInput}

\end{sphinxuseclass}
\begin{sphinxuseclass}{cell}\begin{sphinxVerbatimInput}

\begin{sphinxuseclass}{cell_input}
\begin{sphinxVerbatim}[commandchars=\\\{\}]
\PYG{c+c1}{\PYGZsh{}\PYGZsh{} FUNCIÓN DE CALIBRACIÓN}

\PYG{c+c1}{\PYGZsh{} implementa literalmente el diagrama de decisiones: }
\PYG{c+c1}{\PYGZsh{} \PYGZhy{} usar calibración vigente (permanente) salvo que el usuario fuerce recalibración;}
\PYG{c+c1}{\PYGZsh{} \PYGZhy{} si no hay calibración o se fuerza, intenta detectar y calcular; }
\PYG{c+c1}{\PYGZsh{} \PYGZhy{} si falla, `sin\PYGZus{}escala`}

\PYG{k}{def}\PYG{+w}{ }\PYG{n+nf}{run\PYGZus{}calibration\PYGZus{}pipeline\PYGZus{}checkerboard}\PYG{p}{(}
    \PYG{n}{calib\PYGZus{}file}\PYG{p}{:} \PYG{n}{Path}\PYG{p}{,}
    \PYG{n}{img\PYGZus{}path}\PYG{p}{:} \PYG{n}{Path}\PYG{p}{,}
    \PYG{n}{cols\PYGZus{}internal}\PYG{p}{:} \PYG{n+nb}{int}\PYG{p}{,}
    \PYG{n}{rows\PYGZus{}internal}\PYG{p}{:} \PYG{n+nb}{int}\PYG{p}{,}
    \PYG{n}{square\PYGZus{}mm}\PYG{p}{:} \PYG{n+nb}{float}\PYG{p}{,}
    \PYG{n}{aggregation}\PYG{p}{:} \PYG{n+nb}{str} \PYG{o}{=} \PYG{l+s+s2}{\PYGZdq{}}\PYG{l+s+s2}{median}\PYG{l+s+s2}{\PYGZdq{}}\PYG{p}{,}
    \PYG{n}{force\PYGZus{}recalibrate}\PYG{p}{:} \PYG{n+nb}{bool} \PYG{o}{=} \PYG{k+kc}{False}
\PYG{p}{)} \PYG{o}{\PYGZhy{}}\PYG{o}{\PYGZgt{}} \PYG{n}{Dict}\PYG{p}{[}\PYG{n+nb}{str}\PYG{p}{,} \PYG{n}{Any}\PYG{p}{]}\PYG{p}{:}
\PYG{+w}{    }\PYG{l+s+sd}{\PYGZdq{}\PYGZdq{}\PYGZdq{}}
\PYG{l+s+sd}{    UML “M2 \PYGZhy{} Calibración de escala”:}

\PYG{l+s+sd}{    if (¿Calibración vigente?) then (Sí)}
\PYG{l+s+sd}{        :Usar escala px→mm guardada;}
\PYG{l+s+sd}{    else (No)}
\PYG{l+s+sd}{        :Detectar patrón referencia;}
\PYG{l+s+sd}{        if (¿Detectado?) then (Sí)}
\PYG{l+s+sd}{            :Calcular escala px→mm; (y guardar)}
\PYG{l+s+sd}{        else (No)}
\PYG{l+s+sd}{            :Marcar sin\PYGZus{}escala;}

\PYG{l+s+sd}{    \PYGZsq{}Vigente\PYGZsq{} = existe calibración en disco y NO se fuerza recalibración.}
\PYG{l+s+sd}{    \PYGZdq{}\PYGZdq{}\PYGZdq{}}
    \PYG{c+c1}{\PYGZsh{} 1) ¿Calibración \PYGZsq{}vigente\PYGZsq{}?}
    \PYG{k}{if} \PYG{o+ow}{not} \PYG{n}{force\PYGZus{}recalibrate}\PYG{p}{:}
        \PYG{n}{cal} \PYG{o}{=} \PYG{n}{load\PYGZus{}calibration}\PYG{p}{(}\PYG{n}{calib\PYGZus{}file}\PYG{p}{)}
        \PYG{k}{if} \PYG{n}{cal} \PYG{o+ow}{is} \PYG{o+ow}{not} \PYG{k+kc}{None} \PYG{o+ow}{and} \PYG{n}{cal}\PYG{o}{.}\PYG{n}{method} \PYG{o}{==} \PYG{l+s+s2}{\PYGZdq{}}\PYG{l+s+s2}{checkerboard}\PYG{l+s+s2}{\PYGZdq{}}\PYG{p}{:}
            \PYG{k}{return} \PYG{p}{\PYGZob{}}
                \PYG{l+s+s2}{\PYGZdq{}}\PYG{l+s+s2}{estado}\PYG{l+s+s2}{\PYGZdq{}}\PYG{p}{:} \PYG{l+s+s2}{\PYGZdq{}}\PYG{l+s+s2}{vigente}\PYG{l+s+s2}{\PYGZdq{}}\PYG{p}{,}
                \PYG{l+s+s2}{\PYGZdq{}}\PYG{l+s+s2}{mm\PYGZus{}per\PYGZus{}px}\PYG{l+s+s2}{\PYGZdq{}}\PYG{p}{:} \PYG{n}{cal}\PYG{o}{.}\PYG{n}{mm\PYGZus{}per\PYGZus{}px}\PYG{p}{,}
                \PYG{l+s+s2}{\PYGZdq{}}\PYG{l+s+s2}{method}\PYG{l+s+s2}{\PYGZdq{}}\PYG{p}{:} \PYG{n}{cal}\PYG{o}{.}\PYG{n}{method}\PYG{p}{,}
                \PYG{l+s+s2}{\PYGZdq{}}\PYG{l+s+s2}{meta}\PYG{l+s+s2}{\PYGZdq{}}\PYG{p}{:} \PYG{n}{cal}\PYG{o}{.}\PYG{n}{meta}
            \PYG{p}{\PYGZcb{}}

    \PYG{c+c1}{\PYGZsh{} 2) No vigente (o se fuerza) \PYGZhy{}\PYGZgt{} Detectar patrón referencia}
    \PYG{k}{if} \PYG{o+ow}{not} \PYG{n}{img\PYGZus{}path}\PYG{o}{.}\PYG{n}{exists}\PYG{p}{(}\PYG{p}{)}\PYG{p}{:}
        \PYG{k}{return} \PYG{p}{\PYGZob{}}\PYG{l+s+s2}{\PYGZdq{}}\PYG{l+s+s2}{estado}\PYG{l+s+s2}{\PYGZdq{}}\PYG{p}{:} \PYG{l+s+s2}{\PYGZdq{}}\PYG{l+s+s2}{sin\PYGZus{}escala}\PYG{l+s+s2}{\PYGZdq{}}\PYG{p}{,} \PYG{l+s+s2}{\PYGZdq{}}\PYG{l+s+s2}{motivo}\PYG{l+s+s2}{\PYGZdq{}}\PYG{p}{:} \PYG{l+s+sa}{f}\PYG{l+s+s2}{\PYGZdq{}}\PYG{l+s+s2}{Imagen no encontrada: }\PYG{l+s+si}{\PYGZob{}}\PYG{n+nb}{str}\PYG{p}{(}\PYG{n}{img\PYGZus{}path}\PYG{p}{)}\PYG{l+s+si}{\PYGZcb{}}\PYG{l+s+s2}{\PYGZdq{}}\PYG{p}{\PYGZcb{}}

    \PYG{n}{img} \PYG{o}{=} \PYG{n}{cv2}\PYG{o}{.}\PYG{n}{imread}\PYG{p}{(}\PYG{n+nb}{str}\PYG{p}{(}\PYG{n}{img\PYGZus{}path}\PYG{p}{)}\PYG{p}{)}
    \PYG{k}{if} \PYG{n}{img} \PYG{o+ow}{is} \PYG{k+kc}{None}\PYG{p}{:}
        \PYG{k}{return} \PYG{p}{\PYGZob{}}\PYG{l+s+s2}{\PYGZdq{}}\PYG{l+s+s2}{estado}\PYG{l+s+s2}{\PYGZdq{}}\PYG{p}{:} \PYG{l+s+s2}{\PYGZdq{}}\PYG{l+s+s2}{sin\PYGZus{}escala}\PYG{l+s+s2}{\PYGZdq{}}\PYG{p}{,} \PYG{l+s+s2}{\PYGZdq{}}\PYG{l+s+s2}{motivo}\PYG{l+s+s2}{\PYGZdq{}}\PYG{p}{:} \PYG{l+s+sa}{f}\PYG{l+s+s2}{\PYGZdq{}}\PYG{l+s+s2}{No se pudo abrir la imagen: }\PYG{l+s+si}{\PYGZob{}}\PYG{n+nb}{str}\PYG{p}{(}\PYG{n}{img\PYGZus{}path}\PYG{p}{)}\PYG{l+s+si}{\PYGZcb{}}\PYG{l+s+s2}{\PYGZdq{}}\PYG{p}{\PYGZcb{}}

    \PYG{n}{res} \PYG{o}{=} \PYG{n}{detect\PYGZus{}checkerboard\PYGZus{}scale}\PYG{p}{(}
        \PYG{n}{img\PYGZus{}bgr}\PYG{o}{=}\PYG{n}{img}\PYG{p}{,}
        \PYG{n}{cols\PYGZus{}internal}\PYG{o}{=}\PYG{n}{cols\PYGZus{}internal}\PYG{p}{,}
        \PYG{n}{rows\PYGZus{}internal}\PYG{o}{=}\PYG{n}{rows\PYGZus{}internal}\PYG{p}{,}
        \PYG{n}{square\PYGZus{}mm}\PYG{o}{=}\PYG{n}{square\PYGZus{}mm}\PYG{p}{,}
        \PYG{n}{aggregation}\PYG{o}{=}\PYG{n}{aggregation}
    \PYG{p}{)}

    \PYG{c+c1}{\PYGZsh{} 3) ¿Detectado?}
    \PYG{k}{if} \PYG{n}{res} \PYG{o+ow}{is} \PYG{k+kc}{None}\PYG{p}{:}
        \PYG{k}{return} \PYG{p}{\PYGZob{}}\PYG{l+s+s2}{\PYGZdq{}}\PYG{l+s+s2}{estado}\PYG{l+s+s2}{\PYGZdq{}}\PYG{p}{:} \PYG{l+s+s2}{\PYGZdq{}}\PYG{l+s+s2}{sin\PYGZus{}escala}\PYG{l+s+s2}{\PYGZdq{}}\PYG{p}{,} \PYG{l+s+s2}{\PYGZdq{}}\PYG{l+s+s2}{motivo}\PYG{l+s+s2}{\PYGZdq{}}\PYG{p}{:} \PYG{l+s+s2}{\PYGZdq{}}\PYG{l+s+s2}{No se detectó el damero en la imagen}\PYG{l+s+s2}{\PYGZdq{}}\PYG{p}{\PYGZcb{}}

    \PYG{c+c1}{\PYGZsh{} 4) Calcular escala y guardar}
    \PYG{n}{mm\PYGZus{}per\PYGZus{}px}\PYG{p}{,} \PYG{n}{diag} \PYG{o}{=} \PYG{n}{res}
    \PYG{n}{cal} \PYG{o}{=} \PYG{n}{Calibration}\PYG{p}{(}
        \PYG{n}{mm\PYGZus{}per\PYGZus{}px}\PYG{o}{=}\PYG{n+nb}{float}\PYG{p}{(}\PYG{n}{mm\PYGZus{}per\PYGZus{}px}\PYG{p}{)}\PYG{p}{,}
        \PYG{n}{method}\PYG{o}{=}\PYG{l+s+s2}{\PYGZdq{}}\PYG{l+s+s2}{checkerboard}\PYG{l+s+s2}{\PYGZdq{}}\PYG{p}{,}
        \PYG{n}{meta}\PYG{o}{=}\PYG{p}{\PYGZob{}}\PYG{l+s+s2}{\PYGZdq{}}\PYG{l+s+s2}{diagnostics}\PYG{l+s+s2}{\PYGZdq{}}\PYG{p}{:} \PYG{n}{diag}\PYG{p}{\PYGZcb{}}
    \PYG{p}{)}
    \PYG{n}{save\PYGZus{}calibration}\PYG{p}{(}\PYG{n}{calib\PYGZus{}file}\PYG{p}{,} \PYG{n}{cal}\PYG{p}{)}

    \PYG{k}{return} \PYG{p}{\PYGZob{}}
        \PYG{l+s+s2}{\PYGZdq{}}\PYG{l+s+s2}{estado}\PYG{l+s+s2}{\PYGZdq{}}\PYG{p}{:} \PYG{l+s+s2}{\PYGZdq{}}\PYG{l+s+s2}{recalibrado}\PYG{l+s+s2}{\PYGZdq{}}\PYG{p}{,}
        \PYG{l+s+s2}{\PYGZdq{}}\PYG{l+s+s2}{mm\PYGZus{}per\PYGZus{}px}\PYG{l+s+s2}{\PYGZdq{}}\PYG{p}{:} \PYG{n}{cal}\PYG{o}{.}\PYG{n}{mm\PYGZus{}per\PYGZus{}px}\PYG{p}{,}
        \PYG{l+s+s2}{\PYGZdq{}}\PYG{l+s+s2}{method}\PYG{l+s+s2}{\PYGZdq{}}\PYG{p}{:} \PYG{n}{cal}\PYG{o}{.}\PYG{n}{method}\PYG{p}{,}
        \PYG{l+s+s2}{\PYGZdq{}}\PYG{l+s+s2}{meta}\PYG{l+s+s2}{\PYGZdq{}}\PYG{p}{:} \PYG{n}{cal}\PYG{o}{.}\PYG{n}{meta}
    \PYG{p}{\PYGZcb{}}
\end{sphinxVerbatim}

\end{sphinxuseclass}\end{sphinxVerbatimInput}

\end{sphinxuseclass}
\sphinxAtStartPar
Para calibrar el sistema basta con ejecutar una llamada a la función anterior \sphinxcode{\sphinxupquote{run\_calibration\_pipeline\_checkerboard()}} del siguiente modo:

\begin{sphinxuseclass}{cell}\begin{sphinxVerbatimInput}

\begin{sphinxuseclass}{cell_input}
\begin{sphinxVerbatim}[commandchars=\\\{\}]
\PYG{n}{calibration} \PYG{o}{=} \PYG{n}{run\PYGZus{}calibration\PYGZus{}pipeline\PYGZus{}checkerboard}\PYG{p}{(}
    \PYG{n}{calib\PYGZus{}file}\PYG{o}{=}\PYG{n}{CALIB\PYGZus{}FILE}\PYG{p}{,}
    \PYG{n}{img\PYGZus{}path}\PYG{o}{=}\PYG{n}{IMG\PYGZus{}PATH}\PYG{p}{,}
    \PYG{n}{cols\PYGZus{}internal}\PYG{o}{=}\PYG{n}{CHECKERBOARD\PYGZus{}INTERNAL\PYGZus{}COLS}\PYG{p}{,}
    \PYG{n}{rows\PYGZus{}internal}\PYG{o}{=}\PYG{n}{CHECKERBOARD\PYGZus{}INTERNAL\PYGZus{}ROWS}\PYG{p}{,}
    \PYG{n}{square\PYGZus{}mm}\PYG{o}{=}\PYG{n}{CHECKERBOARD\PYGZus{}SQUARE\PYGZus{}MM}\PYG{p}{,}
    \PYG{n}{aggregation}\PYG{o}{=}\PYG{n}{AGGREGATION}\PYG{p}{,}
    \PYG{n}{force\PYGZus{}recalibrate}\PYG{o}{=}\PYG{k+kc}{False}
\PYG{p}{)}
\PYG{n+nb}{print}\PYG{p}{(}\PYG{l+s+s2}{\PYGZdq{}}\PYG{l+s+s2}{Resultado de la calibración: }\PYG{l+s+se}{\PYGZbs{}n}\PYG{l+s+s2}{\PYGZdq{}}\PYG{p}{)}
\PYG{n}{calibration}
\end{sphinxVerbatim}

\end{sphinxuseclass}\end{sphinxVerbatimInput}
\begin{sphinxVerbatimOutput}

\begin{sphinxuseclass}{cell_output}
\begin{sphinxVerbatim}[commandchars=\\\{\}]
Resultado de la calibración:
\end{sphinxVerbatim}

\begin{sphinxVerbatim}[commandchars=\\\{\}]
\PYGZob{}\PYGZsq{}estado\PYGZsq{}: \PYGZsq{}vigente\PYGZsq{},
 \PYGZsq{}mm\PYGZus{}per\PYGZus{}px\PYGZsq{}: 0.05,
 \PYGZsq{}method\PYGZsq{}: \PYGZsq{}checkerboard\PYGZsq{},
 \PYGZsq{}meta\PYGZsq{}: \PYGZob{}\PYGZsq{}diagnostics\PYGZsq{}: \PYGZob{}\PYGZsq{}pattern\PYGZsq{}: \PYGZsq{}checkerboard\PYGZsq{},
   \PYGZsq{}internal\PYGZus{}cols\PYGZsq{}: 9,
   \PYGZsq{}internal\PYGZus{}rows\PYGZsq{}: 9,
   \PYGZsq{}square\PYGZus{}mm\PYGZsq{}: 10.0,
   \PYGZsq{}edges\PYGZus{}count\PYGZsq{}: 144,
   \PYGZsq{}px\PYGZus{}per\PYGZus{}square\PYGZus{}summary\PYGZsq{}: \PYGZob{}\PYGZsq{}mean\PYGZsq{}: 200.0, \PYGZsq{}median\PYGZsq{}: 200.0, \PYGZsq{}std\PYGZsq{}: 0.0\PYGZcb{},
   \PYGZsq{}aggregation\PYGZsq{}: \PYGZsq{}median\PYGZsq{}\PYGZcb{}\PYGZcb{}\PYGZcb{}
\end{sphinxVerbatim}

\end{sphinxuseclass}\end{sphinxVerbatimOutput}

\end{sphinxuseclass}
\sphinxAtStartPar
Este módulo también proporciona algunas \sphinxstylestrong{utilidades de conversión} que permiten, a partir de un calibrado: obtener la escala (\sphinxcode{\sphinxupquote{get\_mm\_per\_px}}), traducir pixeles a mm (\sphinxcode{\sphinxupquote{px\_to\_mm}}) o transformar mm a pixeles (\sphinxcode{\sphinxupquote{mm\_to\_px}}).

\begin{sphinxuseclass}{cell}\begin{sphinxVerbatimInput}

\begin{sphinxuseclass}{cell_input}
\begin{sphinxVerbatim}[commandchars=\\\{\}]
\PYG{c+c1}{\PYGZsh{}\PYGZsh{} UTILIDADES DE CONVERSION}

\PYG{c+c1}{\PYGZsh{} Funciones auxiliares para convertir medidas en el pipeline de visión.}

\PYG{k}{def}\PYG{+w}{ }\PYG{n+nf}{get\PYGZus{}mm\PYGZus{}per\PYGZus{}px}\PYG{p}{(}\PYG{n}{calib\PYGZus{}file}\PYG{p}{:} \PYG{n}{Path} \PYG{o}{=} \PYG{n}{CALIB\PYGZus{}FILE}\PYG{p}{)} \PYG{o}{\PYGZhy{}}\PYG{o}{\PYGZgt{}} \PYG{n+nb}{float}\PYG{p}{:}
\PYG{+w}{    }\PYG{l+s+sd}{\PYGZdq{}\PYGZdq{}\PYGZdq{}Devuelve la escala mm/px leída del archivo de calibración.\PYGZdq{}\PYGZdq{}\PYGZdq{}}
    \PYG{n}{cal} \PYG{o}{=} \PYG{n}{load\PYGZus{}calibration}\PYG{p}{(}\PYG{n}{calib\PYGZus{}file}\PYG{p}{)}
    \PYG{k}{if} \PYG{n}{cal} \PYG{o+ow}{is} \PYG{k+kc}{None}\PYG{p}{:}
        \PYG{k}{raise} \PYG{n+ne}{RuntimeError}\PYG{p}{(}\PYG{l+s+sa}{f}\PYG{l+s+s2}{\PYGZdq{}}\PYG{l+s+s2}{No hay calibración guardada en }\PYG{l+s+si}{\PYGZob{}}\PYG{n}{calib\PYGZus{}file}\PYG{l+s+si}{\PYGZcb{}}\PYG{l+s+s2}{. Ejecuta el pipeline primero.}\PYG{l+s+s2}{\PYGZdq{}}\PYG{p}{)}
    \PYG{k}{return} \PYG{n+nb}{float}\PYG{p}{(}\PYG{n}{cal}\PYG{o}{.}\PYG{n}{mm\PYGZus{}per\PYGZus{}px}\PYG{p}{)}

\PYG{k}{def}\PYG{+w}{ }\PYG{n+nf}{px\PYGZus{}to\PYGZus{}mm}\PYG{p}{(}\PYG{n}{value\PYGZus{}px}\PYG{p}{:} \PYG{n+nb}{float}\PYG{p}{,} \PYG{n}{mm\PYGZus{}per\PYGZus{}px}\PYG{p}{:} \PYG{n+nb}{float}\PYG{p}{)} \PYG{o}{\PYGZhy{}}\PYG{o}{\PYGZgt{}} \PYG{n+nb}{float}\PYG{p}{:}
\PYG{+w}{    }\PYG{l+s+sd}{\PYGZdq{}\PYGZdq{}\PYGZdq{}Convierte longitudes en píxeles a milímetros.\PYGZdq{}\PYGZdq{}\PYGZdq{}}
    \PYG{k}{return} \PYG{n+nb}{float}\PYG{p}{(}\PYG{n}{value\PYGZus{}px}\PYG{p}{)} \PYG{o}{*} \PYG{n+nb}{float}\PYG{p}{(}\PYG{n}{mm\PYGZus{}per\PYGZus{}px}\PYG{p}{)}

\PYG{k}{def}\PYG{+w}{ }\PYG{n+nf}{mm\PYGZus{}to\PYGZus{}px}\PYG{p}{(}\PYG{n}{value\PYGZus{}mm}\PYG{p}{:} \PYG{n+nb}{float}\PYG{p}{,} \PYG{n}{mm\PYGZus{}per\PYGZus{}px}\PYG{p}{:} \PYG{n+nb}{float}\PYG{p}{)} \PYG{o}{\PYGZhy{}}\PYG{o}{\PYGZgt{}} \PYG{n+nb}{float}\PYG{p}{:}
\PYG{+w}{    }\PYG{l+s+sd}{\PYGZdq{}\PYGZdq{}\PYGZdq{}Convierte longitudes en milímetros a píxeles.\PYGZdq{}\PYGZdq{}\PYGZdq{}}
    \PYG{k}{if} \PYG{n}{mm\PYGZus{}per\PYGZus{}px} \PYG{o}{\PYGZlt{}}\PYG{o}{=} \PYG{l+m+mi}{0}\PYG{p}{:}
        \PYG{k}{raise} \PYG{n+ne}{ValueError}\PYG{p}{(}\PYG{l+s+s2}{\PYGZdq{}}\PYG{l+s+s2}{mm\PYGZus{}per\PYGZus{}px debe ser positivo.}\PYG{l+s+s2}{\PYGZdq{}}\PYG{p}{)}
    \PYG{k}{return} \PYG{n+nb}{float}\PYG{p}{(}\PYG{n}{value\PYGZus{}mm}\PYG{p}{)} \PYG{o}{/} \PYG{n+nb}{float}\PYG{p}{(}\PYG{n}{mm\PYGZus{}per\PYGZus{}px}\PYG{p}{)}
\end{sphinxVerbatim}

\end{sphinxuseclass}\end{sphinxVerbatimInput}

\end{sphinxuseclass}
\sphinxAtStartPar
Un ejemplo de uso de estas funciones sería:

\begin{sphinxuseclass}{cell}\begin{sphinxVerbatimInput}

\begin{sphinxuseclass}{cell_input}
\begin{sphinxVerbatim}[commandchars=\\\{\}]
\PYG{n+nb}{print}\PYG{p}{(}\PYG{l+s+sa}{f}\PYG{l+s+s2}{\PYGZdq{}}\PYG{l+s+s2}{Escala actual mm/px: }\PYG{l+s+si}{\PYGZob{}}\PYG{n}{get\PYGZus{}mm\PYGZus{}per\PYGZus{}px}\PYG{p}{(}\PYG{p}{)}\PYG{l+s+si}{\PYGZcb{}}\PYG{l+s+s2}{\PYGZdq{}}\PYG{p}{)}
\PYG{n+nb}{print}\PYG{p}{(}\PYG{l+s+sa}{f}\PYG{l+s+s2}{\PYGZdq{}}\PYG{l+s+s2}{Longitud en milímetros de 1250 px: }\PYG{l+s+si}{\PYGZob{}}\PYG{n}{px\PYGZus{}to\PYGZus{}mm}\PYG{p}{(}\PYG{l+m+mi}{1250}\PYG{p}{,}\PYG{+w}{ }\PYG{n}{get\PYGZus{}mm\PYGZus{}per\PYGZus{}px}\PYG{p}{(}\PYG{p}{)}\PYG{p}{)}\PYG{l+s+si}{\PYGZcb{}}\PYG{l+s+s2}{ mm}\PYG{l+s+s2}{\PYGZdq{}}\PYG{p}{)}
\PYG{n+nb}{print}\PYG{p}{(}\PYG{l+s+sa}{f}\PYG{l+s+s2}{\PYGZdq{}}\PYG{l+s+s2}{Longitud en piexels de 100 mm: }\PYG{l+s+si}{\PYGZob{}}\PYG{n}{mm\PYGZus{}to\PYGZus{}px}\PYG{p}{(}\PYG{l+m+mi}{100}\PYG{p}{,}\PYG{+w}{ }\PYG{n}{get\PYGZus{}mm\PYGZus{}per\PYGZus{}px}\PYG{p}{(}\PYG{p}{)}\PYG{p}{)}\PYG{l+s+si}{\PYGZcb{}}\PYG{l+s+s2}{ pixeles}\PYG{l+s+s2}{\PYGZdq{}}\PYG{p}{)}
\end{sphinxVerbatim}

\end{sphinxuseclass}\end{sphinxVerbatimInput}
\begin{sphinxVerbatimOutput}

\begin{sphinxuseclass}{cell_output}
\begin{sphinxVerbatim}[commandchars=\\\{\}]
Escala actual mm/px: 0.05
Longitud en milímetros de 1250 px: 62.5 mm
Longitud en piexels de 100 mm: 2000.0 pixeles
\end{sphinxVerbatim}

\end{sphinxuseclass}\end{sphinxVerbatimOutput}

\end{sphinxuseclass}
\sphinxAtStartPar
Para garantizar y auditar que el proceso de calibración se realiza adecuadamente se han definido algunas funciones auxiliares:

\sphinxAtStartPar
\sphinxstylestrong{Control visual y verificación del damero}

\begin{sphinxuseclass}{cell}\begin{sphinxVerbatimInput}

\begin{sphinxuseclass}{cell_input}
\begin{sphinxVerbatim}[commandchars=\\\{\}]
\PYG{c+c1}{\PYGZsh{}\PYGZsh{} OVERLAY DE CONTROL VISUAL Y .png DIAGNÓSTICO}

\PYG{c+c1}{\PYGZsh{} Su función es verificar y documentar que la detección ha sido correcta.}

\PYG{k+kn}{from}\PYG{+w}{ }\PYG{n+nn}{datetime}\PYG{+w}{ }\PYG{k+kn}{import} \PYG{n}{datetime}

\PYG{k}{def}\PYG{+w}{ }\PYG{n+nf}{overlay\PYGZus{}checkerboard\PYGZus{}and\PYGZus{}save}\PYG{p}{(}
    \PYG{n}{img\PYGZus{}path}\PYG{p}{:} \PYG{n}{Path}\PYG{p}{,}
    \PYG{n}{out\PYGZus{}path}\PYG{p}{:} \PYG{n}{Path}\PYG{p}{,}
    \PYG{n}{cols\PYGZus{}internal}\PYG{p}{:} \PYG{n+nb}{int}\PYG{p}{,}
    \PYG{n}{rows\PYGZus{}internal}\PYG{p}{:} \PYG{n+nb}{int}\PYG{p}{,}
    \PYG{n}{mm\PYGZus{}per\PYGZus{}px}\PYG{p}{:} \PYG{n+nb}{float} \PYG{o}{=} \PYG{k+kc}{None}\PYG{p}{,}
    \PYG{n}{draw\PYGZus{}rect}\PYG{p}{:} \PYG{n+nb}{bool} \PYG{o}{=} \PYG{k+kc}{True}
\PYG{p}{)} \PYG{o}{\PYGZhy{}}\PYG{o}{\PYGZgt{}} \PYG{n+nb}{dict}\PYG{p}{:}
\PYG{+w}{    }\PYG{l+s+sd}{\PYGZdq{}\PYGZdq{}\PYGZdq{}}
\PYG{l+s+sd}{    Dibuja esquinas del damero y anota mm/px (si disponible). Guarda PNG.}
\PYG{l+s+sd}{    Devuelve dict con \PYGZsq{}ok\PYGZsq{} y detalles/motivo de fallo.}
\PYG{l+s+sd}{    \PYGZdq{}\PYGZdq{}\PYGZdq{}}
    \PYG{k}{if} \PYG{o+ow}{not} \PYG{n}{img\PYGZus{}path}\PYG{o}{.}\PYG{n}{exists}\PYG{p}{(}\PYG{p}{)}\PYG{p}{:}
        \PYG{k}{return} \PYG{p}{\PYGZob{}}\PYG{l+s+s2}{\PYGZdq{}}\PYG{l+s+s2}{ok}\PYG{l+s+s2}{\PYGZdq{}}\PYG{p}{:} \PYG{k+kc}{False}\PYG{p}{,} \PYG{l+s+s2}{\PYGZdq{}}\PYG{l+s+s2}{motivo}\PYG{l+s+s2}{\PYGZdq{}}\PYG{p}{:} \PYG{l+s+sa}{f}\PYG{l+s+s2}{\PYGZdq{}}\PYG{l+s+s2}{Imagen no encontrada: }\PYG{l+s+si}{\PYGZob{}}\PYG{n+nb}{str}\PYG{p}{(}\PYG{n}{img\PYGZus{}path}\PYG{p}{)}\PYG{l+s+si}{\PYGZcb{}}\PYG{l+s+s2}{\PYGZdq{}}\PYG{p}{\PYGZcb{}}

    \PYG{n}{img} \PYG{o}{=} \PYG{n}{cv2}\PYG{o}{.}\PYG{n}{imread}\PYG{p}{(}\PYG{n+nb}{str}\PYG{p}{(}\PYG{n}{img\PYGZus{}path}\PYG{p}{)}\PYG{p}{)}
    \PYG{k}{if} \PYG{n}{img} \PYG{o+ow}{is} \PYG{k+kc}{None}\PYG{p}{:}
        \PYG{k}{return} \PYG{p}{\PYGZob{}}\PYG{l+s+s2}{\PYGZdq{}}\PYG{l+s+s2}{ok}\PYG{l+s+s2}{\PYGZdq{}}\PYG{p}{:} \PYG{k+kc}{False}\PYG{p}{,} \PYG{l+s+s2}{\PYGZdq{}}\PYG{l+s+s2}{motivo}\PYG{l+s+s2}{\PYGZdq{}}\PYG{p}{:} \PYG{l+s+sa}{f}\PYG{l+s+s2}{\PYGZdq{}}\PYG{l+s+s2}{No se pudo abrir la imagen: }\PYG{l+s+si}{\PYGZob{}}\PYG{n+nb}{str}\PYG{p}{(}\PYG{n}{img\PYGZus{}path}\PYG{p}{)}\PYG{l+s+si}{\PYGZcb{}}\PYG{l+s+s2}{\PYGZdq{}}\PYG{p}{\PYGZcb{}}

    \PYG{n}{gray} \PYG{o}{=} \PYG{n}{cv2}\PYG{o}{.}\PYG{n}{cvtColor}\PYG{p}{(}\PYG{n}{img}\PYG{p}{,} \PYG{n}{cv2}\PYG{o}{.}\PYG{n}{COLOR\PYGZus{}BGR2GRAY}\PYG{p}{)}
    \PYG{n}{flags} \PYG{o}{=} \PYG{n}{cv2}\PYG{o}{.}\PYG{n}{CALIB\PYGZus{}CB\PYGZus{}ADAPTIVE\PYGZus{}THRESH} \PYG{o}{|} \PYG{n}{cv2}\PYG{o}{.}\PYG{n}{CALIB\PYGZus{}CB\PYGZus{}NORMALIZE\PYGZus{}IMAGE}
    \PYG{n}{found}\PYG{p}{,} \PYG{n}{corners} \PYG{o}{=} \PYG{n}{cv2}\PYG{o}{.}\PYG{n}{findChessboardCorners}\PYG{p}{(}\PYG{n}{gray}\PYG{p}{,} \PYG{p}{(}\PYG{n}{cols\PYGZus{}internal}\PYG{p}{,} \PYG{n}{rows\PYGZus{}internal}\PYG{p}{)}\PYG{p}{,} \PYG{n}{flags}\PYG{p}{)}

    \PYG{n}{vis} \PYG{o}{=} \PYG{n}{img}\PYG{o}{.}\PYG{n}{copy}\PYG{p}{(}\PYG{p}{)}
    \PYG{k}{if} \PYG{o+ow}{not} \PYG{n}{found} \PYG{o+ow}{or} \PYG{n}{corners} \PYG{o+ow}{is} \PYG{k+kc}{None}\PYG{p}{:}
        \PYG{n}{cv2}\PYG{o}{.}\PYG{n}{putText}\PYG{p}{(}\PYG{n}{vis}\PYG{p}{,} \PYG{l+s+s2}{\PYGZdq{}}\PYG{l+s+s2}{Checkerboard NO detectado}\PYG{l+s+s2}{\PYGZdq{}}\PYG{p}{,} \PYG{p}{(}\PYG{l+m+mi}{20}\PYG{p}{,} \PYG{l+m+mi}{40}\PYG{p}{)}\PYG{p}{,}
                    \PYG{n}{cv2}\PYG{o}{.}\PYG{n}{FONT\PYGZus{}HERSHEY\PYGZus{}SIMPLEX}\PYG{p}{,} \PYG{l+m+mf}{1.0}\PYG{p}{,} \PYG{p}{(}\PYG{l+m+mi}{0}\PYG{p}{,} \PYG{l+m+mi}{0}\PYG{p}{,} \PYG{l+m+mi}{255}\PYG{p}{)}\PYG{p}{,} \PYG{l+m+mi}{2}\PYG{p}{,} \PYG{n}{cv2}\PYG{o}{.}\PYG{n}{LINE\PYGZus{}AA}\PYG{p}{)}
        \PYG{n}{out\PYGZus{}path}\PYG{o}{.}\PYG{n}{parent}\PYG{o}{.}\PYG{n}{mkdir}\PYG{p}{(}\PYG{n}{parents}\PYG{o}{=}\PYG{k+kc}{True}\PYG{p}{,} \PYG{n}{exist\PYGZus{}ok}\PYG{o}{=}\PYG{k+kc}{True}\PYG{p}{)}
        \PYG{n}{ok} \PYG{o}{=} \PYG{n}{cv2}\PYG{o}{.}\PYG{n}{imwrite}\PYG{p}{(}\PYG{n+nb}{str}\PYG{p}{(}\PYG{n}{out\PYGZus{}path}\PYG{p}{)}\PYG{p}{,} \PYG{n}{vis}\PYG{p}{)}
        \PYG{k}{return} \PYG{p}{\PYGZob{}}\PYG{l+s+s2}{\PYGZdq{}}\PYG{l+s+s2}{ok}\PYG{l+s+s2}{\PYGZdq{}}\PYG{p}{:} \PYG{n+nb}{bool}\PYG{p}{(}\PYG{n}{ok}\PYG{p}{)}\PYG{p}{,} \PYG{l+s+s2}{\PYGZdq{}}\PYG{l+s+s2}{motivo}\PYG{l+s+s2}{\PYGZdq{}}\PYG{p}{:} \PYG{l+s+s2}{\PYGZdq{}}\PYG{l+s+s2}{No se detecto el damero; PNG guardado con aviso.}\PYG{l+s+s2}{\PYGZdq{}}\PYG{p}{,} \PYG{l+s+s2}{\PYGZdq{}}\PYG{l+s+s2}{out}\PYG{l+s+s2}{\PYGZdq{}}\PYG{p}{:} \PYG{n+nb}{str}\PYG{p}{(}\PYG{n}{out\PYGZus{}path}\PYG{p}{)}\PYG{p}{\PYGZcb{}}

    \PYG{n}{criteria} \PYG{o}{=} \PYG{p}{(}\PYG{n}{cv2}\PYG{o}{.}\PYG{n}{TERM\PYGZus{}CRITERIA\PYGZus{}EPS} \PYG{o}{+} \PYG{n}{cv2}\PYG{o}{.}\PYG{n}{TERM\PYGZus{}CRITERIA\PYGZus{}MAX\PYGZus{}ITER}\PYG{p}{,} \PYG{l+m+mi}{50}\PYG{p}{,} \PYG{l+m+mf}{1e\PYGZhy{}4}\PYG{p}{)}
    \PYG{n}{corners} \PYG{o}{=} \PYG{n}{cv2}\PYG{o}{.}\PYG{n}{cornerSubPix}\PYG{p}{(}\PYG{n}{gray}\PYG{p}{,} \PYG{n}{corners}\PYG{p}{,} \PYG{p}{(}\PYG{l+m+mi}{11}\PYG{p}{,} \PYG{l+m+mi}{11}\PYG{p}{)}\PYG{p}{,} \PYG{p}{(}\PYG{o}{\PYGZhy{}}\PYG{l+m+mi}{1}\PYG{p}{,} \PYG{o}{\PYGZhy{}}\PYG{l+m+mi}{1}\PYG{p}{)}\PYG{p}{,} \PYG{n}{criteria}\PYG{p}{)}

    \PYG{n}{cv2}\PYG{o}{.}\PYG{n}{drawChessboardCorners}\PYG{p}{(}\PYG{n}{vis}\PYG{p}{,} \PYG{p}{(}\PYG{n}{cols\PYGZus{}internal}\PYG{p}{,} \PYG{n}{rows\PYGZus{}internal}\PYG{p}{)}\PYG{p}{,} \PYG{n}{corners}\PYG{p}{,} \PYG{n}{found}\PYG{p}{)}

    \PYG{k}{if} \PYG{n}{draw\PYGZus{}rect}\PYG{p}{:}
        \PYG{n}{pts} \PYG{o}{=} \PYG{n}{corners}\PYG{o}{.}\PYG{n}{reshape}\PYG{p}{(}\PYG{o}{\PYGZhy{}}\PYG{l+m+mi}{1}\PYG{p}{,} \PYG{l+m+mi}{2}\PYG{p}{)}
        \PYG{n}{x\PYGZus{}min}\PYG{p}{,} \PYG{n}{y\PYGZus{}min} \PYG{o}{=} \PYG{n}{np}\PYG{o}{.}\PYG{n}{floor}\PYG{p}{(}\PYG{n}{pts}\PYG{o}{.}\PYG{n}{min}\PYG{p}{(}\PYG{n}{axis}\PYG{o}{=}\PYG{l+m+mi}{0}\PYG{p}{)}\PYG{p}{)}\PYG{o}{.}\PYG{n}{astype}\PYG{p}{(}\PYG{n+nb}{int}\PYG{p}{)}
        \PYG{n}{x\PYGZus{}max}\PYG{p}{,} \PYG{n}{y\PYGZus{}max} \PYG{o}{=} \PYG{n}{np}\PYG{o}{.}\PYG{n}{ceil}\PYG{p}{(}\PYG{n}{pts}\PYG{o}{.}\PYG{n}{max}\PYG{p}{(}\PYG{n}{axis}\PYG{o}{=}\PYG{l+m+mi}{0}\PYG{p}{)}\PYG{p}{)}\PYG{o}{.}\PYG{n}{astype}\PYG{p}{(}\PYG{n+nb}{int}\PYG{p}{)}
        \PYG{n}{cv2}\PYG{o}{.}\PYG{n}{rectangle}\PYG{p}{(}\PYG{n}{vis}\PYG{p}{,} \PYG{p}{(}\PYG{n}{x\PYGZus{}min}\PYG{p}{,} \PYG{n}{y\PYGZus{}min}\PYG{p}{)}\PYG{p}{,} \PYG{p}{(}\PYG{n}{x\PYGZus{}max}\PYG{p}{,} \PYG{n}{y\PYGZus{}max}\PYG{p}{)}\PYG{p}{,} \PYG{p}{(}\PYG{l+m+mi}{255}\PYG{p}{,} \PYG{l+m+mi}{0}\PYG{p}{,} \PYG{l+m+mi}{0}\PYG{p}{)}\PYG{p}{,} \PYG{l+m+mi}{2}\PYG{p}{)}

    \PYG{n}{y0} \PYG{o}{=} \PYG{l+m+mi}{30}
    \PYG{n}{cv2}\PYG{o}{.}\PYG{n}{putText}\PYG{p}{(}\PYG{n}{vis}\PYG{p}{,} \PYG{l+s+sa}{f}\PYG{l+s+s2}{\PYGZdq{}}\PYG{l+s+s2}{Damero }\PYG{l+s+si}{\PYGZob{}}\PYG{n}{cols\PYGZus{}internal}\PYG{l+s+si}{\PYGZcb{}}\PYG{l+s+s2}{x}\PYG{l+s+si}{\PYGZob{}}\PYG{n}{rows\PYGZus{}internal}\PYG{l+s+si}{\PYGZcb{}}\PYG{l+s+s2}{ (intersecciones)}\PYG{l+s+s2}{\PYGZdq{}}\PYG{p}{,} \PYG{p}{(}\PYG{l+m+mi}{20}\PYG{p}{,} \PYG{n}{y0}\PYG{p}{)}\PYG{p}{,}
                \PYG{n}{cv2}\PYG{o}{.}\PYG{n}{FONT\PYGZus{}HERSHEY\PYGZus{}SIMPLEX}\PYG{p}{,} \PYG{l+m+mf}{0.8}\PYG{p}{,} \PYG{p}{(}\PYG{l+m+mi}{50}\PYG{p}{,} \PYG{l+m+mi}{220}\PYG{p}{,} \PYG{l+m+mi}{50}\PYG{p}{)}\PYG{p}{,} \PYG{l+m+mi}{2}\PYG{p}{,} \PYG{n}{cv2}\PYG{o}{.}\PYG{n}{LINE\PYGZus{}AA}\PYG{p}{)}
    \PYG{n}{y0} \PYG{o}{+}\PYG{o}{=} \PYG{l+m+mi}{30}
    \PYG{k}{if} \PYG{n}{mm\PYGZus{}per\PYGZus{}px} \PYG{o+ow}{is} \PYG{o+ow}{not} \PYG{k+kc}{None}\PYG{p}{:}
        \PYG{n}{cv2}\PYG{o}{.}\PYG{n}{putText}\PYG{p}{(}\PYG{n}{vis}\PYG{p}{,} \PYG{l+s+sa}{f}\PYG{l+s+s2}{\PYGZdq{}}\PYG{l+s+s2}{Escala: }\PYG{l+s+si}{\PYGZob{}}\PYG{n}{mm\PYGZus{}per\PYGZus{}px}\PYG{l+s+si}{:}\PYG{l+s+s2}{.6f}\PYG{l+s+si}{\PYGZcb{}}\PYG{l+s+s2}{ mm/px}\PYG{l+s+s2}{\PYGZdq{}}\PYG{p}{,} \PYG{p}{(}\PYG{l+m+mi}{20}\PYG{p}{,} \PYG{n}{y0}\PYG{p}{)}\PYG{p}{,}
                    \PYG{n}{cv2}\PYG{o}{.}\PYG{n}{FONT\PYGZus{}HERSHEY\PYGZus{}SIMPLEX}\PYG{p}{,} \PYG{l+m+mf}{0.8}\PYG{p}{,} \PYG{p}{(}\PYG{l+m+mi}{50}\PYG{p}{,} \PYG{l+m+mi}{220}\PYG{p}{,} \PYG{l+m+mi}{50}\PYG{p}{)}\PYG{p}{,} \PYG{l+m+mi}{2}\PYG{p}{,} \PYG{n}{cv2}\PYG{o}{.}\PYG{n}{LINE\PYGZus{}AA}\PYG{p}{)}
        \PYG{n}{y0} \PYG{o}{+}\PYG{o}{=} \PYG{l+m+mi}{30}
    \PYG{n}{cv2}\PYG{o}{.}\PYG{n}{putText}\PYG{p}{(}\PYG{n}{vis}\PYG{p}{,} \PYG{l+s+sa}{f}\PYG{l+s+s2}{\PYGZdq{}}\PYG{l+s+s2}{Timestamp: }\PYG{l+s+si}{\PYGZob{}}\PYG{n}{datetime}\PYG{o}{.}\PYG{n}{now}\PYG{p}{(}\PYG{p}{)}\PYG{o}{.}\PYG{n}{isoformat}\PYG{p}{(}\PYG{n}{timespec}\PYG{o}{=}\PYG{l+s+s1}{\PYGZsq{}}\PYG{l+s+s1}{seconds}\PYG{l+s+s1}{\PYGZsq{}}\PYG{p}{)}\PYG{l+s+si}{\PYGZcb{}}\PYG{l+s+s2}{\PYGZdq{}}\PYG{p}{,} \PYG{p}{(}\PYG{l+m+mi}{20}\PYG{p}{,} \PYG{n}{y0}\PYG{p}{)}\PYG{p}{,}
                \PYG{n}{cv2}\PYG{o}{.}\PYG{n}{FONT\PYGZus{}HERSHEY\PYGZus{}SIMPLEX}\PYG{p}{,} \PYG{l+m+mf}{0.7}\PYG{p}{,} \PYG{p}{(}\PYG{l+m+mi}{200}\PYG{p}{,} \PYG{l+m+mi}{200}\PYG{p}{,} \PYG{l+m+mi}{200}\PYG{p}{)}\PYG{p}{,} \PYG{l+m+mi}{2}\PYG{p}{,} \PYG{n}{cv2}\PYG{o}{.}\PYG{n}{LINE\PYGZus{}AA}\PYG{p}{)}

    \PYG{n}{out\PYGZus{}path}\PYG{o}{.}\PYG{n}{parent}\PYG{o}{.}\PYG{n}{mkdir}\PYG{p}{(}\PYG{n}{parents}\PYG{o}{=}\PYG{k+kc}{True}\PYG{p}{,} \PYG{n}{exist\PYGZus{}ok}\PYG{o}{=}\PYG{k+kc}{True}\PYG{p}{)}
    \PYG{n}{ok} \PYG{o}{=} \PYG{n}{cv2}\PYG{o}{.}\PYG{n}{imwrite}\PYG{p}{(}\PYG{n+nb}{str}\PYG{p}{(}\PYG{n}{out\PYGZus{}path}\PYG{p}{)}\PYG{p}{,} \PYG{n}{vis}\PYG{p}{)}
    \PYG{k}{return} \PYG{p}{\PYGZob{}}\PYG{l+s+s2}{\PYGZdq{}}\PYG{l+s+s2}{ok}\PYG{l+s+s2}{\PYGZdq{}}\PYG{p}{:} \PYG{n+nb}{bool}\PYG{p}{(}\PYG{n}{ok}\PYG{p}{)}\PYG{p}{,} \PYG{l+s+s2}{\PYGZdq{}}\PYG{l+s+s2}{motivo}\PYG{l+s+s2}{\PYGZdq{}}\PYG{p}{:} \PYG{l+s+s2}{\PYGZdq{}}\PYG{l+s+s2}{\PYGZdq{}} \PYG{k}{if} \PYG{n}{ok} \PYG{k}{else} \PYG{l+s+s2}{\PYGZdq{}}\PYG{l+s+s2}{cv2.imwrite fallo}\PYG{l+s+s2}{\PYGZdq{}}\PYG{p}{,} \PYG{l+s+s2}{\PYGZdq{}}\PYG{l+s+s2}{out}\PYG{l+s+s2}{\PYGZdq{}}\PYG{p}{:} \PYG{n+nb}{str}\PYG{p}{(}\PYG{n}{out\PYGZus{}path}\PYG{p}{)}\PYG{p}{\PYGZcb{}}
\end{sphinxVerbatim}

\end{sphinxuseclass}\end{sphinxVerbatimInput}

\end{sphinxuseclass}
\sphinxAtStartPar
El siguiente código muestra cómo ejecutar la función \sphinxcode{\sphinxupquote{overlay\_checkerboard\_and\_save}}.

\begin{sphinxuseclass}{cell}\begin{sphinxVerbatimInput}

\begin{sphinxuseclass}{cell_input}
\begin{sphinxVerbatim}[commandchars=\\\{\}]
\PYG{k+kn}{from}\PYG{+w}{ }\PYG{n+nn}{pathlib}\PYG{+w}{ }\PYG{k+kn}{import} \PYG{n}{Path}
\PYG{k+kn}{from}\PYG{+w}{ }\PYG{n+nn}{IPython}\PYG{n+nn}{.}\PYG{n+nn}{display}\PYG{+w}{ }\PYG{k+kn}{import} \PYG{n}{Image}\PYG{p}{,} \PYG{n}{display}

\PYG{c+c1}{\PYGZsh{} Lee la escala guardada (mm/px) para anotarla en la imagen}
\PYG{n}{mm\PYGZus{}per\PYGZus{}px\PYGZus{}annot} \PYG{o}{=} \PYG{n}{get\PYGZus{}mm\PYGZus{}per\PYGZus{}px}\PYG{p}{(}\PYG{p}{)}

\PYG{n}{OUT\PYGZus{}PNG} \PYG{o}{=} \PYG{n}{Path}\PYG{p}{(}\PYG{l+s+s2}{\PYGZdq{}}\PYG{l+s+s2}{./diagnosticos/diagnostico\PYGZus{}calibracion.png}\PYG{l+s+s2}{\PYGZdq{}}\PYG{p}{)}

\PYG{n}{overlay\PYGZus{}info} \PYG{o}{=} \PYG{n}{overlay\PYGZus{}checkerboard\PYGZus{}and\PYGZus{}save}\PYG{p}{(}
    \PYG{n}{img\PYGZus{}path}\PYG{o}{=}\PYG{n}{IMG\PYGZus{}PATH}\PYG{p}{,}
    \PYG{n}{out\PYGZus{}path}\PYG{o}{=}\PYG{n}{OUT\PYGZus{}PNG}\PYG{p}{,}
    \PYG{n}{cols\PYGZus{}internal}\PYG{o}{=}\PYG{n}{CHECKERBOARD\PYGZus{}INTERNAL\PYGZus{}COLS}\PYG{p}{,}  \PYG{c+c1}{\PYGZsh{} 9}
    \PYG{n}{rows\PYGZus{}internal}\PYG{o}{=}\PYG{n}{CHECKERBOARD\PYGZus{}INTERNAL\PYGZus{}ROWS}\PYG{p}{,}  \PYG{c+c1}{\PYGZsh{} 9}
    \PYG{n}{mm\PYGZus{}per\PYGZus{}px}\PYG{o}{=}\PYG{n}{mm\PYGZus{}per\PYGZus{}px\PYGZus{}annot}\PYG{p}{,}                 \PYG{c+c1}{\PYGZsh{} se anotará \PYGZdq{}Escala: X mm/px\PYGZdq{}}
    \PYG{n}{draw\PYGZus{}rect}\PYG{o}{=}\PYG{k+kc}{True}
\PYG{p}{)}

\PYG{c+c1}{\PYGZsh{} Mostrar la imagen resultante en el notebook}
\PYG{n}{DISPLAY\PYGZus{}WIDTH} \PYG{o}{=} \PYG{l+m+mi}{800}  \PYG{c+c1}{\PYGZsh{} px}
\PYG{n}{display}\PYG{p}{(}\PYG{n}{Image}\PYG{p}{(}\PYG{n}{filename}\PYG{o}{=}\PYG{n+nb}{str}\PYG{p}{(}\PYG{n}{OUT\PYGZus{}PNG}\PYG{p}{)}\PYG{p}{,} \PYG{n}{embed}\PYG{o}{=}\PYG{k+kc}{True}\PYG{p}{,} \PYG{n}{width}\PYG{o}{=}\PYG{n}{DISPLAY\PYGZus{}WIDTH}\PYG{p}{)}\PYG{p}{)}

\PYG{c+c1}{\PYGZsh{} (Opcional) Ver detalles del overlay/guardado}
\PYG{n}{overlay\PYGZus{}info}
\end{sphinxVerbatim}

\end{sphinxuseclass}\end{sphinxVerbatimInput}
\begin{sphinxVerbatimOutput}

\begin{sphinxuseclass}{cell_output}
\noindent\sphinxincludegraphics[width=800\sphinxpxdimen]{{e4932947a5bc7203d0a0677b9fa23dd59aa50735814d15cced69e7bd7d3496f4}.png}

\begin{sphinxVerbatim}[commandchars=\\\{\}]
\PYGZob{}\PYGZsq{}ok\PYGZsq{}: True, \PYGZsq{}motivo\PYGZsq{}: \PYGZsq{}\PYGZsq{}, \PYGZsq{}out\PYGZsq{}: \PYGZsq{}diagnosticos\PYGZbs{}\PYGZbs{}diagnostico\PYGZus{}calibracion.png\PYGZsq{}\PYGZcb{}
\end{sphinxVerbatim}

\end{sphinxuseclass}\end{sphinxVerbatimOutput}

\end{sphinxuseclass}
\sphinxAtStartPar
\sphinxstylestrong{Validación numérica}

\sphinxAtStartPar
Esta función lee la calibración existente y estima la incertidumbre según la ecuación \eqref{equation:content/01/Modulo-2:incertidumbre_calibracion}

\begin{sphinxuseclass}{cell}\begin{sphinxVerbatimInput}

\begin{sphinxuseclass}{cell_input}
\begin{sphinxVerbatim}[commandchars=\\\{\}]
\PYG{c+c1}{\PYGZsh{}\PYGZsh{} VALIDACIÓN NUMÉRICA}

\PYG{c+c1}{\PYGZsh{} Lee la calibración y reporta mm/px y px/mm estimado la incertidumbre}

\PYG{k}{def}\PYG{+w}{ }\PYG{n+nf}{load\PYGZus{}calibration\PYGZus{}dict}\PYG{p}{(}\PYG{n}{path}\PYG{p}{:} \PYG{n}{Path}\PYG{p}{)} \PYG{o}{\PYGZhy{}}\PYG{o}{\PYGZgt{}} \PYG{n+nb}{dict}\PYG{p}{:}
    \PYG{k}{if} \PYG{o+ow}{not} \PYG{n}{path}\PYG{o}{.}\PYG{n}{exists}\PYG{p}{(}\PYG{p}{)}\PYG{p}{:}
        \PYG{k}{raise} \PYG{n+ne}{FileNotFoundError}\PYG{p}{(}\PYG{l+s+sa}{f}\PYG{l+s+s2}{\PYGZdq{}}\PYG{l+s+s2}{No existe el archivo de calibración: }\PYG{l+s+si}{\PYGZob{}}\PYG{n}{path}\PYG{l+s+si}{\PYGZcb{}}\PYG{l+s+s2}{\PYGZdq{}}\PYG{p}{)}
    \PYG{k}{with} \PYG{n}{path}\PYG{o}{.}\PYG{n}{open}\PYG{p}{(}\PYG{l+s+s2}{\PYGZdq{}}\PYG{l+s+s2}{r}\PYG{l+s+s2}{\PYGZdq{}}\PYG{p}{,} \PYG{n}{encoding}\PYG{o}{=}\PYG{l+s+s2}{\PYGZdq{}}\PYG{l+s+s2}{utf\PYGZhy{}8}\PYG{l+s+s2}{\PYGZdq{}}\PYG{p}{)} \PYG{k}{as} \PYG{n}{f}\PYG{p}{:}
        \PYG{k}{return} \PYG{n}{json}\PYG{o}{.}\PYG{n}{load}\PYG{p}{(}\PYG{n}{f}\PYG{p}{)}

\PYG{k}{def}\PYG{+w}{ }\PYG{n+nf}{validation\PYGZus{}report}\PYG{p}{(}\PYG{n}{calib\PYGZus{}json}\PYG{p}{:} \PYG{n+nb}{dict}\PYG{p}{,} \PYG{n}{assume\PYGZus{}square\PYGZus{}exact\PYGZus{}mm}\PYG{o}{=}\PYG{k+kc}{True}\PYG{p}{,} \PYG{n}{square\PYGZus{}sigma\PYGZus{}mm}\PYG{o}{=}\PYG{l+m+mf}{0.0}\PYG{p}{)}\PYG{p}{:}
\PYG{+w}{    }\PYG{l+s+sd}{\PYGZdq{}\PYGZdq{}\PYGZdq{}}
\PYG{l+s+sd}{    Reporta mm/px, px/mm e incertidumbre estimada.}
\PYG{l+s+sd}{    \PYGZhy{} Si conoces la tolerancia del patrón (p.ej. ±0.02 mm), usa assume\PYGZus{}square\PYGZus{}exact\PYGZus{}mm=False y square\PYGZus{}sigma\PYGZus{}mm=0.02.}
\PYG{l+s+sd}{    \PYGZdq{}\PYGZdq{}\PYGZdq{}}
    \PYG{n}{mm\PYGZus{}per\PYGZus{}px} \PYG{o}{=} \PYG{n+nb}{float}\PYG{p}{(}\PYG{n}{calib\PYGZus{}json}\PYG{p}{[}\PYG{l+s+s2}{\PYGZdq{}}\PYG{l+s+s2}{mm\PYGZus{}per\PYGZus{}px}\PYG{l+s+s2}{\PYGZdq{}}\PYG{p}{]}\PYG{p}{)}
    \PYG{n}{px\PYGZus{}per\PYGZus{}mm} \PYG{o}{=} \PYG{l+m+mf}{1.0} \PYG{o}{/} \PYG{n}{mm\PYGZus{}per\PYGZus{}px} \PYG{k}{if} \PYG{n}{mm\PYGZus{}per\PYGZus{}px} \PYG{o}{\PYGZgt{}} \PYG{l+m+mi}{0} \PYG{k}{else} \PYG{n}{np}\PYG{o}{.}\PYG{n}{nan}

    \PYG{n}{diag} \PYG{o}{=} \PYG{n}{calib\PYGZus{}json}\PYG{o}{.}\PYG{n}{get}\PYG{p}{(}\PYG{l+s+s2}{\PYGZdq{}}\PYG{l+s+s2}{meta}\PYG{l+s+s2}{\PYGZdq{}}\PYG{p}{,} \PYG{p}{\PYGZob{}}\PYG{p}{\PYGZcb{}}\PYG{p}{)}\PYG{o}{.}\PYG{n}{get}\PYG{p}{(}\PYG{l+s+s2}{\PYGZdq{}}\PYG{l+s+s2}{diagnostics}\PYG{l+s+s2}{\PYGZdq{}}\PYG{p}{,} \PYG{p}{\PYGZob{}}\PYG{p}{\PYGZcb{}}\PYG{p}{)}
    \PYG{n}{pattern} \PYG{o}{=} \PYG{n}{diag}\PYG{o}{.}\PYG{n}{get}\PYG{p}{(}\PYG{l+s+s2}{\PYGZdq{}}\PYG{l+s+s2}{pattern}\PYG{l+s+s2}{\PYGZdq{}}\PYG{p}{,} \PYG{l+s+s2}{\PYGZdq{}}\PYG{l+s+s2}{checkerboard}\PYG{l+s+s2}{\PYGZdq{}}\PYG{p}{)}
    \PYG{n}{px\PYGZus{}sum} \PYG{o}{=} \PYG{n}{diag}\PYG{o}{.}\PYG{n}{get}\PYG{p}{(}\PYG{l+s+s2}{\PYGZdq{}}\PYG{l+s+s2}{px\PYGZus{}per\PYGZus{}square\PYGZus{}summary}\PYG{l+s+s2}{\PYGZdq{}}\PYG{p}{,} \PYG{p}{\PYGZob{}}\PYG{p}{\PYGZcb{}}\PYG{p}{)}
    \PYG{n}{aggregation} \PYG{o}{=} \PYG{n}{diag}\PYG{o}{.}\PYG{n}{get}\PYG{p}{(}\PYG{l+s+s2}{\PYGZdq{}}\PYG{l+s+s2}{aggregation}\PYG{l+s+s2}{\PYGZdq{}}\PYG{p}{,} \PYG{l+s+s2}{\PYGZdq{}}\PYG{l+s+s2}{median}\PYG{l+s+s2}{\PYGZdq{}}\PYG{p}{)}
    \PYG{n}{square\PYGZus{}mm} \PYG{o}{=} \PYG{n+nb}{float}\PYG{p}{(}\PYG{n}{diag}\PYG{o}{.}\PYG{n}{get}\PYG{p}{(}\PYG{l+s+s2}{\PYGZdq{}}\PYG{l+s+s2}{square\PYGZus{}mm}\PYG{l+s+s2}{\PYGZdq{}}\PYG{p}{,} \PYG{n}{np}\PYG{o}{.}\PYG{n}{nan}\PYG{p}{)}\PYG{p}{)}

    \PYG{n}{center\PYGZus{}px} \PYG{o}{=} \PYG{n}{px\PYGZus{}sum}\PYG{o}{.}\PYG{n}{get}\PYG{p}{(}\PYG{l+s+s2}{\PYGZdq{}}\PYG{l+s+s2}{median}\PYG{l+s+s2}{\PYGZdq{}} \PYG{k}{if} \PYG{n}{aggregation} \PYG{o}{==} \PYG{l+s+s2}{\PYGZdq{}}\PYG{l+s+s2}{median}\PYG{l+s+s2}{\PYGZdq{}} \PYG{k}{else} \PYG{l+s+s2}{\PYGZdq{}}\PYG{l+s+s2}{mean}\PYG{l+s+s2}{\PYGZdq{}}\PYG{p}{,} \PYG{k+kc}{None}\PYG{p}{)}
    \PYG{n}{std\PYGZus{}px} \PYG{o}{=} \PYG{n}{px\PYGZus{}sum}\PYG{o}{.}\PYG{n}{get}\PYG{p}{(}\PYG{l+s+s2}{\PYGZdq{}}\PYG{l+s+s2}{std}\PYG{l+s+s2}{\PYGZdq{}}\PYG{p}{,} \PYG{k+kc}{None}\PYG{p}{)}
    \PYG{n}{n\PYGZus{}edges} \PYG{o}{=} \PYG{n}{diag}\PYG{o}{.}\PYG{n}{get}\PYG{p}{(}\PYG{l+s+s2}{\PYGZdq{}}\PYG{l+s+s2}{edges\PYGZus{}count}\PYG{l+s+s2}{\PYGZdq{}}\PYG{p}{,} \PYG{k+kc}{None}\PYG{p}{)}

    \PYG{n}{sigma\PYGZus{}s} \PYG{o}{=} \PYG{k+kc}{None}
    \PYG{n}{ci95} \PYG{o}{=} \PYG{p}{(}\PYG{k+kc}{None}\PYG{p}{,} \PYG{k+kc}{None}\PYG{p}{)}
    \PYG{n}{details} \PYG{o}{=} \PYG{p}{\PYGZob{}}\PYG{p}{\PYGZcb{}}

    \PYG{k}{if} \PYG{n}{center\PYGZus{}px} \PYG{o+ow}{and} \PYG{n}{std\PYGZus{}px} \PYG{o+ow}{is} \PYG{o+ow}{not} \PYG{k+kc}{None} \PYG{o+ow}{and} \PYG{n}{center\PYGZus{}px} \PYG{o}{\PYGZgt{}} \PYG{l+m+mi}{0}\PYG{p}{:}
        \PYG{n}{sigma\PYGZus{}square} \PYG{o}{=} \PYG{l+m+mf}{0.0} \PYG{k}{if} \PYG{n}{assume\PYGZus{}square\PYGZus{}exact\PYGZus{}mm} \PYG{k}{else} \PYG{n+nb}{float}\PYG{p}{(}\PYG{n}{square\PYGZus{}sigma\PYGZus{}mm}\PYG{p}{)}

        \PYG{c+c1}{\PYGZsh{} Error estándar del estimador: media vs mediana}
        \PYG{k}{if} \PYG{n}{n\PYGZus{}edges} \PYG{o+ow}{and} \PYG{n}{n\PYGZus{}edges} \PYG{o}{\PYGZgt{}} \PYG{l+m+mi}{1}\PYG{p}{:}
            \PYG{k}{if} \PYG{n}{aggregation} \PYG{o}{==} \PYG{l+s+s2}{\PYGZdq{}}\PYG{l+s+s2}{mean}\PYG{l+s+s2}{\PYGZdq{}}\PYG{p}{:}
                \PYG{n}{se\PYGZus{}px} \PYG{o}{=} \PYG{n}{std\PYGZus{}px} \PYG{o}{/} \PYG{n}{np}\PYG{o}{.}\PYG{n}{sqrt}\PYG{p}{(}\PYG{n}{n\PYGZus{}edges}\PYG{p}{)}
            \PYG{k}{else}\PYG{p}{:}
                \PYG{n}{se\PYGZus{}px} \PYG{o}{=} \PYG{l+m+mf}{1.253} \PYG{o}{*} \PYG{n}{std\PYGZus{}px} \PYG{o}{/} \PYG{n}{np}\PYG{o}{.}\PYG{n}{sqrt}\PYG{p}{(}\PYG{n}{n\PYGZus{}edges}\PYG{p}{)}  \PYG{c+c1}{\PYGZsh{} aprox. SE de la mediana}
        \PYG{k}{else}\PYG{p}{:}
            \PYG{n}{se\PYGZus{}px} \PYG{o}{=} \PYG{n}{std\PYGZus{}px}

        \PYG{c+c1}{\PYGZsh{} Propagación de errores (1er orden)}
        \PYG{n}{rel\PYGZus{}sq} \PYG{o}{=} \PYG{p}{(}\PYG{n}{sigma\PYGZus{}square} \PYG{o}{/} \PYG{n}{square\PYGZus{}mm}\PYG{p}{)}\PYG{o}{*}\PYG{o}{*}\PYG{l+m+mi}{2} \PYG{k}{if} \PYG{p}{(}\PYG{n}{square\PYGZus{}mm} \PYG{o+ow}{and} \PYG{n}{square\PYGZus{}mm} \PYG{o}{\PYGZgt{}} \PYG{l+m+mi}{0}\PYG{p}{)} \PYG{k}{else} \PYG{l+m+mf}{0.0}
        \PYG{n}{rel\PYGZus{}px} \PYG{o}{=} \PYG{p}{(}\PYG{n}{se\PYGZus{}px} \PYG{o}{/} \PYG{n}{center\PYGZus{}px}\PYG{p}{)}\PYG{o}{*}\PYG{o}{*}\PYG{l+m+mi}{2}
        \PYG{n}{rel\PYGZus{}s}  \PYG{o}{=} \PYG{n}{np}\PYG{o}{.}\PYG{n}{sqrt}\PYG{p}{(}\PYG{n}{rel\PYGZus{}sq} \PYG{o}{+} \PYG{n}{rel\PYGZus{}px}\PYG{p}{)}

        \PYG{n}{sigma\PYGZus{}s} \PYG{o}{=} \PYG{n}{rel\PYGZus{}s} \PYG{o}{*} \PYG{n}{mm\PYGZus{}per\PYGZus{}px}
        \PYG{n}{ci95} \PYG{o}{=} \PYG{p}{(}\PYG{n}{mm\PYGZus{}per\PYGZus{}px} \PYG{o}{\PYGZhy{}} \PYG{l+m+mf}{1.96} \PYG{o}{*} \PYG{n}{sigma\PYGZus{}s}\PYG{p}{,} \PYG{n}{mm\PYGZus{}per\PYGZus{}px} \PYG{o}{+} \PYG{l+m+mf}{1.96} \PYG{o}{*} \PYG{n}{sigma\PYGZus{}s}\PYG{p}{)}

        \PYG{n}{details} \PYG{o}{=} \PYG{p}{\PYGZob{}}
            \PYG{l+s+s2}{\PYGZdq{}}\PYG{l+s+s2}{center\PYGZus{}px}\PYG{l+s+s2}{\PYGZdq{}}\PYG{p}{:} \PYG{n}{center\PYGZus{}px}\PYG{p}{,}
            \PYG{l+s+s2}{\PYGZdq{}}\PYG{l+s+s2}{std\PYGZus{}px\PYGZus{}edges}\PYG{l+s+s2}{\PYGZdq{}}\PYG{p}{:} \PYG{n}{std\PYGZus{}px}\PYG{p}{,}
            \PYG{l+s+s2}{\PYGZdq{}}\PYG{l+s+s2}{se\PYGZus{}px\PYGZus{}used}\PYG{l+s+s2}{\PYGZdq{}}\PYG{p}{:} \PYG{n}{se\PYGZus{}px}\PYG{p}{,}
            \PYG{l+s+s2}{\PYGZdq{}}\PYG{l+s+s2}{n\PYGZus{}edges}\PYG{l+s+s2}{\PYGZdq{}}\PYG{p}{:} \PYG{n}{n\PYGZus{}edges}\PYG{p}{,}
            \PYG{l+s+s2}{\PYGZdq{}}\PYG{l+s+s2}{aggregation}\PYG{l+s+s2}{\PYGZdq{}}\PYG{p}{:} \PYG{n}{aggregation}\PYG{p}{,}
            \PYG{l+s+s2}{\PYGZdq{}}\PYG{l+s+s2}{square\PYGZus{}mm}\PYG{l+s+s2}{\PYGZdq{}}\PYG{p}{:} \PYG{n}{square\PYGZus{}mm}\PYG{p}{,}
            \PYG{l+s+s2}{\PYGZdq{}}\PYG{l+s+s2}{sigma\PYGZus{}square\PYGZus{}mm}\PYG{l+s+s2}{\PYGZdq{}}\PYG{p}{:} \PYG{n}{sigma\PYGZus{}square}\PYG{p}{,}
            \PYG{l+s+s2}{\PYGZdq{}}\PYG{l+s+s2}{rel\PYGZus{}uncertainty\PYGZus{}s}\PYG{l+s+s2}{\PYGZdq{}}\PYG{p}{:} \PYG{n}{rel\PYGZus{}s}
        \PYG{p}{\PYGZcb{}}

    \PYG{k}{return} \PYG{p}{\PYGZob{}}
        \PYG{l+s+s2}{\PYGZdq{}}\PYG{l+s+s2}{mm\PYGZus{}per\PYGZus{}px}\PYG{l+s+s2}{\PYGZdq{}}\PYG{p}{:} \PYG{n}{mm\PYGZus{}per\PYGZus{}px}\PYG{p}{,}
        \PYG{l+s+s2}{\PYGZdq{}}\PYG{l+s+s2}{px\PYGZus{}per\PYGZus{}mm}\PYG{l+s+s2}{\PYGZdq{}}\PYG{p}{:} \PYG{n}{px\PYGZus{}per\PYGZus{}mm}\PYG{p}{,}
        \PYG{l+s+s2}{\PYGZdq{}}\PYG{l+s+s2}{method}\PYG{l+s+s2}{\PYGZdq{}}\PYG{p}{:} \PYG{n}{calib\PYGZus{}json}\PYG{o}{.}\PYG{n}{get}\PYG{p}{(}\PYG{l+s+s2}{\PYGZdq{}}\PYG{l+s+s2}{method}\PYG{l+s+s2}{\PYGZdq{}}\PYG{p}{,} \PYG{l+s+s2}{\PYGZdq{}}\PYG{l+s+s2}{\PYGZdq{}}\PYG{p}{)}\PYG{p}{,}
        \PYG{l+s+s2}{\PYGZdq{}}\PYG{l+s+s2}{pattern}\PYG{l+s+s2}{\PYGZdq{}}\PYG{p}{:} \PYG{n}{pattern}\PYG{p}{,}
        \PYG{l+s+s2}{\PYGZdq{}}\PYG{l+s+s2}{uncertainty\PYGZus{}sigma\PYGZus{}mm\PYGZus{}per\PYGZus{}px}\PYG{l+s+s2}{\PYGZdq{}}\PYG{p}{:} \PYG{n}{sigma\PYGZus{}s}\PYG{p}{,}
        \PYG{l+s+s2}{\PYGZdq{}}\PYG{l+s+s2}{ci95\PYGZus{}mm\PYGZus{}per\PYGZus{}px}\PYG{l+s+s2}{\PYGZdq{}}\PYG{p}{:} \PYG{n}{ci95}\PYG{p}{,}
        \PYG{l+s+s2}{\PYGZdq{}}\PYG{l+s+s2}{diagnostics\PYGZus{}available}\PYG{l+s+s2}{\PYGZdq{}}\PYG{p}{:} \PYG{p}{(}\PYG{n}{std\PYGZus{}px} \PYG{o+ow}{is} \PYG{o+ow}{not} \PYG{k+kc}{None}\PYG{p}{)}\PYG{p}{,}
        \PYG{l+s+s2}{\PYGZdq{}}\PYG{l+s+s2}{details}\PYG{l+s+s2}{\PYGZdq{}}\PYG{p}{:} \PYG{n}{details}
    \PYG{p}{\PYGZcb{}}
\end{sphinxVerbatim}

\end{sphinxuseclass}\end{sphinxVerbatimInput}

\end{sphinxuseclass}
\begin{sphinxuseclass}{cell}\begin{sphinxVerbatimInput}

\begin{sphinxuseclass}{cell_input}
\begin{sphinxVerbatim}[commandchars=\\\{\}]
\PYG{c+c1}{\PYGZsh{} === Ejecutar validación ===}
\PYG{n+nb}{print}\PYG{p}{(}\PYG{l+s+s2}{\PYGZdq{}}\PYG{l+s+s2}{Validación numérica e incertidumbre:}\PYG{l+s+s2}{\PYGZdq{}}\PYG{p}{)}
\PYG{n}{cal\PYGZus{}data} \PYG{o}{=} \PYG{n}{load\PYGZus{}calibration\PYGZus{}dict}\PYG{p}{(}\PYG{n}{CALIB\PYGZus{}FILE}\PYG{p}{)}
\PYG{n}{report} \PYG{o}{=} \PYG{n}{validation\PYGZus{}report}\PYG{p}{(}\PYG{n}{cal\PYGZus{}data}\PYG{p}{,} \PYG{n}{assume\PYGZus{}square\PYGZus{}exact\PYGZus{}mm}\PYG{o}{=}\PYG{k+kc}{True}\PYG{p}{,} \PYG{n}{square\PYGZus{}sigma\PYGZus{}mm}\PYG{o}{=}\PYG{l+m+mf}{0.0}\PYG{p}{)}
\PYG{n}{report}
\end{sphinxVerbatim}

\end{sphinxuseclass}\end{sphinxVerbatimInput}
\begin{sphinxVerbatimOutput}

\begin{sphinxuseclass}{cell_output}
\begin{sphinxVerbatim}[commandchars=\\\{\}]
Validación numérica e incertidumbre:
\end{sphinxVerbatim}

\begin{sphinxVerbatim}[commandchars=\\\{\}]
\PYGZob{}\PYGZsq{}mm\PYGZus{}per\PYGZus{}px\PYGZsq{}: 0.05,
 \PYGZsq{}px\PYGZus{}per\PYGZus{}mm\PYGZsq{}: 20.0,
 \PYGZsq{}method\PYGZsq{}: \PYGZsq{}checkerboard\PYGZsq{},
 \PYGZsq{}pattern\PYGZsq{}: \PYGZsq{}checkerboard\PYGZsq{},
 \PYGZsq{}uncertainty\PYGZus{}sigma\PYGZus{}mm\PYGZus{}per\PYGZus{}px\PYGZsq{}: np.float64(0.0),
 \PYGZsq{}ci95\PYGZus{}mm\PYGZus{}per\PYGZus{}px\PYGZsq{}: (np.float64(0.05), np.float64(0.05)),
 \PYGZsq{}diagnostics\PYGZus{}available\PYGZsq{}: True,
 \PYGZsq{}details\PYGZsq{}: \PYGZob{}\PYGZsq{}center\PYGZus{}px\PYGZsq{}: 200.0,
  \PYGZsq{}std\PYGZus{}px\PYGZus{}edges\PYGZsq{}: 0.0,
  \PYGZsq{}se\PYGZus{}px\PYGZus{}used\PYGZsq{}: np.float64(0.0),
  \PYGZsq{}n\PYGZus{}edges\PYGZsq{}: 144,
  \PYGZsq{}aggregation\PYGZsq{}: \PYGZsq{}median\PYGZsq{},
  \PYGZsq{}square\PYGZus{}mm\PYGZsq{}: 10.0,
  \PYGZsq{}sigma\PYGZus{}square\PYGZus{}mm\PYGZsq{}: 0.0,
  \PYGZsq{}rel\PYGZus{}uncertainty\PYGZus{}s\PYGZsq{}: np.float64(0.0)\PYGZcb{}\PYGZcb{}
\end{sphinxVerbatim}

\end{sphinxuseclass}\end{sphinxVerbatimOutput}

\end{sphinxuseclass}
\sphinxAtStartPar
\sphinxstylestrong{(Re)calibración forzada}
El usuario puede querer forzar una nueva calibración. La función \sphinxcode{\sphinxupquote{recalibrar\_y\_generar\_png()}} fuerza recálculo con el damero y genera un PNG de control visual.

\begin{sphinxuseclass}{cell}\begin{sphinxVerbatimInput}

\begin{sphinxuseclass}{cell_input}
\begin{sphinxVerbatim}[commandchars=\\\{\}]
\PYG{c+c1}{\PYGZsh{}\PYGZsh{} UTILIDAD DE RECALIBRACION FORZADA CON .png DE DIAGNOSTICO}

\PYG{k}{def}\PYG{+w}{ }\PYG{n+nf}{recalibrar\PYGZus{}y\PYGZus{}generar\PYGZus{}png}\PYG{p}{(}
    \PYG{n}{calib\PYGZus{}file}\PYG{p}{:} \PYG{n}{Path}\PYG{p}{,}
    \PYG{n}{img\PYGZus{}path}\PYG{p}{:} \PYG{n}{Path}\PYG{p}{,}
    \PYG{n}{cols\PYGZus{}internal}\PYG{p}{:} \PYG{n+nb}{int}\PYG{p}{,}
    \PYG{n}{rows\PYGZus{}internal}\PYG{p}{:} \PYG{n+nb}{int}\PYG{p}{,}
    \PYG{n}{square\PYGZus{}mm}\PYG{p}{:} \PYG{n+nb}{float}\PYG{p}{,}
    \PYG{n}{aggregation}\PYG{p}{:} \PYG{n+nb}{str} \PYG{o}{=} \PYG{l+s+s2}{\PYGZdq{}}\PYG{l+s+s2}{median}\PYG{l+s+s2}{\PYGZdq{}}\PYG{p}{,}
    \PYG{n}{out\PYGZus{}png}\PYG{p}{:} \PYG{n}{Path} \PYG{o}{=} \PYG{n}{Path}\PYG{p}{(}\PYG{l+s+s2}{\PYGZdq{}}\PYG{l+s+s2}{./diagnosticos/diagnostico\PYGZus{}calibracion.png}\PYG{l+s+s2}{\PYGZdq{}}\PYG{p}{)}\PYG{p}{,}
    \PYG{n}{timestamp\PYGZus{}bgr}\PYG{p}{:} \PYG{n+nb}{tuple} \PYG{o}{=} \PYG{p}{(}\PYG{l+m+mi}{0}\PYG{p}{,} \PYG{l+m+mi}{0}\PYG{p}{,} \PYG{l+m+mi}{255}\PYG{p}{)}\PYG{p}{,}   \PYG{c+c1}{\PYGZsh{} ROJO en BGR (por defecto)}
    \PYG{n}{box\PYGZus{}bgr}\PYG{p}{:} \PYG{n+nb}{tuple} \PYG{o}{=} \PYG{p}{(}\PYG{l+m+mi}{255}\PYG{p}{,} \PYG{l+m+mi}{0}\PYG{p}{,} \PYG{l+m+mi}{0}\PYG{p}{)}\PYG{p}{,}         \PYG{c+c1}{\PYGZsh{} Azul para rectángulo envolvente}
    \PYG{n}{text\PYGZus{}bgr}\PYG{p}{:} \PYG{n+nb}{tuple} \PYG{o}{=} \PYG{p}{(}\PYG{l+m+mi}{50}\PYG{p}{,} \PYG{l+m+mi}{220}\PYG{p}{,} \PYG{l+m+mi}{50}\PYG{p}{)}       \PYG{c+c1}{\PYGZsh{} Verde para textos informativos}
\PYG{p}{)} \PYG{o}{\PYGZhy{}}\PYG{o}{\PYGZgt{}} \PYG{n+nb}{dict}\PYG{p}{:}
\PYG{+w}{    }\PYG{l+s+sd}{\PYGZdq{}\PYGZdq{}\PYGZdq{}}
\PYG{l+s+sd}{    Fuerza la recalibración usando el damero de \PYGZsq{}img\PYGZus{}path\PYGZsq{} y genera un PNG de diagnóstico.}
\PYG{l+s+sd}{    \PYGZhy{} Dibuja esquinas detectadas, rectángulo envolvente y anota mm/px.}
\PYG{l+s+sd}{    \PYGZhy{} Pinta el timestamp en ROJO (BGR=(0,0,255) por defecto).}
\PYG{l+s+sd}{    }
\PYG{l+s+sd}{    Parámetros}
\PYG{l+s+sd}{    \PYGZhy{}\PYGZhy{}\PYGZhy{}\PYGZhy{}\PYGZhy{}\PYGZhy{}\PYGZhy{}\PYGZhy{}\PYGZhy{}\PYGZhy{}}
\PYG{l+s+sd}{    calib\PYGZus{}file : Path}
\PYG{l+s+sd}{        Ruta del JSON donde se guardará la calibración (mm/px, meta).}
\PYG{l+s+sd}{    img\PYGZus{}path : Path}
\PYG{l+s+sd}{        Ruta de la imagen del damero (10x10 → 9x9 intersecciones internas).}
\PYG{l+s+sd}{    cols\PYGZus{}internal, rows\PYGZus{}internal : int}
\PYG{l+s+sd}{        Intersecciones internas del damero (p.ej., 9x9 para 10x10 cuadrados).}
\PYG{l+s+sd}{    square\PYGZus{}mm : float}
\PYG{l+s+sd}{        Tamaño físico del lado de cada cuadrado (mm), p.ej. 10.0.}
\PYG{l+s+sd}{    aggregation : \PYGZob{}\PYGZdq{}median\PYGZdq{},\PYGZdq{}mean\PYGZdq{}\PYGZcb{}}
\PYG{l+s+sd}{        Agregación para estimar px por cuadrado (robusta por defecto: \PYGZdq{}median\PYGZdq{}).}
\PYG{l+s+sd}{    out\PYGZus{}png : Path}
\PYG{l+s+sd}{        Ruta de salida del PNG de diagnóstico.}
\PYG{l+s+sd}{    timestamp\PYGZus{}bgr : tuple}
\PYG{l+s+sd}{        Color BGR para el timestamp (rojo por defecto).}
\PYG{l+s+sd}{    box\PYGZus{}bgr : tuple}
\PYG{l+s+sd}{        Color BGR del rectángulo envolvente (azul por defecto).}
\PYG{l+s+sd}{    text\PYGZus{}bgr : tuple}
\PYG{l+s+sd}{        Color BGR para textos informativos (verde por defecto).}
\PYG{l+s+sd}{    }
\PYG{l+s+sd}{    Returns}
\PYG{l+s+sd}{    \PYGZhy{}\PYGZhy{}\PYGZhy{}\PYGZhy{}\PYGZhy{}\PYGZhy{}\PYGZhy{}}
\PYG{l+s+sd}{    dict con claves:}
\PYG{l+s+sd}{      \PYGZhy{} \PYGZdq{}recal\PYGZdq{}: dict con el resultado de la recalibración (estado, mm\PYGZus{}per\PYGZus{}px, meta)}
\PYG{l+s+sd}{      \PYGZhy{} \PYGZdq{}overlay\PYGZdq{}: dict con \PYGZob{}\PYGZdq{}ok\PYGZdq{}: bool, \PYGZdq{}out\PYGZdq{}: str(ruta\PYGZus{}png)\PYGZcb{} (o motivo de fallo)}
\PYG{l+s+sd}{      \PYGZhy{} \PYGZdq{}mm\PYGZus{}per\PYGZus{}px\PYGZdq{}: float o None (si no disponible)}
\PYG{l+s+sd}{      \PYGZhy{} \PYGZdq{}estado\PYGZdq{}: str (e.g., \PYGZdq{}recalibrado\PYGZdq{}, \PYGZdq{}vigente\PYGZdq{}, \PYGZdq{}sin\PYGZus{}escala\PYGZdq{})}
\PYG{l+s+sd}{    \PYGZdq{}\PYGZdq{}\PYGZdq{}}
    \PYG{k+kn}{from}\PYG{+w}{ }\PYG{n+nn}{datetime}\PYG{+w}{ }\PYG{k+kn}{import} \PYG{n}{datetime}

    \PYG{c+c1}{\PYGZsh{} 1) Forzar recalibración (usa la función del pipeline del notebook)}
    \PYG{n}{recal} \PYG{o}{=} \PYG{n}{run\PYGZus{}calibration\PYGZus{}pipeline\PYGZus{}checkerboard}\PYG{p}{(}
        \PYG{n}{calib\PYGZus{}file}\PYG{o}{=}\PYG{n}{calib\PYGZus{}file}\PYG{p}{,}
        \PYG{n}{img\PYGZus{}path}\PYG{o}{=}\PYG{n}{img\PYGZus{}path}\PYG{p}{,}
        \PYG{n}{cols\PYGZus{}internal}\PYG{o}{=}\PYG{n}{cols\PYGZus{}internal}\PYG{p}{,}
        \PYG{n}{rows\PYGZus{}internal}\PYG{o}{=}\PYG{n}{rows\PYGZus{}internal}\PYG{p}{,}
        \PYG{n}{square\PYGZus{}mm}\PYG{o}{=}\PYG{n}{square\PYGZus{}mm}\PYG{p}{,}
        \PYG{n}{aggregation}\PYG{o}{=}\PYG{n}{aggregation}\PYG{p}{,}
        \PYG{n}{force\PYGZus{}recalibrate}\PYG{o}{=}\PYG{k+kc}{True}
    \PYG{p}{)}

    \PYG{c+c1}{\PYGZsh{} 2) Preparar overlay (detectar de nuevo para dibujar) con TIMESTAMP ROJO}
    \PYG{k}{if} \PYG{o+ow}{not} \PYG{n}{img\PYGZus{}path}\PYG{o}{.}\PYG{n}{exists}\PYG{p}{(}\PYG{p}{)}\PYG{p}{:}
        \PYG{k}{return} \PYG{p}{\PYGZob{}}
            \PYG{l+s+s2}{\PYGZdq{}}\PYG{l+s+s2}{recal}\PYG{l+s+s2}{\PYGZdq{}}\PYG{p}{:} \PYG{n}{recal}\PYG{p}{,}
            \PYG{l+s+s2}{\PYGZdq{}}\PYG{l+s+s2}{overlay}\PYG{l+s+s2}{\PYGZdq{}}\PYG{p}{:} \PYG{p}{\PYGZob{}}\PYG{l+s+s2}{\PYGZdq{}}\PYG{l+s+s2}{ok}\PYG{l+s+s2}{\PYGZdq{}}\PYG{p}{:} \PYG{k+kc}{False}\PYG{p}{,} \PYG{l+s+s2}{\PYGZdq{}}\PYG{l+s+s2}{motivo}\PYG{l+s+s2}{\PYGZdq{}}\PYG{p}{:} \PYG{l+s+sa}{f}\PYG{l+s+s2}{\PYGZdq{}}\PYG{l+s+s2}{Imagen no encontrada: }\PYG{l+s+si}{\PYGZob{}}\PYG{n+nb}{str}\PYG{p}{(}\PYG{n}{img\PYGZus{}path}\PYG{p}{)}\PYG{l+s+si}{\PYGZcb{}}\PYG{l+s+s2}{\PYGZdq{}}\PYG{p}{\PYGZcb{}}\PYG{p}{,}
            \PYG{l+s+s2}{\PYGZdq{}}\PYG{l+s+s2}{mm\PYGZus{}per\PYGZus{}px}\PYG{l+s+s2}{\PYGZdq{}}\PYG{p}{:} \PYG{n+nb}{float}\PYG{p}{(}\PYG{n}{recal}\PYG{p}{[}\PYG{l+s+s2}{\PYGZdq{}}\PYG{l+s+s2}{mm\PYGZus{}per\PYGZus{}px}\PYG{l+s+s2}{\PYGZdq{}}\PYG{p}{]}\PYG{p}{)} \PYG{k}{if} \PYG{n+nb}{isinstance}\PYG{p}{(}\PYG{n}{recal}\PYG{p}{,} \PYG{n+nb}{dict}\PYG{p}{)} \PYG{o+ow}{and} \PYG{l+s+s2}{\PYGZdq{}}\PYG{l+s+s2}{mm\PYGZus{}per\PYGZus{}px}\PYG{l+s+s2}{\PYGZdq{}} \PYG{o+ow}{in} \PYG{n}{recal} \PYG{k}{else} \PYG{k+kc}{None}\PYG{p}{,}
            \PYG{l+s+s2}{\PYGZdq{}}\PYG{l+s+s2}{estado}\PYG{l+s+s2}{\PYGZdq{}}\PYG{p}{:} \PYG{n}{recal}\PYG{o}{.}\PYG{n}{get}\PYG{p}{(}\PYG{l+s+s2}{\PYGZdq{}}\PYG{l+s+s2}{estado}\PYG{l+s+s2}{\PYGZdq{}}\PYG{p}{)} \PYG{k}{if} \PYG{n+nb}{isinstance}\PYG{p}{(}\PYG{n}{recal}\PYG{p}{,} \PYG{n+nb}{dict}\PYG{p}{)} \PYG{k}{else} \PYG{k+kc}{None}
        \PYG{p}{\PYGZcb{}}

    \PYG{n}{img} \PYG{o}{=} \PYG{n}{cv2}\PYG{o}{.}\PYG{n}{imread}\PYG{p}{(}\PYG{n+nb}{str}\PYG{p}{(}\PYG{n}{img\PYGZus{}path}\PYG{p}{)}\PYG{p}{)}
    \PYG{k}{if} \PYG{n}{img} \PYG{o+ow}{is} \PYG{k+kc}{None}\PYG{p}{:}
        \PYG{k}{return} \PYG{p}{\PYGZob{}}
            \PYG{l+s+s2}{\PYGZdq{}}\PYG{l+s+s2}{recal}\PYG{l+s+s2}{\PYGZdq{}}\PYG{p}{:} \PYG{n}{recal}\PYG{p}{,}
            \PYG{l+s+s2}{\PYGZdq{}}\PYG{l+s+s2}{overlay}\PYG{l+s+s2}{\PYGZdq{}}\PYG{p}{:} \PYG{p}{\PYGZob{}}\PYG{l+s+s2}{\PYGZdq{}}\PYG{l+s+s2}{ok}\PYG{l+s+s2}{\PYGZdq{}}\PYG{p}{:} \PYG{k+kc}{False}\PYG{p}{,} \PYG{l+s+s2}{\PYGZdq{}}\PYG{l+s+s2}{motivo}\PYG{l+s+s2}{\PYGZdq{}}\PYG{p}{:} \PYG{l+s+sa}{f}\PYG{l+s+s2}{\PYGZdq{}}\PYG{l+s+s2}{No se pudo abrir la imagen: }\PYG{l+s+si}{\PYGZob{}}\PYG{n+nb}{str}\PYG{p}{(}\PYG{n}{img\PYGZus{}path}\PYG{p}{)}\PYG{l+s+si}{\PYGZcb{}}\PYG{l+s+s2}{\PYGZdq{}}\PYG{p}{\PYGZcb{}}\PYG{p}{,}
            \PYG{l+s+s2}{\PYGZdq{}}\PYG{l+s+s2}{mm\PYGZus{}per\PYGZus{}px}\PYG{l+s+s2}{\PYGZdq{}}\PYG{p}{:} \PYG{n+nb}{float}\PYG{p}{(}\PYG{n}{recal}\PYG{p}{[}\PYG{l+s+s2}{\PYGZdq{}}\PYG{l+s+s2}{mm\PYGZus{}per\PYGZus{}px}\PYG{l+s+s2}{\PYGZdq{}}\PYG{p}{]}\PYG{p}{)} \PYG{k}{if} \PYG{n+nb}{isinstance}\PYG{p}{(}\PYG{n}{recal}\PYG{p}{,} \PYG{n+nb}{dict}\PYG{p}{)} \PYG{o+ow}{and} \PYG{l+s+s2}{\PYGZdq{}}\PYG{l+s+s2}{mm\PYGZus{}per\PYGZus{}px}\PYG{l+s+s2}{\PYGZdq{}} \PYG{o+ow}{in} \PYG{n}{recal} \PYG{k}{else} \PYG{k+kc}{None}\PYG{p}{,}
            \PYG{l+s+s2}{\PYGZdq{}}\PYG{l+s+s2}{estado}\PYG{l+s+s2}{\PYGZdq{}}\PYG{p}{:} \PYG{n}{recal}\PYG{o}{.}\PYG{n}{get}\PYG{p}{(}\PYG{l+s+s2}{\PYGZdq{}}\PYG{l+s+s2}{estado}\PYG{l+s+s2}{\PYGZdq{}}\PYG{p}{)} \PYG{k}{if} \PYG{n+nb}{isinstance}\PYG{p}{(}\PYG{n}{recal}\PYG{p}{,} \PYG{n+nb}{dict}\PYG{p}{)} \PYG{k}{else} \PYG{k+kc}{None}
        \PYG{p}{\PYGZcb{}}

    \PYG{n}{gray} \PYG{o}{=} \PYG{n}{cv2}\PYG{o}{.}\PYG{n}{cvtColor}\PYG{p}{(}\PYG{n}{img}\PYG{p}{,} \PYG{n}{cv2}\PYG{o}{.}\PYG{n}{COLOR\PYGZus{}BGR2GRAY}\PYG{p}{)}
    \PYG{n}{flags} \PYG{o}{=} \PYG{n}{cv2}\PYG{o}{.}\PYG{n}{CALIB\PYGZus{}CB\PYGZus{}ADAPTIVE\PYGZus{}THRESH} \PYG{o}{|} \PYG{n}{cv2}\PYG{o}{.}\PYG{n}{CALIB\PYGZus{}CB\PYGZus{}NORMALIZE\PYGZus{}IMAGE}
    \PYG{n}{found}\PYG{p}{,} \PYG{n}{corners} \PYG{o}{=} \PYG{n}{cv2}\PYG{o}{.}\PYG{n}{findChessboardCorners}\PYG{p}{(}\PYG{n}{gray}\PYG{p}{,} \PYG{p}{(}\PYG{n}{cols\PYGZus{}internal}\PYG{p}{,} \PYG{n}{rows\PYGZus{}internal}\PYG{p}{)}\PYG{p}{,} \PYG{n}{flags}\PYG{p}{)}

    \PYG{n}{vis} \PYG{o}{=} \PYG{n}{img}\PYG{o}{.}\PYG{n}{copy}\PYG{p}{(}\PYG{p}{)}
    \PYG{k}{if} \PYG{o+ow}{not} \PYG{n}{found} \PYG{o+ow}{or} \PYG{n}{corners} \PYG{o+ow}{is} \PYG{k+kc}{None}\PYG{p}{:}
        \PYG{n}{cv2}\PYG{o}{.}\PYG{n}{putText}\PYG{p}{(}
            \PYG{n}{vis}\PYG{p}{,} \PYG{l+s+s2}{\PYGZdq{}}\PYG{l+s+s2}{Checkerboard NO detectado}\PYG{l+s+s2}{\PYGZdq{}}\PYG{p}{,} \PYG{p}{(}\PYG{l+m+mi}{20}\PYG{p}{,} \PYG{l+m+mi}{40}\PYG{p}{)}\PYG{p}{,}
            \PYG{n}{cv2}\PYG{o}{.}\PYG{n}{FONT\PYGZus{}HERSHEY\PYGZus{}SIMPLEX}\PYG{p}{,} \PYG{l+m+mf}{1.0}\PYG{p}{,} \PYG{p}{(}\PYG{l+m+mi}{0}\PYG{p}{,} \PYG{l+m+mi}{0}\PYG{p}{,} \PYG{l+m+mi}{255}\PYG{p}{)}\PYG{p}{,} \PYG{l+m+mi}{2}\PYG{p}{,} \PYG{n}{cv2}\PYG{o}{.}\PYG{n}{LINE\PYGZus{}AA}
        \PYG{p}{)}
        \PYG{c+c1}{\PYGZsh{} timestamp en rojo aunque no haya detección}
        \PYG{n}{cv2}\PYG{o}{.}\PYG{n}{putText}\PYG{p}{(}
            \PYG{n}{vis}\PYG{p}{,} \PYG{l+s+sa}{f}\PYG{l+s+s2}{\PYGZdq{}}\PYG{l+s+s2}{Timestamp: }\PYG{l+s+si}{\PYGZob{}}\PYG{n}{datetime}\PYG{o}{.}\PYG{n}{now}\PYG{p}{(}\PYG{p}{)}\PYG{o}{.}\PYG{n}{isoformat}\PYG{p}{(}\PYG{n}{timespec}\PYG{o}{=}\PYG{l+s+s1}{\PYGZsq{}}\PYG{l+s+s1}{seconds}\PYG{l+s+s1}{\PYGZsq{}}\PYG{p}{)}\PYG{l+s+si}{\PYGZcb{}}\PYG{l+s+s2}{\PYGZdq{}}\PYG{p}{,} \PYG{p}{(}\PYG{l+m+mi}{20}\PYG{p}{,} \PYG{l+m+mi}{80}\PYG{p}{)}\PYG{p}{,}
            \PYG{n}{cv2}\PYG{o}{.}\PYG{n}{FONT\PYGZus{}HERSHEY\PYGZus{}SIMPLEX}\PYG{p}{,} \PYG{l+m+mf}{0.7}\PYG{p}{,} \PYG{n}{timestamp\PYGZus{}bgr}\PYG{p}{,} \PYG{l+m+mi}{2}\PYG{p}{,} \PYG{n}{cv2}\PYG{o}{.}\PYG{n}{LINE\PYGZus{}AA}
        \PYG{p}{)}
    \PYG{k}{else}\PYG{p}{:}
        \PYG{c+c1}{\PYGZsh{} Refinar esquinas y dibujarlas}
        \PYG{n}{criteria} \PYG{o}{=} \PYG{p}{(}\PYG{n}{cv2}\PYG{o}{.}\PYG{n}{TERM\PYGZus{}CRITERIA\PYGZus{}EPS} \PYG{o}{+} \PYG{n}{cv2}\PYG{o}{.}\PYG{n}{TERM\PYGZus{}CRITERIA\PYGZus{}MAX\PYGZus{}ITER}\PYG{p}{,} \PYG{l+m+mi}{50}\PYG{p}{,} \PYG{l+m+mf}{1e\PYGZhy{}4}\PYG{p}{)}
        \PYG{n}{corners} \PYG{o}{=} \PYG{n}{cv2}\PYG{o}{.}\PYG{n}{cornerSubPix}\PYG{p}{(}\PYG{n}{gray}\PYG{p}{,} \PYG{n}{corners}\PYG{p}{,} \PYG{p}{(}\PYG{l+m+mi}{11}\PYG{p}{,} \PYG{l+m+mi}{11}\PYG{p}{)}\PYG{p}{,} \PYG{p}{(}\PYG{o}{\PYGZhy{}}\PYG{l+m+mi}{1}\PYG{p}{,} \PYG{o}{\PYGZhy{}}\PYG{l+m+mi}{1}\PYG{p}{)}\PYG{p}{,} \PYG{n}{criteria}\PYG{p}{)}
        \PYG{n}{cv2}\PYG{o}{.}\PYG{n}{drawChessboardCorners}\PYG{p}{(}\PYG{n}{vis}\PYG{p}{,} \PYG{p}{(}\PYG{n}{cols\PYGZus{}internal}\PYG{p}{,} \PYG{n}{rows\PYGZus{}internal}\PYG{p}{)}\PYG{p}{,} \PYG{n}{corners}\PYG{p}{,} \PYG{n}{found}\PYG{p}{)}

        \PYG{c+c1}{\PYGZsh{} Rectángulo envolvente}
        \PYG{n}{pts} \PYG{o}{=} \PYG{n}{corners}\PYG{o}{.}\PYG{n}{reshape}\PYG{p}{(}\PYG{o}{\PYGZhy{}}\PYG{l+m+mi}{1}\PYG{p}{,} \PYG{l+m+mi}{2}\PYG{p}{)}
        \PYG{n}{x\PYGZus{}min}\PYG{p}{,} \PYG{n}{y\PYGZus{}min} \PYG{o}{=} \PYG{n}{np}\PYG{o}{.}\PYG{n}{floor}\PYG{p}{(}\PYG{n}{pts}\PYG{o}{.}\PYG{n}{min}\PYG{p}{(}\PYG{n}{axis}\PYG{o}{=}\PYG{l+m+mi}{0}\PYG{p}{)}\PYG{p}{)}\PYG{o}{.}\PYG{n}{astype}\PYG{p}{(}\PYG{n+nb}{int}\PYG{p}{)}
        \PYG{n}{x\PYGZus{}max}\PYG{p}{,} \PYG{n}{y\PYGZus{}max} \PYG{o}{=} \PYG{n}{np}\PYG{o}{.}\PYG{n}{ceil}\PYG{p}{(}\PYG{n}{pts}\PYG{o}{.}\PYG{n}{max}\PYG{p}{(}\PYG{n}{axis}\PYG{o}{=}\PYG{l+m+mi}{0}\PYG{p}{)}\PYG{p}{)}\PYG{o}{.}\PYG{n}{astype}\PYG{p}{(}\PYG{n+nb}{int}\PYG{p}{)}
        \PYG{n}{cv2}\PYG{o}{.}\PYG{n}{rectangle}\PYG{p}{(}\PYG{n}{vis}\PYG{p}{,} \PYG{p}{(}\PYG{n}{x\PYGZus{}min}\PYG{p}{,} \PYG{n}{y\PYGZus{}min}\PYG{p}{)}\PYG{p}{,} \PYG{p}{(}\PYG{n}{x\PYGZus{}max}\PYG{p}{,} \PYG{n}{y\PYGZus{}max}\PYG{p}{)}\PYG{p}{,} \PYG{n}{box\PYGZus{}bgr}\PYG{p}{,} \PYG{l+m+mi}{2}\PYG{p}{)}

        \PYG{c+c1}{\PYGZsh{} Textos informativos (verde por defecto)}
        \PYG{n}{y0} \PYG{o}{=} \PYG{l+m+mi}{30}
        \PYG{n}{cv2}\PYG{o}{.}\PYG{n}{putText}\PYG{p}{(}
            \PYG{n}{vis}\PYG{p}{,} \PYG{l+s+sa}{f}\PYG{l+s+s2}{\PYGZdq{}}\PYG{l+s+s2}{Damero }\PYG{l+s+si}{\PYGZob{}}\PYG{n}{cols\PYGZus{}internal}\PYG{l+s+si}{\PYGZcb{}}\PYG{l+s+s2}{x}\PYG{l+s+si}{\PYGZob{}}\PYG{n}{rows\PYGZus{}internal}\PYG{l+s+si}{\PYGZcb{}}\PYG{l+s+s2}{ (intersecciones)}\PYG{l+s+s2}{\PYGZdq{}}\PYG{p}{,} \PYG{p}{(}\PYG{l+m+mi}{20}\PYG{p}{,} \PYG{n}{y0}\PYG{p}{)}\PYG{p}{,}
            \PYG{n}{cv2}\PYG{o}{.}\PYG{n}{FONT\PYGZus{}HERSHEY\PYGZus{}SIMPLEX}\PYG{p}{,} \PYG{l+m+mf}{0.8}\PYG{p}{,} \PYG{n}{text\PYGZus{}bgr}\PYG{p}{,} \PYG{l+m+mi}{2}\PYG{p}{,} \PYG{n}{cv2}\PYG{o}{.}\PYG{n}{LINE\PYGZus{}AA}
        \PYG{p}{)}
        \PYG{n}{y0} \PYG{o}{+}\PYG{o}{=} \PYG{l+m+mi}{30}
        \PYG{n}{mmpp} \PYG{o}{=} \PYG{n+nb}{float}\PYG{p}{(}\PYG{n}{recal}\PYG{p}{[}\PYG{l+s+s2}{\PYGZdq{}}\PYG{l+s+s2}{mm\PYGZus{}per\PYGZus{}px}\PYG{l+s+s2}{\PYGZdq{}}\PYG{p}{]}\PYG{p}{)} \PYG{k}{if} \PYG{n+nb}{isinstance}\PYG{p}{(}\PYG{n}{recal}\PYG{p}{,} \PYG{n+nb}{dict}\PYG{p}{)} \PYG{o+ow}{and} \PYG{l+s+s2}{\PYGZdq{}}\PYG{l+s+s2}{mm\PYGZus{}per\PYGZus{}px}\PYG{l+s+s2}{\PYGZdq{}} \PYG{o+ow}{in} \PYG{n}{recal} \PYG{k}{else} \PYG{k+kc}{None}
        \PYG{k}{if} \PYG{n}{mmpp} \PYG{o+ow}{is} \PYG{o+ow}{not} \PYG{k+kc}{None}\PYG{p}{:}
            \PYG{n}{cv2}\PYG{o}{.}\PYG{n}{putText}\PYG{p}{(}
                \PYG{n}{vis}\PYG{p}{,} \PYG{l+s+sa}{f}\PYG{l+s+s2}{\PYGZdq{}}\PYG{l+s+s2}{Escala: }\PYG{l+s+si}{\PYGZob{}}\PYG{n}{mmpp}\PYG{l+s+si}{:}\PYG{l+s+s2}{.6f}\PYG{l+s+si}{\PYGZcb{}}\PYG{l+s+s2}{ mm/px}\PYG{l+s+s2}{\PYGZdq{}}\PYG{p}{,} \PYG{p}{(}\PYG{l+m+mi}{20}\PYG{p}{,} \PYG{n}{y0}\PYG{p}{)}\PYG{p}{,}
                \PYG{n}{cv2}\PYG{o}{.}\PYG{n}{FONT\PYGZus{}HERSHEY\PYGZus{}SIMPLEX}\PYG{p}{,} \PYG{l+m+mf}{0.8}\PYG{p}{,} \PYG{n}{text\PYGZus{}bgr}\PYG{p}{,} \PYG{l+m+mi}{2}\PYG{p}{,} \PYG{n}{cv2}\PYG{o}{.}\PYG{n}{LINE\PYGZus{}AA}
            \PYG{p}{)}
            \PYG{n}{y0} \PYG{o}{+}\PYG{o}{=} \PYG{l+m+mi}{30}

        \PYG{c+c1}{\PYGZsh{} TIMESTAMP en ROJO}
        \PYG{n}{cv2}\PYG{o}{.}\PYG{n}{putText}\PYG{p}{(}
            \PYG{n}{vis}\PYG{p}{,} \PYG{l+s+sa}{f}\PYG{l+s+s2}{\PYGZdq{}}\PYG{l+s+s2}{Timestamp: }\PYG{l+s+si}{\PYGZob{}}\PYG{n}{datetime}\PYG{o}{.}\PYG{n}{now}\PYG{p}{(}\PYG{p}{)}\PYG{o}{.}\PYG{n}{isoformat}\PYG{p}{(}\PYG{n}{timespec}\PYG{o}{=}\PYG{l+s+s1}{\PYGZsq{}}\PYG{l+s+s1}{seconds}\PYG{l+s+s1}{\PYGZsq{}}\PYG{p}{)}\PYG{l+s+si}{\PYGZcb{}}\PYG{l+s+s2}{\PYGZdq{}}\PYG{p}{,} \PYG{p}{(}\PYG{l+m+mi}{20}\PYG{p}{,} \PYG{n}{y0}\PYG{p}{)}\PYG{p}{,}
            \PYG{n}{cv2}\PYG{o}{.}\PYG{n}{FONT\PYGZus{}HERSHEY\PYGZus{}SIMPLEX}\PYG{p}{,} \PYG{l+m+mf}{0.7}\PYG{p}{,} \PYG{n}{timestamp\PYGZus{}bgr}\PYG{p}{,} \PYG{l+m+mi}{2}\PYG{p}{,} \PYG{n}{cv2}\PYG{o}{.}\PYG{n}{LINE\PYGZus{}AA}
        \PYG{p}{)}

    \PYG{c+c1}{\PYGZsh{} 3) Guardar PNG}
    \PYG{n}{out\PYGZus{}png}\PYG{o}{.}\PYG{n}{parent}\PYG{o}{.}\PYG{n}{mkdir}\PYG{p}{(}\PYG{n}{parents}\PYG{o}{=}\PYG{k+kc}{True}\PYG{p}{,} \PYG{n}{exist\PYGZus{}ok}\PYG{o}{=}\PYG{k+kc}{True}\PYG{p}{)}
    \PYG{n}{ok} \PYG{o}{=} \PYG{n}{cv2}\PYG{o}{.}\PYG{n}{imwrite}\PYG{p}{(}\PYG{n+nb}{str}\PYG{p}{(}\PYG{n}{out\PYGZus{}png}\PYG{p}{)}\PYG{p}{,} \PYG{n}{vis}\PYG{p}{)}

    \PYG{k}{return} \PYG{p}{\PYGZob{}}
        \PYG{l+s+s2}{\PYGZdq{}}\PYG{l+s+s2}{recal}\PYG{l+s+s2}{\PYGZdq{}}\PYG{p}{:} \PYG{n}{recal}\PYG{p}{,}
        \PYG{l+s+s2}{\PYGZdq{}}\PYG{l+s+s2}{overlay}\PYG{l+s+s2}{\PYGZdq{}}\PYG{p}{:} \PYG{p}{\PYGZob{}}\PYG{l+s+s2}{\PYGZdq{}}\PYG{l+s+s2}{ok}\PYG{l+s+s2}{\PYGZdq{}}\PYG{p}{:} \PYG{n+nb}{bool}\PYG{p}{(}\PYG{n}{ok}\PYG{p}{)}\PYG{p}{,} \PYG{l+s+s2}{\PYGZdq{}}\PYG{l+s+s2}{out}\PYG{l+s+s2}{\PYGZdq{}}\PYG{p}{:} \PYG{n+nb}{str}\PYG{p}{(}\PYG{n}{out\PYGZus{}png}\PYG{p}{)} \PYG{k}{if} \PYG{n}{ok} \PYG{k}{else} \PYG{k+kc}{None}\PYG{p}{\PYGZcb{}}\PYG{p}{,}
        \PYG{l+s+s2}{\PYGZdq{}}\PYG{l+s+s2}{mm\PYGZus{}per\PYGZus{}px}\PYG{l+s+s2}{\PYGZdq{}}\PYG{p}{:} \PYG{n+nb}{float}\PYG{p}{(}\PYG{n}{recal}\PYG{p}{[}\PYG{l+s+s2}{\PYGZdq{}}\PYG{l+s+s2}{mm\PYGZus{}per\PYGZus{}px}\PYG{l+s+s2}{\PYGZdq{}}\PYG{p}{]}\PYG{p}{)} \PYG{k}{if} \PYG{n+nb}{isinstance}\PYG{p}{(}\PYG{n}{recal}\PYG{p}{,} \PYG{n+nb}{dict}\PYG{p}{)} \PYG{o+ow}{and} \PYG{l+s+s2}{\PYGZdq{}}\PYG{l+s+s2}{mm\PYGZus{}per\PYGZus{}px}\PYG{l+s+s2}{\PYGZdq{}} \PYG{o+ow}{in} \PYG{n}{recal} \PYG{k}{else} \PYG{k+kc}{None}\PYG{p}{,}
        \PYG{l+s+s2}{\PYGZdq{}}\PYG{l+s+s2}{estado}\PYG{l+s+s2}{\PYGZdq{}}\PYG{p}{:} \PYG{n}{recal}\PYG{o}{.}\PYG{n}{get}\PYG{p}{(}\PYG{l+s+s2}{\PYGZdq{}}\PYG{l+s+s2}{estado}\PYG{l+s+s2}{\PYGZdq{}}\PYG{p}{)} \PYG{k}{if} \PYG{n+nb}{isinstance}\PYG{p}{(}\PYG{n}{recal}\PYG{p}{,} \PYG{n+nb}{dict}\PYG{p}{)} \PYG{k}{else} \PYG{k+kc}{None}
    \PYG{p}{\PYGZcb{}}
\end{sphinxVerbatim}

\end{sphinxuseclass}\end{sphinxVerbatimInput}

\end{sphinxuseclass}
\sphinxAtStartPar
Un ejemplo de uso de esta función de recalibrado forzado sería:

\begin{sphinxuseclass}{cell}\begin{sphinxVerbatimInput}

\begin{sphinxuseclass}{cell_input}
\begin{sphinxVerbatim}[commandchars=\\\{\}]
\PYG{n}{OUT\PYGZus{}PNG} \PYG{o}{=} \PYG{n}{Path}\PYG{p}{(}\PYG{l+s+s2}{\PYGZdq{}}\PYG{l+s+s2}{./diagnosticos/recalibrado.png}\PYG{l+s+s2}{\PYGZdq{}}\PYG{p}{)}
\PYG{n}{res} \PYG{o}{=} \PYG{n}{recalibrar\PYGZus{}y\PYGZus{}generar\PYGZus{}png}\PYG{p}{(}
    \PYG{n}{calib\PYGZus{}file}\PYG{o}{=}\PYG{n}{CALIB\PYGZus{}FILE}\PYG{p}{,}
    \PYG{n}{img\PYGZus{}path}\PYG{o}{=}\PYG{n}{IMG\PYGZus{}PATH}\PYG{p}{,}
    \PYG{n}{cols\PYGZus{}internal}\PYG{o}{=}\PYG{n}{CHECKERBOARD\PYGZus{}INTERNAL\PYGZus{}COLS}\PYG{p}{,}
    \PYG{n}{rows\PYGZus{}internal}\PYG{o}{=}\PYG{n}{CHECKERBOARD\PYGZus{}INTERNAL\PYGZus{}ROWS}\PYG{p}{,}
    \PYG{n}{square\PYGZus{}mm}\PYG{o}{=}\PYG{n}{CHECKERBOARD\PYGZus{}SQUARE\PYGZus{}MM}\PYG{p}{,}
    \PYG{n}{aggregation}\PYG{o}{=}\PYG{n}{AGGREGATION}\PYG{p}{,}
    \PYG{n}{out\PYGZus{}png}\PYG{o}{=}\PYG{n}{OUT\PYGZus{}PNG}
\PYG{p}{)}

\PYG{n}{res}

\PYG{c+c1}{\PYGZsh{} Visualización del .png de (re)calibrado}
\PYG{n}{DISPLAY\PYGZus{}WIDTH} \PYG{o}{=} \PYG{l+m+mi}{800}  \PYG{c+c1}{\PYGZsh{} px}
\PYG{n}{display}\PYG{p}{(}\PYG{n}{Image}\PYG{p}{(}\PYG{n}{filename}\PYG{o}{=}\PYG{n+nb}{str}\PYG{p}{(}\PYG{n}{OUT\PYGZus{}PNG}\PYG{p}{)}\PYG{p}{,} \PYG{n}{embed}\PYG{o}{=}\PYG{k+kc}{True}\PYG{p}{,} \PYG{n}{width}\PYG{o}{=}\PYG{n}{DISPLAY\PYGZus{}WIDTH}\PYG{p}{)}\PYG{p}{)}
\end{sphinxVerbatim}

\end{sphinxuseclass}\end{sphinxVerbatimInput}
\begin{sphinxVerbatimOutput}

\begin{sphinxuseclass}{cell_output}
\noindent\sphinxincludegraphics[width=800\sphinxpxdimen]{{e9eeb1de0a4605fa31fafa9fa1aef09bbf312616aaa9c03ed22fe7e55e629e11}.png}

\end{sphinxuseclass}\end{sphinxVerbatimOutput}

\end{sphinxuseclass}
\sphinxstepscope


\section{M3 \sphinxhyphen{} Imagen}
\label{\detokenize{content/01/Modulo-3:m3-imagen}}\label{\detokenize{content/01/Modulo-3::doc}}
\sphinxAtStartPar
Este módulo es el núcleo del sistema de visión artificial. Operativamente está compuesto por cuatro bloques funcionales que siguen el flujo recogido en la siguiente imagen:

\begin{figure}[htbp]
\centering
\capstart

\noindent\sphinxincludegraphics[width=0.250\linewidth]{{Modulo-3}.png}
\caption{Bloques funcionales del módulo de imagen (visión artificial)}\label{\detokenize{content/01/Modulo-3:figura-wp1-imagen-6}}\end{figure}

\sphinxAtStartPar
El motor de este módulo se sustenta sobre el algoritmo GrabCut. GrabCut es un método semiautomático para segmentar objetos en una imagen, es decir, para separar una región de interés del resto de la escena. Fue propuesto en 2004 por Rother, Kolmogorov y Blake {[}\sphinxhref{https://doi.org/10.1145/1015706.1015720}{Rother et al., 2004}{]}, y se ha convertido en una herramienta clásica en visión por computadora debido a su equilibrio entre precisión y facilidad de uso.

\sphinxAtStartPar
En su planteamiento original, el algoritmo necesita una pequeña intervención del usuario para trazar un rectángulo que contenga aproximadamente el objeto que se desea extraer. A partir de ahí, GrabCut trabaja de forma iterativa para refinar la segmentación sin necesidad de más entradas, aunque también permite correcciones manuales si es necesario.

\sphinxAtStartPar
Matemáticamente, GrabCut modela los colores del objeto (primer plano o foreground) y del entorno (fondo o background) mediante modelos de mezcla gaussiana (GMM, por sus siglas en inglés){[}\sphinxhref{https://doi.org/10.1007/978-1-4615-7566-5}{Bishop, 2006}{]}. Estos modelos permiten representar distribuciones complejas de color como la combinación de varias campanas gaussianas, cada una capturando un tono o textura característica. Así, en lugar de asumir que todos los píxeles del objeto tienen el mismo color, el GMM reconoce que pueden existir múltiples modos (o grupos) de colores dentro de la misma región.

\sphinxAtStartPar
Además del color, GrabCut también considera la estructura espacial de la imagen: píxeles cercanos tienden a pertenecer a la misma clase. Esta idea se modela mediante un campo aleatorio de Markov (MRF), que penaliza soluciones en las que píxeles vecinos tienen etiquetas muy diferentes sin una justificación clara en los datos (por ejemplo, un cambio brusco de color). De esta forma, se evitan segmentaciones irregulares o con ruido {[}\sphinxhref{https://doi.org/10.1007/978-1-84882-437-9}{Li, 2009}{]}.

\sphinxAtStartPar
Todo esto se combina en una función de energía que mide lo “buena” que es una segmentación propuesta. Minimizar esta energía equivale a encontrar la máscara que mejor explica los colores observados (usando los GMMs) y que, al mismo tiempo, sea espacialmente coherente (gracias al MRF).

\sphinxAtStartPar
En el contexto de FLATCLASS este algoritmo se ha adaptado para no contar con intervención humana inicial. Para aislar el objeto de interés (lenguado) se ha usado una cinta transportadora de color azul, cuyo tono varía ligeramente debido a iluminación escasa, no homogénea, sombras, presencia de gotas de agua o textura del material. Dado que el objeto y el fondo tienen colores claramente distintos, y aún asumiendo que el fondo no es uniforme, se puede indicar al algoritmo que el fondo es predominantemente azul y que el objeto (lenguado) ocupa aproximadamente el centro de la imagen.

\begin{figure}[htbp]
\centering
\capstart

\noindent\sphinxincludegraphics[width=0.500\linewidth]{{P22C-ML_0_calibrated}.png}
\caption{Imagen real de lenguado obtenida con el sistema de captura}\label{\detokenize{content/01/Modulo-3:figura-wp1-imagen-7}}\end{figure}

\sphinxAtStartPar
Para mejorar la detección, el \sphinxstyleemphasis{pipeline} acorta el área de exploración al rectángulo que engloba la cinta y el pez, definiendo de forma automática la región de interés (ROI). Para ello se usa detección cromática robusta de la cinta azul y en un análisis perfilado por columnas que localiza el tramo contiguo donde la cobertura del color azul es dominante a lo largo de toda la altura de la imagen. Primero, el prefiltrado (NLMeans + bilateral + CLAHE) estabiliza ruido e iluminación para que el tono azul sea más estable espectralmente (el bilateral preserva bordes {[}\sphinxhref{https://www.academia.edu/103435418/Bilateral\_filtering\_for\_gray\_and\_color\_images}{Tomasi et al., 1998}{]}. A continuación, en espacio HSV se estima de forma adaptativa el intervalo cromático del azul calculando el pico del histograma del matiz (H) en una banda central y con saturación alta (S) —estrategia inspirada en histogramas de color para indexación/segmentación—, lo que compensa variaciones de cámara e iluminación. Con ese umbral dinámico se genera una máscara binaria “azul”; sobre ella, para cada columna \(j\) se computa la razón de cobertura azul \(r_j=\dfrac{1}{H} \sum^H_{y=1} {blue(y,j)}\). Se marcan como “cinta” las columnas con \(r_j \geq \tau\) y se elige el tramo contíguo más largo \([x_1,x_2]\) de columnas que cumplen el criterio. Esto elimina de forma natural las columnas metálicas laterales correspondientes al soporte de la cámara (fotografía ), que o bien tienen \(r_j\) bajo o aparecen como tramos cortos no contiguos.

\sphinxAtStartPar
En vertical se repite el razonamiento sobre filas dentro de \([x_1,x_2]\) para obtener \([y_1,y_2]\). El ROI final es el rectángulo acolchado \(R=(x,y,w,h)\) que engloba \([x_1,x_2] \times [y_1,y_2]\). Esta determinación geométrica evita depender de cierres morfológicos agresivos (que pueden “invadir” el metal) y se apoya en principios de morfología matemática y conectividad {[}\sphinxhref{https://link.springer.com/book/10.1007/978-3-662-05088-0}{Soille, 2003}{]}. La ROI así calculada se utiliza después para restringir GrabCut, etiquetando como “probable fondo” los bordes de la imagen y como “probable primer plano” la región central, al tiempo que se definen en el código tonos de azul como semillas reforzadas para GrabCut, garantizando que la segmentación psoterior se concentre en el pez y la cinta {[}\sphinxhref{https://doi.org/10.1145/1015706.1015720}{Rother et al., 2004}{]}.

\begin{sphinxuseclass}{cell}\begin{sphinxVerbatimInput}

\begin{sphinxuseclass}{cell_input}
\begin{sphinxVerbatim}[commandchars=\\\{\}]
\PYG{k+kn}{import}\PYG{+w}{ }\PYG{n+nn}{cv2}
\PYG{k+kn}{import}\PYG{+w}{ }\PYG{n+nn}{json}
\PYG{k+kn}{import}\PYG{+w}{ }\PYG{n+nn}{numpy}\PYG{+w}{ }\PYG{k}{as}\PYG{+w}{ }\PYG{n+nn}{np}
\PYG{k+kn}{import}\PYG{+w}{ }\PYG{n+nn}{pandas}\PYG{+w}{ }\PYG{k}{as}\PYG{+w}{ }\PYG{n+nn}{pd}
\PYG{k+kn}{from}\PYG{+w}{ }\PYG{n+nn}{pathlib}\PYG{+w}{ }\PYG{k+kn}{import} \PYG{n}{Path}
\PYG{k+kn}{import}\PYG{+w}{ }\PYG{n+nn}{matplotlib}\PYG{n+nn}{.}\PYG{n+nn}{pyplot}\PYG{+w}{ }\PYG{k}{as}\PYG{+w}{ }\PYG{n+nn}{plt}

\PYG{c+c1}{\PYGZsh{} \PYGZhy{}\PYGZhy{}\PYGZhy{}\PYGZhy{}\PYGZhy{}\PYGZhy{}\PYGZhy{}\PYGZhy{}\PYGZhy{}\PYGZhy{}\PYGZhy{}\PYGZhy{}\PYGZhy{}\PYGZhy{}\PYGZhy{}\PYGZhy{}\PYGZhy{}\PYGZhy{}\PYGZhy{}\PYGZhy{}\PYGZhy{}\PYGZhy{}\PYGZhy{}\PYGZhy{}\PYGZhy{}}
\PYG{c+c1}{\PYGZsh{} CONFIG}
\PYG{c+c1}{\PYGZsh{} \PYGZhy{}\PYGZhy{}\PYGZhy{}\PYGZhy{}\PYGZhy{}\PYGZhy{}\PYGZhy{}\PYGZhy{}\PYGZhy{}\PYGZhy{}\PYGZhy{}\PYGZhy{}\PYGZhy{}\PYGZhy{}\PYGZhy{}\PYGZhy{}\PYGZhy{}\PYGZhy{}\PYGZhy{}\PYGZhy{}\PYGZhy{}\PYGZhy{}\PYGZhy{}\PYGZhy{}\PYGZhy{}}
\PYG{n}{CONFIG} \PYG{o}{=} \PYG{p}{\PYGZob{}}
    \PYG{l+s+s2}{\PYGZdq{}}\PYG{l+s+s2}{img\PYGZus{}path}\PYG{l+s+s2}{\PYGZdq{}}\PYG{p}{:} \PYG{l+s+s2}{\PYGZdq{}}\PYG{l+s+s2}{./original.png}\PYG{l+s+s2}{\PYGZdq{}}\PYG{p}{,}
    \PYG{l+s+s2}{\PYGZdq{}}\PYG{l+s+s2}{out\PYGZus{}prefix}\PYG{l+s+s2}{\PYGZdq{}}\PYG{p}{:} \PYG{l+s+s2}{\PYGZdq{}}\PYG{l+s+s2}{./out/sole\PYGZus{}}\PYG{l+s+s2}{\PYGZdq{}}\PYG{p}{,}
    \PYG{l+s+s2}{\PYGZdq{}}\PYG{l+s+s2}{pad\PYGZus{}roi}\PYG{l+s+s2}{\PYGZdq{}}\PYG{p}{:} \PYG{l+m+mi}{12}\PYG{p}{,}

    \PYG{c+c1}{\PYGZsh{} Prefilter}
    \PYG{l+s+s2}{\PYGZdq{}}\PYG{l+s+s2}{nlmeans\PYGZus{}h}\PYG{l+s+s2}{\PYGZdq{}}\PYG{p}{:} \PYG{l+m+mi}{5}\PYG{p}{,}
    \PYG{l+s+s2}{\PYGZdq{}}\PYG{l+s+s2}{bilateral\PYGZus{}d}\PYG{l+s+s2}{\PYGZdq{}}\PYG{p}{:} \PYG{l+m+mi}{9}\PYG{p}{,} \PYG{l+s+s2}{\PYGZdq{}}\PYG{l+s+s2}{bilateral\PYGZus{}sigmaC}\PYG{l+s+s2}{\PYGZdq{}}\PYG{p}{:} \PYG{l+m+mi}{50}\PYG{p}{,} \PYG{l+s+s2}{\PYGZdq{}}\PYG{l+s+s2}{bilateral\PYGZus{}sigmaS}\PYG{l+s+s2}{\PYGZdq{}}\PYG{p}{:} \PYG{l+m+mi}{50}\PYG{p}{,}
    \PYG{l+s+s2}{\PYGZdq{}}\PYG{l+s+s2}{clahe\PYGZus{}clip}\PYG{l+s+s2}{\PYGZdq{}}\PYG{p}{:} \PYG{l+m+mf}{2.0}\PYG{p}{,} \PYG{l+s+s2}{\PYGZdq{}}\PYG{l+s+s2}{clahe\PYGZus{}tiles}\PYG{l+s+s2}{\PYGZdq{}}\PYG{p}{:} \PYG{p}{(}\PYG{l+m+mi}{8}\PYG{p}{,} \PYG{l+m+mi}{8}\PYG{p}{)}\PYG{p}{,}

    \PYG{c+c1}{\PYGZsh{} Dynamic blue threshold}
    \PYG{l+s+s2}{\PYGZdq{}}\PYG{l+s+s2}{blue\PYGZus{}sat\PYGZus{}min}\PYG{l+s+s2}{\PYGZdq{}}\PYG{p}{:} \PYG{l+m+mi}{80}\PYG{p}{,}
    \PYG{l+s+s2}{\PYGZdq{}}\PYG{l+s+s2}{blue\PYGZus{}half\PYGZus{}window}\PYG{l+s+s2}{\PYGZdq{}}\PYG{p}{:} \PYG{l+m+mi}{15}\PYG{p}{,}

    \PYG{c+c1}{\PYGZsh{} ROI by blue column coverage}
    \PYG{l+s+s2}{\PYGZdq{}}\PYG{l+s+s2}{belt\PYGZus{}col\PYGZus{}cover\PYGZus{}ratio}\PYG{l+s+s2}{\PYGZdq{}}\PYG{p}{:} \PYG{l+m+mf}{0.22}\PYG{p}{,}
    \PYG{l+s+s2}{\PYGZdq{}}\PYG{l+s+s2}{belt\PYGZus{}min\PYGZus{}width\PYGZus{}px}\PYG{l+s+s2}{\PYGZdq{}}\PYG{p}{:} \PYG{l+m+mi}{200}\PYG{p}{,}

    \PYG{c+c1}{\PYGZsh{} GrabCut}
    \PYG{l+s+s2}{\PYGZdq{}}\PYG{l+s+s2}{pr\PYGZus{}fg\PYGZus{}kernel}\PYG{l+s+s2}{\PYGZdq{}}\PYG{p}{:} \PYG{p}{(}\PYG{l+m+mi}{7}\PYG{p}{,} \PYG{l+m+mi}{19}\PYG{p}{)}\PYG{p}{,}
    \PYG{l+s+s2}{\PYGZdq{}}\PYG{l+s+s2}{sure\PYGZus{}fg\PYGZus{}pct\PYGZus{}contrast}\PYG{l+s+s2}{\PYGZdq{}}\PYG{p}{:} \PYG{l+m+mi}{70}\PYG{p}{,}
    \PYG{l+s+s2}{\PYGZdq{}}\PYG{l+s+s2}{sure\PYGZus{}fg\PYGZus{}pct\PYGZus{}dt}\PYG{l+s+s2}{\PYGZdq{}}\PYG{p}{:} \PYG{l+m+mi}{60}\PYG{p}{,}
    \PYG{l+s+s2}{\PYGZdq{}}\PYG{l+s+s2}{gc\PYGZus{}iters}\PYG{l+s+s2}{\PYGZdq{}}\PYG{p}{:} \PYG{l+m+mi}{12}\PYG{p}{,}

    \PYG{c+c1}{\PYGZsh{} Guard\PYGZhy{}rails (BG) inside ROI}
    \PYG{l+s+s2}{\PYGZdq{}}\PYG{l+s+s2}{guard\PYGZus{}frac\PYGZus{}w}\PYG{l+s+s2}{\PYGZdq{}}\PYG{p}{:} \PYG{l+m+mf}{0.03}\PYG{p}{,}
    \PYG{l+s+s2}{\PYGZdq{}}\PYG{l+s+s2}{guard\PYGZus{}min\PYGZus{}px}\PYG{l+s+s2}{\PYGZdq{}}\PYG{p}{:} \PYG{l+m+mi}{14}\PYG{p}{,}
    \PYG{l+s+s2}{\PYGZdq{}}\PYG{l+s+s2}{guard\PYGZus{}max\PYGZus{}px}\PYG{l+s+s2}{\PYGZdq{}}\PYG{p}{:} \PYG{l+m+mi}{48}\PYG{p}{,}

    \PYG{c+c1}{\PYGZsh{} Postprocess}
    \PYG{l+s+s2}{\PYGZdq{}}\PYG{l+s+s2}{post\PYGZus{}close\PYGZus{}kernel}\PYG{l+s+s2}{\PYGZdq{}}\PYG{p}{:} \PYG{p}{(}\PYG{l+m+mi}{9}\PYG{p}{,} \PYG{l+m+mi}{13}\PYG{p}{)}\PYG{p}{,}

    \PYG{c+c1}{\PYGZsh{} Candidate filter \PYGZam{} shape prior (STRICT)}
    \PYG{l+s+s2}{\PYGZdq{}}\PYG{l+s+s2}{min\PYGZus{}area\PYGZus{}px2}\PYG{l+s+s2}{\PYGZdq{}}\PYG{p}{:} \PYG{l+m+mi}{5000}\PYG{p}{,}
    \PYG{l+s+s2}{\PYGZdq{}}\PYG{l+s+s2}{min\PYGZus{}width\PYGZus{}px}\PYG{l+s+s2}{\PYGZdq{}}\PYG{p}{:} \PYG{l+m+mi}{32}\PYG{p}{,}
    \PYG{l+s+s2}{\PYGZdq{}}\PYG{l+s+s2}{edge\PYGZus{}clear\PYGZus{}px}\PYG{l+s+s2}{\PYGZdq{}}\PYG{p}{:} \PYG{l+m+mi}{16}\PYG{p}{,}
    \PYG{l+s+s2}{\PYGZdq{}}\PYG{l+s+s2}{shape\PYGZus{}mu\PYGZus{}log\PYGZus{}aspect}\PYG{l+s+s2}{\PYGZdq{}}\PYG{p}{:} \PYG{l+m+mf}{0.92}\PYG{p}{,}  \PYG{c+c1}{\PYGZsh{} \PYGZti{}aspect 2.5}
    \PYG{l+s+s2}{\PYGZdq{}}\PYG{l+s+s2}{shape\PYGZus{}sigma}\PYG{l+s+s2}{\PYGZdq{}}\PYG{p}{:} \PYG{l+m+mf}{0.35}\PYG{p}{,}

    \PYG{c+c1}{\PYGZsh{} Relaxed fallback deltas (used if strict fails)}
    \PYG{l+s+s2}{\PYGZdq{}}\PYG{l+s+s2}{relax}\PYG{l+s+s2}{\PYGZdq{}}\PYG{p}{:} \PYG{p}{\PYGZob{}}
        \PYG{l+s+s2}{\PYGZdq{}}\PYG{l+s+s2}{min\PYGZus{}area\PYGZus{}px2}\PYG{l+s+s2}{\PYGZdq{}}\PYG{p}{:} \PYG{l+m+mi}{1500}\PYG{p}{,}
        \PYG{l+s+s2}{\PYGZdq{}}\PYG{l+s+s2}{min\PYGZus{}width\PYGZus{}px}\PYG{l+s+s2}{\PYGZdq{}}\PYG{p}{:} \PYG{l+m+mi}{14}\PYG{p}{,}
        \PYG{l+s+s2}{\PYGZdq{}}\PYG{l+s+s2}{edge\PYGZus{}clear\PYGZus{}px}\PYG{l+s+s2}{\PYGZdq{}}\PYG{p}{:} \PYG{l+m+mi}{6}\PYG{p}{,}
        \PYG{l+s+s2}{\PYGZdq{}}\PYG{l+s+s2}{shape\PYGZus{}sigma}\PYG{l+s+s2}{\PYGZdq{}}\PYG{p}{:} \PYG{l+m+mf}{0.70}\PYG{p}{,}
        \PYG{l+s+s2}{\PYGZdq{}}\PYG{l+s+s2}{allow\PYGZus{}touch\PYGZus{}edges}\PYG{l+s+s2}{\PYGZdq{}}\PYG{p}{:} \PYG{k+kc}{True}
    \PYG{p}{\PYGZcb{}}\PYG{p}{,}

    \PYG{c+c1}{\PYGZsh{} Optional metric scale}
    \PYG{l+s+s2}{\PYGZdq{}}\PYG{l+s+s2}{scale\PYGZus{}json}\PYG{l+s+s2}{\PYGZdq{}}\PYG{p}{:} \PYG{l+s+s2}{\PYGZdq{}}\PYG{l+s+s2}{./calibracion\PYGZus{}px\PYGZus{}mm.json}\PYG{l+s+s2}{\PYGZdq{}}\PYG{p}{,}

    \PYG{c+c1}{\PYGZsh{} \PYGZhy{}\PYGZhy{}\PYGZhy{} switches \PYGZhy{}\PYGZhy{}\PYGZhy{}}
    \PYG{l+s+s2}{\PYGZdq{}}\PYG{l+s+s2}{debug}\PYG{l+s+s2}{\PYGZdq{}}\PYG{p}{:} \PYG{k+kc}{True}\PYG{p}{,}     \PYG{c+c1}{\PYGZsh{} PNGs on/off}
    \PYG{l+s+s2}{\PYGZdq{}}\PYG{l+s+s2}{save\PYGZus{}csv}\PYG{l+s+s2}{\PYGZdq{}}\PYG{p}{:} \PYG{k+kc}{True}    \PYG{c+c1}{\PYGZsh{} CSV on/off}
\PYG{p}{\PYGZcb{}}

\PYG{c+c1}{\PYGZsh{} \PYGZhy{}\PYGZhy{}\PYGZhy{}\PYGZhy{}\PYGZhy{}\PYGZhy{}\PYGZhy{}\PYGZhy{}\PYGZhy{}\PYGZhy{}\PYGZhy{}\PYGZhy{}\PYGZhy{}\PYGZhy{}\PYGZhy{}\PYGZhy{}\PYGZhy{}\PYGZhy{}\PYGZhy{}\PYGZhy{}\PYGZhy{}\PYGZhy{}\PYGZhy{}\PYGZhy{}\PYGZhy{}}
\PYG{c+c1}{\PYGZsh{} Utilities}
\PYG{c+c1}{\PYGZsh{} \PYGZhy{}\PYGZhy{}\PYGZhy{}\PYGZhy{}\PYGZhy{}\PYGZhy{}\PYGZhy{}\PYGZhy{}\PYGZhy{}\PYGZhy{}\PYGZhy{}\PYGZhy{}\PYGZhy{}\PYGZhy{}\PYGZhy{}\PYGZhy{}\PYGZhy{}\PYGZhy{}\PYGZhy{}\PYGZhy{}\PYGZhy{}\PYGZhy{}\PYGZhy{}\PYGZhy{}\PYGZhy{}}
\PYG{k}{def}\PYG{+w}{ }\PYG{n+nf}{\PYGZus{}dbg}\PYG{p}{(}\PYG{p}{)} \PYG{o}{\PYGZhy{}}\PYG{o}{\PYGZgt{}} \PYG{n+nb}{bool}\PYG{p}{:}
    \PYG{k}{return} \PYG{n+nb}{bool}\PYG{p}{(}\PYG{n}{CONFIG}\PYG{o}{.}\PYG{n}{get}\PYG{p}{(}\PYG{l+s+s2}{\PYGZdq{}}\PYG{l+s+s2}{debug}\PYG{l+s+s2}{\PYGZdq{}}\PYG{p}{,} \PYG{k+kc}{False}\PYG{p}{)}\PYG{p}{)}

\PYG{k}{def}\PYG{+w}{ }\PYG{n+nf}{\PYGZus{}ensure\PYGZus{}dir}\PYG{p}{(}\PYG{n}{path}\PYG{p}{:} \PYG{n+nb}{str}\PYG{p}{)}\PYG{p}{:}
    \PYG{n}{Path}\PYG{p}{(}\PYG{n}{path}\PYG{p}{)}\PYG{o}{.}\PYG{n}{parent}\PYG{o}{.}\PYG{n}{mkdir}\PYG{p}{(}\PYG{n}{parents}\PYG{o}{=}\PYG{k+kc}{True}\PYG{p}{,} \PYG{n}{exist\PYGZus{}ok}\PYG{o}{=}\PYG{k+kc}{True}\PYG{p}{)}

\PYG{k}{def}\PYG{+w}{ }\PYG{n+nf}{save\PYGZus{}img}\PYG{p}{(}\PYG{n}{name}\PYG{p}{:} \PYG{n+nb}{str}\PYG{p}{,} \PYG{n}{img}\PYG{p}{)} \PYG{o}{\PYGZhy{}}\PYG{o}{\PYGZgt{}} \PYG{n+nb}{str}\PYG{p}{:}
\PYG{+w}{    }\PYG{l+s+sd}{\PYGZdq{}\PYGZdq{}\PYGZdq{}Unconditional file save (internal).\PYGZdq{}\PYGZdq{}\PYGZdq{}}
    \PYG{n}{p} \PYG{o}{=} \PYG{n}{CONFIG}\PYG{p}{[}\PYG{l+s+s2}{\PYGZdq{}}\PYG{l+s+s2}{out\PYGZus{}prefix}\PYG{l+s+s2}{\PYGZdq{}}\PYG{p}{]} \PYG{o}{+} \PYG{n}{name}
    \PYG{n}{\PYGZus{}ensure\PYGZus{}dir}\PYG{p}{(}\PYG{n}{p}\PYG{p}{)}
    \PYG{k}{if} \PYG{n}{img}\PYG{o}{.}\PYG{n}{ndim} \PYG{o}{==} \PYG{l+m+mi}{2}\PYG{p}{:}
        \PYG{n}{cv2}\PYG{o}{.}\PYG{n}{imwrite}\PYG{p}{(}\PYG{n}{p}\PYG{p}{,} \PYG{n}{img}\PYG{p}{)}
    \PYG{k}{else}\PYG{p}{:}
        \PYG{n}{cv2}\PYG{o}{.}\PYG{n}{imwrite}\PYG{p}{(}\PYG{n}{p}\PYG{p}{,} \PYG{n}{img} \PYG{k}{if} \PYG{n}{img}\PYG{o}{.}\PYG{n}{shape}\PYG{p}{[}\PYG{l+m+mi}{2}\PYG{p}{]} \PYG{o}{==} \PYG{l+m+mi}{3} \PYG{k}{else} \PYG{n}{img}\PYG{p}{[}\PYG{o}{.}\PYG{o}{.}\PYG{o}{.}\PYG{p}{,} \PYG{p}{:}\PYG{l+m+mi}{3}\PYG{p}{]}\PYG{p}{)}
    \PYG{k}{return} \PYG{n}{p}

\PYG{k}{def}\PYG{+w}{ }\PYG{n+nf}{save\PYGZus{}img\PYGZus{}dbg}\PYG{p}{(}\PYG{n}{name}\PYG{p}{:} \PYG{n+nb}{str}\PYG{p}{,} \PYG{n}{img}\PYG{p}{)}\PYG{p}{:}
\PYG{+w}{    }\PYG{l+s+sd}{\PYGZdq{}\PYGZdq{}\PYGZdq{}Save only if CONFIG[\PYGZsq{}debug\PYGZsq{}] is True; otherwise return None.\PYGZdq{}\PYGZdq{}\PYGZdq{}}
    \PYG{k}{if} \PYG{o+ow}{not} \PYG{n}{\PYGZus{}dbg}\PYG{p}{(}\PYG{p}{)}\PYG{p}{:}
        \PYG{k}{return} \PYG{k+kc}{None}
    \PYG{k}{return} \PYG{n}{save\PYGZus{}img}\PYG{p}{(}\PYG{n}{name}\PYG{p}{,} \PYG{n}{img}\PYG{p}{)}

\PYG{k}{def}\PYG{+w}{ }\PYG{n+nf}{save\PYGZus{}csv\PYGZus{}dbg}\PYG{p}{(}\PYG{n}{df}\PYG{p}{:} \PYG{n}{pd}\PYG{o}{.}\PYG{n}{DataFrame}\PYG{p}{,} \PYG{n}{name}\PYG{p}{:} \PYG{n+nb}{str}\PYG{p}{)}\PYG{p}{:}
\PYG{+w}{    }\PYG{l+s+sd}{\PYGZdq{}\PYGZdq{}\PYGZdq{}Save CSV only if CONFIG[\PYGZsq{}save\PYGZus{}csv\PYGZsq{}] is True; returns path or None.\PYGZdq{}\PYGZdq{}\PYGZdq{}}
    \PYG{k}{if} \PYG{o+ow}{not} \PYG{n}{CONFIG}\PYG{o}{.}\PYG{n}{get}\PYG{p}{(}\PYG{l+s+s2}{\PYGZdq{}}\PYG{l+s+s2}{save\PYGZus{}csv}\PYG{l+s+s2}{\PYGZdq{}}\PYG{p}{,} \PYG{k+kc}{False}\PYG{p}{)}\PYG{p}{:}
        \PYG{k}{return} \PYG{k+kc}{None}
    \PYG{n}{out\PYGZus{}path} \PYG{o}{=} \PYG{n}{CONFIG}\PYG{p}{[}\PYG{l+s+s2}{\PYGZdq{}}\PYG{l+s+s2}{out\PYGZus{}prefix}\PYG{l+s+s2}{\PYGZdq{}}\PYG{p}{]} \PYG{o}{+} \PYG{n}{name}
    \PYG{n}{\PYGZus{}ensure\PYGZus{}dir}\PYG{p}{(}\PYG{n}{out\PYGZus{}path}\PYG{p}{)}
    \PYG{n}{df}\PYG{o}{.}\PYG{n}{to\PYGZus{}csv}\PYG{p}{(}\PYG{n}{out\PYGZus{}path}\PYG{p}{,} \PYG{n}{index}\PYG{o}{=}\PYG{k+kc}{False}\PYG{p}{)}
    \PYG{k}{return} \PYG{n}{out\PYGZus{}path}

\PYG{k}{def}\PYG{+w}{ }\PYG{n+nf}{get\PYGZus{}scale\PYGZus{}mm\PYGZus{}per\PYGZus{}px}\PYG{p}{(}\PYG{p}{)} \PYG{o}{\PYGZhy{}}\PYG{o}{\PYGZgt{}} \PYG{n+nb}{float} \PYG{o}{|} \PYG{k+kc}{None}\PYG{p}{:}
    \PYG{k}{try}\PYG{p}{:}
        \PYG{k+kn}{from}\PYG{+w}{ }\PYG{n+nn}{get\PYGZus{}mm\PYGZus{}per\PYGZus{}px}\PYG{+w}{ }\PYG{k+kn}{import} \PYG{n}{get\PYGZus{}mm\PYGZus{}per\PYGZus{}px}  \PYG{c+c1}{\PYGZsh{} type: ignore}
        \PYG{n}{val} \PYG{o}{=} \PYG{n+nb}{float}\PYG{p}{(}\PYG{n}{get\PYGZus{}mm\PYGZus{}per\PYGZus{}px}\PYG{p}{(}\PYG{p}{)}\PYG{p}{)}
        \PYG{k}{if} \PYG{n}{val} \PYG{o}{\PYGZgt{}} \PYG{l+m+mi}{0}\PYG{p}{:}
            \PYG{k}{return} \PYG{n}{val}
    \PYG{k}{except} \PYG{n+ne}{Exception}\PYG{p}{:}
        \PYG{k}{pass}
    \PYG{n}{p} \PYG{o}{=} \PYG{n}{Path}\PYG{p}{(}\PYG{n}{CONFIG}\PYG{p}{[}\PYG{l+s+s2}{\PYGZdq{}}\PYG{l+s+s2}{scale\PYGZus{}json}\PYG{l+s+s2}{\PYGZdq{}}\PYG{p}{]}\PYG{p}{)}
    \PYG{k}{if} \PYG{n}{p}\PYG{o}{.}\PYG{n}{exists}\PYG{p}{(}\PYG{p}{)}\PYG{p}{:}
        \PYG{k}{try}\PYG{p}{:}
            \PYG{n}{data} \PYG{o}{=} \PYG{n}{json}\PYG{o}{.}\PYG{n}{loads}\PYG{p}{(}\PYG{n}{p}\PYG{o}{.}\PYG{n}{read\PYGZus{}text}\PYG{p}{(}\PYG{p}{)}\PYG{p}{)}
            \PYG{n}{val} \PYG{o}{=} \PYG{n+nb}{float}\PYG{p}{(}\PYG{n}{data}\PYG{o}{.}\PYG{n}{get}\PYG{p}{(}\PYG{l+s+s2}{\PYGZdq{}}\PYG{l+s+s2}{mm\PYGZus{}per\PYGZus{}px}\PYG{l+s+s2}{\PYGZdq{}}\PYG{p}{,} \PYG{l+m+mi}{0}\PYG{p}{)}\PYG{p}{)}
            \PYG{k}{if} \PYG{n}{val} \PYG{o}{\PYGZgt{}} \PYG{l+m+mi}{0}\PYG{p}{:}
                \PYG{k}{return} \PYG{n}{val}
        \PYG{k}{except} \PYG{n+ne}{Exception}\PYG{p}{:}
            \PYG{k}{pass}
    \PYG{k}{return} \PYG{k+kc}{None}

\PYG{c+c1}{\PYGZsh{} \PYGZhy{}\PYGZhy{}\PYGZhy{}\PYGZhy{} Morphology helpers to avoid kernel errors \PYGZhy{}\PYGZhy{}\PYGZhy{}\PYGZhy{}}
\PYG{k}{def}\PYG{+w}{ }\PYG{n+nf}{\PYGZus{}morph}\PYG{p}{(}\PYG{n}{img}\PYG{p}{:} \PYG{n}{np}\PYG{o}{.}\PYG{n}{ndarray}\PYG{p}{,} \PYG{n}{op}\PYG{p}{:} \PYG{n+nb}{int}\PYG{p}{,} \PYG{n}{shape}\PYG{p}{:} \PYG{n+nb}{int}\PYG{p}{,} \PYG{n}{ksize}\PYG{p}{:} \PYG{n+nb}{tuple}\PYG{p}{[}\PYG{n+nb}{int}\PYG{p}{,}\PYG{n+nb}{int}\PYG{p}{]}\PYG{p}{,} \PYG{n}{iters}\PYG{p}{:} \PYG{n+nb}{int} \PYG{o}{=} \PYG{l+m+mi}{1}\PYG{p}{)} \PYG{o}{\PYGZhy{}}\PYG{o}{\PYGZgt{}} \PYG{n}{np}\PYG{o}{.}\PYG{n}{ndarray}\PYG{p}{:}
\PYG{+w}{    }\PYG{l+s+sd}{\PYGZdq{}\PYGZdq{}\PYGZdq{}}
\PYG{l+s+sd}{    Robust wrapper for cv2.morphologyEx:}
\PYG{l+s+sd}{    \PYGZhy{} Ensures uint8 input}
\PYG{l+s+sd}{    \PYGZhy{} Always builds a valid kernel}
\PYG{l+s+sd}{    \PYGZhy{} Uses positional args for compatibility}
\PYG{l+s+sd}{    \PYGZdq{}\PYGZdq{}\PYGZdq{}}
    \PYG{k}{if} \PYG{n}{img}\PYG{o}{.}\PYG{n}{dtype} \PYG{o}{!=} \PYG{n}{np}\PYG{o}{.}\PYG{n}{uint8}\PYG{p}{:}
        \PYG{k}{if} \PYG{n}{img}\PYG{o}{.}\PYG{n}{dtype} \PYG{o}{==} \PYG{n+nb}{bool}\PYG{p}{:}
            \PYG{n}{img} \PYG{o}{=} \PYG{n}{img}\PYG{o}{.}\PYG{n}{astype}\PYG{p}{(}\PYG{n}{np}\PYG{o}{.}\PYG{n}{uint8}\PYG{p}{)} \PYG{o}{*} \PYG{l+m+mi}{255}
        \PYG{k}{else}\PYG{p}{:}
            \PYG{n}{img} \PYG{o}{=} \PYG{n}{img}\PYG{o}{.}\PYG{n}{astype}\PYG{p}{(}\PYG{n}{np}\PYG{o}{.}\PYG{n}{uint8}\PYG{p}{)}
    \PYG{n}{k} \PYG{o}{=} \PYG{n}{cv2}\PYG{o}{.}\PYG{n}{getStructuringElement}\PYG{p}{(}\PYG{n}{shape}\PYG{p}{,} \PYG{p}{(}\PYG{n+nb}{int}\PYG{p}{(}\PYG{n}{ksize}\PYG{p}{[}\PYG{l+m+mi}{0}\PYG{p}{]}\PYG{p}{)}\PYG{p}{,} \PYG{n+nb}{int}\PYG{p}{(}\PYG{n}{ksize}\PYG{p}{[}\PYG{l+m+mi}{1}\PYG{p}{]}\PYG{p}{)}\PYG{p}{)}\PYG{p}{)}
    \PYG{k}{return} \PYG{n}{cv2}\PYG{o}{.}\PYG{n}{morphologyEx}\PYG{p}{(}\PYG{n}{img}\PYG{p}{,} \PYG{n}{op}\PYG{p}{,} \PYG{n}{k}\PYG{p}{,} \PYG{k+kc}{None}\PYG{p}{,} \PYG{k+kc}{None}\PYG{p}{,} \PYG{n+nb}{int}\PYG{p}{(}\PYG{n}{iters}\PYG{p}{)}\PYG{p}{,} \PYG{n}{cv2}\PYG{o}{.}\PYG{n}{BORDER\PYGZus{}CONSTANT}\PYG{p}{,} \PYG{l+m+mi}{0}\PYG{p}{)}

\PYG{k}{def}\PYG{+w}{ }\PYG{n+nf}{\PYGZus{}open}\PYG{p}{(}\PYG{n}{img}\PYG{p}{:} \PYG{n}{np}\PYG{o}{.}\PYG{n}{ndarray}\PYG{p}{,} \PYG{n}{shape}\PYG{p}{:} \PYG{n+nb}{int}\PYG{p}{,} \PYG{n}{ksize}\PYG{p}{:} \PYG{n+nb}{tuple}\PYG{p}{[}\PYG{n+nb}{int}\PYG{p}{,}\PYG{n+nb}{int}\PYG{p}{]}\PYG{p}{,} \PYG{n}{iters}\PYG{p}{:} \PYG{n+nb}{int} \PYG{o}{=} \PYG{l+m+mi}{1}\PYG{p}{)} \PYG{o}{\PYGZhy{}}\PYG{o}{\PYGZgt{}} \PYG{n}{np}\PYG{o}{.}\PYG{n}{ndarray}\PYG{p}{:}
    \PYG{k}{return} \PYG{n}{\PYGZus{}morph}\PYG{p}{(}\PYG{n}{img}\PYG{p}{,} \PYG{n}{cv2}\PYG{o}{.}\PYG{n}{MORPH\PYGZus{}OPEN}\PYG{p}{,} \PYG{n}{shape}\PYG{p}{,} \PYG{n}{ksize}\PYG{p}{,} \PYG{n}{iters}\PYG{p}{)}

\PYG{k}{def}\PYG{+w}{ }\PYG{n+nf}{\PYGZus{}close}\PYG{p}{(}\PYG{n}{img}\PYG{p}{:} \PYG{n}{np}\PYG{o}{.}\PYG{n}{ndarray}\PYG{p}{,} \PYG{n}{shape}\PYG{p}{:} \PYG{n+nb}{int}\PYG{p}{,} \PYG{n}{ksize}\PYG{p}{:} \PYG{n+nb}{tuple}\PYG{p}{[}\PYG{n+nb}{int}\PYG{p}{,}\PYG{n+nb}{int}\PYG{p}{]}\PYG{p}{,} \PYG{n}{iters}\PYG{p}{:} \PYG{n+nb}{int} \PYG{o}{=} \PYG{l+m+mi}{1}\PYG{p}{)} \PYG{o}{\PYGZhy{}}\PYG{o}{\PYGZgt{}} \PYG{n}{np}\PYG{o}{.}\PYG{n}{ndarray}\PYG{p}{:}
    \PYG{k}{return} \PYG{n}{\PYGZus{}morph}\PYG{p}{(}\PYG{n}{img}\PYG{p}{,} \PYG{n}{cv2}\PYG{o}{.}\PYG{n}{MORPH\PYGZus{}CLOSE}\PYG{p}{,} \PYG{n}{shape}\PYG{p}{,} \PYG{n}{ksize}\PYG{p}{,} \PYG{n}{iters}\PYG{p}{)}

\PYG{c+c1}{\PYGZsh{} \PYGZhy{}\PYGZhy{}\PYGZhy{}\PYGZhy{}\PYGZhy{}\PYGZhy{}\PYGZhy{}\PYGZhy{}\PYGZhy{}\PYGZhy{}\PYGZhy{}\PYGZhy{}\PYGZhy{}\PYGZhy{}\PYGZhy{}\PYGZhy{}\PYGZhy{}\PYGZhy{}\PYGZhy{}\PYGZhy{}\PYGZhy{}\PYGZhy{}\PYGZhy{}\PYGZhy{}\PYGZhy{}}
\PYG{c+c1}{\PYGZsh{} Pipeline blocks}
\PYG{c+c1}{\PYGZsh{} \PYGZhy{}\PYGZhy{}\PYGZhy{}\PYGZhy{}\PYGZhy{}\PYGZhy{}\PYGZhy{}\PYGZhy{}\PYGZhy{}\PYGZhy{}\PYGZhy{}\PYGZhy{}\PYGZhy{}\PYGZhy{}\PYGZhy{}\PYGZhy{}\PYGZhy{}\PYGZhy{}\PYGZhy{}\PYGZhy{}\PYGZhy{}\PYGZhy{}\PYGZhy{}\PYGZhy{}\PYGZhy{}}
\PYG{k}{def}\PYG{+w}{ }\PYG{n+nf}{prefilter}\PYG{p}{(}\PYG{n}{bgr}\PYG{p}{:} \PYG{n}{np}\PYG{o}{.}\PYG{n}{ndarray}\PYG{p}{)} \PYG{o}{\PYGZhy{}}\PYG{o}{\PYGZgt{}} \PYG{n}{np}\PYG{o}{.}\PYG{n}{ndarray}\PYG{p}{:}
    \PYG{n}{den} \PYG{o}{=} \PYG{n}{cv2}\PYG{o}{.}\PYG{n}{fastNlMeansDenoisingColored}\PYG{p}{(}
        \PYG{n}{bgr}\PYG{p}{,} \PYG{k+kc}{None}\PYG{p}{,} \PYG{n}{h}\PYG{o}{=}\PYG{n}{CONFIG}\PYG{p}{[}\PYG{l+s+s2}{\PYGZdq{}}\PYG{l+s+s2}{nlmeans\PYGZus{}h}\PYG{l+s+s2}{\PYGZdq{}}\PYG{p}{]}\PYG{p}{,} \PYG{n}{hColor}\PYG{o}{=}\PYG{n}{CONFIG}\PYG{p}{[}\PYG{l+s+s2}{\PYGZdq{}}\PYG{l+s+s2}{nlmeans\PYGZus{}h}\PYG{l+s+s2}{\PYGZdq{}}\PYG{p}{]}\PYG{p}{,}
        \PYG{n}{templateWindowSize}\PYG{o}{=}\PYG{l+m+mi}{7}\PYG{p}{,} \PYG{n}{searchWindowSize}\PYG{o}{=}\PYG{l+m+mi}{21}
    \PYG{p}{)}
    \PYG{n}{den} \PYG{o}{=} \PYG{n}{cv2}\PYG{o}{.}\PYG{n}{bilateralFilter}\PYG{p}{(}
        \PYG{n}{den}\PYG{p}{,} \PYG{n}{d}\PYG{o}{=}\PYG{n}{CONFIG}\PYG{p}{[}\PYG{l+s+s2}{\PYGZdq{}}\PYG{l+s+s2}{bilateral\PYGZus{}d}\PYG{l+s+s2}{\PYGZdq{}}\PYG{p}{]}\PYG{p}{,}
        \PYG{n}{sigmaColor}\PYG{o}{=}\PYG{n}{CONFIG}\PYG{p}{[}\PYG{l+s+s2}{\PYGZdq{}}\PYG{l+s+s2}{bilateral\PYGZus{}sigmaC}\PYG{l+s+s2}{\PYGZdq{}}\PYG{p}{]}\PYG{p}{,} \PYG{n}{sigmaSpace}\PYG{o}{=}\PYG{n}{CONFIG}\PYG{p}{[}\PYG{l+s+s2}{\PYGZdq{}}\PYG{l+s+s2}{bilateral\PYGZus{}sigmaS}\PYG{l+s+s2}{\PYGZdq{}}\PYG{p}{]}
    \PYG{p}{)}
    \PYG{n}{hsv} \PYG{o}{=} \PYG{n}{cv2}\PYG{o}{.}\PYG{n}{cvtColor}\PYG{p}{(}\PYG{n}{den}\PYG{p}{,} \PYG{n}{cv2}\PYG{o}{.}\PYG{n}{COLOR\PYGZus{}BGR2HSV}\PYG{p}{)}
    \PYG{n}{h}\PYG{p}{,} \PYG{n}{s}\PYG{p}{,} \PYG{n}{v} \PYG{o}{=} \PYG{n}{cv2}\PYG{o}{.}\PYG{n}{split}\PYG{p}{(}\PYG{n}{hsv}\PYG{p}{)}
    \PYG{n}{clahe} \PYG{o}{=} \PYG{n}{cv2}\PYG{o}{.}\PYG{n}{createCLAHE}\PYG{p}{(}\PYG{n}{CONFIG}\PYG{p}{[}\PYG{l+s+s2}{\PYGZdq{}}\PYG{l+s+s2}{clahe\PYGZus{}clip}\PYG{l+s+s2}{\PYGZdq{}}\PYG{p}{]}\PYG{p}{,} \PYG{n}{CONFIG}\PYG{p}{[}\PYG{l+s+s2}{\PYGZdq{}}\PYG{l+s+s2}{clahe\PYGZus{}tiles}\PYG{l+s+s2}{\PYGZdq{}}\PYG{p}{]}\PYG{p}{)}
    \PYG{n}{v} \PYG{o}{=} \PYG{n}{clahe}\PYG{o}{.}\PYG{n}{apply}\PYG{p}{(}\PYG{n}{v}\PYG{p}{)}
    \PYG{n}{out} \PYG{o}{=} \PYG{n}{cv2}\PYG{o}{.}\PYG{n}{cvtColor}\PYG{p}{(}\PYG{n}{cv2}\PYG{o}{.}\PYG{n}{merge}\PYG{p}{(}\PYG{p}{[}\PYG{n}{h}\PYG{p}{,} \PYG{n}{s}\PYG{p}{,} \PYG{n}{v}\PYG{p}{]}\PYG{p}{)}\PYG{p}{,} \PYG{n}{cv2}\PYG{o}{.}\PYG{n}{COLOR\PYGZus{}HSV2BGR}\PYG{p}{)}
    \PYG{n}{save\PYGZus{}img\PYGZus{}dbg}\PYG{p}{(}\PYG{l+s+s2}{\PYGZdq{}}\PYG{l+s+s2}{01\PYGZus{}prefilter.png}\PYG{l+s+s2}{\PYGZdq{}}\PYG{p}{,} \PYG{n}{out}\PYG{p}{)}
    \PYG{k}{return} \PYG{n}{out}

\PYG{k}{def}\PYG{+w}{ }\PYG{n+nf}{dynamic\PYGZus{}blue\PYGZus{}threshold}\PYG{p}{(}\PYG{n}{hsv}\PYG{p}{:} \PYG{n}{np}\PYG{o}{.}\PYG{n}{ndarray}\PYG{p}{)}\PYG{p}{:}
    \PYG{n}{H}\PYG{p}{,} \PYG{n}{S}\PYG{p}{,} \PYG{n}{\PYGZus{}} \PYG{o}{=} \PYG{n}{cv2}\PYG{o}{.}\PYG{n}{split}\PYG{p}{(}\PYG{n}{hsv}\PYG{p}{)}
    \PYG{n}{Hh}\PYG{p}{,} \PYG{n}{Wh} \PYG{o}{=} \PYG{n}{H}\PYG{o}{.}\PYG{n}{shape}
    \PYG{n}{x1}\PYG{p}{,} \PYG{n}{x2} \PYG{o}{=} \PYG{n+nb}{int}\PYG{p}{(}\PYG{n}{Wh} \PYG{o}{*} \PYG{l+m+mf}{0.32}\PYG{p}{)}\PYG{p}{,} \PYG{n+nb}{int}\PYG{p}{(}\PYG{n}{Wh} \PYG{o}{*} \PYG{l+m+mf}{0.68}\PYG{p}{)}
    \PYG{n}{band} \PYG{o}{=} \PYG{n}{H}\PYG{p}{[}\PYG{p}{:}\PYG{p}{,} \PYG{n}{x1}\PYG{p}{:}\PYG{n}{x2}\PYG{p}{]}\PYG{p}{;} \PYG{n}{band\PYGZus{}s} \PYG{o}{=} \PYG{n}{S}\PYG{p}{[}\PYG{p}{:}\PYG{p}{,} \PYG{n}{x1}\PYG{p}{:}\PYG{n}{x2}\PYG{p}{]}
    \PYG{n}{mask\PYGZus{}sat} \PYG{o}{=} \PYG{n}{band\PYGZus{}s} \PYG{o}{\PYGZgt{}} \PYG{n}{CONFIG}\PYG{p}{[}\PYG{l+s+s2}{\PYGZdq{}}\PYG{l+s+s2}{blue\PYGZus{}sat\PYGZus{}min}\PYG{l+s+s2}{\PYGZdq{}}\PYG{p}{]}
    \PYG{n}{hh} \PYG{o}{=} \PYG{n}{band}\PYG{p}{[}\PYG{n}{mask\PYGZus{}sat}\PYG{p}{]}\PYG{o}{.}\PYG{n}{flatten}\PYG{p}{(}\PYG{p}{)}
    \PYG{k}{if} \PYG{n}{hh}\PYG{o}{.}\PYG{n}{size} \PYG{o}{==} \PYG{l+m+mi}{0}\PYG{p}{:}
        \PYG{n}{lower} \PYG{o}{=} \PYG{n}{np}\PYG{o}{.}\PYG{n}{array}\PYG{p}{(}\PYG{p}{[}\PYG{l+m+mi}{95}\PYG{p}{,} \PYG{l+m+mi}{100}\PYG{p}{,} \PYG{l+m+mi}{30}\PYG{p}{]}\PYG{p}{,} \PYG{n}{np}\PYG{o}{.}\PYG{n}{uint8}\PYG{p}{)}
        \PYG{n}{upper} \PYG{o}{=} \PYG{n}{np}\PYG{o}{.}\PYG{n}{array}\PYG{p}{(}\PYG{p}{[}\PYG{l+m+mi}{125}\PYG{p}{,} \PYG{l+m+mi}{255}\PYG{p}{,} \PYG{l+m+mi}{255}\PYG{p}{]}\PYG{p}{,} \PYG{n}{np}\PYG{o}{.}\PYG{n}{uint8}\PYG{p}{)}
    \PYG{k}{else}\PYG{p}{:}
        \PYG{n}{hist} \PYG{o}{=} \PYG{n}{cv2}\PYG{o}{.}\PYG{n}{calcHist}\PYG{p}{(}\PYG{p}{[}\PYG{n}{hh}\PYG{o}{.}\PYG{n}{astype}\PYG{p}{(}\PYG{n}{np}\PYG{o}{.}\PYG{n}{uint8}\PYG{p}{)}\PYG{p}{]}\PYG{p}{,} \PYG{p}{[}\PYG{l+m+mi}{0}\PYG{p}{]}\PYG{p}{,} \PYG{k+kc}{None}\PYG{p}{,} \PYG{p}{[}\PYG{l+m+mi}{180}\PYG{p}{]}\PYG{p}{,} \PYG{p}{[}\PYG{l+m+mi}{0}\PYG{p}{,} \PYG{l+m+mi}{180}\PYG{p}{]}\PYG{p}{)}\PYG{o}{.}\PYG{n}{flatten}\PYG{p}{(}\PYG{p}{)}
        \PYG{n}{peak} \PYG{o}{=} \PYG{n+nb}{int}\PYG{p}{(}\PYG{n}{np}\PYG{o}{.}\PYG{n}{argmax}\PYG{p}{(}\PYG{n}{hist}\PYG{p}{)}\PYG{p}{)}\PYG{p}{;} \PYG{n}{hw} \PYG{o}{=} \PYG{n}{CONFIG}\PYG{p}{[}\PYG{l+s+s2}{\PYGZdq{}}\PYG{l+s+s2}{blue\PYGZus{}half\PYGZus{}window}\PYG{l+s+s2}{\PYGZdq{}}\PYG{p}{]}
        \PYG{n}{lower} \PYG{o}{=} \PYG{n}{np}\PYG{o}{.}\PYG{n}{array}\PYG{p}{(}\PYG{p}{[}\PYG{n+nb}{max}\PYG{p}{(}\PYG{l+m+mi}{0}\PYG{p}{,} \PYG{n}{peak} \PYG{o}{\PYGZhy{}} \PYG{n}{hw}\PYG{p}{)}\PYG{p}{,} \PYG{l+m+mi}{80}\PYG{p}{,} \PYG{l+m+mi}{30}\PYG{p}{]}\PYG{p}{,} \PYG{n}{np}\PYG{o}{.}\PYG{n}{uint8}\PYG{p}{)}
        \PYG{n}{upper} \PYG{o}{=} \PYG{n}{np}\PYG{o}{.}\PYG{n}{array}\PYG{p}{(}\PYG{p}{[}\PYG{n+nb}{min}\PYG{p}{(}\PYG{l+m+mi}{179}\PYG{p}{,} \PYG{n}{peak} \PYG{o}{+} \PYG{n}{hw}\PYG{p}{)}\PYG{p}{,} \PYG{l+m+mi}{255}\PYG{p}{,} \PYG{l+m+mi}{255}\PYG{p}{]}\PYG{p}{,} \PYG{n}{np}\PYG{o}{.}\PYG{n}{uint8}\PYG{p}{)}
    \PYG{k}{return} \PYG{n}{lower}\PYG{p}{,} \PYG{n}{upper}

\PYG{k}{def}\PYG{+w}{ }\PYG{n+nf}{\PYGZus{}longest\PYGZus{}true\PYGZus{}run}\PYG{p}{(}\PYG{n}{vec\PYGZus{}bool}\PYG{p}{:} \PYG{n}{np}\PYG{o}{.}\PYG{n}{ndarray}\PYG{p}{)}\PYG{p}{:}
    \PYG{n}{best\PYGZus{}i} \PYG{o}{=} \PYG{n}{best\PYGZus{}j} \PYG{o}{=} \PYG{o}{\PYGZhy{}}\PYG{l+m+mi}{1}\PYG{p}{;} \PYG{n}{cur\PYGZus{}i} \PYG{o}{=} \PYG{o}{\PYGZhy{}}\PYG{l+m+mi}{1}
    \PYG{k}{for} \PYG{n}{k}\PYG{p}{,} \PYG{n}{v} \PYG{o+ow}{in} \PYG{n+nb}{enumerate}\PYG{p}{(}\PYG{n}{vec\PYGZus{}bool}\PYG{p}{)}\PYG{p}{:}
        \PYG{k}{if} \PYG{n}{v}\PYG{p}{:}
            \PYG{k}{if} \PYG{n}{cur\PYGZus{}i} \PYG{o}{\PYGZlt{}} \PYG{l+m+mi}{0}\PYG{p}{:} \PYG{n}{cur\PYGZus{}i} \PYG{o}{=} \PYG{n}{k}
        \PYG{k}{else}\PYG{p}{:}
            \PYG{k}{if} \PYG{n}{cur\PYGZus{}i} \PYG{o}{\PYGZgt{}}\PYG{o}{=} \PYG{l+m+mi}{0}\PYG{p}{:}
                \PYG{k}{if} \PYG{n}{best\PYGZus{}i} \PYG{o}{\PYGZlt{}} \PYG{l+m+mi}{0} \PYG{o+ow}{or} \PYG{p}{(}\PYG{n}{k}\PYG{o}{\PYGZhy{}}\PYG{l+m+mi}{1} \PYG{o}{\PYGZhy{}} \PYG{n}{cur\PYGZus{}i}\PYG{p}{)} \PYG{o}{\PYGZgt{}} \PYG{p}{(}\PYG{n}{best\PYGZus{}j} \PYG{o}{\PYGZhy{}} \PYG{n}{best\PYGZus{}i}\PYG{p}{)}\PYG{p}{:}
                    \PYG{n}{best\PYGZus{}i}\PYG{p}{,} \PYG{n}{best\PYGZus{}j} \PYG{o}{=} \PYG{n}{cur\PYGZus{}i}\PYG{p}{,} \PYG{n}{k}\PYG{o}{\PYGZhy{}}\PYG{l+m+mi}{1}
                \PYG{n}{cur\PYGZus{}i} \PYG{o}{=} \PYG{o}{\PYGZhy{}}\PYG{l+m+mi}{1}
    \PYG{k}{if} \PYG{n}{cur\PYGZus{}i} \PYG{o}{\PYGZgt{}}\PYG{o}{=} \PYG{l+m+mi}{0}\PYG{p}{:}
        \PYG{n}{k} \PYG{o}{=} \PYG{n+nb}{len}\PYG{p}{(}\PYG{n}{vec\PYGZus{}bool}\PYG{p}{)}
        \PYG{k}{if} \PYG{n}{best\PYGZus{}i} \PYG{o}{\PYGZlt{}} \PYG{l+m+mi}{0} \PYG{o+ow}{or} \PYG{p}{(}\PYG{n}{k}\PYG{o}{\PYGZhy{}}\PYG{l+m+mi}{1} \PYG{o}{\PYGZhy{}} \PYG{n}{cur\PYGZus{}i}\PYG{p}{)} \PYG{o}{\PYGZgt{}} \PYG{p}{(}\PYG{n}{best\PYGZus{}j} \PYG{o}{\PYGZhy{}} \PYG{n}{best\PYGZus{}i}\PYG{p}{)}\PYG{p}{:}
            \PYG{n}{best\PYGZus{}i}\PYG{p}{,} \PYG{n}{best\PYGZus{}j} \PYG{o}{=} \PYG{n}{cur\PYGZus{}i}\PYG{p}{,} \PYG{n}{k}\PYG{o}{\PYGZhy{}}\PYG{l+m+mi}{1}
    \PYG{k}{return} \PYG{k+kc}{None} \PYG{k}{if} \PYG{n}{best\PYGZus{}i} \PYG{o}{\PYGZlt{}} \PYG{l+m+mi}{0} \PYG{k}{else} \PYG{p}{(}\PYG{n}{best\PYGZus{}i}\PYG{p}{,} \PYG{n}{best\PYGZus{}j}\PYG{p}{)}

\PYG{k}{def}\PYG{+w}{ }\PYG{n+nf}{belt\PYGZus{}roi\PYGZus{}from\PYGZus{}blue}\PYG{p}{(}\PYG{n}{hsv}\PYG{p}{:} \PYG{n}{np}\PYG{o}{.}\PYG{n}{ndarray}\PYG{p}{,} \PYG{n}{lower}\PYG{p}{:} \PYG{n}{np}\PYG{o}{.}\PYG{n}{ndarray}\PYG{p}{,} \PYG{n}{upper}\PYG{p}{:} \PYG{n}{np}\PYG{o}{.}\PYG{n}{ndarray}\PYG{p}{)}\PYG{p}{:}
\PYG{+w}{    }\PYG{l+s+sd}{\PYGZdq{}\PYGZdq{}\PYGZdq{}}
\PYG{l+s+sd}{    ROI = rectángulo sobre la cinta:}
\PYG{l+s+sd}{    \PYGZhy{} Umbral HSV → máscara azul}
\PYG{l+s+sd}{    \PYGZhy{} Cobertura azul por columna y tramo contiguo más largo}
\PYG{l+s+sd}{    \PYGZhy{} Rango vertical por filas en ese tramo}
\PYG{l+s+sd}{    \PYGZdq{}\PYGZdq{}\PYGZdq{}}
    \PYG{n}{H}\PYG{p}{,} \PYG{n}{W} \PYG{o}{=} \PYG{n}{hsv}\PYG{o}{.}\PYG{n}{shape}\PYG{p}{[}\PYG{p}{:}\PYG{l+m+mi}{2}\PYG{p}{]}
    \PYG{n}{blue\PYGZus{}raw} \PYG{o}{=} \PYG{n}{cv2}\PYG{o}{.}\PYG{n}{inRange}\PYG{p}{(}\PYG{n}{hsv}\PYG{p}{,} \PYG{n}{lower}\PYG{p}{,} \PYG{n}{upper}\PYG{p}{)}\PYG{o}{.}\PYG{n}{astype}\PYG{p}{(}\PYG{n}{np}\PYG{o}{.}\PYG{n}{uint8}\PYG{p}{)}
    \PYG{n}{blue} \PYG{o}{=} \PYG{n}{\PYGZus{}open}\PYG{p}{(}\PYG{n}{blue\PYGZus{}raw}\PYG{p}{,} \PYG{n}{cv2}\PYG{o}{.}\PYG{n}{MORPH\PYGZus{}RECT}\PYG{p}{,} \PYG{p}{(}\PYG{l+m+mi}{3}\PYG{p}{,} \PYG{l+m+mi}{3}\PYG{p}{)}\PYG{p}{,} \PYG{l+m+mi}{1}\PYG{p}{)}
    \PYG{n}{blue} \PYG{o}{=} \PYG{n}{cv2}\PYG{o}{.}\PYG{n}{medianBlur}\PYG{p}{(}\PYG{n}{blue}\PYG{p}{,} \PYG{l+m+mi}{3}\PYG{p}{)}

    \PYG{n}{col\PYGZus{}counts} \PYG{o}{=} \PYG{p}{(}\PYG{n}{blue} \PYG{o}{\PYGZgt{}} \PYG{l+m+mi}{0}\PYG{p}{)}\PYG{o}{.}\PYG{n}{sum}\PYG{p}{(}\PYG{n}{axis}\PYG{o}{=}\PYG{l+m+mi}{0}\PYG{p}{)}\PYG{o}{.}\PYG{n}{astype}\PYG{p}{(}\PYG{n}{np}\PYG{o}{.}\PYG{n}{int32}\PYG{p}{)}
    \PYG{n}{cover\PYGZus{}ratio} \PYG{o}{=} \PYG{n}{col\PYGZus{}counts} \PYG{o}{/} \PYG{n+nb}{float}\PYG{p}{(}\PYG{n}{H}\PYG{p}{)}
    \PYG{n}{col\PYGZus{}mask} \PYG{o}{=} \PYG{n}{cover\PYGZus{}ratio} \PYG{o}{\PYGZgt{}}\PYG{o}{=} \PYG{n}{CONFIG}\PYG{p}{[}\PYG{l+s+s2}{\PYGZdq{}}\PYG{l+s+s2}{belt\PYGZus{}col\PYGZus{}cover\PYGZus{}ratio}\PYG{l+s+s2}{\PYGZdq{}}\PYG{p}{]}
    \PYG{n}{run} \PYG{o}{=} \PYG{n}{\PYGZus{}longest\PYGZus{}true\PYGZus{}run}\PYG{p}{(}\PYG{n}{col\PYGZus{}mask}\PYG{p}{)}
    \PYG{k}{if} \PYG{n}{run} \PYG{o+ow}{is} \PYG{k+kc}{None}\PYG{p}{:}
        \PYG{k}{raise} \PYG{n+ne}{RuntimeError}\PYG{p}{(}\PYG{l+s+s2}{\PYGZdq{}}\PYG{l+s+s2}{No se identificó ningún tramo de cinta por cobertura azul.}\PYG{l+s+s2}{\PYGZdq{}}\PYG{p}{)}
    \PYG{n}{x1}\PYG{p}{,} \PYG{n}{x2} \PYG{o}{=} \PYG{n}{run}
    \PYG{k}{if} \PYG{p}{(}\PYG{n}{x2} \PYG{o}{\PYGZhy{}} \PYG{n}{x1} \PYG{o}{+} \PYG{l+m+mi}{1}\PYG{p}{)} \PYG{o}{\PYGZlt{}} \PYG{n}{CONFIG}\PYG{p}{[}\PYG{l+s+s2}{\PYGZdq{}}\PYG{l+s+s2}{belt\PYGZus{}min\PYGZus{}width\PYGZus{}px}\PYG{l+s+s2}{\PYGZdq{}}\PYG{p}{]}\PYG{p}{:}
        \PYG{n}{relax} \PYG{o}{=} \PYG{n+nb}{max}\PYG{p}{(}\PYG{l+m+mf}{0.02}\PYG{p}{,} \PYG{l+m+mf}{0.5} \PYG{o}{*} \PYG{n}{CONFIG}\PYG{p}{[}\PYG{l+s+s2}{\PYGZdq{}}\PYG{l+s+s2}{belt\PYGZus{}col\PYGZus{}cover\PYGZus{}ratio}\PYG{l+s+s2}{\PYGZdq{}}\PYG{p}{]}\PYG{p}{)}
        \PYG{n}{col\PYGZus{}mask\PYGZus{}relaxed} \PYG{o}{=} \PYG{n}{cover\PYGZus{}ratio} \PYG{o}{\PYGZgt{}}\PYG{o}{=} \PYG{p}{(}\PYG{n}{CONFIG}\PYG{p}{[}\PYG{l+s+s2}{\PYGZdq{}}\PYG{l+s+s2}{belt\PYGZus{}col\PYGZus{}cover\PYGZus{}ratio}\PYG{l+s+s2}{\PYGZdq{}}\PYG{p}{]} \PYG{o}{\PYGZhy{}} \PYG{n}{relax}\PYG{p}{)}
        \PYG{n}{run2} \PYG{o}{=} \PYG{n}{\PYGZus{}longest\PYGZus{}true\PYGZus{}run}\PYG{p}{(}\PYG{n}{col\PYGZus{}mask\PYGZus{}relaxed}\PYG{p}{)}
        \PYG{k}{if} \PYG{n}{run2} \PYG{o+ow}{is} \PYG{k+kc}{None} \PYG{o+ow}{or} \PYG{p}{(}\PYG{n}{run2}\PYG{p}{[}\PYG{l+m+mi}{1}\PYG{p}{]} \PYG{o}{\PYGZhy{}} \PYG{n}{run2}\PYG{p}{[}\PYG{l+m+mi}{0}\PYG{p}{]} \PYG{o}{+} \PYG{l+m+mi}{1}\PYG{p}{)} \PYG{o}{\PYGZlt{}} \PYG{n}{CONFIG}\PYG{p}{[}\PYG{l+s+s2}{\PYGZdq{}}\PYG{l+s+s2}{belt\PYGZus{}min\PYGZus{}width\PYGZus{}px}\PYG{l+s+s2}{\PYGZdq{}}\PYG{p}{]}\PYG{p}{:}
            \PYG{k}{raise} \PYG{n+ne}{RuntimeError}\PYG{p}{(}\PYG{l+s+s2}{\PYGZdq{}}\PYG{l+s+s2}{La anchura estimada de la cinta es demasiado pequeña.}\PYG{l+s+s2}{\PYGZdq{}}\PYG{p}{)}
        \PYG{n}{x1}\PYG{p}{,} \PYG{n}{x2} \PYG{o}{=} \PYG{n}{run2}

    \PYG{n}{blue\PYGZus{}slice} \PYG{o}{=} \PYG{n}{blue}\PYG{p}{[}\PYG{p}{:}\PYG{p}{,} \PYG{n}{x1}\PYG{p}{:}\PYG{n}{x2} \PYG{o}{+} \PYG{l+m+mi}{1}\PYG{p}{]}
    \PYG{n}{row\PYGZus{}counts} \PYG{o}{=} \PYG{p}{(}\PYG{n}{blue\PYGZus{}slice} \PYG{o}{\PYGZgt{}} \PYG{l+m+mi}{0}\PYG{p}{)}\PYG{o}{.}\PYG{n}{sum}\PYG{p}{(}\PYG{n}{axis}\PYG{o}{=}\PYG{l+m+mi}{1}\PYG{p}{)}\PYG{o}{.}\PYG{n}{astype}\PYG{p}{(}\PYG{n}{np}\PYG{o}{.}\PYG{n}{int32}\PYG{p}{)}
    \PYG{n}{row\PYGZus{}mask} \PYG{o}{=} \PYG{n}{row\PYGZus{}counts} \PYG{o}{\PYGZgt{}}\PYG{o}{=} \PYG{n+nb}{max}\PYG{p}{(}\PYG{l+m+mi}{1}\PYG{p}{,} \PYG{n+nb}{int}\PYG{p}{(}\PYG{l+m+mf}{0.02} \PYG{o}{*} \PYG{p}{(}\PYG{n}{x2} \PYG{o}{\PYGZhy{}} \PYG{n}{x1} \PYG{o}{+} \PYG{l+m+mi}{1}\PYG{p}{)}\PYG{p}{)}\PYG{p}{)}
    \PYG{n}{row\PYGZus{}run} \PYG{o}{=} \PYG{n}{\PYGZus{}longest\PYGZus{}true\PYGZus{}run}\PYG{p}{(}\PYG{n}{row\PYGZus{}mask}\PYG{p}{)}
    \PYG{n}{y1}\PYG{p}{,} \PYG{n}{y2} \PYG{o}{=} \PYG{p}{(}\PYG{l+m+mi}{0}\PYG{p}{,} \PYG{n}{H} \PYG{o}{\PYGZhy{}} \PYG{l+m+mi}{1}\PYG{p}{)} \PYG{k}{if} \PYG{n}{row\PYGZus{}run} \PYG{o+ow}{is} \PYG{k+kc}{None} \PYG{k}{else} \PYG{n}{row\PYGZus{}run}

    \PYG{n}{pad} \PYG{o}{=} \PYG{n}{CONFIG}\PYG{p}{[}\PYG{l+s+s2}{\PYGZdq{}}\PYG{l+s+s2}{pad\PYGZus{}roi}\PYG{l+s+s2}{\PYGZdq{}}\PYG{p}{]}
    \PYG{n}{x} \PYG{o}{=} \PYG{n+nb}{max}\PYG{p}{(}\PYG{l+m+mi}{0}\PYG{p}{,} \PYG{n}{x1} \PYG{o}{\PYGZhy{}} \PYG{n}{pad}\PYG{p}{)}\PYG{p}{;} \PYG{n}{y} \PYG{o}{=} \PYG{n+nb}{max}\PYG{p}{(}\PYG{l+m+mi}{0}\PYG{p}{,} \PYG{n}{y1} \PYG{o}{\PYGZhy{}} \PYG{n}{pad}\PYG{p}{)}
    \PYG{n}{w} \PYG{o}{=} \PYG{n+nb}{min}\PYG{p}{(}\PYG{n}{W} \PYG{o}{\PYGZhy{}} \PYG{n}{x}\PYG{p}{,} \PYG{p}{(}\PYG{n}{x2} \PYG{o}{\PYGZhy{}} \PYG{n}{x1} \PYG{o}{+} \PYG{l+m+mi}{1}\PYG{p}{)} \PYG{o}{+} \PYG{l+m+mi}{2} \PYG{o}{*} \PYG{n}{pad}\PYG{p}{)}
    \PYG{n}{h} \PYG{o}{=} \PYG{n+nb}{min}\PYG{p}{(}\PYG{n}{H} \PYG{o}{\PYGZhy{}} \PYG{n}{y}\PYG{p}{,} \PYG{p}{(}\PYG{n}{y2} \PYG{o}{\PYGZhy{}} \PYG{n}{y1} \PYG{o}{+} \PYG{l+m+mi}{1}\PYG{p}{)} \PYG{o}{+} \PYG{l+m+mi}{2} \PYG{o}{*} \PYG{n}{pad}\PYG{p}{)}

    \PYG{n}{dbg} \PYG{o}{=} \PYG{n}{cv2}\PYG{o}{.}\PYG{n}{cvtColor}\PYG{p}{(}\PYG{n}{hsv}\PYG{p}{,} \PYG{n}{cv2}\PYG{o}{.}\PYG{n}{COLOR\PYGZus{}HSV2BGR}\PYG{p}{)}
    \PYG{n}{cv2}\PYG{o}{.}\PYG{n}{rectangle}\PYG{p}{(}\PYG{n}{dbg}\PYG{p}{,} \PYG{p}{(}\PYG{n}{x}\PYG{p}{,} \PYG{n}{y}\PYG{p}{)}\PYG{p}{,} \PYG{p}{(}\PYG{n}{x} \PYG{o}{+} \PYG{n}{w}\PYG{p}{,} \PYG{n}{y} \PYG{o}{+} \PYG{n}{h}\PYG{p}{)}\PYG{p}{,} \PYG{p}{(}\PYG{l+m+mi}{255}\PYG{p}{,} \PYG{l+m+mi}{0}\PYG{p}{,} \PYG{l+m+mi}{255}\PYG{p}{)}\PYG{p}{,} \PYG{l+m+mi}{3}\PYG{p}{)}
    \PYG{n}{vis} \PYG{o}{=} \PYG{n}{dbg}\PYG{o}{.}\PYG{n}{copy}\PYG{p}{(}\PYG{p}{)}
    \PYG{n}{vis}\PYG{p}{[}\PYG{p}{:}\PYG{p}{,} \PYG{n}{col\PYGZus{}mask}\PYG{p}{,} \PYG{p}{:}\PYG{p}{]} \PYG{o}{=} \PYG{n}{cv2}\PYG{o}{.}\PYG{n}{addWeighted}\PYG{p}{(}\PYG{n}{vis}\PYG{p}{[}\PYG{p}{:}\PYG{p}{,} \PYG{n}{col\PYGZus{}mask}\PYG{p}{,} \PYG{p}{:}\PYG{p}{]}\PYG{p}{,} \PYG{l+m+mf}{0.5}\PYG{p}{,} \PYG{n}{np}\PYG{o}{.}\PYG{n}{full\PYGZus{}like}\PYG{p}{(}\PYG{n}{vis}\PYG{p}{[}\PYG{p}{:}\PYG{p}{,} \PYG{n}{col\PYGZus{}mask}\PYG{p}{,} \PYG{p}{:}\PYG{p}{]}\PYG{p}{,} \PYG{p}{(}\PYG{l+m+mi}{255}\PYG{p}{,} \PYG{l+m+mi}{0}\PYG{p}{,} \PYG{l+m+mi}{0}\PYG{p}{)}\PYG{p}{)}\PYG{p}{,} \PYG{l+m+mf}{0.5}\PYG{p}{,} \PYG{l+m+mi}{0}\PYG{p}{)}
    \PYG{n}{save\PYGZus{}img\PYGZus{}dbg}\PYG{p}{(}\PYG{l+s+s2}{\PYGZdq{}}\PYG{l+s+s2}{02b\PYGZus{}belt\PYGZus{}columns.png}\PYG{l+s+s2}{\PYGZdq{}}\PYG{p}{,} \PYG{n}{vis}\PYG{p}{)}
    \PYG{n}{save\PYGZus{}img\PYGZus{}dbg}\PYG{p}{(}\PYG{l+s+s2}{\PYGZdq{}}\PYG{l+s+s2}{03\PYGZus{}roi\PYGZus{}debug.png}\PYG{l+s+s2}{\PYGZdq{}}\PYG{p}{,} \PYG{n}{dbg}\PYG{p}{)}
    \PYG{n}{save\PYGZus{}img\PYGZus{}dbg}\PYG{p}{(}\PYG{l+s+s2}{\PYGZdq{}}\PYG{l+s+s2}{02\PYGZus{}blue\PYGZus{}mask.png}\PYG{l+s+s2}{\PYGZdq{}}\PYG{p}{,} \PYG{n}{blue\PYGZus{}raw}\PYG{p}{)}
    \PYG{k}{return} \PYG{p}{(}\PYG{n}{x}\PYG{p}{,} \PYG{n}{y}\PYG{p}{,} \PYG{n}{w}\PYG{p}{,} \PYG{n}{h}\PYG{p}{)}\PYG{p}{,} \PYG{n}{blue}

\PYG{k}{def}\PYG{+w}{ }\PYG{n+nf}{seeds\PYGZus{}in\PYGZus{}roi}\PYG{p}{(}\PYG{n}{bgr\PYGZus{}roi}\PYG{p}{:} \PYG{n}{np}\PYG{o}{.}\PYG{n}{ndarray}\PYG{p}{,} \PYG{n}{lower}\PYG{p}{:} \PYG{n}{np}\PYG{o}{.}\PYG{n}{ndarray}\PYG{p}{,} \PYG{n}{upper}\PYG{p}{:} \PYG{n}{np}\PYG{o}{.}\PYG{n}{ndarray}\PYG{p}{)}\PYG{p}{:}
    \PYG{n}{hsv} \PYG{o}{=} \PYG{n}{cv2}\PYG{o}{.}\PYG{n}{cvtColor}\PYG{p}{(}\PYG{n}{bgr\PYGZus{}roi}\PYG{p}{,} \PYG{n}{cv2}\PYG{o}{.}\PYG{n}{COLOR\PYGZus{}BGR2HSV}\PYG{p}{)}
    \PYG{n}{blue} \PYG{o}{=} \PYG{n}{cv2}\PYG{o}{.}\PYG{n}{inRange}\PYG{p}{(}\PYG{n}{hsv}\PYG{p}{,} \PYG{n}{lower}\PYG{p}{,} \PYG{n}{upper}\PYG{p}{)}\PYG{o}{.}\PYG{n}{astype}\PYG{p}{(}\PYG{n}{np}\PYG{o}{.}\PYG{n}{uint8}\PYG{p}{)}
    \PYG{n}{not\PYGZus{}blue} \PYG{o}{=} \PYG{n}{cv2}\PYG{o}{.}\PYG{n}{bitwise\PYGZus{}not}\PYG{p}{(}\PYG{n}{blue}\PYG{p}{)}

    \PYG{n}{k\PYGZus{}vert} \PYG{o}{=} \PYG{n}{cv2}\PYG{o}{.}\PYG{n}{getStructuringElement}\PYG{p}{(}\PYG{n}{cv2}\PYG{o}{.}\PYG{n}{MORPH\PYGZus{}ELLIPSE}\PYG{p}{,} \PYG{n}{CONFIG}\PYG{p}{[}\PYG{l+s+s2}{\PYGZdq{}}\PYG{l+s+s2}{pr\PYGZus{}fg\PYGZus{}kernel}\PYG{l+s+s2}{\PYGZdq{}}\PYG{p}{]}\PYG{p}{)}
    \PYG{n}{pr\PYGZus{}fg} \PYG{o}{=} \PYG{n}{cv2}\PYG{o}{.}\PYG{n}{dilate}\PYG{p}{(}\PYG{n}{not\PYGZus{}blue}\PYG{p}{,} \PYG{n}{k\PYGZus{}vert}\PYG{p}{,} \PYG{n}{iterations}\PYG{o}{=}\PYG{l+m+mi}{1}\PYG{p}{)}

    \PYG{n}{h}\PYG{p}{,} \PYG{n}{s}\PYG{p}{,} \PYG{n}{\PYGZus{}} \PYG{o}{=} \PYG{n}{cv2}\PYG{o}{.}\PYG{n}{split}\PYG{p}{(}\PYG{n}{hsv}\PYG{p}{)}
    \PYG{n}{belt\PYGZus{}h} \PYG{o}{=} \PYG{n}{h}\PYG{p}{[}\PYG{n}{blue} \PYG{o}{\PYGZgt{}} \PYG{l+m+mi}{0}\PYG{p}{]}\PYG{o}{.}\PYG{n}{astype}\PYG{p}{(}\PYG{n}{np}\PYG{o}{.}\PYG{n}{float32}\PYG{p}{)}\PYG{p}{;} \PYG{n}{belt\PYGZus{}s} \PYG{o}{=} \PYG{n}{s}\PYG{p}{[}\PYG{n}{blue} \PYG{o}{\PYGZgt{}} \PYG{l+m+mi}{0}\PYG{p}{]}\PYG{o}{.}\PYG{n}{astype}\PYG{p}{(}\PYG{n}{np}\PYG{o}{.}\PYG{n}{float32}\PYG{p}{)}
    \PYG{k}{if} \PYG{n}{belt\PYGZus{}h}\PYG{o}{.}\PYG{n}{size}\PYG{p}{:}
        \PYG{n}{mu\PYGZus{}h}\PYG{p}{,} \PYG{n}{mu\PYGZus{}s} \PYG{o}{=} \PYG{n+nb}{float}\PYG{p}{(}\PYG{n}{np}\PYG{o}{.}\PYG{n}{median}\PYG{p}{(}\PYG{n}{belt\PYGZus{}h}\PYG{p}{)}\PYG{p}{)}\PYG{p}{,} \PYG{n+nb}{float}\PYG{p}{(}\PYG{n}{np}\PYG{o}{.}\PYG{n}{median}\PYG{p}{(}\PYG{n}{belt\PYGZus{}s}\PYG{p}{)}\PYG{p}{)}
        \PYG{n}{dh} \PYG{o}{=} \PYG{n}{cv2}\PYG{o}{.}\PYG{n}{absdiff}\PYG{p}{(}\PYG{n}{h}\PYG{o}{.}\PYG{n}{astype}\PYG{p}{(}\PYG{n}{np}\PYG{o}{.}\PYG{n}{int16}\PYG{p}{)}\PYG{p}{,} \PYG{n}{np}\PYG{o}{.}\PYG{n}{full\PYGZus{}like}\PYG{p}{(}\PYG{n}{h}\PYG{p}{,} \PYG{n+nb}{int}\PYG{p}{(}\PYG{n}{mu\PYGZus{}h}\PYG{p}{)}\PYG{p}{,} \PYG{n}{np}\PYG{o}{.}\PYG{n}{int16}\PYG{p}{)}\PYG{p}{)}\PYG{o}{.}\PYG{n}{astype}\PYG{p}{(}\PYG{n}{np}\PYG{o}{.}\PYG{n}{uint8}\PYG{p}{)}
        \PYG{n}{ds} \PYG{o}{=} \PYG{n}{cv2}\PYG{o}{.}\PYG{n}{absdiff}\PYG{p}{(}\PYG{n}{s}\PYG{o}{.}\PYG{n}{astype}\PYG{p}{(}\PYG{n}{np}\PYG{o}{.}\PYG{n}{int16}\PYG{p}{)}\PYG{p}{,} \PYG{n}{np}\PYG{o}{.}\PYG{n}{full\PYGZus{}like}\PYG{p}{(}\PYG{n}{s}\PYG{p}{,} \PYG{n+nb}{int}\PYG{p}{(}\PYG{n}{mu\PYGZus{}s}\PYG{p}{)}\PYG{p}{,} \PYG{n}{np}\PYG{o}{.}\PYG{n}{int16}\PYG{p}{)}\PYG{p}{)}\PYG{o}{.}\PYG{n}{astype}\PYG{p}{(}\PYG{n}{np}\PYG{o}{.}\PYG{n}{uint8}\PYG{p}{)}
        \PYG{n}{contrast} \PYG{o}{=} \PYG{n}{cv2}\PYG{o}{.}\PYG{n}{addWeighted}\PYG{p}{(}\PYG{n}{dh}\PYG{p}{,} \PYG{l+m+mf}{0.7}\PYG{p}{,} \PYG{n}{ds}\PYG{p}{,} \PYG{l+m+mf}{0.3}\PYG{p}{,} \PYG{l+m+mi}{0}\PYG{p}{)}
        \PYG{n}{thr} \PYG{o}{=} \PYG{n+nb}{int}\PYG{p}{(}\PYG{n}{np}\PYG{o}{.}\PYG{n}{percentile}\PYG{p}{(}\PYG{n}{contrast}\PYG{p}{,} \PYG{n}{CONFIG}\PYG{p}{[}\PYG{l+s+s2}{\PYGZdq{}}\PYG{l+s+s2}{sure\PYGZus{}fg\PYGZus{}pct\PYGZus{}contrast}\PYG{l+s+s2}{\PYGZdq{}}\PYG{p}{]}\PYG{p}{)}\PYG{p}{)}
        \PYG{n}{\PYGZus{}}\PYG{p}{,} \PYG{n}{hi} \PYG{o}{=} \PYG{n}{cv2}\PYG{o}{.}\PYG{n}{threshold}\PYG{p}{(}\PYG{n}{contrast}\PYG{p}{,} \PYG{n}{thr}\PYG{p}{,} \PYG{l+m+mi}{255}\PYG{p}{,} \PYG{n}{cv2}\PYG{o}{.}\PYG{n}{THRESH\PYGZus{}BINARY}\PYG{p}{)}
        \PYG{n}{sure\PYGZus{}fg} \PYG{o}{=} \PYG{n}{cv2}\PYG{o}{.}\PYG{n}{bitwise\PYGZus{}and}\PYG{p}{(}\PYG{n}{hi}\PYG{p}{,} \PYG{n}{not\PYGZus{}blue}\PYG{p}{)}
        \PYG{n}{save\PYGZus{}img\PYGZus{}dbg}\PYG{p}{(}\PYG{l+s+s2}{\PYGZdq{}}\PYG{l+s+s2}{07\PYGZus{}contrast.png}\PYG{l+s+s2}{\PYGZdq{}}\PYG{p}{,} \PYG{n}{cv2}\PYG{o}{.}\PYG{n}{normalize}\PYG{p}{(}\PYG{n}{contrast}\PYG{p}{,} \PYG{k+kc}{None}\PYG{p}{,} \PYG{l+m+mi}{0}\PYG{p}{,} \PYG{l+m+mi}{255}\PYG{p}{,} \PYG{n}{cv2}\PYG{o}{.}\PYG{n}{NORM\PYGZus{}MINMAX}\PYG{p}{)}\PYG{p}{)}
    \PYG{k}{else}\PYG{p}{:}
        \PYG{n}{sure\PYGZus{}fg} \PYG{o}{=} \PYG{n}{cv2}\PYG{o}{.}\PYG{n}{erode}\PYG{p}{(}\PYG{n}{pr\PYGZus{}fg}\PYG{p}{,} \PYG{n}{cv2}\PYG{o}{.}\PYG{n}{getStructuringElement}\PYG{p}{(}\PYG{n}{cv2}\PYG{o}{.}\PYG{n}{MORPH\PYGZus{}ELLIPSE}\PYG{p}{,} \PYG{p}{(}\PYG{l+m+mi}{9}\PYG{p}{,} \PYG{l+m+mi}{9}\PYG{p}{)}\PYG{p}{)}\PYG{p}{,} \PYG{l+m+mi}{1}\PYG{p}{)}

    \PYG{n}{dist} \PYG{o}{=} \PYG{n}{cv2}\PYG{o}{.}\PYG{n}{distanceTransform}\PYG{p}{(}\PYG{p}{(}\PYG{n}{pr\PYGZus{}fg} \PYG{o}{\PYGZgt{}} \PYG{l+m+mi}{0}\PYG{p}{)}\PYG{o}{.}\PYG{n}{astype}\PYG{p}{(}\PYG{n}{np}\PYG{o}{.}\PYG{n}{uint8}\PYG{p}{)}\PYG{p}{,} \PYG{n}{cv2}\PYG{o}{.}\PYG{n}{DIST\PYGZus{}L2}\PYG{p}{,} \PYG{l+m+mi}{5}\PYG{p}{)}
    \PYG{k}{if} \PYG{n}{np}\PYG{o}{.}\PYG{n}{any}\PYG{p}{(}\PYG{n}{dist} \PYG{o}{\PYGZgt{}} \PYG{l+m+mi}{0}\PYG{p}{)}\PYG{p}{:}
        \PYG{n}{th} \PYG{o}{=} \PYG{n}{np}\PYG{o}{.}\PYG{n}{percentile}\PYG{p}{(}\PYG{n}{dist}\PYG{p}{[}\PYG{n}{dist} \PYG{o}{\PYGZgt{}} \PYG{l+m+mi}{0}\PYG{p}{]}\PYG{p}{,} \PYG{n}{CONFIG}\PYG{p}{[}\PYG{l+s+s2}{\PYGZdq{}}\PYG{l+s+s2}{sure\PYGZus{}fg\PYGZus{}pct\PYGZus{}dt}\PYG{l+s+s2}{\PYGZdq{}}\PYG{p}{]}\PYG{p}{)}
        \PYG{n}{sure\PYGZus{}fg} \PYG{o}{|}\PYG{o}{=} \PYG{p}{(}\PYG{n}{dist} \PYG{o}{\PYGZgt{}} \PYG{n}{th}\PYG{p}{)}\PYG{o}{.}\PYG{n}{astype}\PYG{p}{(}\PYG{n}{np}\PYG{o}{.}\PYG{n}{uint8}\PYG{p}{)} \PYG{o}{*} \PYG{l+m+mi}{255}

    \PYG{n}{gc\PYGZus{}mask} \PYG{o}{=} \PYG{n}{np}\PYG{o}{.}\PYG{n}{full}\PYG{p}{(}\PYG{n}{blue}\PYG{o}{.}\PYG{n}{shape}\PYG{p}{,} \PYG{n}{cv2}\PYG{o}{.}\PYG{n}{GC\PYGZus{}PR\PYGZus{}BGD}\PYG{p}{,} \PYG{n}{np}\PYG{o}{.}\PYG{n}{uint8}\PYG{p}{)}
    \PYG{n}{gc\PYGZus{}mask}\PYG{p}{[}\PYG{n}{blue} \PYG{o}{\PYGZgt{}} \PYG{l+m+mi}{0}\PYG{p}{]}    \PYG{o}{=} \PYG{n}{cv2}\PYG{o}{.}\PYG{n}{GC\PYGZus{}BGD}
    \PYG{n}{gc\PYGZus{}mask}\PYG{p}{[}\PYG{n}{pr\PYGZus{}fg} \PYG{o}{\PYGZgt{}} \PYG{l+m+mi}{0}\PYG{p}{]}   \PYG{o}{=} \PYG{n}{cv2}\PYG{o}{.}\PYG{n}{GC\PYGZus{}PR\PYGZus{}FGD}
    \PYG{n}{gc\PYGZus{}mask}\PYG{p}{[}\PYG{n}{sure\PYGZus{}fg} \PYG{o}{\PYGZgt{}} \PYG{l+m+mi}{0}\PYG{p}{]} \PYG{o}{=} \PYG{n}{cv2}\PYG{o}{.}\PYG{n}{GC\PYGZus{}FGD}

    \PYG{c+c1}{\PYGZsh{} Horizontal borders as BG}
    \PYG{n}{m} \PYG{o}{=} \PYG{l+m+mi}{6}
    \PYG{n}{gc\PYGZus{}mask}\PYG{p}{[}\PYG{p}{:}\PYG{n}{m}\PYG{p}{,} \PYG{p}{:}\PYG{p}{]}  \PYG{o}{=} \PYG{n}{cv2}\PYG{o}{.}\PYG{n}{GC\PYGZus{}BGD}
    \PYG{n}{gc\PYGZus{}mask}\PYG{p}{[}\PYG{o}{\PYGZhy{}}\PYG{n}{m}\PYG{p}{:}\PYG{p}{,} \PYG{p}{:}\PYG{p}{]} \PYG{o}{=} \PYG{n}{cv2}\PYG{o}{.}\PYG{n}{GC\PYGZus{}BGD}
    \PYG{c+c1}{\PYGZsh{} Lateral guard\PYGZhy{}rails as BG}
    \PYG{n}{W} \PYG{o}{=} \PYG{n}{gc\PYGZus{}mask}\PYG{o}{.}\PYG{n}{shape}\PYG{p}{[}\PYG{l+m+mi}{1}\PYG{p}{]}
    \PYG{n}{g} \PYG{o}{=} \PYG{n+nb}{int}\PYG{p}{(}\PYG{n}{CONFIG}\PYG{p}{[}\PYG{l+s+s2}{\PYGZdq{}}\PYG{l+s+s2}{guard\PYGZus{}frac\PYGZus{}w}\PYG{l+s+s2}{\PYGZdq{}}\PYG{p}{]} \PYG{o}{*} \PYG{n}{W}\PYG{p}{)}
    \PYG{n}{g} \PYG{o}{=} \PYG{n+nb}{max}\PYG{p}{(}\PYG{n}{CONFIG}\PYG{p}{[}\PYG{l+s+s2}{\PYGZdq{}}\PYG{l+s+s2}{guard\PYGZus{}min\PYGZus{}px}\PYG{l+s+s2}{\PYGZdq{}}\PYG{p}{]}\PYG{p}{,} \PYG{n+nb}{min}\PYG{p}{(}\PYG{n}{CONFIG}\PYG{p}{[}\PYG{l+s+s2}{\PYGZdq{}}\PYG{l+s+s2}{guard\PYGZus{}max\PYGZus{}px}\PYG{l+s+s2}{\PYGZdq{}}\PYG{p}{]}\PYG{p}{,} \PYG{n}{g}\PYG{p}{)}\PYG{p}{)}
    \PYG{n}{gc\PYGZus{}mask}\PYG{p}{[}\PYG{p}{:}\PYG{p}{,} \PYG{p}{:}\PYG{n}{g}\PYG{p}{]}  \PYG{o}{=} \PYG{n}{cv2}\PYG{o}{.}\PYG{n}{GC\PYGZus{}BGD}
    \PYG{n}{gc\PYGZus{}mask}\PYG{p}{[}\PYG{p}{:}\PYG{p}{,} \PYG{o}{\PYGZhy{}}\PYG{n}{g}\PYG{p}{:}\PYG{p}{]} \PYG{o}{=} \PYG{n}{cv2}\PYG{o}{.}\PYG{n}{GC\PYGZus{}BGD}

    \PYG{n}{save\PYGZus{}img\PYGZus{}dbg}\PYG{p}{(}\PYG{l+s+s2}{\PYGZdq{}}\PYG{l+s+s2}{04\PYGZus{}pr\PYGZus{}fg.png}\PYG{l+s+s2}{\PYGZdq{}}\PYG{p}{,} \PYG{n}{pr\PYGZus{}fg}\PYG{p}{)}
    \PYG{n}{save\PYGZus{}img\PYGZus{}dbg}\PYG{p}{(}\PYG{l+s+s2}{\PYGZdq{}}\PYG{l+s+s2}{05\PYGZus{}sure\PYGZus{}fg.png}\PYG{l+s+s2}{\PYGZdq{}}\PYG{p}{,} \PYG{n}{sure\PYGZus{}fg}\PYG{p}{)}
    \PYG{n}{save\PYGZus{}img\PYGZus{}dbg}\PYG{p}{(}\PYG{l+s+s2}{\PYGZdq{}}\PYG{l+s+s2}{06\PYGZus{}not\PYGZus{}blue.png}\PYG{l+s+s2}{\PYGZdq{}}\PYG{p}{,} \PYG{n}{not\PYGZus{}blue}\PYG{p}{)}
    \PYG{n}{save\PYGZus{}img\PYGZus{}dbg}\PYG{p}{(}\PYG{l+s+s2}{\PYGZdq{}}\PYG{l+s+s2}{08\PYGZus{}gc\PYGZus{}seeds.png}\PYG{l+s+s2}{\PYGZdq{}}\PYG{p}{,} \PYG{p}{(}\PYG{n}{gc\PYGZus{}mask} \PYG{o}{*} \PYG{p}{(}\PYG{l+m+mi}{255} \PYG{o}{/}\PYG{o}{/} \PYG{l+m+mi}{3}\PYG{p}{)}\PYG{p}{)}\PYG{o}{.}\PYG{n}{astype}\PYG{p}{(}\PYG{n}{np}\PYG{o}{.}\PYG{n}{uint8}\PYG{p}{)}\PYG{p}{)}
    \PYG{k}{return} \PYG{n}{gc\PYGZus{}mask}

\PYG{k}{def}\PYG{+w}{ }\PYG{n+nf}{render\PYGZus{}candidates\PYGZus{}debug}\PYG{p}{(}\PYG{n}{bgr\PYGZus{}base}\PYG{p}{:} \PYG{n}{np}\PYG{o}{.}\PYG{n}{ndarray}\PYG{p}{,}
                            \PYG{n}{cnts}\PYG{p}{:} \PYG{n+nb}{list}\PYG{p}{,}
                            \PYG{n}{scores}\PYG{p}{:} \PYG{n+nb}{list}\PYG{p}{,}
                            \PYG{n}{best\PYGZus{}idx}\PYG{p}{:} \PYG{n+nb}{int} \PYG{o}{|} \PYG{k+kc}{None}\PYG{p}{,}
                            \PYG{n}{roi\PYGZus{}xywh}\PYG{p}{:} \PYG{n+nb}{tuple}\PYG{p}{[}\PYG{n+nb}{int}\PYG{p}{,}\PYG{n+nb}{int}\PYG{p}{,}\PYG{n+nb}{int}\PYG{p}{,}\PYG{n+nb}{int}\PYG{p}{]}\PYG{p}{,}
                            \PYG{n}{tag}\PYG{p}{:} \PYG{n+nb}{str}\PYG{p}{)}\PYG{p}{:}
\PYG{+w}{    }\PYG{l+s+sd}{\PYGZdq{}\PYGZdq{}\PYGZdq{}Writes PNG only in debug; returns path or None.\PYGZdq{}\PYGZdq{}\PYGZdq{}}
    \PYG{k}{if} \PYG{o+ow}{not} \PYG{n}{\PYGZus{}dbg}\PYG{p}{(}\PYG{p}{)}\PYG{p}{:}
        \PYG{k}{return} \PYG{k+kc}{None}
    \PYG{n}{vis} \PYG{o}{=} \PYG{n}{bgr\PYGZus{}base}\PYG{o}{.}\PYG{n}{copy}\PYG{p}{(}\PYG{p}{)}
    \PYG{n}{xR}\PYG{p}{,} \PYG{n}{yR}\PYG{p}{,} \PYG{n}{wR}\PYG{p}{,} \PYG{n}{hR} \PYG{o}{=} \PYG{n}{roi\PYGZus{}xywh}
    \PYG{n}{cv2}\PYG{o}{.}\PYG{n}{rectangle}\PYG{p}{(}\PYG{n}{vis}\PYG{p}{,} \PYG{p}{(}\PYG{n}{xR}\PYG{p}{,} \PYG{n}{yR}\PYG{p}{)}\PYG{p}{,} \PYG{p}{(}\PYG{n}{xR} \PYG{o}{+} \PYG{n}{wR}\PYG{p}{,} \PYG{n}{yR} \PYG{o}{+} \PYG{n}{hR}\PYG{p}{)}\PYG{p}{,} \PYG{p}{(}\PYG{l+m+mi}{255}\PYG{p}{,} \PYG{l+m+mi}{0}\PYG{p}{,} \PYG{l+m+mi}{255}\PYG{p}{)}\PYG{p}{,} \PYG{l+m+mi}{2}\PYG{p}{)}

    \PYG{n}{sc} \PYG{o}{=} \PYG{n}{np}\PYG{o}{.}\PYG{n}{array}\PYG{p}{(}\PYG{n}{scores}\PYG{p}{,} \PYG{n}{dtype}\PYG{o}{=}\PYG{n}{np}\PYG{o}{.}\PYG{n}{float32}\PYG{p}{)} \PYG{k}{if} \PYG{n}{scores} \PYG{k}{else} \PYG{n}{np}\PYG{o}{.}\PYG{n}{array}\PYG{p}{(}\PYG{p}{[}\PYG{p}{]}\PYG{p}{,} \PYG{n}{np}\PYG{o}{.}\PYG{n}{float32}\PYG{p}{)}
    \PYG{k}{if} \PYG{n}{sc}\PYG{o}{.}\PYG{n}{size}\PYG{p}{:}
        \PYG{n}{sc}\PYG{p}{[}\PYG{o}{\PYGZti{}}\PYG{n}{np}\PYG{o}{.}\PYG{n}{isfinite}\PYG{p}{(}\PYG{n}{sc}\PYG{p}{)}\PYG{p}{]} \PYG{o}{=} \PYG{l+m+mf}{0.0}
        \PYG{n}{scn} \PYG{o}{=} \PYG{p}{(}\PYG{n}{sc} \PYG{o}{\PYGZhy{}} \PYG{n}{sc}\PYG{o}{.}\PYG{n}{min}\PYG{p}{(}\PYG{p}{)}\PYG{p}{)} \PYG{o}{/} \PYG{n+nb}{max}\PYG{p}{(}\PYG{l+m+mf}{1e\PYGZhy{}9}\PYG{p}{,} \PYG{n}{sc}\PYG{o}{.}\PYG{n}{max}\PYG{p}{(}\PYG{p}{)} \PYG{o}{\PYGZhy{}} \PYG{n}{sc}\PYG{o}{.}\PYG{n}{min}\PYG{p}{(}\PYG{p}{)}\PYG{p}{)} \PYG{k}{if} \PYG{o+ow}{not} \PYG{n}{np}\PYG{o}{.}\PYG{n}{all}\PYG{p}{(}\PYG{n}{sc} \PYG{o}{==} \PYG{n}{sc}\PYG{p}{[}\PYG{l+m+mi}{0}\PYG{p}{]}\PYG{p}{)} \PYG{k}{else} \PYG{n}{np}\PYG{o}{.}\PYG{n}{zeros\PYGZus{}like}\PYG{p}{(}\PYG{n}{sc}\PYG{p}{)}

    \PYG{k}{for} \PYG{n}{i}\PYG{p}{,} \PYG{n}{c} \PYG{o+ow}{in} \PYG{n+nb}{enumerate}\PYG{p}{(}\PYG{n}{cnts}\PYG{p}{)}\PYG{p}{:}
        \PYG{n}{color} \PYG{o}{=} \PYG{p}{(}\PYG{l+m+mi}{0}\PYG{p}{,} \PYG{l+m+mi}{0}\PYG{p}{,} \PYG{l+m+mi}{255}\PYG{p}{)}\PYG{p}{;} \PYG{n}{thick} \PYG{o}{=} \PYG{l+m+mi}{2}
        \PYG{k}{if} \PYG{n}{best\PYGZus{}idx} \PYG{o+ow}{is} \PYG{o+ow}{not} \PYG{k+kc}{None} \PYG{o+ow}{and} \PYG{n}{i} \PYG{o}{==} \PYG{n}{best\PYGZus{}idx}\PYG{p}{:}
            \PYG{n}{color} \PYG{o}{=} \PYG{p}{(}\PYG{l+m+mi}{0}\PYG{p}{,} \PYG{l+m+mi}{255}\PYG{p}{,} \PYG{l+m+mi}{0}\PYG{p}{)}\PYG{p}{;} \PYG{n}{thick} \PYG{o}{=} \PYG{l+m+mi}{3}
        \PYG{n}{cv2}\PYG{o}{.}\PYG{n}{drawContours}\PYG{p}{(}\PYG{n}{vis}\PYG{p}{,} \PYG{p}{[}\PYG{n}{c}\PYG{p}{]}\PYG{p}{,} \PYG{o}{\PYGZhy{}}\PYG{l+m+mi}{1}\PYG{p}{,} \PYG{n}{color}\PYG{p}{,} \PYG{n}{thick}\PYG{p}{)}
        \PYG{n}{x}\PYG{p}{,} \PYG{n}{y}\PYG{p}{,} \PYG{n}{w}\PYG{p}{,} \PYG{n}{h} \PYG{o}{=} \PYG{n}{cv2}\PYG{o}{.}\PYG{n}{boundingRect}\PYG{p}{(}\PYG{n}{c}\PYG{p}{)}
        \PYG{n}{s} \PYG{o}{=} \PYG{n}{scores}\PYG{p}{[}\PYG{n}{i}\PYG{p}{]} \PYG{k}{if} \PYG{n}{i} \PYG{o}{\PYGZlt{}} \PYG{n+nb}{len}\PYG{p}{(}\PYG{n}{scores}\PYG{p}{)} \PYG{k}{else} \PYG{n+nb}{float}\PYG{p}{(}\PYG{l+s+s2}{\PYGZdq{}}\PYG{l+s+s2}{nan}\PYG{l+s+s2}{\PYGZdq{}}\PYG{p}{)}
        \PYG{n}{txt} \PYG{o}{=} \PYG{l+s+sa}{f}\PYG{l+s+s2}{\PYGZdq{}}\PYG{l+s+s2}{s=}\PYG{l+s+si}{\PYGZob{}}\PYG{n}{s}\PYG{l+s+si}{:}\PYG{l+s+s2}{.0f}\PYG{l+s+si}{\PYGZcb{}}\PYG{l+s+s2}{  }\PYG{l+s+si}{\PYGZob{}}\PYG{n}{w}\PYG{l+s+si}{\PYGZcb{}}\PYG{l+s+s2}{x}\PYG{l+s+si}{\PYGZob{}}\PYG{n}{h}\PYG{l+s+si}{\PYGZcb{}}\PYG{l+s+s2}{\PYGZdq{}}
        \PYG{n}{y\PYGZus{}txt} \PYG{o}{=} \PYG{n+nb}{max}\PYG{p}{(}\PYG{l+m+mi}{15}\PYG{p}{,} \PYG{n}{y} \PYG{o}{\PYGZhy{}} \PYG{l+m+mi}{5}\PYG{p}{)}
        \PYG{p}{(}\PYG{n}{tw}\PYG{p}{,} \PYG{n}{th}\PYG{p}{)}\PYG{p}{,} \PYG{n}{\PYGZus{}} \PYG{o}{=} \PYG{n}{cv2}\PYG{o}{.}\PYG{n}{getTextSize}\PYG{p}{(}\PYG{n}{txt}\PYG{p}{,} \PYG{n}{cv2}\PYG{o}{.}\PYG{n}{FONT\PYGZus{}HERSHEY\PYGZus{}SIMPLEX}\PYG{p}{,} \PYG{l+m+mf}{0.45}\PYG{p}{,} \PYG{l+m+mi}{1}\PYG{p}{)}
        \PYG{n}{cv2}\PYG{o}{.}\PYG{n}{rectangle}\PYG{p}{(}\PYG{n}{vis}\PYG{p}{,} \PYG{p}{(}\PYG{n}{x}\PYG{p}{,} \PYG{n}{y\PYGZus{}txt} \PYG{o}{\PYGZhy{}} \PYG{n}{th} \PYG{o}{\PYGZhy{}} \PYG{l+m+mi}{4}\PYG{p}{)}\PYG{p}{,} \PYG{p}{(}\PYG{n}{x} \PYG{o}{+} \PYG{n}{tw} \PYG{o}{+} \PYG{l+m+mi}{4}\PYG{p}{,} \PYG{n}{y\PYGZus{}txt} \PYG{o}{+} \PYG{l+m+mi}{2}\PYG{p}{)}\PYG{p}{,} \PYG{p}{(}\PYG{l+m+mi}{0}\PYG{p}{,} \PYG{l+m+mi}{0}\PYG{p}{,} \PYG{l+m+mi}{0}\PYG{p}{)}\PYG{p}{,} \PYG{o}{\PYGZhy{}}\PYG{l+m+mi}{1}\PYG{p}{)}
        \PYG{n}{cv2}\PYG{o}{.}\PYG{n}{putText}\PYG{p}{(}\PYG{n}{vis}\PYG{p}{,} \PYG{n}{txt}\PYG{p}{,} \PYG{p}{(}\PYG{n}{x} \PYG{o}{+} \PYG{l+m+mi}{2}\PYG{p}{,} \PYG{n}{y\PYGZus{}txt}\PYG{p}{)}\PYG{p}{,} \PYG{n}{cv2}\PYG{o}{.}\PYG{n}{FONT\PYGZus{}HERSHEY\PYGZus{}SIMPLEX}\PYG{p}{,} \PYG{l+m+mf}{0.45}\PYG{p}{,} \PYG{p}{(}\PYG{l+m+mi}{255}\PYG{p}{,} \PYG{l+m+mi}{255}\PYG{p}{,} \PYG{l+m+mi}{255}\PYG{p}{)}\PYG{p}{,} \PYG{l+m+mi}{1}\PYG{p}{,} \PYG{n}{cv2}\PYG{o}{.}\PYG{n}{LINE\PYGZus{}AA}\PYG{p}{)}
        \PYG{k}{if} \PYG{n}{sc}\PYG{o}{.}\PYG{n}{size} \PYG{o+ow}{and} \PYG{p}{(}\PYG{n}{best\PYGZus{}idx} \PYG{o+ow}{is} \PYG{k+kc}{None} \PYG{o+ow}{or} \PYG{n}{i} \PYG{o}{!=} \PYG{n}{best\PYGZus{}idx}\PYG{p}{)}\PYG{p}{:}
            \PYG{n}{alpha} \PYG{o}{=} \PYG{n+nb}{float}\PYG{p}{(}\PYG{l+m+mf}{0.35} \PYG{o}{+} \PYG{l+m+mf}{0.45} \PYG{o}{*} \PYG{n}{scn}\PYG{p}{[}\PYG{n}{i}\PYG{p}{]}\PYG{p}{)}
            \PYG{n}{cv2}\PYG{o}{.}\PYG{n}{drawContours}\PYG{p}{(}\PYG{n}{vis}\PYG{p}{,} \PYG{p}{[}\PYG{n}{c}\PYG{p}{]}\PYG{p}{,} \PYG{o}{\PYGZhy{}}\PYG{l+m+mi}{1}\PYG{p}{,} \PYG{n+nb}{tuple}\PYG{p}{(}\PYG{n+nb}{int}\PYG{p}{(}\PYG{n}{alpha} \PYG{o}{*} \PYG{n}{v}\PYG{p}{)} \PYG{k}{for} \PYG{n}{v} \PYG{o+ow}{in} \PYG{n}{color}\PYG{p}{)}\PYG{p}{,} \PYG{l+m+mi}{2}\PYG{p}{)}

    \PYG{k}{return} \PYG{n}{save\PYGZus{}img}\PYG{p}{(}\PYG{l+s+sa}{f}\PYG{l+s+s2}{\PYGZdq{}}\PYG{l+s+s2}{10\PYGZus{}candidates\PYGZus{}}\PYG{l+s+si}{\PYGZob{}}\PYG{n}{tag}\PYG{l+s+si}{\PYGZcb{}}\PYG{l+s+s2}{.png}\PYG{l+s+s2}{\PYGZdq{}}\PYG{p}{,} \PYG{n}{vis}\PYG{p}{)}

\PYG{k}{def}\PYG{+w}{ }\PYG{n+nf}{contour\PYGZus{}scores}\PYG{p}{(}\PYG{n}{cnts}\PYG{p}{,} \PYG{n}{roi}\PYG{p}{,} \PYG{n}{strict}\PYG{o}{=}\PYG{k+kc}{True}\PYG{p}{)}\PYG{p}{:}
    \PYG{n}{xR}\PYG{p}{,} \PYG{n}{yR}\PYG{p}{,} \PYG{n}{wR}\PYG{p}{,} \PYG{n}{hR} \PYG{o}{=} \PYG{n}{roi}
    \PYG{k}{if} \PYG{n}{strict}\PYG{p}{:}
        \PYG{n}{min\PYGZus{}area} \PYG{o}{=} \PYG{n}{CONFIG}\PYG{p}{[}\PYG{l+s+s2}{\PYGZdq{}}\PYG{l+s+s2}{min\PYGZus{}area\PYGZus{}px2}\PYG{l+s+s2}{\PYGZdq{}}\PYG{p}{]}
        \PYG{n}{min\PYGZus{}width} \PYG{o}{=} \PYG{n}{CONFIG}\PYG{p}{[}\PYG{l+s+s2}{\PYGZdq{}}\PYG{l+s+s2}{min\PYGZus{}width\PYGZus{}px}\PYG{l+s+s2}{\PYGZdq{}}\PYG{p}{]}
        \PYG{n}{edge\PYGZus{}clear} \PYG{o}{=} \PYG{n}{CONFIG}\PYG{p}{[}\PYG{l+s+s2}{\PYGZdq{}}\PYG{l+s+s2}{edge\PYGZus{}clear\PYGZus{}px}\PYG{l+s+s2}{\PYGZdq{}}\PYG{p}{]}
        \PYG{n}{mu} \PYG{o}{=} \PYG{n}{CONFIG}\PYG{p}{[}\PYG{l+s+s2}{\PYGZdq{}}\PYG{l+s+s2}{shape\PYGZus{}mu\PYGZus{}log\PYGZus{}aspect}\PYG{l+s+s2}{\PYGZdq{}}\PYG{p}{]}
        \PYG{n}{sg} \PYG{o}{=} \PYG{n}{CONFIG}\PYG{p}{[}\PYG{l+s+s2}{\PYGZdq{}}\PYG{l+s+s2}{shape\PYGZus{}sigma}\PYG{l+s+s2}{\PYGZdq{}}\PYG{p}{]}
        \PYG{n}{allow\PYGZus{}touch\PYGZus{}edges} \PYG{o}{=} \PYG{k+kc}{False}
    \PYG{k}{else}\PYG{p}{:}
        \PYG{n}{min\PYGZus{}area} \PYG{o}{=} \PYG{n}{CONFIG}\PYG{p}{[}\PYG{l+s+s2}{\PYGZdq{}}\PYG{l+s+s2}{relax}\PYG{l+s+s2}{\PYGZdq{}}\PYG{p}{]}\PYG{p}{[}\PYG{l+s+s2}{\PYGZdq{}}\PYG{l+s+s2}{min\PYGZus{}area\PYGZus{}px2}\PYG{l+s+s2}{\PYGZdq{}}\PYG{p}{]}
        \PYG{n}{min\PYGZus{}width} \PYG{o}{=} \PYG{n}{CONFIG}\PYG{p}{[}\PYG{l+s+s2}{\PYGZdq{}}\PYG{l+s+s2}{relax}\PYG{l+s+s2}{\PYGZdq{}}\PYG{p}{]}\PYG{p}{[}\PYG{l+s+s2}{\PYGZdq{}}\PYG{l+s+s2}{min\PYGZus{}width\PYGZus{}px}\PYG{l+s+s2}{\PYGZdq{}}\PYG{p}{]}
        \PYG{n}{edge\PYGZus{}clear} \PYG{o}{=} \PYG{n}{CONFIG}\PYG{p}{[}\PYG{l+s+s2}{\PYGZdq{}}\PYG{l+s+s2}{relax}\PYG{l+s+s2}{\PYGZdq{}}\PYG{p}{]}\PYG{p}{[}\PYG{l+s+s2}{\PYGZdq{}}\PYG{l+s+s2}{edge\PYGZus{}clear\PYGZus{}px}\PYG{l+s+s2}{\PYGZdq{}}\PYG{p}{]}
        \PYG{n}{mu} \PYG{o}{=} \PYG{n}{CONFIG}\PYG{p}{[}\PYG{l+s+s2}{\PYGZdq{}}\PYG{l+s+s2}{shape\PYGZus{}mu\PYGZus{}log\PYGZus{}aspect}\PYG{l+s+s2}{\PYGZdq{}}\PYG{p}{]}
        \PYG{n}{sg} \PYG{o}{=} \PYG{n}{CONFIG}\PYG{p}{[}\PYG{l+s+s2}{\PYGZdq{}}\PYG{l+s+s2}{relax}\PYG{l+s+s2}{\PYGZdq{}}\PYG{p}{]}\PYG{p}{[}\PYG{l+s+s2}{\PYGZdq{}}\PYG{l+s+s2}{shape\PYGZus{}sigma}\PYG{l+s+s2}{\PYGZdq{}}\PYG{p}{]}
        \PYG{n}{allow\PYGZus{}touch\PYGZus{}edges} \PYG{o}{=} \PYG{n}{CONFIG}\PYG{p}{[}\PYG{l+s+s2}{\PYGZdq{}}\PYG{l+s+s2}{relax}\PYG{l+s+s2}{\PYGZdq{}}\PYG{p}{]}\PYG{p}{[}\PYG{l+s+s2}{\PYGZdq{}}\PYG{l+s+s2}{allow\PYGZus{}touch\PYGZus{}edges}\PYG{l+s+s2}{\PYGZdq{}}\PYG{p}{]}

    \PYG{k}{def}\PYG{+w}{ }\PYG{n+nf}{score}\PYG{p}{(}\PYG{n}{c}\PYG{p}{)}\PYG{p}{:}
        \PYG{n}{A} \PYG{o}{=} \PYG{n}{cv2}\PYG{o}{.}\PYG{n}{contourArea}\PYG{p}{(}\PYG{n}{c}\PYG{p}{)}
        \PYG{k}{if} \PYG{n}{A} \PYG{o}{\PYGZlt{}} \PYG{n}{min\PYGZus{}area}\PYG{p}{:}
            \PYG{k}{return} \PYG{o}{\PYGZhy{}}\PYG{n}{np}\PYG{o}{.}\PYG{n}{inf}
        \PYG{n}{x0}\PYG{p}{,} \PYG{n}{y0}\PYG{p}{,} \PYG{n}{w0}\PYG{p}{,} \PYG{n}{h0} \PYG{o}{=} \PYG{n}{cv2}\PYG{o}{.}\PYG{n}{boundingRect}\PYG{p}{(}\PYG{n}{c}\PYG{p}{)}
        \PYG{k}{if} \PYG{n+nb}{min}\PYG{p}{(}\PYG{n}{w0}\PYG{p}{,} \PYG{n}{h0}\PYG{p}{)} \PYG{o}{\PYGZlt{}} \PYG{n}{min\PYGZus{}width}\PYG{p}{:}
            \PYG{k}{return} \PYG{o}{\PYGZhy{}}\PYG{n}{np}\PYG{o}{.}\PYG{n}{inf}
        \PYG{k}{if} \PYG{o+ow}{not} \PYG{n}{allow\PYGZus{}touch\PYGZus{}edges}\PYG{p}{:}
            \PYG{k}{if} \PYG{p}{(}\PYG{n}{xR} \PYG{o}{+} \PYG{n}{x0}\PYG{p}{)} \PYG{o}{\PYGZlt{}} \PYG{p}{(}\PYG{n}{xR} \PYG{o}{+} \PYG{n}{edge\PYGZus{}clear}\PYG{p}{)} \PYG{o+ow}{or} \PYG{p}{(}\PYG{n}{xR} \PYG{o}{+} \PYG{n}{x0} \PYG{o}{+} \PYG{n}{w0}\PYG{p}{)} \PYG{o}{\PYGZgt{}} \PYG{p}{(}\PYG{n}{xR} \PYG{o}{+} \PYG{n}{wR} \PYG{o}{\PYGZhy{}} \PYG{n}{edge\PYGZus{}clear}\PYG{p}{)}\PYG{p}{:}
                \PYG{k}{return} \PYG{o}{\PYGZhy{}}\PYG{n}{np}\PYG{o}{.}\PYG{n}{inf}
        \PYG{n}{a} \PYG{o}{=} \PYG{n+nb}{max}\PYG{p}{(}\PYG{n}{w0}\PYG{p}{,} \PYG{n}{h0}\PYG{p}{)} \PYG{o}{/} \PYG{n+nb}{max}\PYG{p}{(}\PYG{l+m+mf}{1.0}\PYG{p}{,} \PYG{n+nb}{min}\PYG{p}{(}\PYG{n}{w0}\PYG{p}{,} \PYG{n}{h0}\PYG{p}{)}\PYG{p}{)}
        \PYG{n}{w\PYGZus{}aspect} \PYG{o}{=} \PYG{n}{np}\PYG{o}{.}\PYG{n}{exp}\PYG{p}{(}\PYG{o}{\PYGZhy{}}\PYG{p}{(}\PYG{p}{(}\PYG{n}{np}\PYG{o}{.}\PYG{n}{log}\PYG{p}{(}\PYG{n}{a}\PYG{p}{)} \PYG{o}{\PYGZhy{}} \PYG{n}{mu}\PYG{p}{)} \PYG{o}{*}\PYG{o}{*} \PYG{l+m+mi}{2}\PYG{p}{)} \PYG{o}{/} \PYG{p}{(}\PYG{l+m+mf}{2.0} \PYG{o}{*} \PYG{p}{(}\PYG{n}{sg} \PYG{o}{*}\PYG{o}{*} \PYG{l+m+mi}{2}\PYG{p}{)}\PYG{p}{)}\PYG{p}{)}
        \PYG{n}{hull} \PYG{o}{=} \PYG{n}{cv2}\PYG{o}{.}\PYG{n}{convexHull}\PYG{p}{(}\PYG{n}{c}\PYG{p}{)}
        \PYG{n}{Ah} \PYG{o}{=} \PYG{n}{cv2}\PYG{o}{.}\PYG{n}{contourArea}\PYG{p}{(}\PYG{n}{hull}\PYG{p}{)}
        \PYG{n}{solidity} \PYG{o}{=} \PYG{p}{(}\PYG{n}{A} \PYG{o}{/} \PYG{n}{Ah}\PYG{p}{)} \PYG{k}{if} \PYG{n}{Ah} \PYG{o}{\PYGZgt{}} \PYG{l+m+mi}{0} \PYG{k}{else} \PYG{l+m+mf}{0.0}
        \PYG{n}{w\PYGZus{}sol} \PYG{o}{=} \PYG{n}{np}\PYG{o}{.}\PYG{n}{clip}\PYG{p}{(}\PYG{n}{solidity}\PYG{p}{,} \PYG{l+m+mf}{0.0}\PYG{p}{,} \PYG{l+m+mf}{1.0}\PYG{p}{)}
        \PYG{k}{return} \PYG{n+nb}{float}\PYG{p}{(}\PYG{n}{A} \PYG{o}{*} \PYG{n}{w\PYGZus{}aspect} \PYG{o}{*} \PYG{p}{(}\PYG{l+m+mf}{0.6} \PYG{o}{+} \PYG{l+m+mf}{0.4} \PYG{o}{*} \PYG{n}{w\PYGZus{}sol}\PYG{p}{)}\PYG{p}{)}

    \PYG{k}{return} \PYG{p}{[}\PYG{n}{score}\PYG{p}{(}\PYG{n}{c}\PYG{p}{)} \PYG{k}{for} \PYG{n}{c} \PYG{o+ow}{in} \PYG{n}{cnts}\PYG{p}{]}

\PYG{c+c1}{\PYGZsh{} \PYGZhy{}\PYGZhy{}\PYGZhy{}\PYGZhy{}\PYGZhy{}\PYGZhy{}\PYGZhy{}\PYGZhy{}\PYGZhy{}\PYGZhy{}\PYGZhy{}\PYGZhy{}\PYGZhy{}\PYGZhy{}\PYGZhy{}\PYGZhy{}\PYGZhy{}\PYGZhy{}\PYGZhy{}\PYGZhy{}\PYGZhy{}\PYGZhy{}\PYGZhy{}\PYGZhy{}\PYGZhy{}}
\PYG{c+c1}{\PYGZsh{} Run}
\PYG{c+c1}{\PYGZsh{} \PYGZhy{}\PYGZhy{}\PYGZhy{}\PYGZhy{}\PYGZhy{}\PYGZhy{}\PYGZhy{}\PYGZhy{}\PYGZhy{}\PYGZhy{}\PYGZhy{}\PYGZhy{}\PYGZhy{}\PYGZhy{}\PYGZhy{}\PYGZhy{}\PYGZhy{}\PYGZhy{}\PYGZhy{}\PYGZhy{}\PYGZhy{}\PYGZhy{}\PYGZhy{}\PYGZhy{}\PYGZhy{}}
\PYG{k}{def}\PYG{+w}{ }\PYG{n+nf}{extractor}\PYG{p}{(}\PYG{p}{)}\PYG{p}{:}
    \PYG{n}{IMG\PYGZus{}PATH} \PYG{o}{=} \PYG{n}{CONFIG}\PYG{p}{[}\PYG{l+s+s2}{\PYGZdq{}}\PYG{l+s+s2}{img\PYGZus{}path}\PYG{l+s+s2}{\PYGZdq{}}\PYG{p}{]}
    \PYG{n}{bgr0} \PYG{o}{=} \PYG{n}{cv2}\PYG{o}{.}\PYG{n}{imread}\PYG{p}{(}\PYG{n}{IMG\PYGZus{}PATH}\PYG{p}{,} \PYG{n}{cv2}\PYG{o}{.}\PYG{n}{IMREAD\PYGZus{}COLOR}\PYG{p}{)}
    \PYG{k}{if} \PYG{n}{bgr0} \PYG{o+ow}{is} \PYG{k+kc}{None}\PYG{p}{:}
        \PYG{k}{raise} \PYG{n+ne}{FileNotFoundError}\PYG{p}{(}\PYG{l+s+sa}{f}\PYG{l+s+s2}{\PYGZdq{}}\PYG{l+s+s2}{No se pudo leer la imagen: }\PYG{l+s+si}{\PYGZob{}}\PYG{n}{IMG\PYGZus{}PATH}\PYG{l+s+si}{\PYGZcb{}}\PYG{l+s+s2}{\PYGZdq{}}\PYG{p}{)}

    \PYG{c+c1}{\PYGZsh{} 1) Prefilter}
    \PYG{n}{bgr} \PYG{o}{=} \PYG{n}{prefilter}\PYG{p}{(}\PYG{n}{bgr0}\PYG{p}{)}

    \PYG{c+c1}{\PYGZsh{} 2) Dynamic blue \PYGZam{} ROI}
    \PYG{n}{hsv} \PYG{o}{=} \PYG{n}{cv2}\PYG{o}{.}\PYG{n}{cvtColor}\PYG{p}{(}\PYG{n}{bgr}\PYG{p}{,} \PYG{n}{cv2}\PYG{o}{.}\PYG{n}{COLOR\PYGZus{}BGR2HSV}\PYG{p}{)}
    \PYG{n}{lower}\PYG{p}{,} \PYG{n}{upper} \PYG{o}{=} \PYG{n}{dynamic\PYGZus{}blue\PYGZus{}threshold}\PYG{p}{(}\PYG{n}{hsv}\PYG{p}{)}
    \PYG{n}{roi}\PYG{p}{,} \PYG{n}{\PYGZus{}} \PYG{o}{=} \PYG{n}{belt\PYGZus{}roi\PYGZus{}from\PYGZus{}blue}\PYG{p}{(}\PYG{n}{hsv}\PYG{p}{,} \PYG{n}{lower}\PYG{p}{,} \PYG{n}{upper}\PYG{p}{)}
    \PYG{n}{x}\PYG{p}{,} \PYG{n}{y}\PYG{p}{,} \PYG{n}{w}\PYG{p}{,} \PYG{n}{h} \PYG{o}{=} \PYG{n}{roi}
    \PYG{n}{crop} \PYG{o}{=} \PYG{n}{bgr}\PYG{p}{[}\PYG{n}{y}\PYG{p}{:}\PYG{n}{y} \PYG{o}{+} \PYG{n}{h}\PYG{p}{,} \PYG{n}{x}\PYG{p}{:}\PYG{n}{x} \PYG{o}{+} \PYG{n}{w}\PYG{p}{]}\PYG{o}{.}\PYG{n}{copy}\PYG{p}{(}\PYG{p}{)}

    \PYG{c+c1}{\PYGZsh{} 3) GrabCut seeds + run}
    \PYG{n}{gc\PYGZus{}mask} \PYG{o}{=} \PYG{n}{seeds\PYGZus{}in\PYGZus{}roi}\PYG{p}{(}\PYG{n}{crop}\PYG{p}{,} \PYG{n}{lower}\PYG{p}{,} \PYG{n}{upper}\PYG{p}{)}
    \PYG{n}{bg\PYGZus{}model} \PYG{o}{=} \PYG{n}{np}\PYG{o}{.}\PYG{n}{zeros}\PYG{p}{(}\PYG{p}{(}\PYG{l+m+mi}{1}\PYG{p}{,} \PYG{l+m+mi}{65}\PYG{p}{)}\PYG{p}{,} \PYG{n}{np}\PYG{o}{.}\PYG{n}{float64}\PYG{p}{)}
    \PYG{n}{fg\PYGZus{}model} \PYG{o}{=} \PYG{n}{np}\PYG{o}{.}\PYG{n}{zeros}\PYG{p}{(}\PYG{p}{(}\PYG{l+m+mi}{1}\PYG{p}{,} \PYG{l+m+mi}{65}\PYG{p}{)}\PYG{p}{,} \PYG{n}{np}\PYG{o}{.}\PYG{n}{float64}\PYG{p}{)}
    \PYG{n}{cv2}\PYG{o}{.}\PYG{n}{grabCut}\PYG{p}{(}\PYG{n}{crop}\PYG{p}{,} \PYG{n}{gc\PYGZus{}mask}\PYG{p}{,} \PYG{k+kc}{None}\PYG{p}{,} \PYG{n}{bg\PYGZus{}model}\PYG{p}{,} \PYG{n}{fg\PYGZus{}model}\PYG{p}{,} \PYG{n}{CONFIG}\PYG{p}{[}\PYG{l+s+s2}{\PYGZdq{}}\PYG{l+s+s2}{gc\PYGZus{}iters}\PYG{l+s+s2}{\PYGZdq{}}\PYG{p}{]}\PYG{p}{,} \PYG{n}{cv2}\PYG{o}{.}\PYG{n}{GC\PYGZus{}INIT\PYGZus{}WITH\PYGZus{}MASK}\PYG{p}{)}

    \PYG{c+c1}{\PYGZsh{} 4) Post\PYGZhy{}process mask}
    \PYG{n}{mask\PYGZus{}roi} \PYG{o}{=} \PYG{n}{np}\PYG{o}{.}\PYG{n}{where}\PYG{p}{(}\PYG{p}{(}\PYG{n}{gc\PYGZus{}mask} \PYG{o}{==} \PYG{n}{cv2}\PYG{o}{.}\PYG{n}{GC\PYGZus{}FGD}\PYG{p}{)} \PYG{o}{|} \PYG{p}{(}\PYG{n}{gc\PYGZus{}mask} \PYG{o}{==} \PYG{n}{cv2}\PYG{o}{.}\PYG{n}{GC\PYGZus{}PR\PYGZus{}FGD}\PYG{p}{)}\PYG{p}{,} \PYG{l+m+mi}{1}\PYG{p}{,} \PYG{l+m+mi}{0}\PYG{p}{)}\PYG{o}{.}\PYG{n}{astype}\PYG{p}{(}\PYG{n}{np}\PYG{o}{.}\PYG{n}{uint8}\PYG{p}{)}
    \PYG{n}{save\PYGZus{}img\PYGZus{}dbg}\PYG{p}{(}\PYG{l+s+s2}{\PYGZdq{}}\PYG{l+s+s2}{09\PYGZus{}mask\PYGZus{}roi\PYGZus{}raw.png}\PYG{l+s+s2}{\PYGZdq{}}\PYG{p}{,} \PYG{p}{(}\PYG{n}{mask\PYGZus{}roi} \PYG{o}{*} \PYG{l+m+mi}{255}\PYG{p}{)}\PYG{p}{)}
    \PYG{n}{mask\PYGZus{}roi} \PYG{o}{=} \PYG{n}{\PYGZus{}close}\PYG{p}{(}\PYG{n}{mask\PYGZus{}roi}\PYG{p}{,} \PYG{n}{cv2}\PYG{o}{.}\PYG{n}{MORPH\PYGZus{}ELLIPSE}\PYG{p}{,} \PYG{n}{CONFIG}\PYG{p}{[}\PYG{l+s+s2}{\PYGZdq{}}\PYG{l+s+s2}{post\PYGZus{}close\PYGZus{}kernel}\PYG{l+s+s2}{\PYGZdq{}}\PYG{p}{]}\PYG{p}{,} \PYG{l+m+mi}{1}\PYG{p}{)}
    \PYG{n}{mask\PYGZus{}roi} \PYG{o}{=} \PYG{n}{\PYGZus{}open}\PYG{p}{(}\PYG{n}{mask\PYGZus{}roi}\PYG{p}{,}  \PYG{n}{cv2}\PYG{o}{.}\PYG{n}{MORPH\PYGZus{}ELLIPSE}\PYG{p}{,} \PYG{p}{(}\PYG{l+m+mi}{5}\PYG{p}{,} \PYG{l+m+mi}{5}\PYG{p}{)}\PYG{p}{,} \PYG{l+m+mi}{1}\PYG{p}{)}

    \PYG{n}{full} \PYG{o}{=} \PYG{n}{np}\PYG{o}{.}\PYG{n}{zeros}\PYG{p}{(}\PYG{n}{bgr}\PYG{o}{.}\PYG{n}{shape}\PYG{p}{[}\PYG{p}{:}\PYG{l+m+mi}{2}\PYG{p}{]}\PYG{p}{,} \PYG{n}{np}\PYG{o}{.}\PYG{n}{uint8}\PYG{p}{)}
    \PYG{n}{full}\PYG{p}{[}\PYG{n}{y}\PYG{p}{:}\PYG{n}{y} \PYG{o}{+} \PYG{n}{h}\PYG{p}{,} \PYG{n}{x}\PYG{p}{:}\PYG{n}{x} \PYG{o}{+} \PYG{n}{w}\PYG{p}{]} \PYG{o}{=} \PYG{n}{mask\PYGZus{}roi}

    \PYG{c+c1}{\PYGZsh{} 5) Contours \PYGZam{} scoring (strict → relaxed fallback)}
    \PYG{n}{cnts}\PYG{p}{,} \PYG{n}{\PYGZus{}} \PYG{o}{=} \PYG{n}{cv2}\PYG{o}{.}\PYG{n}{findContours}\PYG{p}{(}\PYG{p}{(}\PYG{n}{full} \PYG{o}{*} \PYG{l+m+mi}{255}\PYG{p}{)}\PYG{o}{.}\PYG{n}{astype}\PYG{p}{(}\PYG{n}{np}\PYG{o}{.}\PYG{n}{uint8}\PYG{p}{)}\PYG{p}{,} \PYG{n}{cv2}\PYG{o}{.}\PYG{n}{RETR\PYGZus{}EXTERNAL}\PYG{p}{,} \PYG{n}{cv2}\PYG{o}{.}\PYG{n}{CHAIN\PYGZus{}APPROX\PYGZus{}NONE}\PYG{p}{)}
    \PYG{n}{scores\PYGZus{}strict} \PYG{o}{=} \PYG{n}{contour\PYGZus{}scores}\PYG{p}{(}\PYG{n}{cnts}\PYG{p}{,} \PYG{n}{roi}\PYG{p}{,} \PYG{n}{strict}\PYG{o}{=}\PYG{k+kc}{True}\PYG{p}{)} \PYG{k}{if} \PYG{n}{cnts} \PYG{k}{else} \PYG{p}{[}\PYG{p}{]}
    \PYG{n}{best\PYGZus{}idx} \PYG{o}{=} \PYG{n+nb}{int}\PYG{p}{(}\PYG{n}{np}\PYG{o}{.}\PYG{n}{argmax}\PYG{p}{(}\PYG{n}{scores\PYGZus{}strict}\PYG{p}{)}\PYG{p}{)} \PYG{k}{if} \PYG{n}{scores\PYGZus{}strict} \PYG{k}{else} \PYG{k+kc}{None}
    \PYG{n}{strict\PYGZus{}ok} \PYG{o}{=} \PYG{n+nb}{bool}\PYG{p}{(}\PYG{n}{scores\PYGZus{}strict}\PYG{p}{)} \PYG{o+ow}{and} \PYG{n}{np}\PYG{o}{.}\PYG{n}{isfinite}\PYG{p}{(}\PYG{n}{scores\PYGZus{}strict}\PYG{p}{[}\PYG{n}{best\PYGZus{}idx}\PYG{p}{]}\PYG{p}{)} \PYG{o+ow}{and} \PYG{p}{(}\PYG{n}{scores\PYGZus{}strict}\PYG{p}{[}\PYG{n}{best\PYGZus{}idx}\PYG{p}{]} \PYG{o}{\PYGZgt{}} \PYG{l+m+mi}{0}\PYG{p}{)}
    \PYG{n}{\PYGZus{}} \PYG{o}{=} \PYG{n}{render\PYGZus{}candidates\PYGZus{}debug}\PYG{p}{(}\PYG{n}{bgr0}\PYG{p}{,} \PYG{n}{cnts}\PYG{p}{,} \PYG{n}{scores\PYGZus{}strict}\PYG{p}{,} \PYG{n}{best\PYGZus{}idx} \PYG{k}{if} \PYG{n}{strict\PYGZus{}ok} \PYG{k}{else} \PYG{k+kc}{None}\PYG{p}{,} \PYG{n}{roi}\PYG{p}{,} \PYG{n}{tag}\PYG{o}{=}\PYG{l+s+s2}{\PYGZdq{}}\PYG{l+s+s2}{strict}\PYG{l+s+s2}{\PYGZdq{}}\PYG{p}{)}

    \PYG{k}{if} \PYG{o+ow}{not} \PYG{n}{strict\PYGZus{}ok}\PYG{p}{:}
        \PYG{n}{scores\PYGZus{}relax} \PYG{o}{=} \PYG{n}{contour\PYGZus{}scores}\PYG{p}{(}\PYG{n}{cnts}\PYG{p}{,} \PYG{n}{roi}\PYG{p}{,} \PYG{n}{strict}\PYG{o}{=}\PYG{k+kc}{False}\PYG{p}{)} \PYG{k}{if} \PYG{n}{cnts} \PYG{k}{else} \PYG{p}{[}\PYG{p}{]}
        \PYG{n}{best\PYGZus{}idx\PYGZus{}relax} \PYG{o}{=} \PYG{n+nb}{int}\PYG{p}{(}\PYG{n}{np}\PYG{o}{.}\PYG{n}{argmax}\PYG{p}{(}\PYG{n}{scores\PYGZus{}relax}\PYG{p}{)}\PYG{p}{)} \PYG{k}{if} \PYG{n}{scores\PYGZus{}relax} \PYG{k}{else} \PYG{k+kc}{None}
        \PYG{n}{relaxed\PYGZus{}ok} \PYG{o}{=} \PYG{n+nb}{bool}\PYG{p}{(}\PYG{n}{scores\PYGZus{}relax}\PYG{p}{)} \PYG{o+ow}{and} \PYG{n}{np}\PYG{o}{.}\PYG{n}{isfinite}\PYG{p}{(}\PYG{n}{scores\PYGZus{}relax}\PYG{p}{[}\PYG{n}{best\PYGZus{}idx\PYGZus{}relax}\PYG{p}{]}\PYG{p}{)} \PYG{o+ow}{and} \PYG{p}{(}\PYG{n}{scores\PYGZus{}relax}\PYG{p}{[}\PYG{n}{best\PYGZus{}idx\PYGZus{}relax}\PYG{p}{]} \PYG{o}{\PYGZgt{}} \PYG{l+m+mi}{0}\PYG{p}{)}
        \PYG{n}{\PYGZus{}} \PYG{o}{=} \PYG{n}{render\PYGZus{}candidates\PYGZus{}debug}\PYG{p}{(}\PYG{n}{bgr0}\PYG{p}{,} \PYG{n}{cnts}\PYG{p}{,} \PYG{n}{scores\PYGZus{}relax}\PYG{p}{,} \PYG{n}{best\PYGZus{}idx\PYGZus{}relax} \PYG{k}{if} \PYG{n}{relaxed\PYGZus{}ok} \PYG{k}{else} \PYG{k+kc}{None}\PYG{p}{,} \PYG{n}{roi}\PYG{p}{,} \PYG{n}{tag}\PYG{o}{=}\PYG{l+s+s2}{\PYGZdq{}}\PYG{l+s+s2}{relaxed}\PYG{l+s+s2}{\PYGZdq{}}\PYG{p}{)}
        \PYG{k}{if} \PYG{o+ow}{not} \PYG{n}{relaxed\PYGZus{}ok}\PYG{p}{:}
            \PYG{k}{if} \PYG{o+ow}{not} \PYG{n}{cnts}\PYG{p}{:}
                \PYG{k}{raise} \PYG{n+ne}{RuntimeError}\PYG{p}{(}\PYG{l+s+s2}{\PYGZdq{}}\PYG{l+s+s2}{No contornos tras segmentación.}\PYG{l+s+s2}{\PYGZdq{}}\PYG{p}{)}
            \PYG{c+c1}{\PYGZsh{} last\PYGZhy{}resort: largest area}
            \PYG{n}{areas} \PYG{o}{=} \PYG{p}{[}\PYG{n}{cv2}\PYG{o}{.}\PYG{n}{contourArea}\PYG{p}{(}\PYG{n}{c}\PYG{p}{)} \PYG{k}{for} \PYG{n}{c} \PYG{o+ow}{in} \PYG{n}{cnts}\PYG{p}{]}
            \PYG{n}{use\PYGZus{}idx} \PYG{o}{=} \PYG{n+nb}{int}\PYG{p}{(}\PYG{n}{np}\PYG{o}{.}\PYG{n}{argmax}\PYG{p}{(}\PYG{n}{areas}\PYG{p}{)}\PYG{p}{)}
        \PYG{k}{else}\PYG{p}{:}
            \PYG{n}{use\PYGZus{}idx} \PYG{o}{=} \PYG{n}{best\PYGZus{}idx\PYGZus{}relax}
        \PYG{n}{used\PYGZus{}scores} \PYG{o}{=} \PYG{n}{scores\PYGZus{}relax}
    \PYG{k}{else}\PYG{p}{:}
        \PYG{n}{use\PYGZus{}idx} \PYG{o}{=} \PYG{n}{best\PYGZus{}idx}
        \PYG{n}{used\PYGZus{}scores} \PYG{o}{=} \PYG{n}{scores\PYGZus{}strict}

    \PYG{n}{cnt} \PYG{o}{=} \PYG{n}{cnts}\PYG{p}{[}\PYG{n}{use\PYGZus{}idx}\PYG{p}{]}

    \PYG{c+c1}{\PYGZsh{} 6) PCA measures}
    \PYG{n}{pts} \PYG{o}{=} \PYG{n}{cnt}\PYG{o}{.}\PYG{n}{reshape}\PYG{p}{(}\PYG{o}{\PYGZhy{}}\PYG{l+m+mi}{1}\PYG{p}{,} \PYG{l+m+mi}{2}\PYG{p}{)}\PYG{o}{.}\PYG{n}{astype}\PYG{p}{(}\PYG{n}{np}\PYG{o}{.}\PYG{n}{float32}\PYG{p}{)}
    \PYG{n}{mean}\PYG{p}{,} \PYG{n}{eigvecs}\PYG{p}{,} \PYG{n}{\PYGZus{}} \PYG{o}{=} \PYG{n}{cv2}\PYG{o}{.}\PYG{n}{PCACompute2}\PYG{p}{(}\PYG{n}{pts}\PYG{p}{,} \PYG{n}{mean}\PYG{o}{=}\PYG{k+kc}{None}\PYG{p}{)}
    \PYG{n}{mean} \PYG{o}{=} \PYG{n}{mean}\PYG{o}{.}\PYG{n}{flatten}\PYG{p}{(}\PYG{p}{)}
    \PYG{n}{R} \PYG{o}{=} \PYG{n}{np}\PYG{o}{.}\PYG{n}{asarray}\PYG{p}{(}\PYG{n}{eigvecs}\PYG{p}{,} \PYG{n}{np}\PYG{o}{.}\PYG{n}{float32}\PYG{p}{)}
    \PYG{n}{pts\PYGZus{}c} \PYG{o}{=} \PYG{n}{pts} \PYG{o}{\PYGZhy{}} \PYG{n}{mean}
    \PYG{n}{pts\PYGZus{}r} \PYG{o}{=} \PYG{n}{pts\PYGZus{}c} \PYG{o}{@} \PYG{n}{R}\PYG{o}{.}\PYG{n}{T}
    \PYG{n}{min\PYGZus{}x}\PYG{p}{,} \PYG{n}{max\PYGZus{}x} \PYG{o}{=} \PYG{n}{np}\PYG{o}{.}\PYG{n}{min}\PYG{p}{(}\PYG{n}{pts\PYGZus{}r}\PYG{p}{[}\PYG{p}{:}\PYG{p}{,} \PYG{l+m+mi}{0}\PYG{p}{]}\PYG{p}{)}\PYG{p}{,} \PYG{n}{np}\PYG{o}{.}\PYG{n}{max}\PYG{p}{(}\PYG{n}{pts\PYGZus{}r}\PYG{p}{[}\PYG{p}{:}\PYG{p}{,} \PYG{l+m+mi}{0}\PYG{p}{]}\PYG{p}{)}
    \PYG{n}{min\PYGZus{}y}\PYG{p}{,} \PYG{n}{max\PYGZus{}y} \PYG{o}{=} \PYG{n}{np}\PYG{o}{.}\PYG{n}{min}\PYG{p}{(}\PYG{n}{pts\PYGZus{}r}\PYG{p}{[}\PYG{p}{:}\PYG{p}{,} \PYG{l+m+mi}{1}\PYG{p}{]}\PYG{p}{)}\PYG{p}{,} \PYG{n}{np}\PYG{o}{.}\PYG{n}{max}\PYG{p}{(}\PYG{n}{pts\PYGZus{}r}\PYG{p}{[}\PYG{p}{:}\PYG{p}{,} \PYG{l+m+mi}{1}\PYG{p}{]}\PYG{p}{)}
    \PYG{n}{length\PYGZus{}px} \PYG{o}{=} \PYG{n+nb}{float}\PYG{p}{(}\PYG{n}{max\PYGZus{}x} \PYG{o}{\PYGZhy{}} \PYG{n}{min\PYGZus{}x}\PYG{p}{)}
    \PYG{n}{width\PYGZus{}px}  \PYG{o}{=} \PYG{n+nb}{float}\PYG{p}{(}\PYG{n}{max\PYGZus{}y} \PYG{o}{\PYGZhy{}} \PYG{n}{min\PYGZus{}y}\PYG{p}{)}
    \PYG{n}{area\PYGZus{}px2}  \PYG{o}{=} \PYG{n+nb}{float}\PYG{p}{(}\PYG{n}{cv2}\PYG{o}{.}\PYG{n}{contourArea}\PYG{p}{(}\PYG{n}{cnt}\PYG{p}{)}\PYG{p}{)}

    \PYG{c+c1}{\PYGZsh{} 7) Overlay}
    \PYG{n}{rgb} \PYG{o}{=} \PYG{n}{cv2}\PYG{o}{.}\PYG{n}{cvtColor}\PYG{p}{(}\PYG{n}{bgr0}\PYG{p}{,} \PYG{n}{cv2}\PYG{o}{.}\PYG{n}{COLOR\PYGZus{}BGR2RGB}\PYG{p}{)}
    \PYG{n}{overlay} \PYG{o}{=} \PYG{n}{rgb}\PYG{o}{.}\PYG{n}{copy}\PYG{p}{(}\PYG{p}{)}
    \PYG{n}{box\PYGZus{}r} \PYG{o}{=} \PYG{n}{np}\PYG{o}{.}\PYG{n}{array}\PYG{p}{(}\PYG{p}{[}\PYG{p}{[}\PYG{n}{min\PYGZus{}x}\PYG{p}{,} \PYG{n}{min\PYGZus{}y}\PYG{p}{]}\PYG{p}{,} \PYG{p}{[}\PYG{n}{max\PYGZus{}x}\PYG{p}{,} \PYG{n}{min\PYGZus{}y}\PYG{p}{]}\PYG{p}{,} \PYG{p}{[}\PYG{n}{max\PYGZus{}x}\PYG{p}{,} \PYG{n}{max\PYGZus{}y}\PYG{p}{]}\PYG{p}{,} \PYG{p}{[}\PYG{n}{min\PYGZus{}x}\PYG{p}{,} \PYG{n}{max\PYGZus{}y}\PYG{p}{]}\PYG{p}{]}\PYG{p}{,} \PYG{n}{np}\PYG{o}{.}\PYG{n}{float32}\PYG{p}{)}
    \PYG{n}{box\PYGZus{}img} \PYG{o}{=} \PYG{p}{(}\PYG{n}{box\PYGZus{}r} \PYG{o}{@} \PYG{n}{R}\PYG{p}{)} \PYG{o}{+} \PYG{n}{mean}
    \PYG{n}{box\PYGZus{}img} \PYG{o}{=} \PYG{n}{np}\PYG{o}{.}\PYG{n}{int32}\PYG{p}{(}\PYG{n}{box\PYGZus{}img}\PYG{p}{)}
    \PYG{n}{cv2}\PYG{o}{.}\PYG{n}{drawContours}\PYG{p}{(}\PYG{n}{overlay}\PYG{p}{,} \PYG{p}{[}\PYG{n}{cnt}\PYG{p}{]}\PYG{p}{,} \PYG{o}{\PYGZhy{}}\PYG{l+m+mi}{1}\PYG{p}{,} \PYG{p}{(}\PYG{l+m+mi}{0}\PYG{p}{,} \PYG{l+m+mi}{255}\PYG{p}{,} \PYG{l+m+mi}{0}\PYG{p}{)}\PYG{p}{,} \PYG{l+m+mi}{3}\PYG{p}{)}
    \PYG{n}{cv2}\PYG{o}{.}\PYG{n}{polylines}\PYG{p}{(}\PYG{n}{overlay}\PYG{p}{,} \PYG{p}{[}\PYG{n}{box\PYGZus{}img}\PYG{p}{]}\PYG{p}{,} \PYG{k+kc}{True}\PYG{p}{,} \PYG{p}{(}\PYG{l+m+mi}{255}\PYG{p}{,} \PYG{l+m+mi}{255}\PYG{p}{,} \PYG{l+m+mi}{0}\PYG{p}{)}\PYG{p}{,} \PYG{l+m+mi}{2}\PYG{p}{)}
    \PYG{n}{i\PYGZus{}max} \PYG{o}{=} \PYG{n}{np}\PYG{o}{.}\PYG{n}{argmax}\PYG{p}{(}\PYG{n}{pts\PYGZus{}r}\PYG{p}{[}\PYG{p}{:}\PYG{p}{,} \PYG{l+m+mi}{0}\PYG{p}{]}\PYG{p}{)}\PYG{p}{;} \PYG{n}{i\PYGZus{}min} \PYG{o}{=} \PYG{n}{np}\PYG{o}{.}\PYG{n}{argmin}\PYG{p}{(}\PYG{n}{pts\PYGZus{}r}\PYG{p}{[}\PYG{p}{:}\PYG{p}{,} \PYG{l+m+mi}{0}\PYG{p}{]}\PYG{p}{)}
    \PYG{n}{p1} \PYG{o}{=} \PYG{p}{(}\PYG{n}{np}\PYG{o}{.}\PYG{n}{array}\PYG{p}{(}\PYG{p}{[}\PYG{n}{pts\PYGZus{}r}\PYG{p}{[}\PYG{n}{i\PYGZus{}min}\PYG{p}{,} \PYG{l+m+mi}{0}\PYG{p}{]}\PYG{p}{,} \PYG{l+m+mf}{0.0}\PYG{p}{]}\PYG{p}{,} \PYG{n}{np}\PYG{o}{.}\PYG{n}{float32}\PYG{p}{)} \PYG{o}{@} \PYG{n}{R}\PYG{p}{)} \PYG{o}{+} \PYG{n}{mean}
    \PYG{n}{p2} \PYG{o}{=} \PYG{p}{(}\PYG{n}{np}\PYG{o}{.}\PYG{n}{array}\PYG{p}{(}\PYG{p}{[}\PYG{n}{pts\PYGZus{}r}\PYG{p}{[}\PYG{n}{i\PYGZus{}max}\PYG{p}{,} \PYG{l+m+mi}{0}\PYG{p}{]}\PYG{p}{,} \PYG{l+m+mf}{0.0}\PYG{p}{]}\PYG{p}{,} \PYG{n}{np}\PYG{o}{.}\PYG{n}{float32}\PYG{p}{)} \PYG{o}{@} \PYG{n}{R}\PYG{p}{)} \PYG{o}{+} \PYG{n}{mean}
    \PYG{n}{cv2}\PYG{o}{.}\PYG{n}{line}\PYG{p}{(}\PYG{n}{overlay}\PYG{p}{,} \PYG{n+nb}{tuple}\PYG{p}{(}\PYG{n}{np}\PYG{o}{.}\PYG{n}{int32}\PYG{p}{(}\PYG{n}{p1}\PYG{p}{)}\PYG{p}{)}\PYG{p}{,} \PYG{n+nb}{tuple}\PYG{p}{(}\PYG{n}{np}\PYG{o}{.}\PYG{n}{int32}\PYG{p}{(}\PYG{n}{p2}\PYG{p}{)}\PYG{p}{)}\PYG{p}{,} \PYG{p}{(}\PYG{l+m+mi}{255}\PYG{p}{,} \PYG{l+m+mi}{0}\PYG{p}{,} \PYG{l+m+mi}{0}\PYG{p}{)}\PYG{p}{,} \PYG{l+m+mi}{2}\PYG{p}{)}
    \PYG{n}{cv2}\PYG{o}{.}\PYG{n}{rectangle}\PYG{p}{(}\PYG{n}{overlay}\PYG{p}{,} \PYG{p}{(}\PYG{n}{x}\PYG{p}{,} \PYG{n}{y}\PYG{p}{)}\PYG{p}{,} \PYG{p}{(}\PYG{n}{x} \PYG{o}{+} \PYG{n}{w}\PYG{p}{,} \PYG{n}{y} \PYG{o}{+} \PYG{n}{h}\PYG{p}{)}\PYG{p}{,} \PYG{p}{(}\PYG{l+m+mi}{255}\PYG{p}{,} \PYG{l+m+mi}{0}\PYG{p}{,} \PYG{l+m+mi}{255}\PYG{p}{)}\PYG{p}{,} \PYG{l+m+mi}{2}\PYG{p}{)}

    \PYG{c+c1}{\PYGZsh{} 8) Units}
    \PYG{n}{mm\PYGZus{}per\PYGZus{}px} \PYG{o}{=} \PYG{n}{get\PYGZus{}scale\PYGZus{}mm\PYGZus{}per\PYGZus{}px}\PYG{p}{(}\PYG{p}{)}
    \PYG{k}{if} \PYG{n}{mm\PYGZus{}per\PYGZus{}px} \PYG{o+ow}{and} \PYG{n}{np}\PYG{o}{.}\PYG{n}{isfinite}\PYG{p}{(}\PYG{n}{mm\PYGZus{}per\PYGZus{}px}\PYG{p}{)} \PYG{o+ow}{and} \PYG{n}{mm\PYGZus{}per\PYGZus{}px} \PYG{o}{\PYGZgt{}} \PYG{l+m+mi}{0}\PYG{p}{:}
        \PYG{n}{length\PYGZus{}mm} \PYG{o}{=} \PYG{n}{length\PYGZus{}px} \PYG{o}{*} \PYG{n}{mm\PYGZus{}per\PYGZus{}px}
        \PYG{n}{width\PYGZus{}mm}  \PYG{o}{=} \PYG{n}{width\PYGZus{}px}  \PYG{o}{*} \PYG{n}{mm\PYGZus{}per\PYGZus{}px}
        \PYG{n}{area\PYGZus{}mm2}  \PYG{o}{=} \PYG{n}{area\PYGZus{}px2}  \PYG{o}{*} \PYG{p}{(}\PYG{n}{mm\PYGZus{}per\PYGZus{}px} \PYG{o}{*}\PYG{o}{*} \PYG{l+m+mi}{2}\PYG{p}{)}
    \PYG{k}{else}\PYG{p}{:}
        \PYG{n}{mm\PYGZus{}per\PYGZus{}px} \PYG{o}{=} \PYG{n+nb}{float}\PYG{p}{(}\PYG{l+s+s2}{\PYGZdq{}}\PYG{l+s+s2}{nan}\PYG{l+s+s2}{\PYGZdq{}}\PYG{p}{)}\PYG{p}{;} \PYG{n}{length\PYGZus{}mm} \PYG{o}{=} \PYG{n}{width\PYGZus{}mm} \PYG{o}{=} \PYG{n}{area\PYGZus{}mm2} \PYG{o}{=} \PYG{n+nb}{float}\PYG{p}{(}\PYG{l+s+s2}{\PYGZdq{}}\PYG{l+s+s2}{nan}\PYG{l+s+s2}{\PYGZdq{}}\PYG{p}{)}

    \PYG{c+c1}{\PYGZsh{} 9) Optional outputs}
    \PYG{n}{mask\PYGZus{}full} \PYG{o}{=} \PYG{p}{(}\PYG{n}{full} \PYG{o}{*} \PYG{l+m+mi}{255}\PYG{p}{)}\PYG{o}{.}\PYG{n}{astype}\PYG{p}{(}\PYG{n}{np}\PYG{o}{.}\PYG{n}{uint8}\PYG{p}{)}
    \PYG{n}{mask\PYGZus{}path} \PYG{o}{=} \PYG{n}{save\PYGZus{}img\PYGZus{}dbg}\PYG{p}{(}\PYG{l+s+s2}{\PYGZdq{}}\PYG{l+s+s2}{mask\PYGZus{}debug.png}\PYG{l+s+s2}{\PYGZdq{}}\PYG{p}{,} \PYG{n}{mask\PYGZus{}full}\PYG{p}{)}
    \PYG{n}{overlay\PYGZus{}path} \PYG{o}{=} \PYG{n}{save\PYGZus{}img\PYGZus{}dbg}\PYG{p}{(}\PYG{l+s+s2}{\PYGZdq{}}\PYG{l+s+s2}{overlay\PYGZus{}debug.png}\PYG{l+s+s2}{\PYGZdq{}}\PYG{p}{,} \PYG{n}{cv2}\PYG{o}{.}\PYG{n}{cvtColor}\PYG{p}{(}\PYG{n}{overlay}\PYG{p}{,} \PYG{n}{cv2}\PYG{o}{.}\PYG{n}{COLOR\PYGZus{}RGB2BGR}\PYG{p}{)}\PYG{p}{)}

    \PYG{n}{df} \PYG{o}{=} \PYG{n}{pd}\PYG{o}{.}\PYG{n}{DataFrame}\PYG{p}{(}\PYG{p}{[}\PYG{p}{\PYGZob{}}
        \PYG{l+s+s2}{\PYGZdq{}}\PYG{l+s+s2}{image\PYGZus{}path}\PYG{l+s+s2}{\PYGZdq{}}\PYG{p}{:} \PYG{n}{IMG\PYGZus{}PATH}\PYG{p}{,}
        \PYG{l+s+s2}{\PYGZdq{}}\PYG{l+s+s2}{roi\PYGZus{}x}\PYG{l+s+s2}{\PYGZdq{}}\PYG{p}{:} \PYG{n}{x}\PYG{p}{,} \PYG{l+s+s2}{\PYGZdq{}}\PYG{l+s+s2}{roi\PYGZus{}y}\PYG{l+s+s2}{\PYGZdq{}}\PYG{p}{:} \PYG{n}{y}\PYG{p}{,} \PYG{l+s+s2}{\PYGZdq{}}\PYG{l+s+s2}{roi\PYGZus{}w}\PYG{l+s+s2}{\PYGZdq{}}\PYG{p}{:} \PYG{n}{w}\PYG{p}{,} \PYG{l+s+s2}{\PYGZdq{}}\PYG{l+s+s2}{roi\PYGZus{}h}\PYG{l+s+s2}{\PYGZdq{}}\PYG{p}{:} \PYG{n}{h}\PYG{p}{,}
        \PYG{l+s+s2}{\PYGZdq{}}\PYG{l+s+s2}{length\PYGZus{}px}\PYG{l+s+s2}{\PYGZdq{}}\PYG{p}{:} \PYG{n}{length\PYGZus{}px}\PYG{p}{,} \PYG{l+s+s2}{\PYGZdq{}}\PYG{l+s+s2}{width\PYGZus{}px}\PYG{l+s+s2}{\PYGZdq{}}\PYG{p}{:} \PYG{n}{width\PYGZus{}px}\PYG{p}{,} \PYG{l+s+s2}{\PYGZdq{}}\PYG{l+s+s2}{area\PYGZus{}px2}\PYG{l+s+s2}{\PYGZdq{}}\PYG{p}{:} \PYG{n}{area\PYGZus{}px2}\PYG{p}{,}
        \PYG{l+s+s2}{\PYGZdq{}}\PYG{l+s+s2}{mm\PYGZus{}per\PYGZus{}px}\PYG{l+s+s2}{\PYGZdq{}}\PYG{p}{:} \PYG{n+nb}{float}\PYG{p}{(}\PYG{n}{mm\PYGZus{}per\PYGZus{}px}\PYG{p}{)} \PYG{k}{if} \PYG{n}{np}\PYG{o}{.}\PYG{n}{isfinite}\PYG{p}{(}\PYG{n}{mm\PYGZus{}per\PYGZus{}px}\PYG{p}{)} \PYG{k}{else} \PYG{n}{np}\PYG{o}{.}\PYG{n}{nan}\PYG{p}{,}
        \PYG{l+s+s2}{\PYGZdq{}}\PYG{l+s+s2}{length\PYGZus{}mm}\PYG{l+s+s2}{\PYGZdq{}}\PYG{p}{:} \PYG{n+nb}{float}\PYG{p}{(}\PYG{n}{length\PYGZus{}mm}\PYG{p}{)} \PYG{k}{if} \PYG{n}{np}\PYG{o}{.}\PYG{n}{isfinite}\PYG{p}{(}\PYG{n}{length\PYGZus{}mm}\PYG{p}{)} \PYG{k}{else} \PYG{n}{np}\PYG{o}{.}\PYG{n}{nan}\PYG{p}{,}
        \PYG{l+s+s2}{\PYGZdq{}}\PYG{l+s+s2}{width\PYGZus{}mm}\PYG{l+s+s2}{\PYGZdq{}}\PYG{p}{:} \PYG{n+nb}{float}\PYG{p}{(}\PYG{n}{width\PYGZus{}mm}\PYG{p}{)} \PYG{k}{if} \PYG{n}{np}\PYG{o}{.}\PYG{n}{isfinite}\PYG{p}{(}\PYG{n}{width\PYGZus{}mm}\PYG{p}{)} \PYG{k}{else} \PYG{n}{np}\PYG{o}{.}\PYG{n}{nan}\PYG{p}{,}
        \PYG{l+s+s2}{\PYGZdq{}}\PYG{l+s+s2}{area\PYGZus{}mm2}\PYG{l+s+s2}{\PYGZdq{}}\PYG{p}{:} \PYG{n+nb}{float}\PYG{p}{(}\PYG{n}{area\PYGZus{}mm2}\PYG{p}{)} \PYG{k}{if} \PYG{n}{np}\PYG{o}{.}\PYG{n}{isfinite}\PYG{p}{(}\PYG{n}{area\PYGZus{}mm2}\PYG{p}{)} \PYG{k}{else} \PYG{n}{np}\PYG{o}{.}\PYG{n}{nan}\PYG{p}{,}
        \PYG{l+s+s2}{\PYGZdq{}}\PYG{l+s+s2}{best\PYGZus{}score}\PYG{l+s+s2}{\PYGZdq{}}\PYG{p}{:} \PYG{n+nb}{float}\PYG{p}{(}\PYG{n}{used\PYGZus{}scores}\PYG{p}{[}\PYG{n}{use\PYGZus{}idx}\PYG{p}{]}\PYG{p}{)} \PYG{k}{if} \PYG{p}{(}\PYG{n}{used\PYGZus{}scores} \PYG{o+ow}{and} \PYG{n}{np}\PYG{o}{.}\PYG{n}{isfinite}\PYG{p}{(}\PYG{n}{used\PYGZus{}scores}\PYG{p}{[}\PYG{n}{use\PYGZus{}idx}\PYG{p}{]}\PYG{p}{)}\PYG{p}{)} \PYG{k}{else} \PYG{n}{np}\PYG{o}{.}\PYG{n}{nan}
    \PYG{p}{\PYGZcb{}}\PYG{p}{]}\PYG{p}{)}
    \PYG{n}{csv\PYGZus{}path} \PYG{o}{=} \PYG{n}{save\PYGZus{}csv\PYGZus{}dbg}\PYG{p}{(}\PYG{n}{df}\PYG{p}{,} \PYG{l+s+s2}{\PYGZdq{}}\PYG{l+s+s2}{measurements\PYGZus{}debug.csv}\PYG{l+s+s2}{\PYGZdq{}}\PYG{p}{)}

    \PYG{k}{return} \PYG{p}{\PYGZob{}}
        \PYG{l+s+s2}{\PYGZdq{}}\PYG{l+s+s2}{roi}\PYG{l+s+s2}{\PYGZdq{}}\PYG{p}{:} \PYG{p}{(}\PYG{n}{x}\PYG{p}{,} \PYG{n}{y}\PYG{p}{,} \PYG{n}{w}\PYG{p}{,} \PYG{n}{h}\PYG{p}{)}\PYG{p}{,}
        \PYG{l+s+s2}{\PYGZdq{}}\PYG{l+s+s2}{length\PYGZus{}px}\PYG{l+s+s2}{\PYGZdq{}}\PYG{p}{:} \PYG{n}{length\PYGZus{}px}\PYG{p}{,} \PYG{l+s+s2}{\PYGZdq{}}\PYG{l+s+s2}{width\PYGZus{}px}\PYG{l+s+s2}{\PYGZdq{}}\PYG{p}{:} \PYG{n}{width\PYGZus{}px}\PYG{p}{,} \PYG{l+s+s2}{\PYGZdq{}}\PYG{l+s+s2}{area\PYGZus{}px2}\PYG{l+s+s2}{\PYGZdq{}}\PYG{p}{:} \PYG{n}{area\PYGZus{}px2}\PYG{p}{,}
        \PYG{l+s+s2}{\PYGZdq{}}\PYG{l+s+s2}{mm\PYGZus{}per\PYGZus{}px}\PYG{l+s+s2}{\PYGZdq{}}\PYG{p}{:} \PYG{n}{mm\PYGZus{}per\PYGZus{}px}\PYG{p}{,} \PYG{l+s+s2}{\PYGZdq{}}\PYG{l+s+s2}{length\PYGZus{}mm}\PYG{l+s+s2}{\PYGZdq{}}\PYG{p}{:} \PYG{n}{length\PYGZus{}mm}\PYG{p}{,} \PYG{l+s+s2}{\PYGZdq{}}\PYG{l+s+s2}{width\PYGZus{}mm}\PYG{l+s+s2}{\PYGZdq{}}\PYG{p}{:} \PYG{n}{width\PYGZus{}mm}\PYG{p}{,} \PYG{l+s+s2}{\PYGZdq{}}\PYG{l+s+s2}{area\PYGZus{}mm2}\PYG{l+s+s2}{\PYGZdq{}}\PYG{p}{:} \PYG{n}{area\PYGZus{}mm2}\PYG{p}{,}
        \PYG{l+s+s2}{\PYGZdq{}}\PYG{l+s+s2}{mask\PYGZus{}img}\PYG{l+s+s2}{\PYGZdq{}}\PYG{p}{:} \PYG{n}{mask\PYGZus{}full}\PYG{p}{,} \PYG{l+s+s2}{\PYGZdq{}}\PYG{l+s+s2}{overlay\PYGZus{}img}\PYG{l+s+s2}{\PYGZdq{}}\PYG{p}{:} \PYG{n}{overlay}\PYG{p}{,}
        \PYG{l+s+s2}{\PYGZdq{}}\PYG{l+s+s2}{mask\PYGZus{}path}\PYG{l+s+s2}{\PYGZdq{}}\PYG{p}{:} \PYG{n}{mask\PYGZus{}path}\PYG{p}{,} \PYG{l+s+s2}{\PYGZdq{}}\PYG{l+s+s2}{overlay\PYGZus{}path}\PYG{l+s+s2}{\PYGZdq{}}\PYG{p}{:} \PYG{n}{overlay\PYGZus{}path}\PYG{p}{,} \PYG{l+s+s2}{\PYGZdq{}}\PYG{l+s+s2}{csv\PYGZus{}path}\PYG{l+s+s2}{\PYGZdq{}}\PYG{p}{:} \PYG{n}{csv\PYGZus{}path}\PYG{p}{,}
        \PYG{l+s+s2}{\PYGZdq{}}\PYG{l+s+s2}{df}\PYG{l+s+s2}{\PYGZdq{}}\PYG{p}{:} \PYG{n}{df}
    \PYG{p}{\PYGZcb{}}
\end{sphinxVerbatim}

\end{sphinxuseclass}\end{sphinxVerbatimInput}

\end{sphinxuseclass}
\sphinxAtStartPar
El algoritmo de visión desarrollado se sustenta sobre una seria de funciones, cada una responsable de resolver el bloque operativo definido en la figura \hyperref[\detokenize{content/01/Modulo-3:figura-wp1-imagen-6}]{Fig.\@ \ref{\detokenize{content/01/Modulo-3:figura-wp1-imagen-6}}}.


\subsection{Preprocesado}
\label{\detokenize{content/01/Modulo-3:preprocesado}}
\sphinxAtStartPar
Este bloque funcional se sustenta sobre las siguientes funciones:
\begin{itemize}
\item {}
\sphinxAtStartPar
\sphinxstylestrong{prefilter(bgr).} Toma como entrada una imagen BGR (\sphinxcode{\sphinxupquote{np.ndarray}}) y devuelve una versión prefiltrada en BGR tras aplicar NLMeans (reducción de ruido no local), filtro bilateral (suavizado que preserva bordes) y CLAHE en el canal V (HSV). Su utilidad es mejorar la relación señal\sphinxhyphen{}ruido y homogeneizar la iluminación antes de la segmentación, estabilizando los modelos de apariencia usados por GrabCut y reduciendo artefactos de borde; en términos prácticos, atenúa alta frecuencia sin degradar discontinuidades relevantes (cabeza/cola del pez frente a la cinta).

\item {}
\sphinxAtStartPar
\sphinxstylestrong{dynamic\_blue\_threshold(hsv).} Recibe una imagen en HSV y retorna dos vectores \sphinxcode{\sphinxupquote{lower}} y \sphinxcode{\sphinxupquote{upper}} (umbrales HSV) que acotan dinámicamente el color azul de la cinta. Estima el pico del histograma del canal H en una franja vertical central (filtrada por saturación mínima) y construye un intervalo \([\,H_0\pm\Delta H\,]\). Su utilidad es adaptar la detección del fondo (cinta) a cambios de cámara/iluminación, reforzando la separación fondo (azul) / no\sphinxhyphen{}fondo para las semillas de GrabCut.

\item {}
\sphinxAtStartPar
\sphinxstylestrong{\_longest\_true\_run(vec\_bool)} Realiza una búsqueda lineal del tramo contiguo máximo de verdaderos. Como parámetro de entrada necesita un vector booleano 1D (np.ndarray) y devuelve \((i, j)\) (índices inclusive) del segmento más largo con \sphinxstyleemphasis{True}, o \sphinxstyleemphasis{None} si no existe. Se usa para fijar límites horizontales (columnas) y verticales (filas) de la cinta.

\item {}
\sphinxAtStartPar
\sphinxstylestrong{belt\_roi\_from\_blue(hsv, lower, upper).} Recibe la imagen HSV y los umbrales \sphinxcode{\sphinxupquote{lower/upper}}; devuelve la ROI de la cinta como tupla \sphinxcode{\sphinxupquote{(x, y, w, h)}} y la máscara binaria \sphinxcode{\sphinxupquote{blue}}. Umbraliza el azul, limpia con operaciones morfológicas (apertura/cierre) y toma el \sphinxstyleemphasis{bounding box} de la mayor componente conexa, expandiéndolo con \sphinxstyleemphasis{padding}. Su utilidad es anclar espacialmente el problema: restringe la segmentación a la banda transportadora (banda a banda), evitando que estructuras ajenas (p. ej., metálicas) perturben la optimización.

\item {}
\sphinxAtStartPar
\sphinxstylestrong{\_morph(img, op, shape, ksize, iters=1)} Se trata de un envoltorio de la interfaz de OpenCV \sphinxcode{\sphinxupquote{cv2.morphologyEx}} usada para aplicar operaciones morfológicas apertura, cierre, gradiente morfológico, tophat, blackhat, hit\sphinxhyphen{}or\sphinxhyphen{}miss) sobre la imagen. Como entrada necesita: img (np.uint8 o convertible), op (operación de OpenCV), shape (tipo de elemento estructurante), ksize (tupla de tamaño), iters (int) retornando la imagen procesada (np.ndarray) y evitando errores de resolución de sobrecarga.

\item {}
\sphinxAtStartPar
\sphinxstylestrong{\_open(img, shape, ksize, iters=1)} Aplica una apertura morfológica especializada con \sphinxcode{\sphinxupquote{cv2.MORPH\_OPEN}}. Se usa para eliminar ruido fino preservando estructuras mayores.

\item {}
\sphinxAtStartPar
\sphinxstylestrong{\_close(img, shape, ksize, iters=1)} Aplica un cierre morfológico especializado. Misma signatura que \sphinxcode{\sphinxupquote{\_open}} pero con \sphinxcode{\sphinxupquote{cv2.MORPH\_CLOSE}}. Rellena huecos/pequeñas discontinuidades en máscaras binarizadas; devuelve np.ndarray.

\end{itemize}


\subsection{Segmentación}
\label{\detokenize{content/01/Modulo-3:segmentacion}}\begin{itemize}
\item {}
\sphinxAtStartPar
\sphinxstylestrong{seeds\_in\_roi(bgr\_roi, lower, upper).} Recibe el recorte BGR de la ROI y los umbrales HSV; devuelve \sphinxcode{\sphinxupquote{gc\_mask}}, una máscara de estados entera con etiquetas de GrabCut \{BG seguro, PR\_BG, PR\_FG, FG\}. Construye las semillas así: (i) \sphinxstylestrong{BG seguro} = píxeles azules; (ii) \sphinxstylestrong{PR\_FG} = no\sphinxhyphen{}azul dilatado con kernel anisotrópico (alto y estrecho) para conectar zonas finas (cola/cabeza); (iii) \sphinxstylestrong{FG seguro} = no\sphinxhyphen{}azul con alto contraste cromático respecto al azul (combinación de distancias en H y S umbraladas por percentil), reforzado mediante \sphinxstylestrong{distance transform} sobre PR\_FG; además fija el borde de la ROI como BG. Su utilidad es condicionar favorablemente la minimización de la energía de GrabCut (término de datos GMM + término de suavidad por corte en grafo), reduciendo fugas y pérdidas en extremos.

\end{itemize}


\subsection{Selección y medición}
\label{\detokenize{content/01/Modulo-3:seleccion-y-medicion}}\begin{itemize}
\item {}
\sphinxAtStartPar
\sphinxstylestrong{contour\_scores(cnts, roi, strict=True)} Realiza una ponderación morfológica de candidatos. Como entredas necesita: contornos \sphinxcode{\sphinxupquote{cnts}}, ROI \sphinxcode{\sphinxupquote{(x,y,w,h)}} y un crietrio de severidad. Para cada contorno calcula \sphinxcode{\sphinxupquote{score = área × prior(log\sphinxhyphen{}aspecto) × solidez}}, aplicando filtros de área, anchura mínima y proximidad a bordes (relajables en modo \sphinxstyleemphasis{non\sphinxhyphen{}strict}). Devuelve una lista de \sphinxcode{\sphinxupquote{float}} alineada con \sphinxcode{\sphinxupquote{cnts}}.

\item {}
\sphinxAtStartPar
\sphinxstylestrong{extractor().} No recibe argumentos y devuelve: \sphinxcode{\sphinxupquote{overlay\_path}}, \sphinxcode{\sphinxupquote{mask\_path}}, \sphinxcode{\sphinxupquote{csv\_path}} (rutas), \sphinxcode{\sphinxupquote{length\_px}}, \sphinxcode{\sphinxupquote{width\_px}}, \sphinxcode{\sphinxupquote{area\_px2}} (magnitudes en píxeles), \sphinxcode{\sphinxupquote{mm\_per\_px}} (si existe) y la \sphinxcode{\sphinxupquote{roi}} empleada. Orquesta todo el pipeline: carga y prefiltra la imagen; estima el umbral azul dinámico y extrae la ROI de la cinta; genera semillas y ejecuta GrabCut con iteraciones configurables; aplica postproceso morfológico (cierre/apertura); selecciona el contorno del pez por área + alargamiento; calcula \sphinxstylestrong{longitud} y \sphinxstylestrong{anchura} proyectando el contorno sobre los autovectores de la \sphinxstylestrong{PCA} (extensión \(\max-\min\) en cada eje) y el \sphinxstylestrong{área} con \sphinxcode{\sphinxupquote{cv2.contourArea}}; finalmente guarda las salidas (overlay, máscara y CSV) y convierte a unidades métricas si hay factor de escala disponible.

\end{itemize}


\subsection{Otras funciones}
\label{\detokenize{content/01/Modulo-3:otras-funciones}}
\sphinxAtStartPar
El pipeline también proporciona algunas funciones auxiliares como:
\begin{itemize}
\item {}
\sphinxAtStartPar
\sphinxstylestrong{render\_candidates\_debug(bgr\_base, cnts, scores, best\_idx, roi\_xywh, tag)} Es una función útil en una auditoría visual del criterio de selección. Como entradas necesita: imagen base, lista de contornos, lista de scores, índice del mejor (o \sphinxcode{\sphinxupquote{None}}), ROI \sphinxcode{\sphinxupquote{(x,y,w,h)}} y etiqueta de texto. Dibuja ROI y contornos (verde el seleccionado, rojo el resto) con anotación de puntuación y guarda el fichero .PNG sólo en modo \sphinxcode{\sphinxupquote{debug}}, devolviendo su ruta o \sphinxcode{\sphinxupquote{None}}.

\item {}
\sphinxAtStartPar
\sphinxstylestrong{save\_img(name, img)} Guarda las imágenes obtendias en los procesos anteriores. Como variables de entrada usa: \sphinxcode{\sphinxupquote{name (str)}} e \sphinxcode{\sphinxupquote{img (np.ndarray, BGR o 1 canal)}}. Escribe el array en disco en con el parámetro \sphinxcode{\sphinxupquote{CONFIG{[}"out\_prefix"{]}+name}} y devuelve la ruta escrita \sphinxcode{\sphinxupquote{(str)}}. Esta función centraliza la serialización de artefactos visuales.

\item {}
\sphinxAtStartPar
\sphinxstylestrong{save\_img\_dbg(name, img)} Esta función es similar a la anterior(recibe los mismos parámetros que \sphinxcode{\sphinxupquote{save\_img}}) pero sólo guarda si \sphinxcode{\sphinxupquote{CONFIG{[}"debug"{]}==True}}. Devuelve la ruta escrita o \sphinxcode{\sphinxupquote{None}} si no hubo escritura. Implementa además el patrón de side\sphinxhyphen{}effects controlados por flag.

\item {}
\sphinxAtStartPar
\sphinxstylestrong{save\_csv\_dbg(df, name)} Garantiza la persistencia de los resultados del proceso en formato CSV sólo si \sphinxcode{\sphinxupquote{CONFIG{[}"save\_csv"{]}==True}}. Los parámetros de entreda son un \sphinxcode{\sphinxupquote{df (DataFrame de Pandas)}} y el nombre \sphinxcode{\sphinxupquote{name (str)}} del fichero. Guarda el CSV en \sphinxcode{\sphinxupquote{CONFIG{[}"out\_prefix"{]}+name}} y devuelve la ruta o \sphinxcode{\sphinxupquote{None}}. Permite desacoplar cómputo en memoria de la E/S de informes.

\item {}
\sphinxAtStartPar
\sphinxstylestrong{get\_scale\_mm\_per\_px().} No recibe argumentos y devuelve el factor escalar \sphinxcode{\sphinxupquote{mm\_per\_px}} (flotante) o \sphinxcode{\sphinxupquote{None}} si no está disponible. Internamente busca primero una función externa del usuario \sphinxcode{\sphinxupquote{get\_mm\_per\_px()}} y, en su defecto, un fichero \sphinxcode{\sphinxupquote{calibracion\_px\_mm.json}} con la clave \sphinxcode{\sphinxupquote{"mm\_per\_px"}}, que debería existir tras haber realizado una calibración. Su utilidad es habilitar la conversión métrica de las magnitudes geométricas estimadas en píxeles —longitud, anchura y área.

\end{itemize}

\sphinxAtStartPar
Para ejecutar el pipeline basta con:

\begin{sphinxuseclass}{cell}\begin{sphinxVerbatimInput}

\begin{sphinxuseclass}{cell_input}
\begin{sphinxVerbatim}[commandchars=\\\{\}]
\PYG{c+c1}{\PYGZsh{} \PYGZhy{}\PYGZhy{}\PYGZhy{}\PYGZhy{}\PYGZhy{}\PYGZhy{}\PYGZhy{}\PYGZhy{}\PYGZhy{}\PYGZhy{}\PYGZhy{}\PYGZhy{}\PYGZhy{}\PYGZhy{}\PYGZhy{}\PYGZhy{}\PYGZhy{}\PYGZhy{}\PYGZhy{}\PYGZhy{}\PYGZhy{}\PYGZhy{}\PYGZhy{}\PYGZhy{}\PYGZhy{}}
\PYG{c+c1}{\PYGZsh{} Execute \PYGZam{} (optional) display}
\PYG{c+c1}{\PYGZsh{} \PYGZhy{}\PYGZhy{}\PYGZhy{}\PYGZhy{}\PYGZhy{}\PYGZhy{}\PYGZhy{}\PYGZhy{}\PYGZhy{}\PYGZhy{}\PYGZhy{}\PYGZhy{}\PYGZhy{}\PYGZhy{}\PYGZhy{}\PYGZhy{}\PYGZhy{}\PYGZhy{}\PYGZhy{}\PYGZhy{}\PYGZhy{}\PYGZhy{}\PYGZhy{}\PYGZhy{}\PYGZhy{}}
\PYG{n}{res} \PYG{o}{=} \PYG{n}{extractor}\PYG{p}{(}\PYG{p}{)}

\PYG{c+c1}{\PYGZsh{} Visualización en notebook (no guarda archivos por sí misma)}
\PYG{n}{plt}\PYG{o}{.}\PYG{n}{figure}\PYG{p}{(}\PYG{n}{figsize}\PYG{o}{=}\PYG{p}{(}\PYG{l+m+mi}{12}\PYG{p}{,} \PYG{l+m+mi}{5}\PYG{p}{)}\PYG{p}{)}
\PYG{n}{plt}\PYG{o}{.}\PYG{n}{subplot}\PYG{p}{(}\PYG{l+m+mi}{1}\PYG{p}{,} \PYG{l+m+mi}{2}\PYG{p}{,} \PYG{l+m+mi}{1}\PYG{p}{)}\PYG{p}{;} \PYG{n}{plt}\PYG{o}{.}\PYG{n}{title}\PYG{p}{(}\PYG{l+s+s2}{\PYGZdq{}}\PYG{l+s+s2}{Overlay final}\PYG{l+s+s2}{\PYGZdq{}}\PYG{p}{)}\PYG{p}{;} \PYG{n}{plt}\PYG{o}{.}\PYG{n}{imshow}\PYG{p}{(}\PYG{n}{res}\PYG{p}{[}\PYG{l+s+s2}{\PYGZdq{}}\PYG{l+s+s2}{overlay\PYGZus{}img}\PYG{l+s+s2}{\PYGZdq{}}\PYG{p}{]}\PYG{p}{)}\PYG{p}{;} \PYG{n}{plt}\PYG{o}{.}\PYG{n}{axis}\PYG{p}{(}\PYG{l+s+s2}{\PYGZdq{}}\PYG{l+s+s2}{off}\PYG{l+s+s2}{\PYGZdq{}}\PYG{p}{)}
\PYG{n}{plt}\PYG{o}{.}\PYG{n}{subplot}\PYG{p}{(}\PYG{l+m+mi}{1}\PYG{p}{,} \PYG{l+m+mi}{2}\PYG{p}{,} \PYG{l+m+mi}{2}\PYG{p}{)}\PYG{p}{;} \PYG{n}{plt}\PYG{o}{.}\PYG{n}{title}\PYG{p}{(}\PYG{l+s+s2}{\PYGZdq{}}\PYG{l+s+s2}{Máscara final}\PYG{l+s+s2}{\PYGZdq{}}\PYG{p}{)}\PYG{p}{;} \PYG{n}{plt}\PYG{o}{.}\PYG{n}{imshow}\PYG{p}{(}\PYG{n}{res}\PYG{p}{[}\PYG{l+s+s2}{\PYGZdq{}}\PYG{l+s+s2}{mask\PYGZus{}img}\PYG{l+s+s2}{\PYGZdq{}}\PYG{p}{]}\PYG{p}{,} \PYG{n}{cmap}\PYG{o}{=}\PYG{l+s+s2}{\PYGZdq{}}\PYG{l+s+s2}{gray}\PYG{l+s+s2}{\PYGZdq{}}\PYG{p}{)}\PYG{p}{;} \PYG{n}{plt}\PYG{o}{.}\PYG{n}{axis}\PYG{p}{(}\PYG{l+s+s2}{\PYGZdq{}}\PYG{l+s+s2}{off}\PYG{l+s+s2}{\PYGZdq{}}\PYG{p}{)}
\PYG{n}{plt}\PYG{o}{.}\PYG{n}{tight\PYGZus{}layout}\PYG{p}{(}\PYG{p}{)}

\PYG{n+nb}{print}\PYG{p}{(}\PYG{l+s+sa}{f}\PYG{l+s+s2}{\PYGZdq{}}\PYG{l+s+s2}{Área de interés (ROI): }\PYG{l+s+si}{\PYGZob{}}\PYG{n}{res}\PYG{p}{[}\PYG{l+s+s1}{\PYGZsq{}}\PYG{l+s+s1}{roi}\PYG{l+s+s1}{\PYGZsq{}}\PYG{p}{]}\PYG{l+s+si}{\PYGZcb{}}\PYG{l+s+s2}{\PYGZdq{}}\PYG{p}{)}
\PYG{n+nb}{print}\PYG{p}{(}\PYG{l+s+sa}{f}\PYG{l+s+s2}{\PYGZdq{}}\PYG{l+s+s2}{Medidas detectadas: L=}\PYG{l+s+si}{\PYGZob{}}\PYG{n}{res}\PYG{p}{[}\PYG{l+s+s1}{\PYGZsq{}}\PYG{l+s+s1}{length\PYGZus{}px}\PYG{l+s+s1}{\PYGZsq{}}\PYG{p}{]}\PYG{l+s+si}{:}\PYG{l+s+s2}{.1f}\PYG{l+s+si}{\PYGZcb{}}\PYG{l+s+s2}{px  W=}\PYG{l+s+si}{\PYGZob{}}\PYG{n}{res}\PYG{p}{[}\PYG{l+s+s1}{\PYGZsq{}}\PYG{l+s+s1}{width\PYGZus{}px}\PYG{l+s+s1}{\PYGZsq{}}\PYG{p}{]}\PYG{l+s+si}{:}\PYG{l+s+s2}{.1f}\PYG{l+s+si}{\PYGZcb{}}\PYG{l+s+s2}{px  A=}\PYG{l+s+si}{\PYGZob{}}\PYG{n}{res}\PYG{p}{[}\PYG{l+s+s1}{\PYGZsq{}}\PYG{l+s+s1}{area\PYGZus{}px2}\PYG{l+s+s1}{\PYGZsq{}}\PYG{p}{]}\PYG{l+s+si}{:}\PYG{l+s+s2}{.1f}\PYG{l+s+si}{\PYGZcb{}}\PYG{l+s+s2}{px\PYGZca{}2}\PYG{l+s+s2}{\PYGZdq{}}\PYG{p}{)}
\PYG{k}{if} \PYG{n}{np}\PYG{o}{.}\PYG{n}{isfinite}\PYG{p}{(}\PYG{n}{res}\PYG{p}{[}\PYG{l+s+s2}{\PYGZdq{}}\PYG{l+s+s2}{mm\PYGZus{}per\PYGZus{}px}\PYG{l+s+s2}{\PYGZdq{}}\PYG{p}{]}\PYG{p}{)}\PYG{p}{:}
    \PYG{n+nb}{print}\PYG{p}{(}\PYG{l+s+sa}{f}\PYG{l+s+s2}{\PYGZdq{}}\PYG{l+s+s2}{Medidas reales: L=}\PYG{l+s+si}{\PYGZob{}}\PYG{n}{res}\PYG{p}{[}\PYG{l+s+s1}{\PYGZsq{}}\PYG{l+s+s1}{length\PYGZus{}mm}\PYG{l+s+s1}{\PYGZsq{}}\PYG{p}{]}\PYG{l+s+si}{:}\PYG{l+s+s2}{.2f}\PYG{l+s+si}{\PYGZcb{}}\PYG{l+s+s2}{ mm  W=}\PYG{l+s+si}{\PYGZob{}}\PYG{n}{res}\PYG{p}{[}\PYG{l+s+s1}{\PYGZsq{}}\PYG{l+s+s1}{width\PYGZus{}mm}\PYG{l+s+s1}{\PYGZsq{}}\PYG{p}{]}\PYG{l+s+si}{:}\PYG{l+s+s2}{.2f}\PYG{l+s+si}{\PYGZcb{}}\PYG{l+s+s2}{ mm  A=}\PYG{l+s+si}{\PYGZob{}}\PYG{n}{res}\PYG{p}{[}\PYG{l+s+s1}{\PYGZsq{}}\PYG{l+s+s1}{area\PYGZus{}mm2}\PYG{l+s+s1}{\PYGZsq{}}\PYG{p}{]}\PYG{l+s+si}{:}\PYG{l+s+s2}{.2f}\PYG{l+s+si}{\PYGZcb{}}\PYG{l+s+s2}{ mm\PYGZca{}2  Resolución }\PYG{l+s+si}{\PYGZob{}}\PYG{n}{res}\PYG{p}{[}\PYG{l+s+s1}{\PYGZsq{}}\PYG{l+s+s1}{mm\PYGZus{}per\PYGZus{}px}\PYG{l+s+s1}{\PYGZsq{}}\PYG{p}{]}\PYG{l+s+si}{:}\PYG{l+s+s2}{.6f}\PYG{l+s+si}{\PYGZcb{}}\PYG{l+s+s2}{ mm/px}\PYG{l+s+s2}{\PYGZdq{}}\PYG{p}{)}
\PYG{n+nb}{print}\PYG{p}{(}\PYG{l+s+sa}{f}\PYG{l+s+s2}{\PYGZdq{}}\PYG{l+s+s2}{PNG overlay guardado: }\PYG{l+s+si}{\PYGZob{}}\PYG{n}{res}\PYG{p}{[}\PYG{l+s+s1}{\PYGZsq{}}\PYG{l+s+s1}{overlay\PYGZus{}path}\PYG{l+s+s1}{\PYGZsq{}}\PYG{p}{]}\PYG{l+s+si}{\PYGZcb{}}\PYG{l+s+s2}{\PYGZdq{}}\PYG{p}{)}
\PYG{n+nb}{print}\PYG{p}{(}\PYG{l+s+sa}{f}\PYG{l+s+s2}{\PYGZdq{}}\PYG{l+s+s2}{PNG máscara guardado: }\PYG{l+s+si}{\PYGZob{}}\PYG{n}{res}\PYG{p}{[}\PYG{l+s+s1}{\PYGZsq{}}\PYG{l+s+s1}{mask\PYGZus{}path}\PYG{l+s+s1}{\PYGZsq{}}\PYG{p}{]}\PYG{l+s+si}{\PYGZcb{}}\PYG{l+s+s2}{\PYGZdq{}}\PYG{p}{)}
\PYG{n+nb}{print}\PYG{p}{(}\PYG{l+s+sa}{f}\PYG{l+s+s2}{\PYGZdq{}}\PYG{l+s+s2}{CSV datos: }\PYG{l+s+si}{\PYGZob{}}\PYG{n}{res}\PYG{p}{[}\PYG{l+s+s1}{\PYGZsq{}}\PYG{l+s+s1}{csv\PYGZus{}path}\PYG{l+s+s1}{\PYGZsq{}}\PYG{p}{]}\PYG{l+s+si}{\PYGZcb{}}\PYG{l+s+s2}{\PYGZdq{}}\PYG{p}{)}
\end{sphinxVerbatim}

\end{sphinxuseclass}\end{sphinxVerbatimInput}
\begin{sphinxVerbatimOutput}

\begin{sphinxuseclass}{cell_output}
\begin{sphinxVerbatim}[commandchars=\\\{\}]
Área de interés (ROI): (368, 0, 1114, 1080)
Medidas detectadas: L=745.1px  W=293.4px  A=152108.0px\PYGZca{}2
Medidas reales: L=37.25 mm  W=14.67 mm  A=380.27 mm\PYGZca{}2  Resolución 0.050000 mm/px
PNG overlay guardado: ./out/sole\PYGZus{}overlay\PYGZus{}debug.png
PNG máscara guardado: ./out/sole\PYGZus{}mask\PYGZus{}debug.png
CSV datos: ./out/sole\PYGZus{}measurements\PYGZus{}debug.csv
\end{sphinxVerbatim}

\noindent\sphinxincludegraphics{{2a29c2c420da628fe8d4d169efb8261f0e80cf00ed791003e75cd91b39b231fd}.png}

\end{sphinxuseclass}\end{sphinxVerbatimOutput}

\end{sphinxuseclass}
\sphinxAtStartPar
Como se puede apreciar en las imágenes del proceso, el \sphinxstyleemphasis{pipeline} define adecuadamente el área de interés al ancho de la cinta (recuadro color violeta) y es capaz detectar el pez (recuadro amarillo) y obtener el contorno (verde) y las medidas del mismo.

\sphinxstepscope


\section{M4 \sphinxhyphen{} Salida y persistencia}
\label{\detokenize{content/01/Modulo-4:m4-salida-y-persistencia}}\label{\detokenize{content/01/Modulo-4::doc}}
\sphinxAtStartPar
El último de los bloques definido en el flujograma \hyperref[\detokenize{content/01/Imagen:figura-wp1-imagen-1}]{Fig.\@ \ref{\detokenize{content/01/Imagen:figura-wp1-imagen-1}}} es el responsable de la ejecución del pipeline de procesado de imagen y el registro en MongoDB de los datos obtenidos del pipeline.

\sphinxAtStartPar
Para ello se ha generado un flujo en Node\sphinxhyphen{}RED que recibe la imagen obtenida (ver módulo M1\sphinxhyphen{}Captura) y ejecuta el código python de extracción de imágenes. La información de salida del pipeline es recogida y estructurada adecuadamente para su inclusión en la base de datos. La definición formal del JSON es la siguiente:

\begin{sphinxVerbatim}[commandchars=\\\{\}]
\PYG{p}{\PYGZob{}}
\PYG{+w}{  }\PYG{n+nt}{\PYGZdq{}\PYGZdl{}schema\PYGZdq{}}\PYG{p}{:}\PYG{+w}{ }\PYG{l+s+s2}{\PYGZdq{}https://json\PYGZhy{}schema.org/draft/2020\PYGZhy{}12/schema\PYGZdq{}}\PYG{p}{,}
\PYG{+w}{  }\PYG{n+nt}{\PYGZdq{}title\PYGZdq{}}\PYG{p}{:}\PYG{+w}{ }\PYG{l+s+s2}{\PYGZdq{}FishMeasurement\PYGZdq{}}\PYG{p}{,}
\PYG{+w}{  }\PYG{n+nt}{\PYGZdq{}type\PYGZdq{}}\PYG{p}{:}\PYG{+w}{ }\PYG{l+s+s2}{\PYGZdq{}object\PYGZdq{}}\PYG{p}{,}
\PYG{+w}{  }\PYG{n+nt}{\PYGZdq{}properties\PYGZdq{}}\PYG{p}{:}\PYG{+w}{ }\PYG{p}{\PYGZob{}}
\PYG{+w}{    }\PYG{n+nt}{\PYGZdq{}timestamp\PYGZdq{}}\PYG{p}{:}\PYG{+w}{ }\PYG{p}{\PYGZob{}}
\PYG{+w}{      }\PYG{n+nt}{\PYGZdq{}type\PYGZdq{}}\PYG{p}{:}\PYG{+w}{ }\PYG{l+s+s2}{\PYGZdq{}string\PYGZdq{}}\PYG{p}{,}
\PYG{+w}{      }\PYG{n+nt}{\PYGZdq{}format\PYGZdq{}}\PYG{p}{:}\PYG{+w}{ }\PYG{l+s+s2}{\PYGZdq{}date\PYGZhy{}time\PYGZdq{}}\PYG{p}{,}
\PYG{+w}{      }\PYG{n+nt}{\PYGZdq{}description\PYGZdq{}}\PYG{p}{:}\PYG{+w}{ }\PYG{l+s+s2}{\PYGZdq{}Marca temporal ISO 8601 de la medición (ej. 2025\PYGZhy{}10\PYGZhy{}15T09:30:00Z)\PYGZdq{}}
\PYG{+w}{    }\PYG{p}{\PYGZcb{},}
\PYG{+w}{    }\PYG{n+nt}{\PYGZdq{}metrics\PYGZdq{}}\PYG{p}{:}\PYG{+w}{ }\PYG{p}{\PYGZob{}}
\PYG{+w}{      }\PYG{n+nt}{\PYGZdq{}type\PYGZdq{}}\PYG{p}{:}\PYG{+w}{ }\PYG{l+s+s2}{\PYGZdq{}object\PYGZdq{}}\PYG{p}{,}
\PYG{+w}{      }\PYG{n+nt}{\PYGZdq{}properties\PYGZdq{}}\PYG{p}{:}\PYG{+w}{ }\PYG{p}{\PYGZob{}}
\PYG{+w}{        }\PYG{n+nt}{\PYGZdq{}pixels\PYGZdq{}}\PYG{p}{:}\PYG{+w}{ }\PYG{p}{\PYGZob{}}
\PYG{+w}{          }\PYG{n+nt}{\PYGZdq{}type\PYGZdq{}}\PYG{p}{:}\PYG{+w}{ }\PYG{l+s+s2}{\PYGZdq{}object\PYGZdq{}}\PYG{p}{,}
\PYG{+w}{          }\PYG{n+nt}{\PYGZdq{}properties\PYGZdq{}}\PYG{p}{:}\PYG{+w}{ }\PYG{p}{\PYGZob{}}
\PYG{+w}{            }\PYG{n+nt}{\PYGZdq{}long\PYGZdq{}}\PYG{p}{:}\PYG{+w}{ }\PYG{p}{\PYGZob{}}
\PYG{+w}{              }\PYG{n+nt}{\PYGZdq{}type\PYGZdq{}}\PYG{p}{:}\PYG{+w}{ }\PYG{l+s+s2}{\PYGZdq{}number\PYGZdq{}}\PYG{p}{,}
\PYG{+w}{              }\PYG{n+nt}{\PYGZdq{}multipleOf\PYGZdq{}}\PYG{p}{:}\PYG{+w}{ }\PYG{l+m+mf}{0.01}\PYG{p}{,}
\PYG{+w}{              }\PYG{n+nt}{\PYGZdq{}description\PYGZdq{}}\PYG{p}{:}\PYG{+w}{ }\PYG{l+s+s2}{\PYGZdq{}Longitud en píxeles (2 decimales)\PYGZdq{}}
\PYG{+w}{            }\PYG{p}{\PYGZcb{},}
\PYG{+w}{            }\PYG{n+nt}{\PYGZdq{}width\PYGZdq{}}\PYG{p}{:}\PYG{+w}{ }\PYG{p}{\PYGZob{}}
\PYG{+w}{              }\PYG{n+nt}{\PYGZdq{}type\PYGZdq{}}\PYG{p}{:}\PYG{+w}{ }\PYG{l+s+s2}{\PYGZdq{}number\PYGZdq{}}\PYG{p}{,}
\PYG{+w}{              }\PYG{n+nt}{\PYGZdq{}multipleOf\PYGZdq{}}\PYG{p}{:}\PYG{+w}{ }\PYG{l+m+mf}{0.01}\PYG{p}{,}
\PYG{+w}{              }\PYG{n+nt}{\PYGZdq{}description\PYGZdq{}}\PYG{p}{:}\PYG{+w}{ }\PYG{l+s+s2}{\PYGZdq{}Anchura en píxeles (2 decimales)\PYGZdq{}}
\PYG{+w}{            }\PYG{p}{\PYGZcb{},}
\PYG{+w}{            }\PYG{n+nt}{\PYGZdq{}surface\PYGZdq{}}\PYG{p}{:}\PYG{+w}{ }\PYG{p}{\PYGZob{}}
\PYG{+w}{              }\PYG{n+nt}{\PYGZdq{}type\PYGZdq{}}\PYG{p}{:}\PYG{+w}{ }\PYG{l+s+s2}{\PYGZdq{}number\PYGZdq{}}\PYG{p}{,}
\PYG{+w}{              }\PYG{n+nt}{\PYGZdq{}multipleOf\PYGZdq{}}\PYG{p}{:}\PYG{+w}{ }\PYG{l+m+mf}{0.01}\PYG{p}{,}
\PYG{+w}{              }\PYG{n+nt}{\PYGZdq{}description\PYGZdq{}}\PYG{p}{:}\PYG{+w}{ }\PYG{l+s+s2}{\PYGZdq{}Superficie en píxeles cuadrados (2 decimales)\PYGZdq{}}
\PYG{+w}{            }\PYG{p}{\PYGZcb{}}
\PYG{+w}{          }\PYG{p}{\PYGZcb{},}
\PYG{+w}{          }\PYG{n+nt}{\PYGZdq{}required\PYGZdq{}}\PYG{p}{:}\PYG{+w}{ }\PYG{p}{[}\PYG{l+s+s2}{\PYGZdq{}long\PYGZdq{}}\PYG{p}{,}\PYG{+w}{ }\PYG{l+s+s2}{\PYGZdq{}width\PYGZdq{}}\PYG{p}{,}\PYG{+w}{ }\PYG{l+s+s2}{\PYGZdq{}surface\PYGZdq{}}\PYG{p}{]}
\PYG{+w}{        }\PYG{p}{\PYGZcb{},}
\PYG{+w}{        }\PYG{n+nt}{\PYGZdq{}mm\PYGZdq{}}\PYG{p}{:}\PYG{+w}{ }\PYG{p}{\PYGZob{}}
\PYG{+w}{          }\PYG{n+nt}{\PYGZdq{}type\PYGZdq{}}\PYG{p}{:}\PYG{+w}{ }\PYG{l+s+s2}{\PYGZdq{}object\PYGZdq{}}\PYG{p}{,}
\PYG{+w}{          }\PYG{n+nt}{\PYGZdq{}properties\PYGZdq{}}\PYG{p}{:}\PYG{+w}{ }\PYG{p}{\PYGZob{}}
\PYG{+w}{            }\PYG{n+nt}{\PYGZdq{}long\PYGZdq{}}\PYG{p}{:}\PYG{+w}{ }\PYG{p}{\PYGZob{}}
\PYG{+w}{              }\PYG{n+nt}{\PYGZdq{}type\PYGZdq{}}\PYG{p}{:}\PYG{+w}{ }\PYG{l+s+s2}{\PYGZdq{}number\PYGZdq{}}\PYG{p}{,}
\PYG{+w}{              }\PYG{n+nt}{\PYGZdq{}multipleOf\PYGZdq{}}\PYG{p}{:}\PYG{+w}{ }\PYG{l+m+mf}{0.01}\PYG{p}{,}
\PYG{+w}{              }\PYG{n+nt}{\PYGZdq{}description\PYGZdq{}}\PYG{p}{:}\PYG{+w}{ }\PYG{l+s+s2}{\PYGZdq{}Longitud en milímetros (2 decimales)\PYGZdq{}}
\PYG{+w}{            }\PYG{p}{\PYGZcb{},}
\PYG{+w}{            }\PYG{n+nt}{\PYGZdq{}width\PYGZdq{}}\PYG{p}{:}\PYG{+w}{ }\PYG{p}{\PYGZob{}}
\PYG{+w}{              }\PYG{n+nt}{\PYGZdq{}type\PYGZdq{}}\PYG{p}{:}\PYG{+w}{ }\PYG{l+s+s2}{\PYGZdq{}number\PYGZdq{}}\PYG{p}{,}
\PYG{+w}{              }\PYG{n+nt}{\PYGZdq{}multipleOf\PYGZdq{}}\PYG{p}{:}\PYG{+w}{ }\PYG{l+m+mf}{0.01}\PYG{p}{,}
\PYG{+w}{              }\PYG{n+nt}{\PYGZdq{}description\PYGZdq{}}\PYG{p}{:}\PYG{+w}{ }\PYG{l+s+s2}{\PYGZdq{}Anchura en milímetros (2 decimales)\PYGZdq{}}
\PYG{+w}{            }\PYG{p}{\PYGZcb{},}
\PYG{+w}{            }\PYG{n+nt}{\PYGZdq{}surface\PYGZdq{}}\PYG{p}{:}\PYG{+w}{ }\PYG{p}{\PYGZob{}}
\PYG{+w}{              }\PYG{n+nt}{\PYGZdq{}type\PYGZdq{}}\PYG{p}{:}\PYG{+w}{ }\PYG{l+s+s2}{\PYGZdq{}number\PYGZdq{}}\PYG{p}{,}
\PYG{+w}{              }\PYG{n+nt}{\PYGZdq{}multipleOf\PYGZdq{}}\PYG{p}{:}\PYG{+w}{ }\PYG{l+m+mf}{0.01}\PYG{p}{,}
\PYG{+w}{              }\PYG{n+nt}{\PYGZdq{}description\PYGZdq{}}\PYG{p}{:}\PYG{+w}{ }\PYG{l+s+s2}{\PYGZdq{}Superficie en milímetros cuadrados (2 decimales)\PYGZdq{}}
\PYG{+w}{            }\PYG{p}{\PYGZcb{}}
\PYG{+w}{          }\PYG{p}{\PYGZcb{},}
\PYG{+w}{          }\PYG{n+nt}{\PYGZdq{}required\PYGZdq{}}\PYG{p}{:}\PYG{+w}{ }\PYG{p}{[}\PYG{l+s+s2}{\PYGZdq{}long\PYGZdq{}}\PYG{p}{,}\PYG{+w}{ }\PYG{l+s+s2}{\PYGZdq{}width\PYGZdq{}}\PYG{p}{,}\PYG{+w}{ }\PYG{l+s+s2}{\PYGZdq{}surface\PYGZdq{}}\PYG{p}{]}
\PYG{+w}{        }\PYG{p}{\PYGZcb{}}
\PYG{+w}{      }\PYG{p}{\PYGZcb{},}
\PYG{+w}{      }\PYG{n+nt}{\PYGZdq{}required\PYGZdq{}}\PYG{p}{:}\PYG{+w}{ }\PYG{p}{[}\PYG{l+s+s2}{\PYGZdq{}pixels\PYGZdq{}}\PYG{p}{,}\PYG{+w}{ }\PYG{l+s+s2}{\PYGZdq{}mm\PYGZdq{}}\PYG{p}{]}
\PYG{+w}{    }\PYG{p}{\PYGZcb{},}
\PYG{+w}{    }\PYG{n+nt}{\PYGZdq{}size\PYGZdq{}}\PYG{p}{:}\PYG{+w}{ }\PYG{p}{\PYGZob{}}
\PYG{+w}{      }\PYG{n+nt}{\PYGZdq{}type\PYGZdq{}}\PYG{p}{:}\PYG{+w}{ }\PYG{l+s+s2}{\PYGZdq{}string\PYGZdq{}}\PYG{p}{,}
\PYG{+w}{      }\PYG{n+nt}{\PYGZdq{}enum\PYGZdq{}}\PYG{p}{:}\PYG{+w}{ }\PYG{p}{[}\PYG{l+s+s2}{\PYGZdq{}small\PYGZdq{}}\PYG{p}{,}\PYG{+w}{ }\PYG{l+s+s2}{\PYGZdq{}medium\PYGZdq{}}\PYG{p}{,}\PYG{+w}{ }\PYG{l+s+s2}{\PYGZdq{}large\PYGZdq{}}\PYG{p}{],}
\PYG{+w}{      }\PYG{n+nt}{\PYGZdq{}description\PYGZdq{}}\PYG{p}{:}\PYG{+w}{ }\PYG{l+s+s2}{\PYGZdq{}Clasificación del pez según sus dimensiones\PYGZdq{}}
\PYG{+w}{    }\PYG{p}{\PYGZcb{}}
\PYG{+w}{  }\PYG{p}{\PYGZcb{},}
\PYG{+w}{  }\PYG{n+nt}{\PYGZdq{}required\PYGZdq{}}\PYG{p}{:}\PYG{+w}{ }\PYG{p}{[}\PYG{l+s+s2}{\PYGZdq{}timestamp\PYGZdq{}}\PYG{p}{,}\PYG{+w}{ }\PYG{l+s+s2}{\PYGZdq{}metrics\PYGZdq{}}\PYG{p}{,}\PYG{+w}{ }\PYG{l+s+s2}{\PYGZdq{}size\PYGZdq{}}\PYG{p}{]}
\PYG{p}{\PYGZcb{}}
\end{sphinxVerbatim}

\sphinxAtStartPar
El flujo de Node\sphinxhyphen{}Red responsable de salida y persistencia de los datos es el que se recoge en la siguiente figura.

\begin{figure}[htbp]
\centering
\capstart

\noindent\sphinxincludegraphics[width=1.000\linewidth]{{Modulo-4_nodered}.png}
\caption{Flujo Node\sphinxhyphen{}RED para persistencia de datos de tamaño}\label{\detokenize{content/01/Modulo-4:figura-wp1-imagen-8}}\end{figure}

\sphinxstepscope


\part{PT2 \sphinxhyphen{} Datasets}

\sphinxstepscope


\chapter{Obtención y etiquetado del dataset}
\label{\detokenize{content/02/Dataset:obtencion-y-etiquetado-del-dataset}}\label{\detokenize{content/02/Dataset::doc}}
\begin{sphinxadmonition}{note}{Resumen}

\sphinxAtStartPar
Este artículo presenta la metodología utilizada en la recopilación de un conjunto de datos o \sphinxstyleemphasis{dataset}  representativo que relacione las dimensiones corporales de los lenguados con su peso real en condiciones controladas. Para ello se ha contado con la colaboración del equipo de producción de Safistela S.A (\sphinxhref{https://seaeight.eu/seaeight}{Sea8}) para garantizar los controles de calidad estadísticas y el cumplimiento de los estándares éticos en investigación acuícola.

\sphinxAtStartPar
\sphinxstylestrong{Entregable}: E2.1\\
\sphinxstylestrong{Versión}: 1.0\\
\sphinxstylestrong{Autor}: Javier Álvarez Osuna\\
\sphinxstylestrong{Email}: javier.osuna@fishfarmfeeder.com\\
\sphinxstylestrong{ORCID}: \sphinxhref{https://orcid.org/0000-0001-7063-1279}{0000\sphinxhyphen{}0001\sphinxhyphen{}7063\sphinxhyphen{}1279}\\
\sphinxstylestrong{Licencia}: CC\sphinxhyphen{}BY\sphinxhyphen{}4.0\\
\sphinxstylestrong{Código proyecto}: IG408M.2025.000.000072

\begin{figure}[H]
\centering

\noindent\sphinxincludegraphics[width=1.000\linewidth]{{FLATCLASS_logo_publicidad}.png}
\end{figure}
\end{sphinxadmonition}


\section{Introducción}
\label{\detokenize{content/02/Dataset:introduccion}}
\sphinxAtStartPar
El desarrollo de modelos precisos para la estimación de peso en peces planos, como el lenguado (Solea spp.), requiere de conjuntos de datos morfométricos robustos que capturen la relación entre las dimensiones corporales y el peso real en condiciones controladas. Estas relaciones son fundamentales para aplicaciones en acuicultura, donde la monitorización no invasiva del crecimiento optimiza el manejo productivo y reduce el estrés en los organismos {[}\sphinxhref{https://doi.org/10.1371/journal.pone.0157890}{Takács et al., 2015}{]}. En particular, los juveniles entre 70 \sphinxhyphen{} 90 días representan una fase crítica del desarrollo, en la que variaciones en la longitud, altura y anchura pueden correlacionarse significativamente con cambios en la biomasa, siempre que las mediciones se realicen con precisión y minimizando perturbaciones {[}\sphinxhref{https://doi.org/10.1016/j.aquaculture.2011.03.011}{Costas et al., 2013}{]}.

\sphinxAtStartPar
No obstante, la adquisición de estos datos enfrenta un desafío metodológico clave: equilibrar la exactitud de las mediciones con el bienestar animal. Estudios previos han demostrado que la manipulación repetida de ejemplares juveniles induce estrés agudo, afectando no solo su fisiología, sino también la fiabilidad de los datos morfométricos al alterar su postura natural {[}\sphinxhref{https://link.springer.com/article/10.1007/s10695-011-9518-8}{Martins et al., 2012}{]}. Por ello, el protocolo para la obtención de los valores morfométricos necesarios se diseñó bajo el principio de \sphinxstylestrong{“medida única estandarizada”}, respaldado por controles de calidad estadísticos y técnicas de sedación suave, para garantizar tanto la representatividad de los datos como el cumplimiento de los estándares éticos en investigación acuícola {[}\sphinxhref{https://doi.org/10.1016/j.applanim.2004.02.003}{Conte, 2004}{]}. La metodología descrita a continuación proporciona un marco reproducible para obtener datos reales confiables, esenciales para el desarrollo de algoritmos de estimación de peso en lenguados.


\section{Métodología para la adquisición de datos morfométricos}
\label{\detokenize{content/02/Dataset:metodologia-para-la-adquisicion-de-datos-morfometricos}}

\subsection{Preparación de los Ejemplares}
\label{\detokenize{content/02/Dataset:preparacion-de-los-ejemplares}}
\sphinxAtStartPar
Los juveniles de lenguado (90 días post\sphinxhyphen{}eclosión) fueron previamente aclimatados durante 24 h en tanques de cuarentena con condiciones controladas (temperatura: 18 ± 1°C, salinidad: 35 ± 1 ppt, oxígeno disuelto: >6 mg/L) para minimizar el estrés asociado al manejo {[}\sphinxhref{https://link.springer.com/article/10.1007/s10695-011-9518-8}{Martins et al., 2012}{]}. Antes de las mediciones, se realizó un ayuno de 12 h para evitar variaciones en el peso corporal debido a la ingesta de alimento.


\subsection{Protocolo de Medición}
\label{\detokenize{content/02/Dataset:protocolo-de-medicion}}
\sphinxAtStartPar
Las mediciones morfométricas (peso, longitud total, altura y anchura) se realizaron siguiendo un enfoque de medida única por variable, fundamentado en criterios de bienestar animal y reproducibilidad estadística. Cada individuo fue manipulado por un único operador entrenado, utilizando instrumental calibrado (balanza digital de precisión ±0.01 g y calibrador digital ±0.01 cm). Para garantizar la estandarización, la longitud total se midió desde el rostro hasta el extremo del lóbulo caudal en posición natural, mientras que la altura y anchura se registraron en la región de máxima dimensión corporal {[}\sphinxhref{https://www.researchgate.net/publication/282421080\_Advanced\_Techniques\_for\_Morphometric\_Analysis\_in\_Fish}{Tonna 2015}{]}.

\sphinxAtStartPar
La decisión de no replicar las mediciones se basó en evidencias que demuestran que la manipulación repetida induce estrés agudo en estadios juveniles, afectando su homeostasis osmótica y aumentando la mortalidad post\sphinxhyphen{}manejo (Costas et al., 2013). Las réplicas requieren mayor tiempo fuera del agua, generando hipoxia y acidosis tisular. Estudios en \sphinxstyleemphasis{Solea senegalensis} muestran que exposiciones > 30 segundos alteran el equilibrio osmótico {[}\sphinxhref{https://doi.org/10.1016/j.aquaculture.2011.03.011}{Costas et al., 2013}{]}. Además, cada contacto adicional incrementa el riesgo de lesiones (ej. descamación, daño en aletas) y eleva los niveles de cortisol, afectando su fisiología y crecimiento posterior {[}\sphinxhref{https://doi.org/10.1016/j.applanim.2004.02.003}{Conte, 2004}{]}.

\sphinxAtStartPar
Para mitigar riesgos, se empleó sedación suave, protocolo validado previamente por el personal facultativo de Safistela S.A (Sea8). Cada ejemplar fue medido en un tiempo máximo de 20 segundos fuera del agua, sobre una superficie húmeda para reducir la pérdida de mucus.


\subsection{Consideraciones Éticas y de Bienestar}
\label{\detokenize{content/02/Dataset:consideraciones-eticas-y-de-bienestar}}
\sphinxAtStartPar
El protocolo cumplió con las directrices de la UE (2010/63/EU) para la reducción del estrés en animales de experimentación. Se evitó la manipulación repetida y se monitorizaron signos de estrés post\sphinxhyphen{}manejo (ej. natación errática o falta de alimentación durante 2 h posteriores). Los ejemplares fueron reintegrados inmediatamente a tanques de recuperación con flujo de agua continuo y sombreado para minimizar perturbaciones.


\subsection{Estructura de los datos}
\label{\detokenize{content/02/Dataset:estructura-de-los-datos}}
\sphinxAtStartPar
La base de datos se organizó en una hoja de Excel con 209 registros y 5 variables (columnas)


\begin{savenotes}\sphinxattablestart
\sphinxthistablewithglobalstyle
\centering
\begin{tabulary}{\linewidth}[t]{TTTT}
\sphinxtoprule
\sphinxstyletheadfamily
\sphinxAtStartPar
Variable
&\sphinxstyletheadfamily
\sphinxAtStartPar
Tipo de Dato
&\sphinxstyletheadfamily
\sphinxAtStartPar
Unidad
&\sphinxstyletheadfamily
\sphinxAtStartPar
Descripción
\\
\sphinxmidrule
\sphinxtableatstartofbodyhook
\sphinxAtStartPar
ID\_Ejemplar
&
\sphinxAtStartPar
Numérico
&
\sphinxAtStartPar
\sphinxhyphen{}
&
\sphinxAtStartPar
Número entero correlativo.
\\
\sphinxhline
\sphinxAtStartPar
Peso
&
\sphinxAtStartPar
Numérico
&
\sphinxAtStartPar
gramos (g)
&
\sphinxAtStartPar
Peso corporal.
\\
\sphinxhline
\sphinxAtStartPar
Longitud
&
\sphinxAtStartPar
Numérico
&
\sphinxAtStartPar
cm
&
\sphinxAtStartPar
Longitud total.
\\
\sphinxhline
\sphinxAtStartPar
Altura
&
\sphinxAtStartPar
Numérico
&
\sphinxAtStartPar
cm
&
\sphinxAtStartPar
Altura máxima del cuerpo.
\\
\sphinxhline
\sphinxAtStartPar
Anchura
&
\sphinxAtStartPar
Numérico
&
\sphinxAtStartPar
cm
&
\sphinxAtStartPar
Anchura máxima del cuerpo.
\\
\sphinxbottomrule
\end{tabulary}
\sphinxtableafterendhook\par
\sphinxattableend\end{savenotes}


\section{Inspección general del conjunto de datos}
\label{\detokenize{content/02/Dataset:inspeccion-general-del-conjunto-de-datos}}
\begin{sphinxuseclass}{cell}\begin{sphinxVerbatimInput}

\begin{sphinxuseclass}{cell_input}
\begin{sphinxVerbatim}[commandchars=\\\{\}]
\PYG{k+kn}{import}\PYG{+w}{ }\PYG{n+nn}{pandas}\PYG{+w}{ }\PYG{k}{as}\PYG{+w}{ }\PYG{n+nn}{pd}
\PYG{k+kn}{import}\PYG{+w}{ }\PYG{n+nn}{numpy}\PYG{+w}{ }\PYG{k}{as}\PYG{+w}{ }\PYG{n+nn}{np}
\PYG{k+kn}{import}\PYG{+w}{ }\PYG{n+nn}{matplotlib}\PYG{n+nn}{.}\PYG{n+nn}{pyplot}\PYG{+w}{ }\PYG{k}{as}\PYG{+w}{ }\PYG{n+nn}{plt}
\PYG{k+kn}{import}\PYG{+w}{ }\PYG{n+nn}{seaborn}\PYG{+w}{ }\PYG{k}{as}\PYG{+w}{ }\PYG{n+nn}{sns}

\PYG{c+c1}{\PYGZsh{} Leer el dataset}
\PYG{n}{file\PYGZus{}path} \PYG{o}{=} \PYG{l+s+s1}{\PYGZsq{}}\PYG{l+s+s1}{.././data/Dimensiones\PYGZus{}lenguado.xlsx}\PYG{l+s+s1}{\PYGZsq{}}
\PYG{n}{df} \PYG{o}{=} \PYG{n}{pd}\PYG{o}{.}\PYG{n}{read\PYGZus{}excel}\PYG{p}{(}\PYG{n}{file\PYGZus{}path}\PYG{p}{)}

\PYG{n+nb}{print}\PYG{p}{(}\PYG{l+s+s2}{\PYGZdq{}}\PYG{l+s+s2}{Forma:}\PYG{l+s+s2}{\PYGZdq{}}\PYG{p}{,} \PYG{n}{df}\PYG{o}{.}\PYG{n}{shape}\PYG{p}{)}
\PYG{n+nb}{print}\PYG{p}{(}\PYG{l+s+s2}{\PYGZdq{}}\PYG{l+s+se}{\PYGZbs{}n}\PYG{l+s+s2}{ Visualización de los 20 primeros registros}\PYG{l+s+s2}{\PYGZdq{}}\PYG{p}{)}
\PYG{n}{display}\PYG{p}{(}\PYG{n}{df}\PYG{o}{.}\PYG{n}{head}\PYG{p}{(}\PYG{l+m+mi}{15}\PYG{p}{)}\PYG{p}{)}
\PYG{n+nb}{print}\PYG{p}{(}\PYG{l+s+s2}{\PYGZdq{}}\PYG{l+s+se}{\PYGZbs{}n}\PYG{l+s+s2}{ Estadística básica}\PYG{l+s+s2}{\PYGZdq{}}\PYG{p}{)}
\PYG{n}{display}\PYG{p}{(}\PYG{n}{df}\PYG{o}{.}\PYG{n}{describe}\PYG{p}{(}\PYG{p}{)}\PYG{o}{.}\PYG{n}{transpose}\PYG{p}{(}\PYG{p}{)}\PYG{p}{)}
\end{sphinxVerbatim}

\end{sphinxuseclass}\end{sphinxVerbatimInput}
\begin{sphinxVerbatimOutput}

\begin{sphinxuseclass}{cell_output}
\begin{sphinxVerbatim}[commandchars=\\\{\}]
Forma: (209, 4)

 Visualización de los 20 primeros registros
\end{sphinxVerbatim}

\begin{sphinxVerbatim}[commandchars=\\\{\}]
    Peso (g)  Longitud (cm)  Anchura (cm)  Altura (cm)
0       0.46            3.3           1.3          0.2
1       1.08            4.5           1.1          0.3
2       0.67            3.9           1.5          0.2
3       0.98            4.4           1.7          0.3
4       0.93            4.2           1.8          0.3
5       1.89            4.5           2.0          0.4
6       1.60            5.1           1.8          0.3
7       1.90            5.0           2.0          0.3
8       1.59            5.4           1.9          0.3
9       1.67            5.4           1.9          0.3
10      1.91            5.4           1.9          0.3
11      1.94            5.4           1.9          0.3
12      2.08            5.5           1.9          0.3
13      1.72            5.4           2.0          0.3
14      1.77            5.4           2.0          0.3
\end{sphinxVerbatim}

\begin{sphinxVerbatim}[commandchars=\\\{\}]
 Estadística básica
\end{sphinxVerbatim}

\begin{sphinxVerbatim}[commandchars=\\\{\}]
               count      mean       std   min   25\PYGZpc{}   50\PYGZpc{}   75\PYGZpc{}    max
Peso (g)       209.0  5.344641  3.635514  0.46  2.82  4.29  6.96  21.98
Longitud (cm)  209.0  7.244498  1.505369  3.30  6.10  7.00  8.30  11.40
Anchura (cm)   209.0  2.788995  0.698297  1.10  2.30  2.70  3.30   5.20
Altura (cm)    209.0  0.462679  0.117033  0.20  0.40  0.50  0.50   0.90
\end{sphinxVerbatim}

\end{sphinxuseclass}\end{sphinxVerbatimOutput}

\end{sphinxuseclass}

\subsection{Distribuciones univariantes}
\label{\detokenize{content/02/Dataset:distribuciones-univariantes}}
\begin{sphinxuseclass}{cell}\begin{sphinxVerbatimInput}

\begin{sphinxuseclass}{cell_input}
\begin{sphinxVerbatim}[commandchars=\\\{\}]
\PYG{c+c1}{\PYGZsh{} DISTRIBUCIONES UNIVARIANTES}
\PYG{c+c1}{\PYGZsh{} \PYGZhy{}\PYGZhy{}\PYGZhy{}\PYGZhy{}\PYGZhy{}\PYGZhy{}\PYGZhy{}\PYGZhy{}\PYGZhy{}\PYGZhy{}\PYGZhy{}\PYGZhy{}\PYGZhy{}\PYGZhy{}\PYGZhy{}\PYGZhy{}\PYGZhy{}\PYGZhy{}\PYGZhy{}\PYGZhy{}\PYGZhy{}\PYGZhy{}\PYGZhy{}\PYGZhy{}\PYGZhy{}\PYGZhy{}\PYGZhy{}}
\PYG{n}{numeric\PYGZus{}cols} \PYG{o}{=} \PYG{p}{[}\PYG{l+s+s1}{\PYGZsq{}}\PYG{l+s+s1}{Peso (g)}\PYG{l+s+s1}{\PYGZsq{}}\PYG{p}{,} \PYG{l+s+s1}{\PYGZsq{}}\PYG{l+s+s1}{Longitud (cm)}\PYG{l+s+s1}{\PYGZsq{}}\PYG{p}{,} \PYG{l+s+s1}{\PYGZsq{}}\PYG{l+s+s1}{Anchura (cm)}\PYG{l+s+s1}{\PYGZsq{}}\PYG{p}{,} \PYG{l+s+s1}{\PYGZsq{}}\PYG{l+s+s1}{Altura (cm)}\PYG{l+s+s1}{\PYGZsq{}}\PYG{p}{]}

\PYG{k}{for} \PYG{n}{col} \PYG{o+ow}{in} \PYG{n}{numeric\PYGZus{}cols}\PYG{p}{:}
    \PYG{n}{fig}\PYG{p}{,} \PYG{n}{ax} \PYG{o}{=} \PYG{n}{plt}\PYG{o}{.}\PYG{n}{subplots}\PYG{p}{(}\PYG{l+m+mi}{1}\PYG{p}{,} \PYG{l+m+mi}{2}\PYG{p}{,} \PYG{n}{figsize}\PYG{o}{=}\PYG{p}{(}\PYG{l+m+mi}{10}\PYG{p}{,} \PYG{l+m+mi}{4}\PYG{p}{)}\PYG{p}{)}
    \PYG{n}{sns}\PYG{o}{.}\PYG{n}{histplot}\PYG{p}{(}\PYG{n}{df}\PYG{p}{[}\PYG{n}{col}\PYG{p}{]}\PYG{p}{,} \PYG{n}{kde}\PYG{o}{=}\PYG{k+kc}{True}\PYG{p}{,} \PYG{n}{ax}\PYG{o}{=}\PYG{n}{ax}\PYG{p}{[}\PYG{l+m+mi}{0}\PYG{p}{]}\PYG{p}{,} \PYG{n}{color}\PYG{o}{=}\PYG{l+s+s2}{\PYGZdq{}}\PYG{l+s+s2}{\PYGZsh{}3EADB0}\PYG{l+s+s2}{\PYGZdq{}}\PYG{p}{)}
    \PYG{n}{ax}\PYG{p}{[}\PYG{l+m+mi}{0}\PYG{p}{]}\PYG{o}{.}\PYG{n}{set\PYGZus{}title}\PYG{p}{(}\PYG{l+s+sa}{f}\PYG{l+s+s2}{\PYGZdq{}}\PYG{l+s+s2}{Histograma \PYGZhy{} }\PYG{l+s+si}{\PYGZob{}}\PYG{n}{col}\PYG{l+s+si}{\PYGZcb{}}\PYG{l+s+s2}{\PYGZdq{}}\PYG{p}{)}
    \PYG{n}{sns}\PYG{o}{.}\PYG{n}{violinplot}\PYG{p}{(}\PYG{n}{y}\PYG{o}{=}\PYG{n}{df}\PYG{p}{[}\PYG{n}{col}\PYG{p}{]}\PYG{p}{,} \PYG{n}{ax}\PYG{o}{=}\PYG{n}{ax}\PYG{p}{[}\PYG{l+m+mi}{1}\PYG{p}{]}\PYG{p}{,} \PYG{n}{inner}\PYG{o}{=}\PYG{l+s+s2}{\PYGZdq{}}\PYG{l+s+s2}{quartile}\PYG{l+s+s2}{\PYGZdq{}}\PYG{p}{)}
    \PYG{n}{ax}\PYG{p}{[}\PYG{l+m+mi}{1}\PYG{p}{]}\PYG{o}{.}\PYG{n}{set\PYGZus{}title}\PYG{p}{(}\PYG{l+s+sa}{f}\PYG{l+s+s2}{\PYGZdq{}}\PYG{l+s+s2}{Violinplot \PYGZhy{} }\PYG{l+s+si}{\PYGZob{}}\PYG{n}{col}\PYG{l+s+si}{\PYGZcb{}}\PYG{l+s+s2}{\PYGZdq{}}\PYG{p}{)}
    \PYG{n}{plt}\PYG{o}{.}\PYG{n}{tight\PYGZus{}layout}\PYG{p}{(}\PYG{p}{)}
\end{sphinxVerbatim}

\end{sphinxuseclass}\end{sphinxVerbatimInput}
\begin{sphinxVerbatimOutput}

\begin{sphinxuseclass}{cell_output}
\noindent\sphinxincludegraphics{{164f9ed551f0c0f5142355ae586e9e2a19570f5ab9d8ddc807f3ed5c875958b0}.png}

\noindent\sphinxincludegraphics{{ea37e27b71cebb779007811c459223cf4824825a22d820ffcf24b04cd9fd6df0}.png}

\noindent\sphinxincludegraphics{{bdc4e67ea39df3fbf665dee1abf4e8ad27ad41ab749842123b4935f7905084ea}.png}

\noindent\sphinxincludegraphics{{11cd6f82662f825d1cadf8bbbf6e4a31fc5546ce95626629d71aa89663b628e5}.png}

\end{sphinxuseclass}\end{sphinxVerbatimOutput}

\end{sphinxuseclass}

\begin{savenotes}\sphinxattablestart
\sphinxthistablewithglobalstyle
\centering
\begin{tabulary}{\linewidth}[t]{TTT}
\sphinxtoprule
\sphinxstyletheadfamily
\sphinxAtStartPar
\sphinxstylestrong{Variable}
&\sphinxstyletheadfamily
\sphinxAtStartPar
\sphinxstylestrong{Forma del violín}
&\sphinxstyletheadfamily
\sphinxAtStartPar
\sphinxstylestrong{Interpretación}
\\
\sphinxmidrule
\sphinxtableatstartofbodyhook
\sphinxAtStartPar
\sphinxstylestrong{Peso (g)}
&
\sphinxAtStartPar
Ancho entre \sphinxstylestrong{3 – 7 g}, cola larga y estrecha hasta ≈ 22 g, varios puntos fuera del bigote superior.
&
\sphinxAtStartPar
Distribución \sphinxstylestrong{fuertemente asimétrica a la derecha}: la mayoría son ejemplares pequeños‑medios y unos pocos muy pesados (outliers).
\\
\sphinxhline
\sphinxAtStartPar
\sphinxstylestrong{Longitud (cm)}
&
\sphinxAtStartPar
Violín casi simétrico, más ancho en 6 – 8 cm, cola superior corta, sin puntos fuera del bigote.
&
\sphinxAtStartPar
Distribución \sphinxstylestrong{unimodal y estable}; no hay valores extremos claros, buena candidata a predictor base.
\\
\sphinxhline
\sphinxAtStartPar
\sphinxstylestrong{Anchura (cm)}
&
\sphinxAtStartPar
Pico ancho en 2,5 – 3,5 cm, cola superior hasta 5,2 cm con muy poca densidad, 1‑2 puntos aislados.
&
\sphinxAtStartPar
Forma similar a Longitud pero con \sphinxstylestrong{cola alta leve}; unos pocos peces muy anchos podrían ser anomalías o errores.
\\
\sphinxhline
\sphinxAtStartPar
\sphinxstylestrong{Altura (cm)}
&
\sphinxAtStartPar
Violín estrecho (0,4 – 0,5 cm habituales), cola fina hasta 0,9 cm, varios puntos aislados arriba.
&
\sphinxAtStartPar
Variable con \sphinxstylestrong{rango pequeño}; valores altos destacan como outliers y podrían indicar errores de medición o casos excepcionales.
\\
\sphinxbottomrule
\end{tabulary}
\sphinxtableafterendhook\par
\sphinxattableend\end{savenotes}


\subsection{Distribuciones bivariantes}
\label{\detokenize{content/02/Dataset:distribuciones-bivariantes}}
\begin{sphinxuseclass}{cell}\begin{sphinxVerbatimInput}

\begin{sphinxuseclass}{cell_input}
\begin{sphinxVerbatim}[commandchars=\\\{\}]
\PYG{n}{sns}\PYG{o}{.}\PYG{n}{pairplot}\PYG{p}{(}\PYG{n}{df}\PYG{p}{[}\PYG{n}{numeric\PYGZus{}cols}\PYG{p}{]}\PYG{p}{,} \PYG{n}{diag\PYGZus{}kind}\PYG{o}{=}\PYG{l+s+s2}{\PYGZdq{}}\PYG{l+s+s2}{kde}\PYG{l+s+s2}{\PYGZdq{}}\PYG{p}{)}
\PYG{n}{plt}\PYG{o}{.}\PYG{n}{suptitle}\PYG{p}{(}\PYG{l+s+s2}{\PYGZdq{}}\PYG{l+s+s2}{Pairplot variables morfológicas}\PYG{l+s+s2}{\PYGZdq{}}\PYG{p}{,} \PYG{n}{y}\PYG{o}{=}\PYG{l+m+mf}{1.02}\PYG{p}{)}\PYG{p}{;}
\end{sphinxVerbatim}

\end{sphinxuseclass}\end{sphinxVerbatimInput}
\begin{sphinxVerbatimOutput}

\begin{sphinxuseclass}{cell_output}
\noindent\sphinxincludegraphics{{e30b8a547eca74ca6043545a14d481946cea86b21fa4b65e5479d9d5ae63cd29}.png}

\end{sphinxuseclass}\end{sphinxVerbatimOutput}

\end{sphinxuseclass}
\sphinxAtStartPar
El análisis de las relaciones bivariantes muestra que el peso crece exponencialmente con la Longitud y la Anchura, dibujando nubes curvilíneas donde los puntos se vuelven más dispersos a medida que el pez gana tamaño; eso confirma que la relación no es estrictamente lineal. Longitud y Anchura mantienen entre sí una correlación positiva moderada los peces más largos tienden a ser más anchos mientras que Altura se agrupa fuertemente en torno a un intervalo estrecho y, salvo algunos valores altos anómalos, aporta poca variabilidad adicional. En todas las combinaciones que incluyen Peso se aprecian varios puntos solitarios por encima de la nube principal, lo que corrobora la presencia de \sphinxstyleemphasis{outliers}. No se observan clusters separados que sugieran sub\sphinxhyphen{}poblaciones distintas, sino una gradación continua, de modo que los valores extremos se interpretan mejor como individuos excepcionales (o posibles errores) que como grupos diferenciados.


\subsection{Matriz de correlación}
\label{\detokenize{content/02/Dataset:matriz-de-correlacion}}
\begin{sphinxuseclass}{cell}\begin{sphinxVerbatimInput}

\begin{sphinxuseclass}{cell_input}
\begin{sphinxVerbatim}[commandchars=\\\{\}]
\PYG{c+c1}{\PYGZsh{} 2. Calcular matriz de correlación (método de Pearson por defecto)}
\PYG{n}{corr\PYGZus{}matrix} \PYG{o}{=} \PYG{n}{df}\PYG{o}{.}\PYG{n}{corr}\PYG{p}{(}\PYG{p}{)}

\PYG{c+c1}{\PYGZsh{} 3. Crear el gráfico}
\PYG{n}{plt}\PYG{o}{.}\PYG{n}{figure}\PYG{p}{(}\PYG{n}{figsize}\PYG{o}{=}\PYG{p}{(}\PYG{l+m+mi}{10}\PYG{p}{,} \PYG{l+m+mi}{8}\PYG{p}{)}\PYG{p}{)}
\PYG{n}{sns}\PYG{o}{.}\PYG{n}{heatmap}\PYG{p}{(}\PYG{n}{corr\PYGZus{}matrix}\PYG{p}{,}
            \PYG{n}{annot}\PYG{o}{=}\PYG{k+kc}{True}\PYG{p}{,}       \PYG{c+c1}{\PYGZsh{} Muestra valores en cada celda}
            \PYG{n}{fmt}\PYG{o}{=}\PYG{l+s+s2}{\PYGZdq{}}\PYG{l+s+s2}{.2f}\PYG{l+s+s2}{\PYGZdq{}}\PYG{p}{,}       \PYG{c+c1}{\PYGZsh{} Formato de 2 decimales}
            \PYG{n}{cmap}\PYG{o}{=}\PYG{l+s+s1}{\PYGZsq{}}\PYG{l+s+s1}{plasma}\PYG{l+s+s1}{\PYGZsq{}}\PYG{p}{,}  \PYG{c+c1}{\PYGZsh{} Mapa de colores (rojo\PYGZhy{}negro\PYGZhy{}azul)}
            \PYG{n}{vmin}\PYG{o}{=}\PYG{o}{\PYGZhy{}}\PYG{l+m+mi}{1}\PYG{p}{,} \PYG{n}{vmax}\PYG{o}{=}\PYG{l+m+mi}{1}\PYG{p}{,}  \PYG{c+c1}{\PYGZsh{} Rango de correlación [\PYGZhy{}1, 1]}
            \PYG{n}{linewidths}\PYG{o}{=}\PYG{l+m+mf}{0.5}\PYG{p}{,} \PYG{c+c1}{\PYGZsh{} Grosor de líneas entre celdas}
            \PYG{n}{mask}\PYG{o}{=}\PYG{n}{np}\PYG{o}{.}\PYG{n}{triu}\PYG{p}{(}\PYG{n}{np}\PYG{o}{.}\PYG{n}{ones\PYGZus{}like}\PYG{p}{(}\PYG{n}{corr\PYGZus{}matrix}\PYG{p}{,} \PYG{n}{dtype}\PYG{o}{=}\PYG{n+nb}{bool}\PYG{p}{)}\PYG{p}{)}  \PYG{c+c1}{\PYGZsh{} Mostrar solo triangular inferior}
           \PYG{p}{)}

\PYG{c+c1}{\PYGZsh{} 4. Personalizar}
\PYG{n}{plt}\PYG{o}{.}\PYG{n}{title}\PYG{p}{(}\PYG{l+s+s1}{\PYGZsq{}}\PYG{l+s+s1}{Matriz de Correlación de Variables}\PYG{l+s+s1}{\PYGZsq{}}\PYG{p}{,} \PYG{n}{pad}\PYG{o}{=}\PYG{l+m+mi}{20}\PYG{p}{,} \PYG{n}{fontsize}\PYG{o}{=}\PYG{l+m+mi}{14}\PYG{p}{)}
\PYG{n}{plt}\PYG{o}{.}\PYG{n}{xticks}\PYG{p}{(}\PYG{n}{rotation}\PYG{o}{=}\PYG{l+m+mi}{45}\PYG{p}{)}
\PYG{n}{plt}\PYG{o}{.}\PYG{n}{yticks}\PYG{p}{(}\PYG{n}{rotation}\PYG{o}{=}\PYG{l+m+mi}{0}\PYG{p}{)}
\PYG{n}{plt}\PYG{o}{.}\PYG{n}{tight\PYGZus{}layout}\PYG{p}{(}\PYG{p}{)}
\PYG{n}{plt}\PYG{o}{.}\PYG{n}{show}\PYG{p}{(}\PYG{p}{)}
\end{sphinxVerbatim}

\end{sphinxuseclass}\end{sphinxVerbatimInput}
\begin{sphinxVerbatimOutput}

\begin{sphinxuseclass}{cell_output}
\noindent\sphinxincludegraphics{{c94ee8fc37810363daaf4753953f89542fdf6cec6e58035f4aa80fec2c6082f1}.png}

\end{sphinxuseclass}\end{sphinxVerbatimOutput}

\end{sphinxuseclass}
\sphinxAtStartPar
La matriz de correlación indica que, en el lenguado, el peso está fuertemente asociado con la longitud y la anchura (r≈0,94), mientras que su asociación con la altura (espesor corporal) es claramente menor (r≈0,86). En términos de varianza explicada, longitud y anchura individualmente capturan \textasciitilde{}88\% del comportamiento del peso (r²≈0,8836), frente a \textasciitilde{}74\% para la altura (r²≈0,7396). Esto es coherente con la biomecánica del pez plano: el peso depende del volumen, pero en especies deprimidas dorsoventralmente la variabilidad del área proyectada (aprox. longitud×anchura) domina sobre la variación del espesor. La correlación igualmente alta entre longitud y anchura (r≈0,94) sugiere una morfología con proporciones relativamente estables durante el crecimiento (isometría planiforme) o, al menos, una fuerte covariación de ambas dimensiones; en la práctica, esto implica multicolinealidad si se usan juntas en regresión. La correlación más baja de peso con la altura indica que el espesor es más variable entre individuos (condición corporal, estado nutricional, maduración gonadal), aportando información menos estable sobre el peso que las dimensiones planas. En conjunto, estas correlaciones significan que la masa del lenguado está principalmente determinada por su extensión superficial y que la altura añade variabilidad más idiosincrática.

\begin{sphinxadmonition}{note}{Varianza explicada}

\sphinxAtStartPar
Sea \(Y=\text{Peso}\) y \(X\in\{\text{Longitud},\text{Anchura},\text{Altura}\}\).

\sphinxAtStartPar
En regresión lineal simple se cumple:
\$\(
R^2 = r^2,
\)\(
donde \)r=\textbackslash{}mathrm\{corr\}(X,Y)\$.

\sphinxAtStartPar
Con los valores obtenidos en la matriz de correlación tenemos que:
\$\(
r_{Y,\text{Longitud}}=0.94,\qquad
r_{Y,\text{Anchura}}=0.94,\qquad
r_{Y,\text{Altura}}=0.86.
\)\$

\sphinxAtStartPar
Por tanto,
\begin{itemize}
\item {}
\sphinxAtStartPar
\(R^2_{\text{Peso}\sim\text{Longitud}} = 0.94^2 = 0.8836 \approx 88.36\%\)

\item {}
\sphinxAtStartPar
\(R^2_{\text{Peso}\sim\text{Ancgurad}} = 0.94^2 = 0.8836 \approx 88.36\%\)

\item {}
\sphinxAtStartPar
\(R^2_{\text{Peso}\sim\text{Altura}} = 0.86^2 = 0.7396 \approx 74\%\)

\end{itemize}
\end{sphinxadmonition}

\sphinxstepscope


\chapter{Generación de datos sintéticos}
\label{\detokenize{content/02/Generador:generacion-de-datos-sinteticos}}\label{\detokenize{content/02/Generador::doc}}
\begin{sphinxadmonition}{note}{Resumen}

\sphinxAtStartPar
Este artículo presenta las líneas de trabajo desarrolladas para la generación de un dataset sintético de tamaño y peso de lenguados, con el objetivo de proporcionar una base de datos suficientemente amplia y representativa para el entrenamiento, validación y mejora de modelos predictivos. Se recogen las metodologías utilizadas en la generación de datos sintéticos, incluyendo la modelización estadística de distribuciones empíricas, técnicas de simulación basadas en procesos de crecimiento biológico y enfoques de aprendizaje automático para la síntesis de datos realistas. Además, se analizan los criterios de validación empleados para garantizar que los datos generados reflejen fielmente las tendencias y variabilidad observadas en poblaciones reales de lenguado (\sphinxstyleemphasis{Solea solea}), asegurando así su utilidad en el desarrollo de algoritmos precisos y generalizables para la predicción del peso a partir de variables alometricas.

\sphinxAtStartPar
\sphinxstylestrong{Entregable}: E2.2\\
\sphinxstylestrong{Versión}: 1.0\\
\sphinxstylestrong{Autor}: Javier Álvarez Osuna\\
\sphinxstylestrong{Email}: javier.osuna@fishfarmfeeder.com\\
\sphinxstylestrong{ORCID}: \sphinxhref{https://orcid.org/0000-0001-7063-1279}{0000\sphinxhyphen{}0001\sphinxhyphen{}7063\sphinxhyphen{}1279}\\
\sphinxstylestrong{Licencia}: CC\sphinxhyphen{}BY\sphinxhyphen{}4.0\\
\sphinxstylestrong{Código proyecto}: IG408M.2025.000.000072

\noindent{\hspace*{\fill}\sphinxincludegraphics[width=1.000\linewidth]{{FLATCLASS_logo_publicidad}.png}\hspace*{\fill}}
\end{sphinxadmonition}


\section{Introducción}
\label{\detokenize{content/02/Generador:introduccion}}
\sphinxAtStartPar
En el ámbito de la acuicultura de precisión, la caracterización morfométrica de los peces y su relación con el peso corporal constituye un eje central para la optimización de procesos como la clasificación automática, el control de crecimiento y la dosificación alimentaria. En el caso particular de los peces planos en fase de alevinaje —como el lenguado (Solea solea) o el rodaballo (Scophthalmus maximus)—, las variables morfométricas fundamentales incluyen la longitud corporal, la anchura transversal y la altura dorso\sphinxhyphen{}ventral, parámetros que definen la geometría del individuo y que se presumen relacionados de forma sistemática con la biomasa individual.

\sphinxAtStartPar
La necesidad de disponer de un dataset suficientemente amplio, representativo y multivariado, que relacione estas variables morfométricas con el peso corporal correspondiente, responde a múltiples consideraciones de carácter estadístico, biológico y computacional. Aun en ausencia de un modelo alométrico explícito que relacione de forma determinista dichas variables, es posible anticipar que cualquier estrategia de inferencia o predicción del peso basada en dimensiones requerirá una densidad adecuada de datos en el espacio tridimensional definido por longitud, anchura y altura. Este requisito es crítico para garantizar tanto la fidelidad del ajuste como la capacidad de generalización del modelo aprendido.

\sphinxAtStartPar
Cuando el volumen de datos disponibles es reducido, surgen una serie de limitaciones estructurales:
\begin{itemize}
\item {}
\sphinxAtStartPar
\sphinxstylestrong{Alta varianza en la estimación de parámetros}: La precisión de los modelos predictivos decae significativamente cuando las observaciones son escasas o están mal distribuidas en el dominio de entrada.

\item {}
\sphinxAtStartPar
\sphinxstylestrong{Riesgo de sobreajuste}: En entornos de datos reducidos, los modelos tienden a capturar ruido en lugar de relaciones funcionales genuinas, lo cual compromete la validez externa.

\item {}
\sphinxAtStartPar
\sphinxstylestrong{Cobertura insuficiente del espacio morfométrico}: Se produce una pérdida de representatividad en las regiones marginales del dominio, lo que reduce la capacidad del sistema para extrapolar o interpolar en condiciones reales de producción.

\item {}
\sphinxAtStartPar
\sphinxstylestrong{Sesgos estructurales}: Las muestras pequeñas pueden reflejar sesgos en las condiciones de cría, genética o instrumentación, induciendo patrones espurios no generalizables.

\end{itemize}

\sphinxAtStartPar
Desde la perspectiva de la matemática probabilística, esta problemática puede entenderse mediante el marco de la inferencia bayesiana. En este enfoque, el conocimiento sobre los parámetros \(\theta\) (por ejemplo, la relación entre morfología y peso) se representa como una distribución posterior condicionada a los datos \(D\):
\begin{equation*}
\begin{split}
p(\theta | D) \propto p(D | \theta) \cdot p(\theta)
\end{split}
\end{equation*}
\sphinxAtStartPar
donde \(D\) representa los datos observados. Cuando el tamaño de \(D\) es reducido, la función de verosimilitud \(p(D|a,b)\) tiene una varianza alta, lo que genera estimaciones más inciertas. Al aumentar el tamaño de \(D\) con datos sintéticos plausibles, la estimación de \(p(a,b|D)\) se vuelve más precisa, reduciendo la varianza de los parámetros.

\sphinxAtStartPar
En el contexto de la acuicultura, la recopilación masiva de medidas biométricas de precisión en alevines presenta limitaciones logísticas y económicas significativas. Por tanto, el uso inteligencia artificial para permite simular observaciones adicionales coherentes con la distribución empírica observada, conservando las correlaciones entre las dimensiones corporales y el peso de los individuos.

\sphinxAtStartPar
En este trabajo se abordó la generación de datos sintéticos mediante tres métodos complementarios: \sphinxstylestrong{Gaussian Copula}, \sphinxstylestrong{CTGAN} y \sphinxstylestrong{TVAE}. Esta aproximación multicriterio permite evaluar de manera comparativa la capacidad de cada técnica para replicar no solo las distribuciones marginales de los datos originales, sino también las dependencias no lineales propias de los modelos alométricos, típicos en biología de organismos. La selección final del método más adecuado se basará en un análisis cuantitativo que incluye: (1) métricas de evaluación de la calidad sintética (KS\sphinxhyphen{}test, divergencia KL), (2) precisión en la replicación de las ecuaciones alométricas (RMSE entre valores reales y predichos), y (3) viabilidad computacional. Esta comparación sistemática garantizará que los datos sintéticos amplificados mantengan validez ecológica y estadística, priorizando el método que mejor equilibre fidelidad biológica y escalabilidad, para ser usados en los modelos predictivos de peso.

\sphinxAtStartPar
Los estudios sobre generación de datos sintéticos llevasoa a cabo en el marco de FLATCLASS responden por tanto, a una doble motivación: por un lado, \sphinxstylestrong{ampliar artificialmente el conjunto de datos disponible} para entrenar modelos predictivos del peso a partir de variables morfométricas; y por otro, \sphinxstylestrong{mantener la coherencia estadística y biológica} de los registros generados, minimizando los riesgos de sobreajuste y mejorando la capacidad de generalización de los modelos desarrollados.

\sphinxstepscope


\section{Gaussian Copula}
\label{\detokenize{content/02/GaussianCopula:gaussian-copula}}\label{\detokenize{content/02/GaussianCopula::doc}}
\sphinxAtStartPar
El método Gaussian Copula destaca por su capacidad para modelar dependencias multivariadas entre variables, preservando la estructura estadística de los datos originales. Este enfoque se basa en la teoría de \sphinxstylestrong{cópulas}, que permite desacoplar las distribuciones marginales de las dependencias entre variables, ofreciendo flexibilidad para generar datos sintéticos realistas {[}\sphinxhref{https://arxiv.org/pdf/2009.09471}{Zheng et al., 2020}{]}.

\sphinxAtStartPar
Las \sphinxstylestrong{cópulas} son funciones que vinculan distribuciones marginales univariadas para formar una distribución multivariada. El teorema de Sklar (1959) establece que cualquier distribución conjunta \(H\) de variables aleatorias \((X_1,X_2,...,X_d)\) puede expresarse como:
\begin{equation*}
\begin{split}H(x_1,x_2,...,x_d)=C(F_1(x_1),F_2(x_2),...,F_d(d_d))\end{split}
\end{equation*}
\sphinxAtStartPar
donde \(C\) es la cópula y \(F_i\) son las funciones de distribución marginal. La Gaussian Copula utiliza una cópula derivada de la distribución normal multivariada, definida como:
\begin{equation*}
\begin{split}
C_\Sigma(\mathbf{u}) = \Phi_\Sigma \left( \Phi^{-1}(u_1), \Phi^{-1}(u_2), \dots, \Phi^{-1}(u_d) \right)
\end{split}
\end{equation*}
\sphinxAtStartPar
donde:
\begin{itemize}
\item {}
\sphinxAtStartPar
\(\Phi_\Sigma\): es la función de distribución normal multivariada con matriz de covarianza \(\Sigma\)

\item {}
\sphinxAtStartPar
\(\Phi^{-1}\): es la inversa de la distribución normal estándar (quantile function)

\item {}
\sphinxAtStartPar
\(u=(u_1,u_2,...,u_d)\): vector de probabilidades uniformes en \([0,1]\)

\end{itemize}

\sphinxAtStartPar
Los fundamentos matemáticos anteriormente descritos, incluyendo el teorema de Sklar, la transformación al espacio gaussiano mediante funciones inversas \((Φ^{-1)}\), y el muestreo de datos sintéticos basado en la matriz de covarianza \((Σ)\), están implementados de forma eficiente y optimizada dentro de la librería \sphinxcode{\sphinxupquote{SDV (Synthetic Data Vault)}}. En particular, la clase \sphinxcode{\sphinxupquote{GaussianCopula}} del módulo \sphinxcode{\sphinxupquote{sdv.tabular}} encapsula estos modelos matemáticos, permitiendo la generación de datos tabulares sintéticos que preservan tanto las distribuciones marginales \((F_i)\) como las dependencias multivariadas \((C_Σ)\) presentes en los datos originales. La librería automatiza procesos clave como:
\begin{enumerate}
\sphinxsetlistlabels{\arabic}{enumi}{enumii}{}{.}%
\item {}
\sphinxAtStartPar
La estimación no paramétrica de distribuciones marginales,

\item {}
\sphinxAtStartPar
El cálculo de la matriz de correlación de rangos (rank correlation), y

\item {}
\sphinxAtStartPar
La generación de muestras sintéticas mediante inversión de la cópula \((F_{i}^{-1})\).

\end{enumerate}

\sphinxAtStartPar
Además, SDV incorpora validaciones internas para garantizar que los datos generados mantengan propiedades estadísticas consistentes, siguiendo las mejores prácticas descritas en la literatura {[}\sphinxhref{https://doi.org/10.1109/DSAA.2016.49}{Patki et al., 2016}{]}.

\begin{sphinxuseclass}{cell}\begin{sphinxVerbatimInput}

\begin{sphinxuseclass}{cell_input}
\begin{sphinxVerbatim}[commandchars=\\\{\}]
\PYG{c+c1}{\PYGZsh{} Importar librerías necesarias}
\PYG{k+kn}{import}\PYG{+w}{ }\PYG{n+nn}{pandas}\PYG{+w}{ }\PYG{k}{as}\PYG{+w}{ }\PYG{n+nn}{pd}
\PYG{k+kn}{import}\PYG{+w}{ }\PYG{n+nn}{numpy}\PYG{+w}{ }\PYG{k}{as}\PYG{+w}{ }\PYG{n+nn}{np}
\PYG{k+kn}{import}\PYG{+w}{ }\PYG{n+nn}{scipy}
\PYG{k+kn}{import}\PYG{+w}{ }\PYG{n+nn}{matplotlib}\PYG{n+nn}{.}\PYG{n+nn}{pyplot}\PYG{+w}{ }\PYG{k}{as}\PYG{+w}{ }\PYG{n+nn}{plt}

\PYG{k+kn}{from}\PYG{+w}{ }\PYG{n+nn}{sdv}\PYG{n+nn}{.}\PYG{n+nn}{single\PYGZus{}table}\PYG{+w}{ }\PYG{k+kn}{import} \PYG{n}{GaussianCopulaSynthesizer}
\PYG{k+kn}{from}\PYG{+w}{ }\PYG{n+nn}{sdv}\PYG{n+nn}{.}\PYG{n+nn}{metadata}\PYG{+w}{ }\PYG{k+kn}{import} \PYG{n}{Metadata}
\PYG{k+kn}{from}\PYG{+w}{ }\PYG{n+nn}{pathlib}\PYG{+w}{ }\PYG{k+kn}{import} \PYG{n}{Path}
\PYG{k+kn}{from}\PYG{+w}{ }\PYG{n+nn}{myst\PYGZus{}nb}\PYG{+w}{ }\PYG{k+kn}{import} \PYG{n}{glue}

\PYG{n}{num\PYGZus{}synthetic} \PYG{o}{=} \PYG{l+m+mi}{500} \PYG{c+c1}{\PYGZsh{}variable para definir el número de registros sintéticos}

\PYG{c+c1}{\PYGZsh{} 1. Cargar datos reales}
\PYG{c+c1}{\PYGZsh{}\PYGZhy{}\PYGZhy{}\PYGZhy{}\PYGZhy{}\PYGZhy{}\PYGZhy{}\PYGZhy{}\PYGZhy{}\PYGZhy{}\PYGZhy{}\PYGZhy{}\PYGZhy{}\PYGZhy{}\PYGZhy{}\PYGZhy{}\PYGZhy{}\PYGZhy{}\PYGZhy{}\PYGZhy{}\PYGZhy{}\PYGZhy{}\PYGZhy{}\PYGZhy{}\PYGZhy{}\PYGZhy{}\PYGZhy{}}
\PYG{n}{path\PYGZus{}realData} \PYG{o}{=} \PYG{l+s+s1}{\PYGZsq{}}\PYG{l+s+s1}{.././data/Dimensiones\PYGZus{}lenguado.xlsx}\PYG{l+s+s1}{\PYGZsq{}}
\PYG{n}{real\PYGZus{}data} \PYG{o}{=} \PYG{n}{pd}\PYG{o}{.}\PYG{n}{read\PYGZus{}excel}\PYG{p}{(}\PYG{n}{path\PYGZus{}realData}\PYG{p}{)}

\PYG{c+c1}{\PYGZsh{} 2. Obtener los metadata del dataset}
\PYG{c+c1}{\PYGZsh{}\PYGZhy{}\PYGZhy{}\PYGZhy{}\PYGZhy{}\PYGZhy{}\PYGZhy{}\PYGZhy{}\PYGZhy{}\PYGZhy{}\PYGZhy{}\PYGZhy{}\PYGZhy{}\PYGZhy{}\PYGZhy{}\PYGZhy{}\PYGZhy{}\PYGZhy{}\PYGZhy{}\PYGZhy{}\PYGZhy{}\PYGZhy{}\PYGZhy{}\PYGZhy{}\PYGZhy{}\PYGZhy{}\PYGZhy{}\PYGZhy{}\PYGZhy{}\PYGZhy{}\PYGZhy{}\PYGZhy{}\PYGZhy{}\PYGZhy{}\PYGZhy{}\PYGZhy{}\PYGZhy{}\PYGZhy{}\PYGZhy{}}

\PYG{n}{metadata\PYGZus{}path} \PYG{o}{=} \PYG{n}{Path}\PYG{p}{(}\PYG{l+s+s1}{\PYGZsq{}}\PYG{l+s+s1}{.././data/metadata\PYGZus{}lenguado.json}\PYG{l+s+s1}{\PYGZsq{}}\PYG{p}{)}

\PYG{k}{if} \PYG{n}{metadata\PYGZus{}path}\PYG{o}{.}\PYG{n}{exists}\PYG{p}{(}\PYG{p}{)}\PYG{p}{:}
    \PYG{c+c1}{\PYGZsh{} Cargar metadatos desde el JSON (evita el warning)}
    \PYG{n}{metadata} \PYG{o}{=} \PYG{n}{Metadata}\PYG{o}{.}\PYG{n}{load\PYGZus{}from\PYGZus{}json}\PYG{p}{(}\PYG{n}{metadata\PYGZus{}path}\PYG{p}{)}
    \PYG{n+nb}{print}\PYG{p}{(}\PYG{l+s+s2}{\PYGZdq{}}\PYG{l+s+s2}{Metadatos cargados desde JSON.}\PYG{l+s+se}{\PYGZbs{}n}\PYG{l+s+s2}{\PYGZdq{}}\PYG{p}{)}
\PYG{k}{else}\PYG{p}{:}
    \PYG{c+c1}{\PYGZsh{} Detectar metadatos y guardarlos en JSON para futuras ejecuciones}
    \PYG{n}{metadata} \PYG{o}{=} \PYG{n}{Metadata}\PYG{o}{.}\PYG{n}{detect\PYGZus{}from\PYGZus{}dataframe}\PYG{p}{(}\PYG{n}{real\PYGZus{}data}\PYG{p}{)}
    \PYG{n}{metadata}\PYG{o}{.}\PYG{n}{save\PYGZus{}to\PYGZus{}json}\PYG{p}{(}\PYG{n}{metadata\PYGZus{}path}\PYG{p}{)}
    \PYG{n+nb}{print}\PYG{p}{(}\PYG{l+s+s2}{\PYGZdq{}}\PYG{l+s+s2}{Metadatos detectados y guardados en JSON.}\PYG{l+s+se}{\PYGZbs{}n}\PYG{l+s+s2}{\PYGZdq{}}\PYG{p}{)}

\PYG{c+c1}{\PYGZsh{} 3. Sintetizar datos}
\PYG{c+c1}{\PYGZsh{}\PYGZhy{}\PYGZhy{}\PYGZhy{}\PYGZhy{}\PYGZhy{}\PYGZhy{}\PYGZhy{}\PYGZhy{}\PYGZhy{}\PYGZhy{}\PYGZhy{}\PYGZhy{}\PYGZhy{}\PYGZhy{}\PYGZhy{}\PYGZhy{}\PYGZhy{}\PYGZhy{}\PYGZhy{}\PYGZhy{}\PYGZhy{}\PYGZhy{}\PYGZhy{}\PYGZhy{}\PYGZhy{}\PYGZhy{}\PYGZhy{}\PYGZhy{}\PYGZhy{}\PYGZhy{}\PYGZhy{}\PYGZhy{}\PYGZhy{}\PYGZhy{}\PYGZhy{}\PYGZhy{}\PYGZhy{}\PYGZhy{}\PYGZhy{}}
\PYG{n}{synthesizer} \PYG{o}{=} \PYG{n}{GaussianCopulaSynthesizer}\PYG{p}{(}
    \PYG{n}{metadata} \PYG{o}{=} \PYG{n}{metadata}\PYG{p}{,}
    \PYG{n}{enforce\PYGZus{}min\PYGZus{}max\PYGZus{}values} \PYG{o}{=} \PYG{k+kc}{True}\PYG{p}{,} \PYG{c+c1}{\PYGZsh{} Forzar datos realísticos}
    \PYG{n}{enforce\PYGZus{}rounding} \PYG{o}{=} \PYG{k+kc}{True} \PYG{c+c1}{\PYGZsh{} Redondeos automáticos}
    \PYG{p}{)}

\PYG{n}{synthesizer}\PYG{o}{.}\PYG{n}{fit}\PYG{p}{(}\PYG{n}{real\PYGZus{}data}\PYG{p}{)}
\PYG{n}{synthetic\PYGZus{}data} \PYG{o}{=} \PYG{n}{synthesizer}\PYG{o}{.}\PYG{n}{sample}\PYG{p}{(}\PYG{n}{num\PYGZus{}rows}\PYG{o}{=} \PYG{n}{num\PYGZus{}synthetic}\PYG{p}{)}\PYG{p}{;}
\PYG{n}{display}\PYG{p}{(}\PYG{n}{synthetic\PYGZus{}data}\PYG{o}{.}\PYG{n}{head}\PYG{p}{(}\PYG{l+m+mi}{10}\PYG{p}{)}\PYG{p}{)}
\PYG{n+nb}{print}\PYG{p}{(}\PYG{l+s+sa}{f}\PYG{l+s+s2}{\PYGZdq{}}\PYG{l+s+s2}{Se muestran 10 primeros registros sintéticos de un total de }\PYG{l+s+si}{\PYGZob{}}\PYG{n}{num\PYGZus{}synthetic}\PYG{l+s+si}{\PYGZcb{}}\PYG{l+s+s2}{\PYGZdq{}}\PYG{p}{)}


\PYG{c+c1}{\PYGZsh{} 4. Guardar satos sintéticos}
\PYG{c+c1}{\PYGZsh{}\PYGZhy{}\PYGZhy{}\PYGZhy{}\PYGZhy{}\PYGZhy{}\PYGZhy{}\PYGZhy{}\PYGZhy{}\PYGZhy{}\PYGZhy{}\PYGZhy{}\PYGZhy{}\PYGZhy{}\PYGZhy{}\PYGZhy{}\PYGZhy{}\PYGZhy{}\PYGZhy{}\PYGZhy{}\PYGZhy{}\PYGZhy{}\PYGZhy{}\PYGZhy{}\PYGZhy{}\PYGZhy{}\PYGZhy{}\PYGZhy{}\PYGZhy{}\PYGZhy{}\PYGZhy{}\PYGZhy{}\PYGZhy{}\PYGZhy{}\PYGZhy{}\PYGZhy{}\PYGZhy{}\PYGZhy{}}
\PYG{n}{path\PYGZus{}syntheticData} \PYG{o}{=} \PYG{l+s+s2}{\PYGZdq{}}\PYG{l+s+s2}{.././data/SyntheticGaussianCopula.xlsx}\PYG{l+s+s2}{\PYGZdq{}}
\PYG{n}{synthetic\PYGZus{}data}\PYG{o}{.}\PYG{n}{to\PYGZus{}excel}\PYG{p}{(}\PYG{n}{path\PYGZus{}syntheticData}\PYG{p}{,} \PYG{n}{index}\PYG{o}{=}\PYG{k+kc}{False}\PYG{p}{,} \PYG{n}{engine}\PYG{o}{=}\PYG{l+s+s1}{\PYGZsq{}}\PYG{l+s+s1}{openpyxl}\PYG{l+s+s1}{\PYGZsq{}}\PYG{p}{)}

\PYG{c+c1}{\PYGZsh{} 5. Validación calidad datos sintéticos}
\PYG{c+c1}{\PYGZsh{}\PYGZhy{}\PYGZhy{}\PYGZhy{}\PYGZhy{}\PYGZhy{}\PYGZhy{}\PYGZhy{}\PYGZhy{}\PYGZhy{}\PYGZhy{}\PYGZhy{}\PYGZhy{}\PYGZhy{}\PYGZhy{}\PYGZhy{}\PYGZhy{}\PYGZhy{}\PYGZhy{}\PYGZhy{}\PYGZhy{}\PYGZhy{}\PYGZhy{}\PYGZhy{}\PYGZhy{}\PYGZhy{}\PYGZhy{}\PYGZhy{}\PYGZhy{}\PYGZhy{}\PYGZhy{}\PYGZhy{}\PYGZhy{}\PYGZhy{}\PYGZhy{}\PYGZhy{}\PYGZhy{}\PYGZhy{}\PYGZhy{}\PYGZhy{}\PYGZhy{}}
\PYG{k+kn}{from}\PYG{+w}{ }\PYG{n+nn}{sdv}\PYG{n+nn}{.}\PYG{n+nn}{evaluation}\PYG{n+nn}{.}\PYG{n+nn}{single\PYGZus{}table}\PYG{+w}{ }\PYG{k+kn}{import} \PYG{n}{run\PYGZus{}diagnostic}\PYG{p}{,} \PYG{n}{evaluate\PYGZus{}quality}
\PYG{k+kn}{from}\PYG{+w}{ }\PYG{n+nn}{sdv}\PYG{n+nn}{.}\PYG{n+nn}{evaluation}\PYG{n+nn}{.}\PYG{n+nn}{single\PYGZus{}table}\PYG{+w}{ }\PYG{k+kn}{import} \PYG{n}{get\PYGZus{}column\PYGZus{}plot}

\PYG{c+c1}{\PYGZsh{}\PYGZsh{} Diagnóstico básico}

\PYG{n}{diagnostic} \PYG{o}{=} \PYG{n}{run\PYGZus{}diagnostic}\PYG{p}{(}\PYG{n}{real\PYGZus{}data}\PYG{p}{,} \PYG{n}{synthetic\PYGZus{}data}\PYG{p}{,} \PYG{n}{metadata}\PYG{p}{,} \PYG{k+kc}{False}\PYG{p}{)} \PYG{c+c1}{\PYGZsh{} Se ejecuta el diagnóstico sin mostrar en pantalla}
\PYG{n+nb}{print}\PYG{p}{(}\PYG{l+s+s2}{\PYGZdq{}}\PYG{l+s+se}{\PYGZbs{}n}\PYG{l+s+s2}{DIAGNÓSTICO BÁSICO CALIDAD DATOS}\PYG{l+s+se}{\PYGZbs{}n}\PYG{l+s+s2}{\PYGZdq{}}\PYG{p}{)}
\PYG{n+nb}{print}\PYG{p}{(}\PYG{n}{diagnostic}\PYG{o}{.}\PYG{n}{get\PYGZus{}details}\PYG{p}{(}\PYG{n}{property\PYGZus{}name}\PYG{o}{=}\PYG{l+s+s1}{\PYGZsq{}}\PYG{l+s+s1}{Data Validity}\PYG{l+s+s1}{\PYGZsq{}}\PYG{p}{)}\PYG{p}{)}
\end{sphinxVerbatim}

\end{sphinxuseclass}\end{sphinxVerbatimInput}
\begin{sphinxVerbatimOutput}

\begin{sphinxuseclass}{cell_output}
\begin{sphinxVerbatim}[commandchars=\\\{\}]
Metadatos cargados desde JSON.
\end{sphinxVerbatim}

\begin{sphinxVerbatim}[commandchars=\\\{\}]
   Peso (g)  Longitud (cm)  Anchura (cm)  Altura (cm)
0      8.74            9.0           3.5          0.5
1      7.84            8.7           3.5          0.5
2      2.40            6.5           2.3          0.4
3      7.45            8.3           3.4          0.6
4      7.16            8.1           3.1          0.5
5      2.95            6.1           2.6          0.4
6     15.51           10.9           4.9          0.8
7      4.72            6.9           2.8          0.5
8      7.60            8.9           3.1          0.6
9      3.01            6.3           2.1          0.4
\end{sphinxVerbatim}

\begin{sphinxVerbatim}[commandchars=\\\{\}]
Se muestran 10 primeros registros sintéticos de un total de 500

DIAGNÓSTICO BÁSICO CALIDAD DATOS

          Column             Metric  Score
0       Peso (g)  BoundaryAdherence    1.0
1  Longitud (cm)  BoundaryAdherence    1.0
2   Anchura (cm)  BoundaryAdherence    1.0
3    Altura (cm)  BoundaryAdherence    1.0
\end{sphinxVerbatim}

\end{sphinxuseclass}\end{sphinxVerbatimOutput}

\end{sphinxuseclass}
\sphinxAtStartPar
La métrica \sphinxcode{\sphinxupquote{BoundaryAdherence}} verifica si los valores sintéticos se mantienen dentro de los rangos mínimos y máximos de los datos reales, devolviendo el porcentaje de filas válidas. En las siguientes gráficas podemos visualizar la similitud entre los datos reales y los sintéticos obtenidos mediante Gaussian Copula.

\begin{sphinxuseclass}{cell}\begin{sphinxVerbatimInput}

\begin{sphinxuseclass}{cell_input}
\begin{sphinxVerbatim}[commandchars=\\\{\}]
\PYG{c+c1}{\PYGZsh{}\PYGZsh{} Visualización de datos}

\PYG{n}{columnas} \PYG{o}{=} \PYG{p}{[}\PYG{l+s+s2}{\PYGZdq{}}\PYG{l+s+s2}{Peso (g)}\PYG{l+s+s2}{\PYGZdq{}}\PYG{p}{,} \PYG{l+s+s2}{\PYGZdq{}}\PYG{l+s+s2}{Longitud (cm)}\PYG{l+s+s2}{\PYGZdq{}}\PYG{p}{,} \PYG{l+s+s2}{\PYGZdq{}}\PYG{l+s+s2}{Anchura (cm)}\PYG{l+s+s2}{\PYGZdq{}}\PYG{p}{,} \PYG{l+s+s2}{\PYGZdq{}}\PYG{l+s+s2}{Altura (cm)}\PYG{l+s+s2}{\PYGZdq{}}\PYG{p}{]}

\PYG{k+kn}{from}\PYG{+w}{ }\PYG{n+nn}{sdv}\PYG{n+nn}{.}\PYG{n+nn}{evaluation}\PYG{n+nn}{.}\PYG{n+nn}{single\PYGZus{}table}\PYG{+w}{ }\PYG{k+kn}{import} \PYG{n}{get\PYGZus{}column\PYGZus{}plot}
\PYG{k+kn}{import}\PYG{+w}{ }\PYG{n+nn}{seaborn}\PYG{+w}{ }\PYG{k}{as}\PYG{+w}{ }\PYG{n+nn}{sns}

\PYG{n}{fig}\PYG{p}{,} \PYG{n}{axs} \PYG{o}{=} \PYG{n}{plt}\PYG{o}{.}\PYG{n}{subplots}\PYG{p}{(}\PYG{l+m+mi}{2}\PYG{p}{,} \PYG{l+m+mi}{2}\PYG{p}{,} \PYG{n}{figsize}\PYG{o}{=}\PYG{p}{(}\PYG{l+m+mi}{15}\PYG{p}{,} \PYG{l+m+mi}{12}\PYG{p}{)}\PYG{p}{)}
\PYG{k}{for} \PYG{n}{i}\PYG{p}{,} \PYG{n}{col} \PYG{o+ow}{in} \PYG{n+nb}{enumerate}\PYG{p}{(}\PYG{n}{columnas}\PYG{p}{)}\PYG{p}{:}
    \PYG{n}{ax} \PYG{o}{=} \PYG{n}{axs}\PYG{p}{[}\PYG{n}{i}\PYG{o}{/}\PYG{o}{/}\PYG{l+m+mi}{2}\PYG{p}{,} \PYG{n}{i}\PYG{o}{\PYGZpc{}}\PYG{k}{2}]
    \PYG{n}{sns}\PYG{o}{.}\PYG{n}{kdeplot}\PYG{p}{(}\PYG{n}{real\PYGZus{}data}\PYG{p}{[}\PYG{n}{col}\PYG{p}{]}\PYG{p}{,} \PYG{n}{ax}\PYG{o}{=}\PYG{n}{ax}\PYG{p}{,} \PYG{n}{color}\PYG{o}{=}\PYG{l+s+s2}{\PYGZdq{}}\PYG{l+s+s2}{\PYGZsh{}1f77b4}\PYG{l+s+s2}{\PYGZdq{}}\PYG{p}{,} \PYG{n}{label}\PYG{o}{=}\PYG{l+s+s2}{\PYGZdq{}}\PYG{l+s+s2}{Real}\PYG{l+s+s2}{\PYGZdq{}}\PYG{p}{,} \PYG{n}{fill}\PYG{o}{=}\PYG{k+kc}{True}\PYG{p}{,} \PYG{n}{alpha}\PYG{o}{=}\PYG{l+m+mf}{0.5}\PYG{p}{)}
    \PYG{n}{sns}\PYG{o}{.}\PYG{n}{kdeplot}\PYG{p}{(}\PYG{n}{synthetic\PYGZus{}data}\PYG{p}{[}\PYG{n}{col}\PYG{p}{]}\PYG{p}{,} \PYG{n}{ax}\PYG{o}{=}\PYG{n}{ax}\PYG{p}{,} \PYG{n}{color}\PYG{o}{=}\PYG{l+s+s2}{\PYGZdq{}}\PYG{l+s+s2}{\PYGZsh{}2ca02c}\PYG{l+s+s2}{\PYGZdq{}}\PYG{p}{,} \PYG{n}{label}\PYG{o}{=}\PYG{l+s+s2}{\PYGZdq{}}\PYG{l+s+s2}{Gaussian Copula}\PYG{l+s+s2}{\PYGZdq{}}\PYG{p}{,} \PYG{n}{fill}\PYG{o}{=}\PYG{k+kc}{True}\PYG{p}{,} \PYG{n}{alpha}\PYG{o}{=}\PYG{l+m+mf}{0.5}\PYG{p}{)}
    \PYG{n}{ax}\PYG{o}{.}\PYG{n}{set\PYGZus{}title}\PYG{p}{(}\PYG{l+s+sa}{f}\PYG{l+s+s1}{\PYGZsq{}}\PYG{l+s+s1}{Distribución de }\PYG{l+s+si}{\PYGZob{}}\PYG{n}{col}\PYG{l+s+si}{\PYGZcb{}}\PYG{l+s+s1}{\PYGZsq{}}\PYG{p}{)}
    \PYG{n}{ax}\PYG{o}{.}\PYG{n}{legend}\PYG{p}{(}\PYG{p}{)}
\PYG{n}{plt}\PYG{o}{.}\PYG{n}{suptitle}\PYG{p}{(}\PYG{l+s+s1}{\PYGZsq{}}\PYG{l+s+s1}{Comparación de datos reales vs. sintéticos Gaussian Copula}\PYG{l+s+s1}{\PYGZsq{}}\PYG{p}{,} \PYG{n}{y}\PYG{o}{=}\PYG{l+m+mf}{1.02}\PYG{p}{)}
\PYG{n}{plt}\PYG{o}{.}\PYG{n}{tight\PYGZus{}layout}\PYG{p}{(}\PYG{p}{)}
\PYG{n}{plt}\PYG{o}{.}\PYG{n}{show}\PYG{p}{(}\PYG{p}{)}

\PYG{c+c1}{\PYGZsh{}\PYGZsh{} Medidas de similiud estadística}

\PYG{n}{quality\PYGZus{}reportGC} \PYG{o}{=} \PYG{n}{evaluate\PYGZus{}quality}\PYG{p}{(}\PYG{n}{real\PYGZus{}data}\PYG{p}{,} \PYG{n}{synthetic\PYGZus{}data}\PYG{o}{.}\PYG{n}{sample}\PYG{p}{(}\PYG{n}{n}\PYG{o}{=}\PYG{n+nb}{len}\PYG{p}{(}\PYG{n}{real\PYGZus{}data}\PYG{p}{)}\PYG{p}{,} \PYG{n}{random\PYGZus{}state}\PYG{o}{=}\PYG{l+m+mi}{42}\PYG{p}{)}\PYG{p}{,} \PYG{n}{metadata}\PYG{p}{,} \PYG{n}{verbose}\PYG{o}{=}\PYG{k+kc}{False}\PYG{p}{)}

\PYG{c+c1}{\PYGZsh{}\PYGZsh{} Obtenemos el valor promedio de la similitud}
\PYG{n}{overall\PYGZus{}score} \PYG{o}{=} \PYG{l+s+sa}{f}\PYG{l+s+s2}{\PYGZdq{}}\PYG{l+s+si}{\PYGZob{}}\PYG{n}{quality\PYGZus{}reportGC}\PYG{o}{.}\PYG{n}{get\PYGZus{}score}\PYG{p}{(}\PYG{p}{)}\PYG{o}{*}\PYG{l+m+mi}{100}\PYG{l+s+si}{:}\PYG{l+s+s2}{.2f}\PYG{l+s+si}{\PYGZcb{}}\PYG{l+s+s2}{\PYGZpc{}}\PYG{l+s+s2}{\PYGZdq{}}
\PYG{n}{glue}\PYG{p}{(}\PYG{l+s+s2}{\PYGZdq{}}\PYG{l+s+s2}{scoreGAN}\PYG{l+s+s2}{\PYGZdq{}}\PYG{p}{,} \PYG{n}{overall\PYGZus{}score}\PYG{p}{,} \PYG{n}{display}\PYG{o}{=}\PYG{k+kc}{False}\PYG{p}{)}\PYG{p}{;} \PYG{c+c1}{\PYGZsh{} Definimos una variable para usar en el texto markdown}
\PYG{n+nb}{print}\PYG{p}{(}\PYG{l+s+s2}{\PYGZdq{}}\PYG{l+s+se}{\PYGZbs{}n}\PYG{l+s+s2}{SIMILITUD ESTADÍSTICA}\PYG{l+s+se}{\PYGZbs{}n}\PYG{l+s+s2}{\PYGZdq{}}\PYG{p}{)}
\PYG{n+nb}{print}\PYG{p}{(}\PYG{l+s+sa}{f}\PYG{l+s+s2}{\PYGZdq{}}\PYG{l+s+s2}{Overall score: }\PYG{l+s+si}{\PYGZob{}}\PYG{n}{overall\PYGZus{}score}\PYG{l+s+si}{\PYGZcb{}}\PYG{l+s+se}{\PYGZbs{}n}\PYG{l+s+s2}{\PYGZdq{}}\PYG{p}{)}

\PYG{c+c1}{\PYGZsh{}\PYGZsh{} Ajuste por columna de forma individual    }
\PYG{n+nb}{print}\PYG{p}{(}\PYG{l+s+s2}{\PYGZdq{}}\PYG{l+s+se}{\PYGZbs{}n}\PYG{l+s+s2}{Resultados de ajuste real vs. sintético por columna de datos}\PYG{l+s+se}{\PYGZbs{}n}\PYG{l+s+s2}{\PYGZdq{}}\PYG{p}{)}
\PYG{n+nb}{print}\PYG{p}{(}\PYG{n}{quality\PYGZus{}reportGC}\PYG{o}{.}\PYG{n}{get\PYGZus{}details}\PYG{p}{(}\PYG{l+s+s1}{\PYGZsq{}}\PYG{l+s+s1}{Column Shapes}\PYG{l+s+s1}{\PYGZsq{}}\PYG{p}{)}\PYG{p}{)}
\end{sphinxVerbatim}

\end{sphinxuseclass}\end{sphinxVerbatimInput}
\begin{sphinxVerbatimOutput}

\begin{sphinxuseclass}{cell_output}
\noindent\sphinxincludegraphics{{cd770d1f412a0d1487bfc8a8aaf98ea59fc768af49407efb4c27a4a80c456b29}.png}

\begin{sphinxVerbatim}[commandchars=\\\{\}]
SIMILITUD ESTADÍSTICA

Overall score: 96.26\PYGZpc{}


Resultados de ajuste real vs. sintético por columna de datos

          Column        Metric     Score
0       Peso (g)  KSComplement  0.928230
1  Longitud (cm)  KSComplement  0.928230
2   Anchura (cm)  KSComplement  0.923445
3    Altura (cm)  KSComplement  0.952153
\end{sphinxVerbatim}

\end{sphinxuseclass}\end{sphinxVerbatimOutput}

\end{sphinxuseclass}
\sphinxAtStartPar
Los resultados obtenidos indican un alto grado de similitud entre los datos sintéticos y reales, con una adecuación del \DUrole{output,text_plain}{‘96.26\%’}. La métrica \sphinxcode{\sphinxupquote{KSComplement}} evalúa el grado de similitud individualmente entre cada columna real y su sintética basándose en la forma de sus distribuciones univariadas (forma, media, dispersión). Los valores obtenidos indican un elevado grado de ajuste entre los datos reales y sintéticos, tal y como también se puede apreciar en las gráficas anteriores.

\sphinxAtStartPar
La siguiente tabla recoge la correlación entre pares de columnas numéricas y determina el grado de similitud entre los datos reales y sintéticos (\sphinxcode{\sphinxupquote{CorrelationSimilarity}}), comparando específicamente las tendencias en sus distribuciones en 2D. Como se desprende de los valores obtenidos, los porcentajes de correlación son, en la mayoría de los casos, \(\approx 1\), lo cual garantiza la validad de los datos sintéticos obtenidos

\begin{sphinxuseclass}{cell}\begin{sphinxVerbatimInput}

\begin{sphinxuseclass}{cell_input}
\begin{sphinxVerbatim}[commandchars=\\\{\}]
\PYG{n+nb}{print}\PYG{p}{(}\PYG{l+s+s2}{\PYGZdq{}}\PYG{l+s+se}{\PYGZbs{}n}\PYG{l+s+s2}{ Correlaciones entre variables real vs. sintética}\PYG{l+s+se}{\PYGZbs{}n}\PYG{l+s+s2}{\PYGZdq{}}\PYG{p}{)}
\PYG{n+nb}{print}\PYG{p}{(}\PYG{n}{quality\PYGZus{}reportGC}\PYG{o}{.}\PYG{n}{get\PYGZus{}details}\PYG{p}{(}\PYG{l+s+s1}{\PYGZsq{}}\PYG{l+s+s1}{Column Pair Trends}\PYG{l+s+s1}{\PYGZsq{}}\PYG{p}{)}\PYG{p}{)}
\PYG{n+nb}{print}\PYG{p}{(}\PYG{l+s+s2}{\PYGZdq{}}\PYG{l+s+se}{\PYGZbs{}n}\PYG{l+s+s2}{\PYGZdq{}}\PYG{p}{)}
\end{sphinxVerbatim}

\end{sphinxuseclass}\end{sphinxVerbatimInput}
\begin{sphinxVerbatimOutput}

\begin{sphinxuseclass}{cell_output}
\begin{sphinxVerbatim}[commandchars=\\\{\}]
 Correlaciones entre variables real vs. sintética

        Column 1       Column 2                 Metric     Score  \PYGZbs{}
0       Peso (g)  Longitud (cm)  CorrelationSimilarity  0.995285
1       Peso (g)   Anchura (cm)  CorrelationSimilarity  0.998692
2       Peso (g)    Altura (cm)  CorrelationSimilarity  0.987694
3  Longitud (cm)   Anchura (cm)  CorrelationSimilarity  0.999846
4  Longitud (cm)    Altura (cm)  CorrelationSimilarity  0.983536
5   Anchura (cm)    Altura (cm)  CorrelationSimilarity  0.987907

   Real Correlation  Synthetic Correlation
0          0.932951               0.942382
1          0.942354               0.939738
2          0.861768               0.837156
3          0.936674               0.936982
4          0.870165               0.837236
5          0.842673               0.818487
\end{sphinxVerbatim}

\end{sphinxuseclass}\end{sphinxVerbatimOutput}

\end{sphinxuseclass}
\sphinxstepscope


\section{Conditional Tabular GAN (CTGAN)}
\label{\detokenize{content/02/CTGAN:conditional-tabular-gan-ctgan}}\label{\detokenize{content/02/CTGAN::doc}}
\sphinxAtStartPar
CTGAN es una variación de las \sphinxstylestrong{Redes Generativas Adversarias (GAN, por sus siglas en inglés)} especialmente diseñado para conjuntos de datos tabulares, una categoría que históricamente ha planteado desafíos significativos en el campo del aprendizaje automático generativo. Las redes antagónicas generativas o redes adversarias generativas (GANs) son un método para la optimización competitivo entre dos redes neuronales, una llamada generadora y otra discriminadora, con el objetivo de conseguir generar nuevas instancias idealmente indistinguibles a las pertenecientes a la distribución de probabilidad de la que derivan los datos de entrenamiento.

\sphinxAtStartPar
Las GANs, originalmente concebidas para tareas de síntesis de imágenes {[}\sphinxhref{https://doi.org/10.1145/3422622}{Goodfellow et al., 2014}{]}, han demostrado en los últimos años una notable capacidad para modelar distribuciones complejas en espacios tabulares multivariantes, especialmente en dominios donde los datos reales son escasos o costosos de obtener. El fundamento teórico general del que derivan, permite su utilización para la generación de cualquier tipo de datos, habiéndose demostrado efectiva en campos diversos como son la visión por computador {[}\sphinxhref{https://doi.org/10.48550/arXiv.1505.03906}{Roy et al., 2015}, \sphinxhref{https://doi.org/10.48550/arXiv.1710.10196}{Karras et al., 2017}{]}, la segmentación semántica {[}\sphinxhref{https://doi.org/10.48550/arXiv.1711.03213}{Hoffman et al., 2017}{]}, la síntesis de series temporales {[}\sphinxhref{https://doi.org/10.48550/arXiv.1806.01875}{Hartmann et al., 2018}{]}, la edición de imagen {[}\sphinxhref{https://doi.org/10.1145/3447648}{Abdal et al., 2020}{]}, el procesamiento del lenguaje natural {[}\sphinxhref{https://doi.org/10.48550/arXiv.1801.07736}{Fedus et al., 2018}{]}, la generación de imagen a partir de texto {[}\sphinxhref{https://doi.org/10.48550/arXiv.2103.00020}{Radford et al., 2021}{]} y más recientemente \sphinxhref{https://doi.org/10.1007/s00778-023-00807-y}{Xu et al. (2023)} demuestran que las GANs superan a técnicas convencionales de síntesis en la preservación de estructuras de dependencia no lineal en variables tabulares reales de alta dimensión.

\sphinxAtStartPar
Para cualquier conjunto de datos, podemos hipotetizar que es posible definir una distribución de probabilidad \(P_{data}\) representativa de la población representada por la muestra formada por el conjunto de datos. De ser esto posible, para cualquier valor de \(x\) será posible establecer un valor \(P_{data}(x)\) que determine la probabilidad de que \(x\) pertenezca a la población. De existir una función de este tipo, sería una función discriminativa que dada una instancia permitiría conocer la probabilidad de pertenencia a la población. Los modelos generativos modelizan la distribución de probabilidad mencionada pero no proporcionan un valor de probabilidad, sino que generan instancias nuevas que pertenecen a distribuciones de probabilidad próximas a la que pretenden asemejar. Las GANs definen un esquema de aprendizaje que facilita la codificación de los atributos definitorios de la distribución de probabilidad en una red neuronal de manera que la red incorpore la información esencial que le permite generar instancias pertenecientes a distribuciones de probabilidad próximas a la que el conjunto de datos que pretende representar.

\sphinxAtStartPar
La arquitectura de las Redes Generativas Adversarias (GAN, por sus siglas en inglés) se basa en la interacción entre dos redes neuronales que trabajan de forma opuesta pero complementaria: una red generadora \((G)\) y una red discriminadora \((D)\). La red generadora tiene como objetivo crear datos sintéticos que imiten con la mayor fidelidad posible los datos reales del conjunto original. Por su parte, la red discriminadora actúa como un detector, cuya tarea es evaluar si una determinada entrada procede del conjunto de datos reales o ha sido generada artificialmente por \(G\).

\sphinxAtStartPar
Durante el proceso de entrenamiento, ambas redes se enfrentan en un proceso competitivo. La generadora intenta “engañar” a la discriminadora creando datos cada vez más realistas, mientras que la discriminadora se entrena para detectar con mayor precisión las falsificaciones. Este enfoque adversarial permite que ambas redes mejoren progresivamente: \(G\) produce datos sintéticos más convincentes y \(D\) refina sus capacidades de detección. Esta dinámica se puede entender como un juego de suma cero, donde el éxito de una red implica el fracaso de la otra, y que teóricamente puede llevar a un punto en el que ninguna de las redes puede mejorar su rendimiento sin afectar negativamente a la otra (equilibrio de Nash).

\begin{figure}[htbp]
\centering
\capstart

\noindent\sphinxincludegraphics[width=1.000\linewidth]{{Modelo_GAN}.png}
\caption{Modelo funcional de una red GAN}\label{\detokenize{content/02/CTGAN:figura-2-1}}\end{figure}

\sphinxAtStartPar
En la figura se muestra un diagrama representativo del proceso de optimización de las GAN. La \sphinxstylestrong{red generadora} \(G(z)\) recibe como entrada un vector de ruido aleatorio \(z\), generado a partir de una distribución conocida \(p_z\), y produce como salida un dato sintético \(x_{fake}\) que intenta imitar los datos reales. La \sphinxstylestrong{red discriminadora} \(D(x)\) recibe como entrada un dato \(x_{real}\)  o \(x_{fake}\) (generado) y devuelve una probabilidad \sphinxcode{\sphinxupquote{D(x)}} entre \(0\) y \(1\) que indica cuán probable es que \(x\) provenga del conjunto real de datos.

\sphinxAtStartPar
Amabas redes se consideran antagónicas dado que sus objetivos son opuestos:
\begin{itemize}
\item {}
\sphinxAtStartPar
\(D\) quiere \sphinxstylestrong{maximizar} la probabilidad de detectar correctamente los datos reales y rechazar los sintéticos.

\item {}
\sphinxAtStartPar
\(G\) quiere \sphinxstylestrong{minimizar} la probabilidad de que \(D\) detecte que sus datos son falsos.

\end{itemize}

\sphinxAtStartPar
El proceso se modela como un \sphinxstylestrong{juego de suma cero} mediante la siguiente función objetivo:
\begin{equation*}
\begin{split}
\min_G \max_D V(D, G) = \mathbb{E}_{x \sim p_{\text{datos}}(x)}[\log D(x)] + \mathbb{E}_{z \sim p_z(z)}[\log(1 - D(G(z)))]
\end{split}
\end{equation*}\begin{itemize}
\item {}
\sphinxAtStartPar
El primer término recompensa a \(D\) por identificar correctamente datos reales.

\item {}
\sphinxAtStartPar
El segundo término recompensa a \(D\) por detectar correctamente los datos generados por \(G\).

\end{itemize}

\sphinxAtStartPar
Mientras tanto, \(G\) intenta minimizar esta función engañando a \(D\), es decir, haciendo que \(D(G(z))\) sea lo más cercano posible a 1.

\sphinxAtStartPar
Durante el entrenamiento, ambas redes mejoran iterativamente. Teóricamente, el proceso puede converger a un \sphinxstylestrong{equilibrio de Nash}, en el que \(G\) genera datos tan similares a los reales que \(D\) no puede distinguir entre ellos, y devuelve aproximadamente:
\begin{equation*}
\begin{split}
D(x) \approx 0.5
\end{split}
\end{equation*}
\sphinxAtStartPar
En ese punto, el sistema ha alcanzado un equilibrio: ni \(G\) ni \(D\) pueden mejorar sin perjudicar a la otra red.

\sphinxAtStartPar
El modelo CTGAN (Conditional Tabular GAN) representa un enfoque avanzado de generación de datos sintéticos específicamente diseñado para conjuntos de datos tabulares, una categoría que históricamente ha planteado desafíos significativos en el campo del aprendizaje automático generativo. A diferencia de las GANs tradicionales, orientadas principalmente a la síntesis de datos en dominios estructurados como imágenes o secuencias temporales, los datos tabulares presentan heterogeneidad tipológica (variables continuas, categóricas y ordinales) y complejas dependencias estadísticas interatributo. CTGAN resuelve esta complejidad mediante una arquitectura condicional que permite preservar la distribución marginal y condicional de cada atributo, logrando generar muestras sintéticas estadísticamente similares y coherentes con los datos originales.

\sphinxAtStartPar
Al igual que las GAN, su diseño se basa en un esquema adversarial compuesto por dos redes neuronales: el generador, que crea datos sintéticos a partir de ruido aleatorio y vectores condicionales, y el discriminador, que evalúa la autenticidad de los datos. Una innovación clave de CTGAN es el uso estratégico de funciones de activación, particularmente ReLU (Rectified Linear Unit) y LeakyReLU, que optimizan el flujo de gradientes durante el entrenamiento.

\sphinxAtStartPar
En el \sphinxstylestrong{generador} se emplea ReLU, para promover una activación fuerte y directa.

\begin{sphinxadmonition}{note}{ReLU ((Rectified Linear Unit)}

\sphinxAtStartPar
La \sphinxstylestrong{función de activación \sphinxcode{\sphinxupquote{ReLU}} (Rectified Linear Unit)} es una de las funciones no lineales más utilizadas en redes neuronales profundas debido a su simplicidad y eficacia computacional. Se define como \(\text{ReLU}(x) = \max(0, x)\), lo que implica que las salidas negativas se eliminan (se convierten en cero) mientras que las positivas se mantienen sin alteración. Esta propiedad introduce no linealidad en la red neuronal, permite que las neuronas se activen solo cuando es necesario y, al mismo tiempo, evita problemas de saturación que se presentan en otras funciones. Sin embargo, \sphinxcode{\sphinxupquote{ReLU}} puede presentar el problema del “apagado de neuronas” (dead neurons), cuando los valores de entrada son negativos de forma persistente, impidiendo que esas neuronas contribuyan al aprendizaje.
\end{sphinxadmonition}

\sphinxAtStartPar
Por otro lado, el \sphinxstylestrong{discriminador} utiliza Leaky ReLU (que anula los valores negarios) para evitar el colapso de unidades activas y asegurar una mejor capacidad de detección de patrones tanto en regiones positivas como negativas del espacio de entrada.

\begin{sphinxadmonition}{note}{Leaky ReLU}

\sphinxAtStartPar
Leaky ReLU es una evolución de la función de activación ReLU especialmente orientada a solventar el problema de “apagado de neuronas” (dead neurons), cuando los valores de entrada son negativos de forma persistente, impidiendo que esas neuronas contribuyan al aprendizaje. Esta función introduce una variante al considerar una pequeña pendiente negativa para los valores menores que cero. Su definición es:
\begin{equation*}
\begin{split}
\text{LeakyReLU}(x) =
\begin{cases}
x & \text{si } x \geq 0 \\
\alpha x & \text{si } x < 0
\end{cases}
\end{split}
\end{equation*}
\sphinxAtStartPar
donde \(\alpha\) es un pequeño valor positivo, típicamente \(0.01\). Esta modificación permite que las neuronas continúen actualizando sus pesos incluso cuando sus entradas son negativas, lo que favorece una mejor propagación del gradiente durante el entrenamiento y mejora la robustez de la red.
\end{sphinxadmonition}

\sphinxAtStartPar
Tanto el generador como el discriminador de la red CTGAN están compuestos por \sphinxstylestrong{cuatro capas ocultas densamente conectadas} (\sphinxstyleemphasis{fully connected layers}) con \sphinxstylestrong{64 neuronass} que responden a un compromiso técnico entre capacidad representacional, estabilidad del entrenamiento y eficiencia computacional, especialmente en el contexto de modelos generativos aplicados a datos tabulares de baja dimensionalidad, como es el caso de las variables morfométricas (longitud, anchura, altura y peso) de juveniles de peces.

\sphinxAtStartPar
Desde un punto de vista práctico, 64 unidades por capa ofrecen una capacidad suficiente para capturar relaciones no lineales complejas entre las variables del espacio latente y las características morfológicas de salida, sin llegar a una sobreparametrización excesiva que pueda inducir sobreajuste o inestabilidad adversarial durante el entrenamiento de la GAN. Esta elección proporciona un número razonable de parámetros entrenables, lo cual resulta adecuado cuando se trabaja con conjuntos de datos de tamaño moderado, como ocurre en nuestro caso.

\sphinxAtStartPar
A este respecto, estudios recientes sobre GANs para datos tabulares —como CTAB\sphinxhyphen{}GAN+ {[}\sphinxhref{https://doi.org/10.3389/fdata.2023.1296508}{Zao et al., 2024}{]} y CTGAN {[}\sphinxhref{https://doi.org/10.1007/s00778-023-00807-y}{Liu et al., 2023}{]}— concluyen que arquitecturas con capas ocultas de entre 64 y 128 neuronas suelen alcanzar un buen equilibrio entre precisión, velocidad de convergencia y estabilidad del discriminador. Por debajo de este umbral (p.e., 16 o 32 neuronas), se puede observar pérdida de capacidad expresiva, mientras que valores superiores (p.e., 256) pueden ser innecesarios para problemas con pocos atributos y generar ruido o fluctuaciones en el aprendizaje entre adversarios.

\begin{sphinxuseclass}{cell}\begin{sphinxVerbatimInput}

\begin{sphinxuseclass}{cell_input}
\begin{sphinxVerbatim}[commandchars=\\\{\}]
\PYG{c+c1}{\PYGZsh{} Importar librerías necesarias}
\PYG{k+kn}{import}\PYG{+w}{ }\PYG{n+nn}{pandas}\PYG{+w}{ }\PYG{k}{as}\PYG{+w}{ }\PYG{n+nn}{pd}
\PYG{k+kn}{import}\PYG{+w}{ }\PYG{n+nn}{numpy}\PYG{+w}{ }\PYG{k}{as}\PYG{+w}{ }\PYG{n+nn}{np}
\PYG{k+kn}{import}\PYG{+w}{ }\PYG{n+nn}{scipy}
\PYG{k+kn}{import}\PYG{+w}{ }\PYG{n+nn}{matplotlib}\PYG{n+nn}{.}\PYG{n+nn}{pyplot}\PYG{+w}{ }\PYG{k}{as}\PYG{+w}{ }\PYG{n+nn}{plt}

\PYG{k+kn}{from}\PYG{+w}{ }\PYG{n+nn}{sdv}\PYG{n+nn}{.}\PYG{n+nn}{single\PYGZus{}table}\PYG{+w}{ }\PYG{k+kn}{import} \PYG{n}{CTGANSynthesizer}
\PYG{k+kn}{from}\PYG{+w}{ }\PYG{n+nn}{sdv}\PYG{n+nn}{.}\PYG{n+nn}{metadata}\PYG{+w}{ }\PYG{k+kn}{import} \PYG{n}{Metadata}
\PYG{k+kn}{from}\PYG{+w}{ }\PYG{n+nn}{pathlib}\PYG{+w}{ }\PYG{k+kn}{import} \PYG{n}{Path}
\PYG{k+kn}{from}\PYG{+w}{ }\PYG{n+nn}{myst\PYGZus{}nb}\PYG{+w}{ }\PYG{k+kn}{import} \PYG{n}{glue}

\PYG{n}{num\PYGZus{}synthetic} \PYG{o}{=} \PYG{l+m+mi}{500} \PYG{c+c1}{\PYGZsh{}variable para definir el número de registros sintéticos}

\PYG{c+c1}{\PYGZsh{} 1. Cargar datos reales}
\PYG{c+c1}{\PYGZsh{}\PYGZhy{}\PYGZhy{}\PYGZhy{}\PYGZhy{}\PYGZhy{}\PYGZhy{}\PYGZhy{}\PYGZhy{}\PYGZhy{}\PYGZhy{}\PYGZhy{}\PYGZhy{}\PYGZhy{}\PYGZhy{}\PYGZhy{}\PYGZhy{}\PYGZhy{}\PYGZhy{}\PYGZhy{}\PYGZhy{}\PYGZhy{}\PYGZhy{}\PYGZhy{}\PYGZhy{}\PYGZhy{}\PYGZhy{}}
\PYG{n}{path\PYGZus{}realData} \PYG{o}{=} \PYG{l+s+s1}{\PYGZsq{}}\PYG{l+s+s1}{.././data/Dimensiones\PYGZus{}lenguado.xlsx}\PYG{l+s+s1}{\PYGZsq{}}
\PYG{n}{real\PYGZus{}data} \PYG{o}{=} \PYG{n}{pd}\PYG{o}{.}\PYG{n}{read\PYGZus{}excel}\PYG{p}{(}\PYG{n}{path\PYGZus{}realData}\PYG{p}{)}

\PYG{c+c1}{\PYGZsh{} 2. Obtener los metadata del dataset}
\PYG{c+c1}{\PYGZsh{}\PYGZhy{}\PYGZhy{}\PYGZhy{}\PYGZhy{}\PYGZhy{}\PYGZhy{}\PYGZhy{}\PYGZhy{}\PYGZhy{}\PYGZhy{}\PYGZhy{}\PYGZhy{}\PYGZhy{}\PYGZhy{}\PYGZhy{}\PYGZhy{}\PYGZhy{}\PYGZhy{}\PYGZhy{}\PYGZhy{}\PYGZhy{}\PYGZhy{}\PYGZhy{}\PYGZhy{}\PYGZhy{}\PYGZhy{}\PYGZhy{}\PYGZhy{}\PYGZhy{}\PYGZhy{}\PYGZhy{}\PYGZhy{}\PYGZhy{}\PYGZhy{}\PYGZhy{}\PYGZhy{}\PYGZhy{}\PYGZhy{}}

\PYG{n}{metadata\PYGZus{}path} \PYG{o}{=} \PYG{n}{Path}\PYG{p}{(}\PYG{l+s+s1}{\PYGZsq{}}\PYG{l+s+s1}{.././data/metadata\PYGZus{}lenguado.json}\PYG{l+s+s1}{\PYGZsq{}}\PYG{p}{)}

\PYG{k}{if} \PYG{n}{metadata\PYGZus{}path}\PYG{o}{.}\PYG{n}{exists}\PYG{p}{(}\PYG{p}{)}\PYG{p}{:}
    \PYG{c+c1}{\PYGZsh{} Cargar metadatos desde el JSON (evita el warning)}
    \PYG{n}{metadata} \PYG{o}{=} \PYG{n}{Metadata}\PYG{o}{.}\PYG{n}{load\PYGZus{}from\PYGZus{}json}\PYG{p}{(}\PYG{n}{metadata\PYGZus{}path}\PYG{p}{)}
    \PYG{n+nb}{print}\PYG{p}{(}\PYG{l+s+s2}{\PYGZdq{}}\PYG{l+s+s2}{Metadatos cargados desde JSON.}\PYG{l+s+s2}{\PYGZdq{}}\PYG{p}{)}
\PYG{k}{else}\PYG{p}{:}
    \PYG{c+c1}{\PYGZsh{} Detectar metadatos y guardarlos en JSON para futuras ejecuciones}
    \PYG{n}{metadata} \PYG{o}{=} \PYG{n}{Metadata}\PYG{o}{.}\PYG{n}{detect\PYGZus{}from\PYGZus{}dataframe}\PYG{p}{(}\PYG{n}{real\PYGZus{}data}\PYG{p}{)}
    \PYG{n}{metadata}\PYG{o}{.}\PYG{n}{save\PYGZus{}to\PYGZus{}json}\PYG{p}{(}\PYG{n}{metadata\PYGZus{}path}\PYG{p}{)}
    \PYG{n+nb}{print}\PYG{p}{(}\PYG{l+s+s2}{\PYGZdq{}}\PYG{l+s+s2}{Metadatos detectados y guardados en JSON.}\PYG{l+s+s2}{\PYGZdq{}}\PYG{p}{)}

\PYG{c+c1}{\PYGZsh{} 3. Entrenar la red CTGAN}
\PYG{c+c1}{\PYGZsh{}\PYGZhy{}\PYGZhy{}\PYGZhy{}\PYGZhy{}\PYGZhy{}\PYGZhy{}\PYGZhy{}\PYGZhy{}\PYGZhy{}\PYGZhy{}\PYGZhy{}\PYGZhy{}\PYGZhy{}\PYGZhy{}\PYGZhy{}\PYGZhy{}\PYGZhy{}\PYGZhy{}\PYGZhy{}\PYGZhy{}\PYGZhy{}\PYGZhy{}\PYGZhy{}\PYGZhy{}\PYGZhy{}\PYGZhy{}\PYGZhy{}}
\PYG{n+nb}{print}\PYG{p}{(}\PYG{l+s+s2}{\PYGZdq{}}\PYG{l+s+s2}{Iniciando CTGAN...}\PYG{l+s+s2}{\PYGZdq{}}\PYG{p}{)}

\PYG{n}{synthesizer} \PYG{o}{=} \PYG{n}{CTGANSynthesizer}\PYG{p}{(}
    \PYG{n}{metadata} \PYG{o}{=} \PYG{n}{metadata}\PYG{p}{,}
    \PYG{n}{enforce\PYGZus{}min\PYGZus{}max\PYGZus{}values} \PYG{o}{=} \PYG{k+kc}{True}\PYG{p}{,} \PYG{c+c1}{\PYGZsh{} Forzar datos realísticos}
    \PYG{n}{enforce\PYGZus{}rounding} \PYG{o}{=} \PYG{k+kc}{True}\PYG{p}{,} \PYG{c+c1}{\PYGZsh{} Redondeos automáticos}
    \PYG{n}{epochs} \PYG{o}{=} \PYG{l+m+mi}{500}\PYG{p}{,}
    \PYG{n}{verbose} \PYG{o}{=} \PYG{k+kc}{True}
\PYG{p}{)}\PYG{p}{;}

\PYG{c+c1}{\PYGZsh{} 4. Generar datos sintéticos}
\PYG{c+c1}{\PYGZsh{}\PYGZhy{}\PYGZhy{}\PYGZhy{}\PYGZhy{}\PYGZhy{}\PYGZhy{}\PYGZhy{}\PYGZhy{}\PYGZhy{}\PYGZhy{}\PYGZhy{}\PYGZhy{}\PYGZhy{}\PYGZhy{}\PYGZhy{}\PYGZhy{}\PYGZhy{}\PYGZhy{}\PYGZhy{}\PYGZhy{}\PYGZhy{}\PYGZhy{}\PYGZhy{}\PYGZhy{}\PYGZhy{}\PYGZhy{}\PYGZhy{}}
\PYG{k+kn}{import}\PYG{+w}{ }\PYG{n+nn}{contextlib}
\PYG{k+kn}{import}\PYG{+w}{ }\PYG{n+nn}{io}

\PYG{n+nb}{print}\PYG{p}{(}\PYG{l+s+s2}{\PYGZdq{}}\PYG{l+s+s2}{Entrenando CTGAN (esto puede tardar varios minutos)...}\PYG{l+s+s2}{\PYGZdq{}}\PYG{p}{)}

\PYG{c+c1}{\PYGZsh{} Suprimir toda la salida estándar para que no se vea al compilar}
\PYG{k}{with} \PYG{n}{contextlib}\PYG{o}{.}\PYG{n}{redirect\PYGZus{}stdout}\PYG{p}{(}\PYG{n}{io}\PYG{o}{.}\PYG{n}{StringIO}\PYG{p}{(}\PYG{p}{)}\PYG{p}{)}\PYG{p}{:}
    \PYG{k}{with} \PYG{n}{contextlib}\PYG{o}{.}\PYG{n}{redirect\PYGZus{}stderr}\PYG{p}{(}\PYG{n}{io}\PYG{o}{.}\PYG{n}{StringIO}\PYG{p}{(}\PYG{p}{)}\PYG{p}{)}\PYG{p}{:}
        \PYG{n}{synthesizer}\PYG{o}{.}\PYG{n}{fit}\PYG{p}{(}\PYG{n}{real\PYGZus{}data}\PYG{p}{)}\PYG{p}{;}


\PYG{n+nb}{print}\PYG{p}{(}\PYG{l+s+s2}{\PYGZdq{}}\PYG{l+s+s2}{Generando datos sintéticos}\PYG{l+s+s2}{\PYGZdq{}}\PYG{p}{)}
\PYG{n}{syntheticGAN\PYGZus{}data} \PYG{o}{=} \PYG{n}{synthesizer}\PYG{o}{.}\PYG{n}{sample}\PYG{p}{(}\PYG{n}{num\PYGZus{}rows}\PYG{o}{=}\PYG{n}{num\PYGZus{}synthetic}\PYG{p}{)}\PYG{p}{;}
\PYG{n}{display}\PYG{p}{(}\PYG{n}{syntheticGAN\PYGZus{}data}\PYG{o}{.}\PYG{n}{head}\PYG{p}{(}\PYG{l+m+mi}{10}\PYG{p}{)}\PYG{p}{)}
\PYG{n+nb}{print}\PYG{p}{(}\PYG{l+s+sa}{f}\PYG{l+s+s2}{\PYGZdq{}}\PYG{l+s+s2}{Se muestran 10 primeros registros sintéticos CTGAN de un total de }\PYG{l+s+si}{\PYGZob{}}\PYG{n}{num\PYGZus{}synthetic}\PYG{l+s+si}{\PYGZcb{}}\PYG{l+s+s2}{\PYGZdq{}}\PYG{p}{)}

\PYG{c+c1}{\PYGZsh{} 5. Guardar satos sintéticos}
\PYG{c+c1}{\PYGZsh{}\PYGZhy{}\PYGZhy{}\PYGZhy{}\PYGZhy{}\PYGZhy{}\PYGZhy{}\PYGZhy{}\PYGZhy{}\PYGZhy{}\PYGZhy{}\PYGZhy{}\PYGZhy{}\PYGZhy{}\PYGZhy{}\PYGZhy{}\PYGZhy{}\PYGZhy{}\PYGZhy{}\PYGZhy{}\PYGZhy{}\PYGZhy{}\PYGZhy{}\PYGZhy{}\PYGZhy{}\PYGZhy{}\PYGZhy{}\PYGZhy{}}

\PYG{n}{path\PYGZus{}syntheticCTGANData} \PYG{o}{=} \PYG{l+s+s2}{\PYGZdq{}}\PYG{l+s+s2}{.././data/SyntheticCTGAN.xlsx}\PYG{l+s+s2}{\PYGZdq{}}
\PYG{n}{syntheticGAN\PYGZus{}data}\PYG{o}{.}\PYG{n}{to\PYGZus{}excel}\PYG{p}{(}\PYG{n}{path\PYGZus{}syntheticCTGANData}\PYG{p}{,} \PYG{n}{index}\PYG{o}{=}\PYG{k+kc}{False}\PYG{p}{,} \PYG{n}{engine}\PYG{o}{=}\PYG{l+s+s1}{\PYGZsq{}}\PYG{l+s+s1}{openpyxl}\PYG{l+s+s1}{\PYGZsq{}}\PYG{p}{)}

\PYG{c+c1}{\PYGZsh{} 6. Validación calidad datos sintéticos}
\PYG{c+c1}{\PYGZsh{}\PYGZhy{}\PYGZhy{}\PYGZhy{}\PYGZhy{}\PYGZhy{}\PYGZhy{}\PYGZhy{}\PYGZhy{}\PYGZhy{}\PYGZhy{}\PYGZhy{}\PYGZhy{}\PYGZhy{}\PYGZhy{}\PYGZhy{}\PYGZhy{}\PYGZhy{}\PYGZhy{}\PYGZhy{}\PYGZhy{}\PYGZhy{}\PYGZhy{}\PYGZhy{}\PYGZhy{}\PYGZhy{}\PYGZhy{}\PYGZhy{}}

\PYG{k+kn}{from}\PYG{+w}{ }\PYG{n+nn}{sdv}\PYG{n+nn}{.}\PYG{n+nn}{evaluation}\PYG{n+nn}{.}\PYG{n+nn}{single\PYGZus{}table}\PYG{+w}{ }\PYG{k+kn}{import} \PYG{n}{run\PYGZus{}diagnostic}\PYG{p}{,} \PYG{n}{evaluate\PYGZus{}quality}
\PYG{k+kn}{from}\PYG{+w}{ }\PYG{n+nn}{sdv}\PYG{n+nn}{.}\PYG{n+nn}{evaluation}\PYG{n+nn}{.}\PYG{n+nn}{single\PYGZus{}table}\PYG{+w}{ }\PYG{k+kn}{import} \PYG{n}{get\PYGZus{}column\PYGZus{}plot}

\PYG{c+c1}{\PYGZsh{}\PYGZsh{} Diagnóstico básico}
\PYG{n}{diagnostic} \PYG{o}{=} \PYG{n}{run\PYGZus{}diagnostic}\PYG{p}{(}\PYG{n}{real\PYGZus{}data}\PYG{p}{,} \PYG{n}{syntheticGAN\PYGZus{}data}\PYG{p}{,} \PYG{n}{metadata}\PYG{p}{,} \PYG{k+kc}{False}\PYG{p}{)}

\PYG{n+nb}{print}\PYG{p}{(}\PYG{l+s+s2}{\PYGZdq{}}\PYG{l+s+se}{\PYGZbs{}n}\PYG{l+s+s2}{DIAGNÓSTICO BÁSICO CALIDAD DATOS}\PYG{l+s+se}{\PYGZbs{}n}\PYG{l+s+s2}{\PYGZdq{}}\PYG{p}{)}
\PYG{n+nb}{print}\PYG{p}{(}\PYG{n}{diagnostic}\PYG{o}{.}\PYG{n}{get\PYGZus{}details}\PYG{p}{(}\PYG{n}{property\PYGZus{}name}\PYG{o}{=}\PYG{l+s+s1}{\PYGZsq{}}\PYG{l+s+s1}{Data Validity}\PYG{l+s+s1}{\PYGZsq{}}\PYG{p}{)}\PYG{p}{)}

\PYG{c+c1}{\PYGZsh{}\PYGZsh{} Visualización de datos}
\PYG{k+kn}{from}\PYG{+w}{ }\PYG{n+nn}{sdv}\PYG{n+nn}{.}\PYG{n+nn}{evaluation}\PYG{n+nn}{.}\PYG{n+nn}{single\PYGZus{}table}\PYG{+w}{ }\PYG{k+kn}{import} \PYG{n}{get\PYGZus{}column\PYGZus{}plot}
\PYG{k+kn}{import}\PYG{+w}{ }\PYG{n+nn}{seaborn}\PYG{+w}{ }\PYG{k}{as}\PYG{+w}{ }\PYG{n+nn}{sns}

\PYG{n}{columnas} \PYG{o}{=} \PYG{p}{[}\PYG{l+s+s2}{\PYGZdq{}}\PYG{l+s+s2}{Peso (g)}\PYG{l+s+s2}{\PYGZdq{}}\PYG{p}{,} \PYG{l+s+s2}{\PYGZdq{}}\PYG{l+s+s2}{Longitud (cm)}\PYG{l+s+s2}{\PYGZdq{}}\PYG{p}{,} \PYG{l+s+s2}{\PYGZdq{}}\PYG{l+s+s2}{Anchura (cm)}\PYG{l+s+s2}{\PYGZdq{}}\PYG{p}{,} \PYG{l+s+s2}{\PYGZdq{}}\PYG{l+s+s2}{Altura (cm)}\PYG{l+s+s2}{\PYGZdq{}}\PYG{p}{]}

\PYG{n}{fig}\PYG{p}{,} \PYG{n}{axs} \PYG{o}{=} \PYG{n}{plt}\PYG{o}{.}\PYG{n}{subplots}\PYG{p}{(}\PYG{l+m+mi}{2}\PYG{p}{,} \PYG{l+m+mi}{2}\PYG{p}{,} \PYG{n}{figsize}\PYG{o}{=}\PYG{p}{(}\PYG{l+m+mi}{15}\PYG{p}{,} \PYG{l+m+mi}{12}\PYG{p}{)}\PYG{p}{)}
\PYG{k}{for} \PYG{n}{i}\PYG{p}{,} \PYG{n}{col} \PYG{o+ow}{in} \PYG{n+nb}{enumerate}\PYG{p}{(}\PYG{n}{columnas}\PYG{p}{)}\PYG{p}{:}
    \PYG{n}{ax} \PYG{o}{=} \PYG{n}{axs}\PYG{p}{[}\PYG{n}{i}\PYG{o}{/}\PYG{o}{/}\PYG{l+m+mi}{2}\PYG{p}{,} \PYG{n}{i}\PYG{o}{\PYGZpc{}}\PYG{k}{2}]
    \PYG{n}{sns}\PYG{o}{.}\PYG{n}{kdeplot}\PYG{p}{(}\PYG{n}{real\PYGZus{}data}\PYG{p}{[}\PYG{n}{col}\PYG{p}{]}\PYG{p}{,} \PYG{n}{ax}\PYG{o}{=}\PYG{n}{ax}\PYG{p}{,} \PYG{n}{color}\PYG{o}{=}\PYG{l+s+s2}{\PYGZdq{}}\PYG{l+s+s2}{\PYGZsh{}1f77b4}\PYG{l+s+s2}{\PYGZdq{}}\PYG{p}{,} \PYG{n}{label}\PYG{o}{=}\PYG{l+s+s2}{\PYGZdq{}}\PYG{l+s+s2}{Real}\PYG{l+s+s2}{\PYGZdq{}}\PYG{p}{,} \PYG{n}{fill}\PYG{o}{=}\PYG{k+kc}{True}\PYG{p}{,} \PYG{n}{alpha}\PYG{o}{=}\PYG{l+m+mf}{0.5}\PYG{p}{)}
    \PYG{n}{sns}\PYG{o}{.}\PYG{n}{kdeplot}\PYG{p}{(}\PYG{n}{syntheticGAN\PYGZus{}data}\PYG{p}{[}\PYG{n}{col}\PYG{p}{]}\PYG{p}{,} \PYG{n}{ax}\PYG{o}{=}\PYG{n}{ax}\PYG{p}{,} \PYG{n}{color}\PYG{o}{=}\PYG{l+s+s2}{\PYGZdq{}}\PYG{l+s+s2}{\PYGZsh{}d62728}\PYG{l+s+s2}{\PYGZdq{}}\PYG{p}{,} \PYG{n}{label}\PYG{o}{=}\PYG{l+s+s2}{\PYGZdq{}}\PYG{l+s+s2}{CTGAN}\PYG{l+s+s2}{\PYGZdq{}}\PYG{p}{,} \PYG{n}{fill}\PYG{o}{=}\PYG{k+kc}{True}\PYG{p}{,} \PYG{n}{alpha}\PYG{o}{=}\PYG{l+m+mf}{0.5}\PYG{p}{)}
    \PYG{n}{ax}\PYG{o}{.}\PYG{n}{set\PYGZus{}title}\PYG{p}{(}\PYG{l+s+sa}{f}\PYG{l+s+s1}{\PYGZsq{}}\PYG{l+s+s1}{Distribución de }\PYG{l+s+si}{\PYGZob{}}\PYG{n}{col}\PYG{l+s+si}{\PYGZcb{}}\PYG{l+s+s1}{\PYGZsq{}}\PYG{p}{)}
    \PYG{n}{ax}\PYG{o}{.}\PYG{n}{legend}\PYG{p}{(}\PYG{p}{)}
\PYG{n}{plt}\PYG{o}{.}\PYG{n}{suptitle}\PYG{p}{(}\PYG{l+s+s1}{\PYGZsq{}}\PYG{l+s+s1}{Comparación de datos reales vs. sintéticos (CTGAN)}\PYG{l+s+s1}{\PYGZsq{}}\PYG{p}{,} \PYG{n}{y}\PYG{o}{=}\PYG{l+m+mf}{1.02}\PYG{p}{)}
\PYG{n}{plt}\PYG{o}{.}\PYG{n}{tight\PYGZus{}layout}\PYG{p}{(}\PYG{p}{)}
\PYG{n}{plt}\PYG{o}{.}\PYG{n}{show}\PYG{p}{(}\PYG{p}{)}

\PYG{c+c1}{\PYGZsh{}\PYGZsh{} Medidas de similiud estadística}

\PYG{n}{quality\PYGZus{}reportGAN} \PYG{o}{=} \PYG{n}{evaluate\PYGZus{}quality}\PYG{p}{(}\PYG{n}{real\PYGZus{}data}\PYG{p}{,} \PYG{n}{syntheticGAN\PYGZus{}data}\PYG{o}{.}\PYG{n}{sample}\PYG{p}{(}\PYG{n}{n}\PYG{o}{=}\PYG{n+nb}{len}\PYG{p}{(}\PYG{n}{real\PYGZus{}data}\PYG{p}{)}\PYG{p}{,} \PYG{n}{random\PYGZus{}state}\PYG{o}{=}\PYG{l+m+mi}{42}\PYG{p}{)}\PYG{p}{,} \PYG{n}{metadata}\PYG{p}{,} \PYG{k+kc}{False}\PYG{p}{)}

\PYG{c+c1}{\PYGZsh{}\PYGZsh{} Obtenemos el valor promedio de la similitud}
\PYG{n}{overall\PYGZus{}scoreGAN} \PYG{o}{=} \PYG{l+s+sa}{f}\PYG{l+s+s2}{\PYGZdq{}}\PYG{l+s+si}{\PYGZob{}}\PYG{n}{quality\PYGZus{}reportGAN}\PYG{o}{.}\PYG{n}{get\PYGZus{}score}\PYG{p}{(}\PYG{p}{)}\PYG{o}{*}\PYG{l+m+mi}{100}\PYG{l+s+si}{:}\PYG{l+s+s2}{.2f}\PYG{l+s+si}{\PYGZcb{}}\PYG{l+s+s2}{\PYGZpc{}}\PYG{l+s+s2}{\PYGZdq{}}
\PYG{n}{glue}\PYG{p}{(}\PYG{l+s+s2}{\PYGZdq{}}\PYG{l+s+s2}{scoreGAN}\PYG{l+s+s2}{\PYGZdq{}}\PYG{p}{,} \PYG{n}{overall\PYGZus{}scoreGAN}\PYG{p}{,} \PYG{n}{display}\PYG{o}{=}\PYG{k+kc}{False}\PYG{p}{)}\PYG{p}{;} \PYG{c+c1}{\PYGZsh{} Definimos una variable para usar en el texto markdown}
\PYG{n+nb}{print}\PYG{p}{(}\PYG{l+s+s2}{\PYGZdq{}}\PYG{l+s+se}{\PYGZbs{}n}\PYG{l+s+s2}{SIMILITUD ESTADÍSTICA}\PYG{l+s+se}{\PYGZbs{}n}\PYG{l+s+s2}{\PYGZdq{}}\PYG{p}{)}
\PYG{n+nb}{print}\PYG{p}{(}\PYG{l+s+sa}{f}\PYG{l+s+s2}{\PYGZdq{}}\PYG{l+s+s2}{Overall score: }\PYG{l+s+si}{\PYGZob{}}\PYG{n}{overall\PYGZus{}scoreGAN}\PYG{l+s+si}{\PYGZcb{}}\PYG{l+s+se}{\PYGZbs{}n}\PYG{l+s+s2}{\PYGZdq{}}\PYG{p}{)}
\end{sphinxVerbatim}

\end{sphinxuseclass}\end{sphinxVerbatimInput}
\begin{sphinxVerbatimOutput}

\begin{sphinxuseclass}{cell_output}
\begin{sphinxVerbatim}[commandchars=\\\{\}]
Metadatos cargados desde JSON.
Iniciando CTGAN...
Entrenando CTGAN (esto puede tardar varios minutos)...
\end{sphinxVerbatim}

\begin{sphinxVerbatim}[commandchars=\\\{\}]
Generando datos sintéticos
\end{sphinxVerbatim}

\begin{sphinxVerbatim}[commandchars=\\\{\}]
   Peso (g)  Longitud (cm)  Anchura (cm)  Altura (cm)
0      0.46            6.7           1.9          0.5
1      3.06            8.4           5.0          0.7
2      6.28           10.0           1.6          0.7
3      0.46            6.3           2.8          0.5
4      6.75            5.6           4.1          0.4
5      2.72            5.1           4.6          0.6
6      9.85           10.7           2.6          0.3
7      0.46            5.3           2.9          0.9
8      1.64           11.0           4.2          0.5
9      1.31            9.0           2.4          0.4
\end{sphinxVerbatim}

\begin{sphinxVerbatim}[commandchars=\\\{\}]
Se muestran 10 primeros registros sintéticos CTGAN de un total de 500

DIAGNÓSTICO BÁSICO CALIDAD DATOS

          Column             Metric  Score
0       Peso (g)  BoundaryAdherence    1.0
1  Longitud (cm)  BoundaryAdherence    1.0
2   Anchura (cm)  BoundaryAdherence    1.0
3    Altura (cm)  BoundaryAdherence    1.0
\end{sphinxVerbatim}

\noindent\sphinxincludegraphics{{101de52b59b3ed49a64f8853b5a8f1a27a413493984983b8ff1813fcfc88f211}.png}

\begin{sphinxVerbatim}[commandchars=\\\{\}]
SIMILITUD ESTADÍSTICA

Overall score: 69.00\PYGZpc{}
\end{sphinxVerbatim}

\end{sphinxuseclass}\end{sphinxVerbatimOutput}

\end{sphinxuseclass}
\begin{sphinxuseclass}{cell}\begin{sphinxVerbatimInput}

\begin{sphinxuseclass}{cell_input}
\begin{sphinxVerbatim}[commandchars=\\\{\}]
\PYG{n+nb}{print}\PYG{p}{(}\PYG{l+s+s2}{\PYGZdq{}}\PYG{l+s+se}{\PYGZbs{}n}\PYG{l+s+s2}{ Distribuciones individuales real vs. sintética}\PYG{l+s+se}{\PYGZbs{}n}\PYG{l+s+s2}{\PYGZdq{}}\PYG{p}{)}
\PYG{n+nb}{print}\PYG{p}{(}\PYG{n}{quality\PYGZus{}reportGAN}\PYG{o}{.}\PYG{n}{get\PYGZus{}details}\PYG{p}{(}\PYG{l+s+s1}{\PYGZsq{}}\PYG{l+s+s1}{Column Shapes}\PYG{l+s+s1}{\PYGZsq{}}\PYG{p}{)}\PYG{p}{)}
\end{sphinxVerbatim}

\end{sphinxuseclass}\end{sphinxVerbatimInput}
\begin{sphinxVerbatimOutput}

\begin{sphinxuseclass}{cell_output}
\begin{sphinxVerbatim}[commandchars=\\\{\}]
 Distribuciones individuales real vs. sintética

          Column        Metric     Score
0       Peso (g)  KSComplement  0.645933
1  Longitud (cm)  KSComplement  0.803828
2   Anchura (cm)  KSComplement  0.803828
3    Altura (cm)  KSComplement  0.870813
\end{sphinxVerbatim}

\end{sphinxuseclass}\end{sphinxVerbatimOutput}

\end{sphinxuseclass}
\begin{sphinxuseclass}{cell}\begin{sphinxVerbatimInput}

\begin{sphinxuseclass}{cell_input}
\begin{sphinxVerbatim}[commandchars=\\\{\}]
\PYG{n+nb}{print}\PYG{p}{(}\PYG{l+s+s2}{\PYGZdq{}}\PYG{l+s+se}{\PYGZbs{}n}\PYG{l+s+s2}{ Correlaciones entre variables real vs. sintética}\PYG{l+s+se}{\PYGZbs{}n}\PYG{l+s+s2}{\PYGZdq{}}\PYG{p}{)}
\PYG{n+nb}{print}\PYG{p}{(}\PYG{n}{quality\PYGZus{}reportGAN}\PYG{o}{.}\PYG{n}{get\PYGZus{}details}\PYG{p}{(}\PYG{l+s+s1}{\PYGZsq{}}\PYG{l+s+s1}{Column Pair Trends}\PYG{l+s+s1}{\PYGZsq{}}\PYG{p}{)}\PYG{p}{)}
\end{sphinxVerbatim}

\end{sphinxuseclass}\end{sphinxVerbatimInput}
\begin{sphinxVerbatimOutput}

\begin{sphinxuseclass}{cell_output}
\begin{sphinxVerbatim}[commandchars=\\\{\}]
 Correlaciones entre variables real vs. sintética

        Column 1       Column 2                 Metric     Score  \PYGZbs{}
0       Peso (g)  Longitud (cm)  CorrelationSimilarity  0.623063
1       Peso (g)   Anchura (cm)  CorrelationSimilarity  0.540390
2       Peso (g)    Altura (cm)  CorrelationSimilarity  0.549910
3  Longitud (cm)   Anchura (cm)  CorrelationSimilarity  0.643393
4  Longitud (cm)    Altura (cm)  CorrelationSimilarity  0.611668
5   Anchura (cm)    Altura (cm)  CorrelationSimilarity  0.624636

   Real Correlation  Synthetic Correlation
0          0.932951               0.179077
1          0.942354               0.023134
2          0.861768              \PYGZhy{}0.038412
3          0.936674               0.223459
4          0.870165               0.093501
5          0.842673               0.091944
\end{sphinxVerbatim}

\end{sphinxuseclass}\end{sphinxVerbatimOutput}

\end{sphinxuseclass}
\sphinxAtStartPar
Como se puede deducir de las tablas anteriores, la generación de datos sintéticos mediante \sphinxstylestrong{CTGAN proporciona unas métricas mucho más pobres} que las obtenidas con el método Gaussian Copula, con una baja correlación con los datos reales y en consecuencia \sphinxstylestrong{invalidando los registros sintéticos obtenidos} de peso, longitud, anchura y altura.

\sphinxstepscope


\section{Tabular Variational Autoencoder (TVAE)}
\label{\detokenize{content/02/TVAE:tabular-variational-autoencoder-tvae}}\label{\detokenize{content/02/TVAE::doc}}
\sphinxAtStartPar
El TVAE (Tabular Variational Autoencoder) es otro método avanzado de inteligencia artificial para generar datos sintéticos realísticos que combina redes neuronales con principios estadísticos. En esencia, funciona como un sistema de compresión y reconstrucción inteligente: primero, un “\sphinxstyleemphasis{encoder}” reduce los datos originales a una representación compacta llamada “espacio latente”, donde se capturan las características esenciales de la información. Luego, un “\sphinxstyleemphasis{decoder}” utiliza esta representación para recrear los datos originales o generar nuevas muestras sintéticas. Lo que diferencia al TVAE de métodos más simples es que este espacio latente no es caótico, sino que sigue una distribución gaussiana controlada, lo que permite generar datos variados pero coherentes con la estructura original.

\sphinxAtStartPar
Matemáticamente, el TVAE se basa en el concepto de “inferencia variacional”, que busca aproximar distribuciones complejas de datos mediante modelos probabilísticos. Durante el entrenamiento, el sistema no solo aprende a reconstruir datos, sino que también minimiza la divergencia entre la distribución del espacio latente y una distribución gaussiana estándar. Esto se logra mediante una función de pérdida que balancea dos objetivos: la precisión en la reconstrucción (para que los datos sintéticos se parezcan a los reales) y la regularización (para que el espacio latente mantenga una estructura estadística ordenada). Este equilibrio evita que el modelo genere datos idénticos a los originales o, por el contrario, produzca información sin sentido.

\sphinxAtStartPar
Para adaptarse específicamente a datos tabulares (como es en nuestro caso), el TVAE incorpora transformaciones especializadas. Así, las variables continuas se modelan como mezclas de distribuciones gaussianas, lo que permite capturar multimodalidad (varios picos en los datos), mientras que las variables categóricas se procesan con codificaciones que preservan sus relaciones. Esta flexibilidad permite al TVAE manejar relaciones no lineales y complejas entre variables, como las típicas en datos biológicos, donde métodos tradicionales basados en correlaciones lineales resultan insuficientes.

\sphinxAtStartPar
La principal ventaja del TVAE frente a métodos como Gaussian Copula radica en su capacidad para aprender patrones intrincados sin depender de suposiciones simplificadas sobre la estructura de los datos. Sin embargo, requiere mayor poder computacional y ajuste fino de hiperparámetros (como la dimensión del espacio latente o el peso de la regularización). Por esto, es ideal para casos donde la calidad de los datos sintéticos es crítica, como en investigación médica o biológica \sphinxhref{https://doi.org/10.48550/arXiv.2205.03257}{{[}Jordon et al., 2022{]}}, donde preservar la complejidad de las distribuciones originales es esencial para validaciones posteriores. Su fundamento matemático lo convierte en una herramienta robusta para expandir conjuntos de datos limitados manteniendo la consistencia estadística de la población original.

\begin{sphinxuseclass}{cell}\begin{sphinxVerbatimInput}

\begin{sphinxuseclass}{cell_input}
\begin{sphinxVerbatim}[commandchars=\\\{\}]
\PYG{k+kn}{import}\PYG{+w}{ }\PYG{n+nn}{pandas}\PYG{+w}{ }\PYG{k}{as}\PYG{+w}{ }\PYG{n+nn}{pd}
\PYG{k+kn}{import}\PYG{+w}{ }\PYG{n+nn}{numpy}\PYG{+w}{ }\PYG{k}{as}\PYG{+w}{ }\PYG{n+nn}{np}
\PYG{k+kn}{from}\PYG{+w}{ }\PYG{n+nn}{sdv}\PYG{n+nn}{.}\PYG{n+nn}{single\PYGZus{}table}\PYG{+w}{ }\PYG{k+kn}{import} \PYG{n}{TVAESynthesizer}
\PYG{k+kn}{from}\PYG{+w}{ }\PYG{n+nn}{sdv}\PYG{n+nn}{.}\PYG{n+nn}{metadata}\PYG{+w}{ }\PYG{k+kn}{import} \PYG{n}{SingleTableMetadata}
\PYG{k+kn}{from}\PYG{+w}{ }\PYG{n+nn}{sdv}\PYG{n+nn}{.}\PYG{n+nn}{evaluation}\PYG{n+nn}{.}\PYG{n+nn}{single\PYGZus{}table}\PYG{+w}{ }\PYG{k+kn}{import} \PYG{n}{run\PYGZus{}diagnostic}\PYG{p}{,} \PYG{n}{evaluate\PYGZus{}quality}
\PYG{k+kn}{from}\PYG{+w}{ }\PYG{n+nn}{sdv}\PYG{n+nn}{.}\PYG{n+nn}{evaluation}\PYG{n+nn}{.}\PYG{n+nn}{single\PYGZus{}table}\PYG{+w}{ }\PYG{k+kn}{import} \PYG{n}{get\PYGZus{}column\PYGZus{}plot}
\PYG{k+kn}{from}\PYG{+w}{ }\PYG{n+nn}{sdv}\PYG{n+nn}{.}\PYG{n+nn}{metadata}\PYG{+w}{ }\PYG{k+kn}{import} \PYG{n}{Metadata}
\PYG{k+kn}{import}\PYG{+w}{ }\PYG{n+nn}{matplotlib}\PYG{n+nn}{.}\PYG{n+nn}{pyplot}\PYG{+w}{ }\PYG{k}{as}\PYG{+w}{ }\PYG{n+nn}{plt}
\PYG{k+kn}{import}\PYG{+w}{ }\PYG{n+nn}{seaborn}\PYG{+w}{ }\PYG{k}{as}\PYG{+w}{ }\PYG{n+nn}{sns}
\PYG{k+kn}{from}\PYG{+w}{ }\PYG{n+nn}{pathlib}\PYG{+w}{ }\PYG{k+kn}{import} \PYG{n}{Path}
\PYG{k+kn}{from}\PYG{+w}{ }\PYG{n+nn}{myst\PYGZus{}nb}\PYG{+w}{ }\PYG{k+kn}{import} \PYG{n}{glue}

\PYG{n}{num\PYGZus{}synthetic} \PYG{o}{=} \PYG{l+m+mi}{500} \PYG{c+c1}{\PYGZsh{}variable para definir el número de registros sintéticos}

\PYG{c+c1}{\PYGZsh{} 1. Cargar datos reales}
\PYG{c+c1}{\PYGZsh{}\PYGZhy{}\PYGZhy{}\PYGZhy{}\PYGZhy{}\PYGZhy{}\PYGZhy{}\PYGZhy{}\PYGZhy{}\PYGZhy{}\PYGZhy{}\PYGZhy{}\PYGZhy{}\PYGZhy{}\PYGZhy{}\PYGZhy{}\PYGZhy{}\PYGZhy{}\PYGZhy{}\PYGZhy{}\PYGZhy{}\PYGZhy{}\PYGZhy{}\PYGZhy{}\PYGZhy{}\PYGZhy{}\PYGZhy{}}
\PYG{n}{path\PYGZus{}realData} \PYG{o}{=} \PYG{l+s+s1}{\PYGZsq{}}\PYG{l+s+s1}{.././data/Dimensiones\PYGZus{}lenguado.xlsx}\PYG{l+s+s1}{\PYGZsq{}}
\PYG{n}{real\PYGZus{}data} \PYG{o}{=} \PYG{n}{pd}\PYG{o}{.}\PYG{n}{read\PYGZus{}excel}\PYG{p}{(}\PYG{n}{path\PYGZus{}realData}\PYG{p}{)}

\PYG{c+c1}{\PYGZsh{} 2. Obtener los metadata del dataset}
\PYG{c+c1}{\PYGZsh{}\PYGZhy{}\PYGZhy{}\PYGZhy{}\PYGZhy{}\PYGZhy{}\PYGZhy{}\PYGZhy{}\PYGZhy{}\PYGZhy{}\PYGZhy{}\PYGZhy{}\PYGZhy{}\PYGZhy{}\PYGZhy{}\PYGZhy{}\PYGZhy{}\PYGZhy{}\PYGZhy{}\PYGZhy{}\PYGZhy{}\PYGZhy{}\PYGZhy{}\PYGZhy{}\PYGZhy{}\PYGZhy{}\PYGZhy{}\PYGZhy{}\PYGZhy{}\PYGZhy{}\PYGZhy{}\PYGZhy{}\PYGZhy{}\PYGZhy{}\PYGZhy{}\PYGZhy{}\PYGZhy{}\PYGZhy{}\PYGZhy{}}

\PYG{n}{metadata\PYGZus{}path} \PYG{o}{=} \PYG{n}{Path}\PYG{p}{(}\PYG{l+s+s1}{\PYGZsq{}}\PYG{l+s+s1}{.././data/metadata\PYGZus{}lenguado.json}\PYG{l+s+s1}{\PYGZsq{}}\PYG{p}{)}

\PYG{k}{if} \PYG{n}{metadata\PYGZus{}path}\PYG{o}{.}\PYG{n}{exists}\PYG{p}{(}\PYG{p}{)}\PYG{p}{:}
    \PYG{c+c1}{\PYGZsh{} Cargar metadatos desde el JSON (evita el warning)}
    \PYG{n}{metadata} \PYG{o}{=} \PYG{n}{Metadata}\PYG{o}{.}\PYG{n}{load\PYGZus{}from\PYGZus{}json}\PYG{p}{(}\PYG{n}{metadata\PYGZus{}path}\PYG{p}{)}
    \PYG{n+nb}{print}\PYG{p}{(}\PYG{l+s+s2}{\PYGZdq{}}\PYG{l+s+s2}{Metadatos cargados desde JSON.}\PYG{l+s+s2}{\PYGZdq{}}\PYG{p}{)}
\PYG{k}{else}\PYG{p}{:}
    \PYG{c+c1}{\PYGZsh{} Detectar metadatos y guardarlos en JSON para futuras ejecuciones}
    \PYG{n}{metadata} \PYG{o}{=} \PYG{n}{Metadata}\PYG{o}{.}\PYG{n}{detect\PYGZus{}from\PYGZus{}dataframe}\PYG{p}{(}\PYG{n}{real\PYGZus{}data}\PYG{p}{)}
    \PYG{n}{metadata}\PYG{o}{.}\PYG{n}{save\PYGZus{}to\PYGZus{}json}\PYG{p}{(}\PYG{n}{metadata\PYGZus{}path}\PYG{p}{)}
    \PYG{n+nb}{print}\PYG{p}{(}\PYG{l+s+s2}{\PYGZdq{}}\PYG{l+s+s2}{Metadatos detectados y guardados en JSON.}\PYG{l+s+s2}{\PYGZdq{}}\PYG{p}{)}

\PYG{c+c1}{\PYGZsh{} 3. Inicializar y entrenar TVAE}
\PYG{c+c1}{\PYGZsh{}\PYGZhy{}\PYGZhy{}\PYGZhy{}\PYGZhy{}\PYGZhy{}\PYGZhy{}\PYGZhy{}\PYGZhy{}\PYGZhy{}\PYGZhy{}\PYGZhy{}\PYGZhy{}\PYGZhy{}\PYGZhy{}\PYGZhy{}\PYGZhy{}\PYGZhy{}\PYGZhy{}\PYGZhy{}\PYGZhy{}\PYGZhy{}\PYGZhy{}\PYGZhy{}\PYGZhy{}\PYGZhy{}\PYGZhy{}\PYGZhy{}\PYGZhy{}\PYGZhy{}\PYGZhy{}\PYGZhy{}\PYGZhy{}}
\PYG{n+nb}{print}\PYG{p}{(}\PYG{l+s+s2}{\PYGZdq{}}\PYG{l+s+s2}{Inicializando TVAE...}\PYG{l+s+s2}{\PYGZdq{}}\PYG{p}{)}
\PYG{n}{tvae\PYGZus{}model} \PYG{o}{=} \PYG{n}{TVAESynthesizer}\PYG{p}{(}
    \PYG{n}{metadata} \PYG{o}{=} \PYG{n}{metadata}\PYG{p}{,}
    \PYG{n}{epochs}\PYG{o}{=}\PYG{l+m+mi}{300}\PYG{p}{,}           \PYG{c+c1}{\PYGZsh{} Número de épocas de entrenamiento}
    \PYG{n}{batch\PYGZus{}size}\PYG{o}{=}\PYG{l+m+mi}{32}\PYG{p}{,}        \PYG{c+c1}{\PYGZsh{} Tamaño del batch}
    \PYG{n}{embedding\PYGZus{}dim}\PYG{o}{=}\PYG{l+m+mi}{128}\PYG{p}{,}    \PYG{c+c1}{\PYGZsh{} Dimensión del espacio latente}
    \PYG{n}{compress\PYGZus{}dims}\PYG{o}{=}\PYG{p}{(}\PYG{l+m+mi}{128}\PYG{p}{,} \PYG{l+m+mi}{128}\PYG{p}{)}\PYG{p}{,}  \PYG{c+c1}{\PYGZsh{} Capas del encoder}
    \PYG{n}{decompress\PYGZus{}dims}\PYG{o}{=}\PYG{p}{(}\PYG{l+m+mi}{128}\PYG{p}{,} \PYG{l+m+mi}{128}\PYG{p}{)}\PYG{p}{,} \PYG{c+c1}{\PYGZsh{} Capas del decoder}
    \PYG{n}{loss\PYGZus{}factor}\PYG{o}{=}\PYG{l+m+mi}{2}\PYG{p}{,}        \PYG{c+c1}{\PYGZsh{} Balance entre pérdida de reconstrucción y KL}
    \PYG{n}{cuda}\PYG{o}{=}\PYG{k+kc}{True}             \PYG{c+c1}{\PYGZsh{} Usar GPU si está disponible}
\PYG{p}{)}

\PYG{n+nb}{print}\PYG{p}{(}\PYG{l+s+s2}{\PYGZdq{}}\PYG{l+s+s2}{Entrenando TVAE (esto puede tomar varios minutos)...}\PYG{l+s+s2}{\PYGZdq{}}\PYG{p}{)}
\PYG{n}{tvae\PYGZus{}model}\PYG{o}{.}\PYG{n}{fit}\PYG{p}{(}\PYG{n}{real\PYGZus{}data}\PYG{p}{)}

\PYG{c+c1}{\PYGZsh{} 4. Generar datos sintéticos}
\PYG{c+c1}{\PYGZsh{}\PYGZhy{}\PYGZhy{}\PYGZhy{}\PYGZhy{}\PYGZhy{}\PYGZhy{}\PYGZhy{}\PYGZhy{}\PYGZhy{}\PYGZhy{}\PYGZhy{}\PYGZhy{}\PYGZhy{}\PYGZhy{}\PYGZhy{}\PYGZhy{}\PYGZhy{}\PYGZhy{}\PYGZhy{}\PYGZhy{}\PYGZhy{}\PYGZhy{}\PYGZhy{}\PYGZhy{}\PYGZhy{}\PYGZhy{}\PYGZhy{}\PYGZhy{}\PYGZhy{}}
\PYG{n+nb}{print}\PYG{p}{(}\PYG{l+s+s2}{\PYGZdq{}}\PYG{l+s+s2}{Generando datos sintéticos...}\PYG{l+s+s2}{\PYGZdq{}}\PYG{p}{)}
\PYG{n}{syntheticTVAE\PYGZus{}data} \PYG{o}{=} \PYG{n}{tvae\PYGZus{}model}\PYG{o}{.}\PYG{n}{sample}\PYG{p}{(}\PYG{n}{num\PYGZus{}rows}\PYG{o}{=}\PYG{l+m+mi}{500}\PYG{p}{)}\PYG{p}{;}
\PYG{n}{display}\PYG{p}{(}\PYG{n}{syntheticTVAE\PYGZus{}data}\PYG{o}{.}\PYG{n}{head}\PYG{p}{(}\PYG{l+m+mi}{10}\PYG{p}{)}\PYG{p}{)}
\PYG{n+nb}{print}\PYG{p}{(}\PYG{l+s+sa}{f}\PYG{l+s+s2}{\PYGZdq{}}\PYG{l+s+s2}{Se muestran 10 primeros registros sintéticos TVAE de un total de }\PYG{l+s+si}{\PYGZob{}}\PYG{n}{num\PYGZus{}synthetic}\PYG{l+s+si}{\PYGZcb{}}\PYG{l+s+s2}{\PYGZdq{}}\PYG{p}{)}

\PYG{c+c1}{\PYGZsh{} 5. Guardar resultados}
\PYG{c+c1}{\PYGZsh{}\PYGZhy{}\PYGZhy{}\PYGZhy{}\PYGZhy{}\PYGZhy{}\PYGZhy{}\PYGZhy{}\PYGZhy{}\PYGZhy{}\PYGZhy{}\PYGZhy{}\PYGZhy{}\PYGZhy{}\PYGZhy{}\PYGZhy{}\PYGZhy{}\PYGZhy{}\PYGZhy{}\PYGZhy{}\PYGZhy{}\PYGZhy{}\PYGZhy{}\PYGZhy{}\PYGZhy{}\PYGZhy{}}
\PYG{n}{path\PYGZus{}syntheticTVAEData} \PYG{o}{=} \PYG{l+s+s2}{\PYGZdq{}}\PYG{l+s+s2}{.././data/SyntheticTVAE.xlsx}\PYG{l+s+s2}{\PYGZdq{}}
\PYG{n}{syntheticTVAE\PYGZus{}data}\PYG{o}{.}\PYG{n}{to\PYGZus{}excel}\PYG{p}{(}\PYG{n}{path\PYGZus{}syntheticTVAEData}\PYG{p}{,} \PYG{n}{index}\PYG{o}{=}\PYG{k+kc}{False}\PYG{p}{,} \PYG{n}{engine}\PYG{o}{=}\PYG{l+s+s1}{\PYGZsq{}}\PYG{l+s+s1}{openpyxl}\PYG{l+s+s1}{\PYGZsq{}}\PYG{p}{)}

\PYG{c+c1}{\PYGZsh{} 6. Evaluación datos sintéticos}
\PYG{c+c1}{\PYGZsh{}\PYGZhy{}\PYGZhy{}\PYGZhy{}\PYGZhy{}\PYGZhy{}\PYGZhy{}\PYGZhy{}\PYGZhy{}\PYGZhy{}\PYGZhy{}\PYGZhy{}\PYGZhy{}\PYGZhy{}\PYGZhy{}\PYGZhy{}\PYGZhy{}\PYGZhy{}\PYGZhy{}\PYGZhy{}\PYGZhy{}\PYGZhy{}\PYGZhy{}\PYGZhy{}\PYGZhy{}\PYGZhy{}\PYGZhy{}\PYGZhy{}\PYGZhy{}\PYGZhy{}\PYGZhy{}\PYGZhy{}}

\PYG{c+c1}{\PYGZsh{}\PYGZsh{} Diagnóstico básico}
\PYG{n}{diagnosticTVAE} \PYG{o}{=} \PYG{n}{run\PYGZus{}diagnostic}\PYG{p}{(}\PYG{n}{real\PYGZus{}data}\PYG{p}{,} \PYG{n}{syntheticTVAE\PYGZus{}data}\PYG{p}{,} \PYG{n}{metadata}\PYG{p}{,} \PYG{k+kc}{False}\PYG{p}{)}
\PYG{n+nb}{print}\PYG{p}{(}\PYG{l+s+s2}{\PYGZdq{}}\PYG{l+s+se}{\PYGZbs{}n}\PYG{l+s+s2}{DIAGNÓSTICO BÁSICO CALIDAD DATOS}\PYG{l+s+se}{\PYGZbs{}n}\PYG{l+s+s2}{\PYGZdq{}}\PYG{p}{)}
\PYG{n+nb}{print}\PYG{p}{(}\PYG{n}{diagnosticTVAE}\PYG{o}{.}\PYG{n}{get\PYGZus{}details}\PYG{p}{(}\PYG{n}{property\PYGZus{}name}\PYG{o}{=}\PYG{l+s+s1}{\PYGZsq{}}\PYG{l+s+s1}{Data Validity}\PYG{l+s+s1}{\PYGZsq{}}\PYG{p}{)}\PYG{p}{)}

\PYG{c+c1}{\PYGZsh{}\PYGZsh{} Visualización datos}
\PYG{c+c1}{\PYGZsh{}\PYGZhy{}\PYGZhy{}\PYGZhy{}\PYGZhy{}\PYGZhy{}\PYGZhy{}\PYGZhy{}\PYGZhy{}\PYGZhy{}\PYGZhy{}\PYGZhy{}\PYGZhy{}\PYGZhy{}\PYGZhy{}\PYGZhy{}\PYGZhy{}\PYGZhy{}\PYGZhy{}\PYGZhy{}\PYGZhy{}\PYGZhy{}\PYGZhy{}\PYGZhy{}\PYGZhy{}\PYGZhy{}\PYGZhy{}\PYGZhy{}\PYGZhy{}\PYGZhy{}\PYGZhy{}\PYGZhy{}\PYGZhy{}}

\PYG{n}{columnas} \PYG{o}{=} \PYG{p}{[}\PYG{l+s+s2}{\PYGZdq{}}\PYG{l+s+s2}{Peso (g)}\PYG{l+s+s2}{\PYGZdq{}}\PYG{p}{,} \PYG{l+s+s2}{\PYGZdq{}}\PYG{l+s+s2}{Longitud (cm)}\PYG{l+s+s2}{\PYGZdq{}}\PYG{p}{,} \PYG{l+s+s2}{\PYGZdq{}}\PYG{l+s+s2}{Anchura (cm)}\PYG{l+s+s2}{\PYGZdq{}}\PYG{p}{,} \PYG{l+s+s2}{\PYGZdq{}}\PYG{l+s+s2}{Altura (cm)}\PYG{l+s+s2}{\PYGZdq{}}\PYG{p}{]}
\PYG{n}{fig}\PYG{p}{,} \PYG{n}{axes} \PYG{o}{=} \PYG{n}{plt}\PYG{o}{.}\PYG{n}{subplots}\PYG{p}{(}\PYG{l+m+mi}{2}\PYG{p}{,} \PYG{l+m+mi}{2}\PYG{p}{,} \PYG{n}{figsize}\PYG{o}{=}\PYG{p}{(}\PYG{l+m+mi}{15}\PYG{p}{,} \PYG{l+m+mi}{10}\PYG{p}{)}\PYG{p}{)}
\PYG{n}{variables} \PYG{o}{=} \PYG{n}{real\PYGZus{}data}\PYG{o}{.}\PYG{n}{columns}

\PYG{k}{for} \PYG{n}{i}\PYG{p}{,} \PYG{n}{var} \PYG{o+ow}{in} \PYG{n+nb}{enumerate}\PYG{p}{(}\PYG{n}{variables}\PYG{p}{)}\PYG{p}{:}
    \PYG{n}{row}\PYG{p}{,} \PYG{n}{col} \PYG{o}{=} \PYG{n}{i} \PYG{o}{/}\PYG{o}{/} \PYG{l+m+mi}{2}\PYG{p}{,} \PYG{n}{i} \PYG{o}{\PYGZpc{}} \PYG{l+m+mi}{2}
    \PYG{n}{sns}\PYG{o}{.}\PYG{n}{kdeplot}\PYG{p}{(}\PYG{n}{real\PYGZus{}data}\PYG{p}{[}\PYG{n}{var}\PYG{p}{]}\PYG{p}{,} \PYG{n}{ax}\PYG{o}{=}\PYG{n}{axes}\PYG{p}{[}\PYG{n}{row}\PYG{p}{,} \PYG{n}{col}\PYG{p}{]}\PYG{p}{,} \PYG{n}{label}\PYG{o}{=}\PYG{l+s+s1}{\PYGZsq{}}\PYG{l+s+s1}{Real}\PYG{l+s+s1}{\PYGZsq{}}\PYG{p}{,} \PYG{n}{fill}\PYG{o}{=}\PYG{k+kc}{True}\PYG{p}{,} \PYG{n}{alpha}\PYG{o}{=}\PYG{l+m+mf}{0.7}\PYG{p}{)}
    \PYG{n}{sns}\PYG{o}{.}\PYG{n}{kdeplot}\PYG{p}{(}\PYG{n}{syntheticTVAE\PYGZus{}data}\PYG{p}{[}\PYG{n}{var}\PYG{p}{]}\PYG{p}{,} \PYG{n}{ax}\PYG{o}{=}\PYG{n}{axes}\PYG{p}{[}\PYG{n}{row}\PYG{p}{,} \PYG{n}{col}\PYG{p}{]}\PYG{p}{,} \PYG{n}{label}\PYG{o}{=}\PYG{l+s+s1}{\PYGZsq{}}\PYG{l+s+s1}{Sintético}\PYG{l+s+s1}{\PYGZsq{}}\PYG{p}{,} \PYG{n}{fill}\PYG{o}{=}\PYG{k+kc}{True}\PYG{p}{,} \PYG{n}{alpha}\PYG{o}{=}\PYG{l+m+mf}{0.7}\PYG{p}{)}
    \PYG{n}{axes}\PYG{p}{[}\PYG{n}{row}\PYG{p}{,} \PYG{n}{col}\PYG{p}{]}\PYG{o}{.}\PYG{n}{set\PYGZus{}title}\PYG{p}{(}\PYG{l+s+sa}{f}\PYG{l+s+s1}{\PYGZsq{}}\PYG{l+s+s1}{Distribución de }\PYG{l+s+si}{\PYGZob{}}\PYG{n}{var}\PYG{l+s+si}{\PYGZcb{}}\PYG{l+s+s1}{\PYGZsq{}}\PYG{p}{)}
    \PYG{n}{axes}\PYG{p}{[}\PYG{n}{row}\PYG{p}{,} \PYG{n}{col}\PYG{p}{]}\PYG{o}{.}\PYG{n}{legend}\PYG{p}{(}\PYG{p}{)}

\PYG{n}{plt}\PYG{o}{.}\PYG{n}{suptitle}\PYG{p}{(}\PYG{l+s+s1}{\PYGZsq{}}\PYG{l+s+s1}{Comparación: Datos Reales vs Sintéticos (TVAE)}\PYG{l+s+s1}{\PYGZsq{}}\PYG{p}{,} \PYG{n}{fontsize}\PYG{o}{=}\PYG{l+m+mi}{16}\PYG{p}{,} \PYG{n}{y}\PYG{o}{=}\PYG{l+m+mf}{0.98}\PYG{p}{)}
\PYG{n}{plt}\PYG{o}{.}\PYG{n}{tight\PYGZus{}layout}\PYG{p}{(}\PYG{p}{)}
\PYG{n}{plt}\PYG{o}{.}\PYG{n}{show}\PYG{p}{(}\PYG{p}{)}


\PYG{c+c1}{\PYGZsh{}\PYGZsh{} Medidas de similiud estadística}

\PYG{n}{quality\PYGZus{}reportTVAE} \PYG{o}{=} \PYG{n}{evaluate\PYGZus{}quality}\PYG{p}{(}\PYG{n}{real\PYGZus{}data}\PYG{p}{,} \PYG{n}{syntheticTVAE\PYGZus{}data}\PYG{o}{.}\PYG{n}{sample}\PYG{p}{(}\PYG{n}{n}\PYG{o}{=}\PYG{n+nb}{len}\PYG{p}{(}\PYG{n}{real\PYGZus{}data}\PYG{p}{)}\PYG{p}{,} \PYG{n}{random\PYGZus{}state}\PYG{o}{=}\PYG{l+m+mi}{42}\PYG{p}{)}\PYG{p}{,} \PYG{n}{metadata}\PYG{p}{,} \PYG{k+kc}{False}\PYG{p}{)}

\PYG{c+c1}{\PYGZsh{}\PYGZsh{} Obtenemos el valor promedio de la similitud}
\PYG{n}{overall\PYGZus{}scoreTVAE} \PYG{o}{=} \PYG{l+s+sa}{f}\PYG{l+s+s2}{\PYGZdq{}}\PYG{l+s+si}{\PYGZob{}}\PYG{n}{quality\PYGZus{}reportTVAE}\PYG{o}{.}\PYG{n}{get\PYGZus{}score}\PYG{p}{(}\PYG{p}{)}\PYG{o}{*}\PYG{l+m+mi}{100}\PYG{l+s+si}{:}\PYG{l+s+s2}{.2f}\PYG{l+s+si}{\PYGZcb{}}\PYG{l+s+s2}{\PYGZpc{}}\PYG{l+s+s2}{\PYGZdq{}}
\PYG{n}{glue}\PYG{p}{(}\PYG{l+s+s2}{\PYGZdq{}}\PYG{l+s+s2}{scoreTVAE}\PYG{l+s+s2}{\PYGZdq{}}\PYG{p}{,} \PYG{n}{overall\PYGZus{}scoreTVAE}\PYG{p}{,} \PYG{n}{display}\PYG{o}{=}\PYG{k+kc}{False}\PYG{p}{)}\PYG{p}{;} \PYG{c+c1}{\PYGZsh{} Definimos una variable para usar en el texto markdown}
\PYG{n+nb}{print}\PYG{p}{(}\PYG{l+s+s2}{\PYGZdq{}}\PYG{l+s+se}{\PYGZbs{}n}\PYG{l+s+s2}{SIMILITUD ESTADÍSTICA}\PYG{l+s+se}{\PYGZbs{}n}\PYG{l+s+s2}{\PYGZdq{}}\PYG{p}{)}
\PYG{n+nb}{print}\PYG{p}{(}\PYG{l+s+sa}{f}\PYG{l+s+s2}{\PYGZdq{}}\PYG{l+s+s2}{Overall score: }\PYG{l+s+si}{\PYGZob{}}\PYG{n}{overall\PYGZus{}scoreTVAE}\PYG{l+s+si}{\PYGZcb{}}\PYG{l+s+se}{\PYGZbs{}n}\PYG{l+s+s2}{\PYGZdq{}}\PYG{p}{)}

\PYG{n+nb}{print}\PYG{p}{(}\PYG{l+s+s2}{\PYGZdq{}}\PYG{l+s+se}{\PYGZbs{}n}\PYG{l+s+s2}{ Distribuciones individuales real vs. sintética}\PYG{l+s+se}{\PYGZbs{}n}\PYG{l+s+s2}{\PYGZdq{}}\PYG{p}{)}
\PYG{n+nb}{print}\PYG{p}{(}\PYG{n}{quality\PYGZus{}reportTVAE}\PYG{o}{.}\PYG{n}{get\PYGZus{}details}\PYG{p}{(}\PYG{l+s+s1}{\PYGZsq{}}\PYG{l+s+s1}{Column Shapes}\PYG{l+s+s1}{\PYGZsq{}}\PYG{p}{)}\PYG{p}{)}

\PYG{n+nb}{print}\PYG{p}{(}\PYG{l+s+s2}{\PYGZdq{}}\PYG{l+s+se}{\PYGZbs{}n}\PYG{l+s+s2}{ Correlaciones entre variables real vs. sintética}\PYG{l+s+se}{\PYGZbs{}n}\PYG{l+s+s2}{\PYGZdq{}}\PYG{p}{)}
\PYG{n+nb}{print}\PYG{p}{(}\PYG{n}{quality\PYGZus{}reportTVAE}\PYG{o}{.}\PYG{n}{get\PYGZus{}details}\PYG{p}{(}\PYG{l+s+s1}{\PYGZsq{}}\PYG{l+s+s1}{Column Pair Trends}\PYG{l+s+s1}{\PYGZsq{}}\PYG{p}{)}\PYG{p}{)}
\end{sphinxVerbatim}

\end{sphinxuseclass}\end{sphinxVerbatimInput}
\begin{sphinxVerbatimOutput}

\begin{sphinxuseclass}{cell_output}
\begin{sphinxVerbatim}[commandchars=\\\{\}]
Metadatos cargados desde JSON.
Inicializando TVAE...
Entrenando TVAE (esto puede tomar varios minutos)...
\end{sphinxVerbatim}

\begin{sphinxVerbatim}[commandchars=\\\{\}]
  File \PYGZdq{}C:\PYGZbs{}Users\PYGZbs{}javie\PYGZbs{}Workspaces\PYGZbs{}base12\PYGZbs{}Lib\PYGZbs{}site\PYGZhy{}packages\PYGZbs{}joblib\PYGZbs{}externals\PYGZbs{}loky\PYGZbs{}backend\PYGZbs{}context.py\PYGZdq{}, line 257, in \PYGZus{}count\PYGZus{}physical\PYGZus{}cores
    cpu\PYGZus{}info = subprocess.run(
               \PYGZca{}\PYGZca{}\PYGZca{}\PYGZca{}\PYGZca{}\PYGZca{}\PYGZca{}\PYGZca{}\PYGZca{}\PYGZca{}\PYGZca{}\PYGZca{}\PYGZca{}\PYGZca{}\PYGZca{}
  File \PYGZdq{}C:\PYGZbs{}Python\PYGZbs{}Python312\PYGZbs{}Lib\PYGZbs{}subprocess.py\PYGZdq{}, line 548, in run
    with Popen(*popenargs, **kwargs) as process:
         \PYGZca{}\PYGZca{}\PYGZca{}\PYGZca{}\PYGZca{}\PYGZca{}\PYGZca{}\PYGZca{}\PYGZca{}\PYGZca{}\PYGZca{}\PYGZca{}\PYGZca{}\PYGZca{}\PYGZca{}\PYGZca{}\PYGZca{}\PYGZca{}\PYGZca{}\PYGZca{}\PYGZca{}\PYGZca{}\PYGZca{}\PYGZca{}\PYGZca{}\PYGZca{}\PYGZca{}
  File \PYGZdq{}C:\PYGZbs{}Python\PYGZbs{}Python312\PYGZbs{}Lib\PYGZbs{}subprocess.py\PYGZdq{}, line 1026, in \PYGZus{}\PYGZus{}init\PYGZus{}\PYGZus{}
    self.\PYGZus{}execute\PYGZus{}child(args, executable, preexec\PYGZus{}fn, close\PYGZus{}fds,
  File \PYGZdq{}C:\PYGZbs{}Python\PYGZbs{}Python312\PYGZbs{}Lib\PYGZbs{}subprocess.py\PYGZdq{}, line 1538, in \PYGZus{}execute\PYGZus{}child
    hp, ht, pid, tid = \PYGZus{}winapi.CreateProcess(executable, args,
                       \PYGZca{}\PYGZca{}\PYGZca{}\PYGZca{}\PYGZca{}\PYGZca{}\PYGZca{}\PYGZca{}\PYGZca{}\PYGZca{}\PYGZca{}\PYGZca{}\PYGZca{}\PYGZca{}\PYGZca{}\PYGZca{}\PYGZca{}\PYGZca{}\PYGZca{}\PYGZca{}\PYGZca{}\PYGZca{}\PYGZca{}\PYGZca{}\PYGZca{}\PYGZca{}\PYGZca{}\PYGZca{}\PYGZca{}\PYGZca{}\PYGZca{}\PYGZca{}\PYGZca{}\PYGZca{}\PYGZca{}\PYGZca{}\PYGZca{}\PYGZca{}\PYGZca{}
\end{sphinxVerbatim}

\begin{sphinxVerbatim}[commandchars=\\\{\}]
Generando datos sintéticos...
\end{sphinxVerbatim}

\begin{sphinxVerbatim}[commandchars=\\\{\}]
   Peso (g)  Longitud (cm)  Anchura (cm)  Altura (cm)
0      2.77            5.9           2.4          0.4
1      3.77            7.3           2.3          0.4
2     10.62            9.3           3.9          0.6
3     11.72            9.5           3.8          0.6
4      4.73            8.2           3.4          0.5
5      3.57            8.3           2.4          0.4
6      2.50            5.2           2.3          0.3
7      2.97            6.2           2.3          0.4
8      7.94            9.3           3.0          0.5
9      2.92            6.0           2.0          0.3
\end{sphinxVerbatim}

\begin{sphinxVerbatim}[commandchars=\\\{\}]
Se muestran 10 primeros registros sintéticos TVAE de un total de 500

DIAGNÓSTICO BÁSICO CALIDAD DATOS

          Column             Metric  Score
0       Peso (g)  BoundaryAdherence    1.0
1  Longitud (cm)  BoundaryAdherence    1.0
2   Anchura (cm)  BoundaryAdherence    1.0
3    Altura (cm)  BoundaryAdherence    1.0
\end{sphinxVerbatim}

\noindent\sphinxincludegraphics{{f66db7e2964f01c988dc71e4e04f37bce31b839357af3158c4fca1afefd83e3a}.png}

\begin{sphinxVerbatim}[commandchars=\\\{\}]
SIMILITUD ESTADÍSTICA

Overall score: 93.71\PYGZpc{}


 Distribuciones individuales real vs. sintética

          Column        Metric     Score
0       Peso (g)  KSComplement  0.861244
1  Longitud (cm)  KSComplement  0.923445
2   Anchura (cm)  KSComplement  0.952153
3    Altura (cm)  KSComplement  0.947368

 Correlaciones entre variables real vs. sintética

        Column 1       Column 2                 Metric     Score  \PYGZbs{}
0       Peso (g)  Longitud (cm)  CorrelationSimilarity  0.921427
1       Peso (g)   Anchura (cm)  CorrelationSimilarity  0.929216
2       Peso (g)    Altura (cm)  CorrelationSimilarity  0.983946
3  Longitud (cm)   Anchura (cm)  CorrelationSimilarity  0.904529
4  Longitud (cm)    Altura (cm)  CorrelationSimilarity  0.986978
5   Anchura (cm)    Altura (cm)  CorrelationSimilarity  0.992524

   Real Correlation  Synthetic Correlation
0          0.932951               0.775805
1          0.942354               0.800787
2          0.861768               0.829660
3          0.936674               0.745732
4          0.870165               0.844121
5          0.842673               0.827720
\end{sphinxVerbatim}

\end{sphinxuseclass}\end{sphinxVerbatimOutput}

\end{sphinxuseclass}
\sphinxAtStartPar
En líneas generales, este método proporciona una similud estadística general del \DUrole{output,text_plain}{‘93.71\%’}. Un análisis con mayor detalle nos revela que la generación de datos sintéticos a nivel de variable individual mantiene razonablemente bien las características originales (\(\approx 90\%\)). Por otro lado, la métrica \sphinxcode{\sphinxupquote{CorrelationSimilarity}} entre pares de variables es elevada aunque inferior al método Gaussian Copula. En lo que se refiere a las correlaciones sintéticas entre variables, los valores porcentuales son bastante menores que los correspondientes referencias con datos reales, lo cual sugiere que los datos obtenidos pueden no ser válidos para entrenamiento de algoritmos de \sphinxstyleemphasis{machine learning}.

\sphinxstepscope


\section{Conclusión}
\label{\detokenize{content/02/Conclusion:conclusion}}\label{\detokenize{content/02/Conclusion::doc}}
\sphinxAtStartPar
En la siguiente gráfica se puede apreciar el comportamiento de los tres métodos de generación de datos sintéticos estudiados.

\begin{sphinxuseclass}{cell}\begin{sphinxVerbatimInput}

\begin{sphinxuseclass}{cell_input}
\begin{sphinxVerbatim}[commandchars=\\\{\}]
\PYG{k+kn}{import}\PYG{+w}{ }\PYG{n+nn}{pandas}\PYG{+w}{ }\PYG{k}{as}\PYG{+w}{ }\PYG{n+nn}{pd}
\PYG{k+kn}{import}\PYG{+w}{ }\PYG{n+nn}{numpy}\PYG{+w}{ }\PYG{k}{as}\PYG{+w}{ }\PYG{n+nn}{np}
\PYG{k+kn}{import}\PYG{+w}{ }\PYG{n+nn}{matplotlib}\PYG{n+nn}{.}\PYG{n+nn}{pyplot}\PYG{+w}{ }\PYG{k}{as}\PYG{+w}{ }\PYG{n+nn}{plt}
\PYG{k+kn}{import}\PYG{+w}{ }\PYG{n+nn}{seaborn}\PYG{+w}{ }\PYG{k}{as}\PYG{+w}{ }\PYG{n+nn}{sns}
\PYG{k+kn}{from}\PYG{+w}{ }\PYG{n+nn}{pathlib}\PYG{+w}{ }\PYG{k+kn}{import} \PYG{n}{Path}
\PYG{k+kn}{from}\PYG{+w}{ }\PYG{n+nn}{myst\PYGZus{}nb}\PYG{+w}{ }\PYG{k+kn}{import} \PYG{n}{glue}

\PYG{n}{real\PYGZus{}data} \PYG{o}{=} \PYG{n}{pd}\PYG{o}{.}\PYG{n}{read\PYGZus{}excel}\PYG{p}{(}\PYG{l+s+s1}{\PYGZsq{}}\PYG{l+s+s1}{.././data/Dimensiones\PYGZus{}lenguado.xlsx}\PYG{l+s+s1}{\PYGZsq{}} \PYG{p}{)}
\PYG{n}{synthetic\PYGZus{}data} \PYG{o}{=} \PYG{n}{pd}\PYG{o}{.}\PYG{n}{read\PYGZus{}excel}\PYG{p}{(}\PYG{l+s+s1}{\PYGZsq{}}\PYG{l+s+s1}{.././data/SyntheticGaussianCopula.xlsx}\PYG{l+s+s1}{\PYGZsq{}} \PYG{p}{)}
\PYG{n}{syntheticGAN\PYGZus{}data} \PYG{o}{=} \PYG{n}{pd}\PYG{o}{.}\PYG{n}{read\PYGZus{}excel}\PYG{p}{(}\PYG{l+s+s1}{\PYGZsq{}}\PYG{l+s+s1}{.././data/SyntheticCTGAN.xlsx}\PYG{l+s+s1}{\PYGZsq{}} \PYG{p}{)}
\PYG{n}{syntheticTVAE\PYGZus{}data} \PYG{o}{=} \PYG{n}{pd}\PYG{o}{.}\PYG{n}{read\PYGZus{}excel}\PYG{p}{(}\PYG{l+s+s1}{\PYGZsq{}}\PYG{l+s+s1}{.././data/SyntheticTVAE.xlsx}\PYG{l+s+s1}{\PYGZsq{}} \PYG{p}{)}

\PYG{n}{columnas} \PYG{o}{=} \PYG{p}{[}\PYG{l+s+s2}{\PYGZdq{}}\PYG{l+s+s2}{Peso (g)}\PYG{l+s+s2}{\PYGZdq{}}\PYG{p}{,} \PYG{l+s+s2}{\PYGZdq{}}\PYG{l+s+s2}{Longitud (cm)}\PYG{l+s+s2}{\PYGZdq{}}\PYG{p}{,} \PYG{l+s+s2}{\PYGZdq{}}\PYG{l+s+s2}{Anchura (cm)}\PYG{l+s+s2}{\PYGZdq{}}\PYG{p}{,} \PYG{l+s+s2}{\PYGZdq{}}\PYG{l+s+s2}{Altura (cm)}\PYG{l+s+s2}{\PYGZdq{}}\PYG{p}{]}

\PYG{n}{fig}\PYG{p}{,} \PYG{n}{axs} \PYG{o}{=} \PYG{n}{plt}\PYG{o}{.}\PYG{n}{subplots}\PYG{p}{(}\PYG{l+m+mi}{2}\PYG{p}{,} \PYG{l+m+mi}{2}\PYG{p}{,} \PYG{n}{figsize}\PYG{o}{=}\PYG{p}{(}\PYG{l+m+mi}{15}\PYG{p}{,} \PYG{l+m+mi}{12}\PYG{p}{)}\PYG{p}{)}
\PYG{k}{for} \PYG{n}{i}\PYG{p}{,} \PYG{n}{col} \PYG{o+ow}{in} \PYG{n+nb}{enumerate}\PYG{p}{(}\PYG{n}{columnas}\PYG{p}{)}\PYG{p}{:}
    \PYG{n}{ax} \PYG{o}{=} \PYG{n}{axs}\PYG{p}{[}\PYG{n}{i}\PYG{o}{/}\PYG{o}{/}\PYG{l+m+mi}{2}\PYG{p}{,} \PYG{n}{i}\PYG{o}{\PYGZpc{}}\PYG{k}{2}]
    \PYG{n}{sns}\PYG{o}{.}\PYG{n}{kdeplot}\PYG{p}{(}\PYG{n}{real\PYGZus{}data}\PYG{p}{[}\PYG{n}{col}\PYG{p}{]}\PYG{p}{,} \PYG{n}{ax}\PYG{o}{=}\PYG{n}{ax}\PYG{p}{,} \PYG{n}{color}\PYG{o}{=}\PYG{l+s+s2}{\PYGZdq{}}\PYG{l+s+s2}{\PYGZsh{}1f77b4}\PYG{l+s+s2}{\PYGZdq{}}\PYG{p}{,} \PYG{n}{label}\PYG{o}{=}\PYG{l+s+s2}{\PYGZdq{}}\PYG{l+s+s2}{Real}\PYG{l+s+s2}{\PYGZdq{}}\PYG{p}{,} \PYG{n}{fill}\PYG{o}{=}\PYG{k+kc}{True}\PYG{p}{,} \PYG{n}{alpha}\PYG{o}{=}\PYG{l+m+mf}{0.5}\PYG{p}{)}
    \PYG{n}{sns}\PYG{o}{.}\PYG{n}{kdeplot}\PYG{p}{(}\PYG{n}{synthetic\PYGZus{}data}\PYG{p}{[}\PYG{n}{col}\PYG{p}{]}\PYG{p}{,} \PYG{n}{ax}\PYG{o}{=}\PYG{n}{ax}\PYG{p}{,} \PYG{n}{color}\PYG{o}{=}\PYG{l+s+s2}{\PYGZdq{}}\PYG{l+s+s2}{\PYGZsh{}2ca02c}\PYG{l+s+s2}{\PYGZdq{}}\PYG{p}{,} \PYG{n}{label}\PYG{o}{=}\PYG{l+s+s2}{\PYGZdq{}}\PYG{l+s+s2}{Gaussian Copula}\PYG{l+s+s2}{\PYGZdq{}}\PYG{p}{,} \PYG{n}{fill}\PYG{o}{=}\PYG{k+kc}{True}\PYG{p}{,} \PYG{n}{alpha}\PYG{o}{=}\PYG{l+m+mf}{0.5}\PYG{p}{)}
    \PYG{n}{sns}\PYG{o}{.}\PYG{n}{kdeplot}\PYG{p}{(}\PYG{n}{syntheticGAN\PYGZus{}data}\PYG{p}{[}\PYG{n}{col}\PYG{p}{]}\PYG{p}{,} \PYG{n}{ax}\PYG{o}{=}\PYG{n}{ax}\PYG{p}{,} \PYG{n}{color}\PYG{o}{=}\PYG{l+s+s2}{\PYGZdq{}}\PYG{l+s+s2}{\PYGZsh{}d62728}\PYG{l+s+s2}{\PYGZdq{}}\PYG{p}{,} \PYG{n}{label}\PYG{o}{=}\PYG{l+s+s2}{\PYGZdq{}}\PYG{l+s+s2}{CTGAN}\PYG{l+s+s2}{\PYGZdq{}}\PYG{p}{,} \PYG{n}{fill}\PYG{o}{=}\PYG{k+kc}{True}\PYG{p}{,} \PYG{n}{alpha}\PYG{o}{=}\PYG{l+m+mf}{0.5}\PYG{p}{)}
    \PYG{n}{sns}\PYG{o}{.}\PYG{n}{kdeplot}\PYG{p}{(}\PYG{n}{syntheticTVAE\PYGZus{}data}\PYG{p}{[}\PYG{n}{col}\PYG{p}{]}\PYG{p}{,} \PYG{n}{ax}\PYG{o}{=}\PYG{n}{ax}\PYG{p}{,} \PYG{n}{color}\PYG{o}{=}\PYG{l+s+s2}{\PYGZdq{}}\PYG{l+s+s2}{\PYGZsh{}9467bd}\PYG{l+s+s2}{\PYGZdq{}}\PYG{p}{,} \PYG{n}{label}\PYG{o}{=}\PYG{l+s+s2}{\PYGZdq{}}\PYG{l+s+s2}{TVAE}\PYG{l+s+s2}{\PYGZdq{}}\PYG{p}{,} \PYG{n}{fill}\PYG{o}{=}\PYG{k+kc}{True}\PYG{p}{,} \PYG{n}{alpha}\PYG{o}{=}\PYG{l+m+mf}{0.5}\PYG{p}{)}
    \PYG{n}{ax}\PYG{o}{.}\PYG{n}{set\PYGZus{}title}\PYG{p}{(}\PYG{l+s+sa}{f}\PYG{l+s+s1}{\PYGZsq{}}\PYG{l+s+s1}{Distribución de }\PYG{l+s+si}{\PYGZob{}}\PYG{n}{col}\PYG{l+s+si}{\PYGZcb{}}\PYG{l+s+s1}{\PYGZsq{}}\PYG{p}{)}
    \PYG{n}{ax}\PYG{o}{.}\PYG{n}{legend}\PYG{p}{(}\PYG{p}{)}
\PYG{n}{plt}\PYG{o}{.}\PYG{n}{suptitle}\PYG{p}{(}\PYG{l+s+s1}{\PYGZsq{}}\PYG{l+s+s1}{Comparación de datos reales vs. sintéticos}\PYG{l+s+s1}{\PYGZsq{}}\PYG{p}{,} \PYG{n}{y}\PYG{o}{=}\PYG{l+m+mf}{1.02}\PYG{p}{)}
\PYG{n}{plt}\PYG{o}{.}\PYG{n}{tight\PYGZus{}layout}\PYG{p}{(}\PYG{p}{)}
\PYG{n}{plt}\PYG{o}{.}\PYG{n}{show}\PYG{p}{(}\PYG{p}{)}
\end{sphinxVerbatim}

\end{sphinxuseclass}\end{sphinxVerbatimInput}
\begin{sphinxVerbatimOutput}

\begin{sphinxuseclass}{cell_output}
\noindent\sphinxincludegraphics{{31bfb5300642f4a944df6453e939e879305fbf3ba30a92813ef0f196681b4199}.png}

\end{sphinxuseclass}\end{sphinxVerbatimOutput}

\end{sphinxuseclass}
\sphinxAtStartPar
Los resultados obtenidos con el método Gaussian Copula son los que mejor se ajustan al dataset real. TVAE también mantiene una adecuada distribución respecto a las variables individuales, pero muestra unos picos de densidad más elevados lo cual indica que está sobreestimando parámetros. Los resultados más pobres se han obtenido con CTGAN. Uno de los principales motivos de estos resulatdos tan bajos radica en que las GANs tabulares aprenden correlaciones mediante adversarial training, que es inestable con muestras pequeñas. {[}\sphinxhref{https://doi.org/10.48550/arXiv.1907.00503}{Xu et al., 2019}{]} recomiendan una población mínima de \(N>1000\) muestras para que poder generalizar.

\sphinxstepscope


\part{PT3 \sphinxhyphen{} Predicción del Peso}

\sphinxstepscope


\chapter{Desarrollo Experimental de un Modelo de IA para la Estimación del Peso}
\label{\detokenize{content/03/Peso:desarrollo-experimental-de-un-modelo-de-ia-para-la-estimacion-del-peso}}\label{\detokenize{content/03/Peso::doc}}
\begin{sphinxadmonition}{note}{Resumen}

\sphinxAtStartPar
La predicción del peso del lenguado (\sphinxstyleemphasis{Solea solea}) a partir de su tamaño es un problema relevante en acuicultura, especialmente para la optimización de la alimentación y el control del crecimiento. Este estudio presenta un método para el desarrollo de un algoritmo de predicción del peso del lenguado a partir de variables morfológicas extraídas mediante visión artificial. Las variables consideradas incluyen la longitud, la anchura y la superficie real del pez, obtenidas automáticamente a partir de imágenes digitales en condiciones controladas. El algoritmo, basado en técnicas de regresión y aprendizaje automático, se ha diseñado para integrarse en un sistema autónomo de clasificación y estimación de peso en tiempo real, con aplicación directa al \sphinxstyleemphasis{gradding} automatizado de alevines en tres categorías de tamaño: pequeño, mediano y grande. Este enfoque proporciona una solución precisa y escalable para optimizar el manejo de peces en sistemas de acuicultura, contribuyendo a una mejora en la eficiencia operativa y el bienestar animal.

\sphinxAtStartPar
\sphinxstylestrong{Entregable}: E3.1\\
\sphinxstylestrong{Versión}: 1.0\\
\sphinxstylestrong{Autor}: Javier Álvarez Osuna\\
\sphinxstylestrong{Email}: javier.osuna@fishfarmfeeder.com\\
\sphinxstylestrong{ORCID}: \sphinxhref{https://orcid.org/0000-0001-7063-1279}{0000\sphinxhyphen{}0001\sphinxhyphen{}7063\sphinxhyphen{}1279}\\
\sphinxstylestrong{Licencia}: CC\sphinxhyphen{}BY\sphinxhyphen{}4.0\\
\sphinxstylestrong{Código proyecto}: IG408M.2025.000.000072

\begin{figure}[H]
\centering

\noindent\sphinxincludegraphics[width=1.000\linewidth]{{FLATCLASS_logo_publicidad}.png}
\end{figure}
\end{sphinxadmonition}


\section{Introducción}
\label{\detokenize{content/03/Peso:introduccion}}
\sphinxAtStartPar
En este capítulo se describen las líneas de trabajo desarrolladas, en el marco del proyecto FLATCLASS, para el desarrollo de un modelo predictivo basado en Inteligencia Artificial (IA) para estimar el peso de ejemplares de lenguado utilizando las variables morfométricas de lenguado obtenidas mediante visión artificial. El propósito del modelo es desarrollar una herramienta no invasiva, escalable y en tiempo real, que cumpla con los requisitos de robustez y precisión necesarios en entornos de producción acuícola. El planteamiento metodológico combina técnicas de aprendizaje supervisado con variables morfométricas clave como longitud, ancho y altura corporal, lo que permite capturar relaciones no lineales con la masa del pescado.

\sphinxAtStartPar
En estudios recientes como el de \sphinxhref{https://doi.org/10.3390/s22145161}{Tengtrairat et al., 2022}, se aplicó una combinación de visión estereoscópica y regresiones profundas para estimar peso de tilapia con un error medio de 30 g y un R² del 0,70. Del mismo modo, \sphinxhref{https://doi.org/10.3390/app13010069}{Lopez‑Tejeida et al., 2023} demostraron que el uso de cámaras infrarrojas NIR y regresión matemática permite reducir el estrés en el pez y mejorar la precisión del modelo de peso con una metodología no invasiva. Estos enfoques validan científicamente la relevancia y aplicabilidad de los métodos que presentamos.

\sphinxAtStartPar
Adicionalmente, trabajos como los de \sphinxhref{https://doi.org/10.48550/arXiv.1909.02710}{Konovalov et al., 2019} y \sphinxhref{https://doi.org/10.48550/arXiv.1909.02710}{Moseli et al., 2024} confirman que las redes neuronales convolucionales permiten segmentar y estimar pesos con errores inferiores al 5\% en poblaciones variadas. Nuestro diseño experimental plantea una estructura reproducible basada en: (1) adquisición de imágenes bajo condiciones controladas, (2) extracción de información morfométrica y etiquetado de muestras, (3) entrenamiento de modelos de IA con validación cruzada, y (4) evaluación de desempeño mediante métricas estadísticas y técnicas de análisis de error. Esto garantiza una base sólida tanto para la validación como para la futura implementación del modelo en escenarios reales.

\sphinxstepscope


\section{Estudio matemático de las relaciones entre las dimensiones morfológicas y el peso}
\label{\detokenize{content/03/Coeficientes:estudio-matematico-de-las-relaciones-entre-las-dimensiones-morfologicas-y-el-peso}}\label{\detokenize{content/03/Coeficientes::doc}}

\subsection{Justificación matemática y biológica}
\label{\detokenize{content/03/Coeficientes:justificacion-matematica-y-biologica}}
\sphinxAtStartPar
La relación entre la longitud y el peso en peces suele seguir la ley cúbica (\(W\propto L^{3}\)), pero en peces planos como el lenguado, la relación superficie\sphinxhyphen{}peso es más adecuada, ya que su cuerpo es achatado y su crecimiento no es isométrico. Los trabajos de {[}\sphinxhref{https://doi.org/10.1111/j.1439-0426.2006.00805.x}{Froese, 2006}{]} y \sphinxhref{https://doi.org/10.1111/j.1751-5823.2012.00179\_26.x}{{[}Haddon et al., 2011{]}} pusieron de manifiesto que en peces planos existe una correlación entre el peso del individuo (\(W\)) y la superficie (\(S\)) del mismo que viene dada por la expresión matemática:
\begin{equation}\label{equation:content/03/Coeficientes:eq_peso-superficie}
\begin{split}W=aS^b\end{split}
\end{equation}
\sphinxAtStartPar
en donde:
\begin{itemize}
\item {}
\sphinxAtStartPar
\(a\) es un coeficiente empírico que depende de la especie y las condiciones de cultivo.

\item {}
\sphinxAtStartPar
\(b\) es un exponente que describe la tasa de crecimiento en función de la superficie.

\end{itemize}

\sphinxAtStartPar
ya que la altura es menos predictiva que la longitud \sphinxhyphen{} anchura (L/A) que sólo contribuye \(\sim5\%\) en modelos morfológicos.

\sphinxAtStartPar
Para obtener valores específicos de ay b, se suelen ajustar datos experimentales usando regresión no lineal minimizando el error cuadrático medio:
\begin{equation}\label{equation:content/03/Coeficientes:eq_ECM}
\begin{split}ECM=\dfrac{1}{N}\sum_{i=1}^N(W_i-aS_i^b)^2\end{split}
\end{equation}
\sphinxAtStartPar
\sphinxhref{https://doi.org/10.1111/j.1095-8649.2011.03223.x}{{[}Torres et al., 2012{]}} recoge que en experimentos realizados con peces planos de las familias Soleidae y Pleuronectidae los coeficientes de la ecuación (1) toman los siguientes valores:
\begin{equation*}
\begin{split}a\approx0,03\quad b\approx1,1-1,4\end{split}
\end{equation*}
\sphinxAtStartPar
habiendo estimado que en el caso concreto del lenguado estos coeficientes son: \(a=0.024\quad b=1.32\)

\sphinxAtStartPar
Los recientes estudios de \sphinxhref{https://doi.org/10.21608/ejabf.2023.322449}{{[}El\sphinxhyphen{}Bokhty et al., 2023{]}} ponen en entredicho la ecuación (1) y sugieren que el crecimiento alométrico positivo del lenguado común (\sphinxstyleemphasis{Solea solea}) responde a la variable morfológica de longitud (\(L\)) que se ajusta a la expresión matemática:
\begin{equation}\label{equation:content/03/Coeficientes:eq_peso-longitud}
\begin{split}W=a\cdot L^b\end{split}
\end{equation}
\sphinxAtStartPar
Concluyendo, en sus trabajos, que los coeficientes de dicha expresión son:
\begin{equation*}
\begin{split}a\approx0,00029\quad b\approx3,3777\end{split}
\end{equation*}
\sphinxAtStartPar
Sin embargo, una investigación realizada en las aguas costeras de Pakistán analizó seis especies de peces planos y encontró variaciones significativas en parámetros como la longitud, la anchura y el peso entre las diferentes especies \sphinxhref{https://doi.org/10.19080/OFOAJ.2021.14.555884}{{[}W. Ali et al., 2021{]}}.  Aunque este estudio no presentó una fórmula específica que relacione el peso con la longitud y la anchura simultáneamente, sugiere que la anchura es una medida morfológica relevante en la evaluación de las relaciones morfométricas en peces planos. Ambas variables podrían relacionarse mediante la siguiente expresión matemática:
\begin{equation}\label{equation:content/03/Coeficientes:eq_peso-longitud_anchura}
\begin{split}W=k\cdot L^a \cdot A^b\end{split}
\end{equation}
\sphinxAtStartPar
en donde:
\begin{itemize}
\item {}
\sphinxAtStartPar
\(W\) = peso,

\item {}
\sphinxAtStartPar
\(L\) = longitud,

\item {}
\sphinxAtStartPar
\(A\) = anchura,

\item {}
\sphinxAtStartPar
\(k\), \(a\), \(b\) = parámetros específicos por especie

\end{itemize}

\sphinxAtStartPar
Por otro lado \sphinxhref{https://doi.org/0.1016/j.aqrep.2021.100676}{{[}Freitas et al.,2021{]}} en sus trabajos con \sphinxstyleemphasis{Paralichthys olivaceus} (falso Halibut del japón) apuntán a que una relación volumétrica es más adecuada si se tiene en cuenta la comprensión dorsoventral. Esta parámetro morfológico actúa como un factor de corrección que hace que el volumen real sea \(\approx70-80\%\) de LxAxH, siendo concretamente de \(0.72\) en el Halibut. Según estos autores el peso de los peces de esta especie se ajustan a la siguiente fórmula:
\begin{equation*}
\begin{split}W=1,05\cdot V^{0,98}\end{split}
\end{equation*}
\sphinxAtStartPar
que responde a la expresión matemática genérica:
\begin{equation}\label{equation:content/03/Coeficientes:eq_peso-volumen}
\begin{split}W=a\cdot (c\cdot V)^b\end{split}
\end{equation}
\sphinxAtStartPar
en donde:
\begin{itemize}
\item {}
\sphinxAtStartPar
\(W\) = peso,

\item {}
\sphinxAtStartPar
\(V\) = volumen obtenido como \(L\cdot A \cdot H\)

\item {}
\sphinxAtStartPar
\(L\) = longitud,

\item {}
\sphinxAtStartPar
\(A\) = anchura,

\item {}
\sphinxAtStartPar
\(H\) = altura,

\item {}
\sphinxAtStartPar
\(c\) = factor de corrección \(\leq 1\)

\item {}
\sphinxAtStartPar
\(a\), \(b\) = parámetros específicos por especie

\end{itemize}

\sphinxAtStartPar
\sphinxhref{https://doi.org/0.1007/s12562-019-01287-2}{{[}Lee et al., 2019{]}} usaron un escáner 3D para medir el volumen real en \sphinxstyleemphasis{Scophthalmus maximus} (Rodaballo) encontrando que el peso de los ejemplares del estudio se ajustan a la expresión:
\begin{equation*}
\begin{split}W=1,12\cdot V-3,25\quad(R^{2}=0,96)\end{split}
\end{equation*}

\subsection{Análisis de modelos peso \sphinxhyphen{} morfología}
\label{\detokenize{content/03/Coeficientes:analisis-de-modelos-peso-morfologia}}
\sphinxAtStartPar
Se dispone de un dataset \sphinxcode{\sphinxupquote{Dimensiones\_lenguado.xls}} con N=200 registros morfométricos de alevines de lenguado, que incluye medidas de longitud, anchura, altura y peso. Considerando la diversidad de enfoques existentes en la literatura científica, se plantea la evaluación comparativa de distintas expresiones matemáticas que describan la relación entre el peso y las variables morfológicas. En particular, se estudiarán modelos clásicos basados en la relación \(W = a \cdot L^b\), \(W = a \cdot S^b\) así como formulaciones extendidas que incorporan múltiples dimensiones corporales, como \(W = k \cdot L^a \cdot A^b \cdot H^c\), con el objetivo de determinar cuál de estas expresiones proporciona un ajuste más preciso a los datos observados. La selección del modelo óptimo se basará en criterios estadísticos de bondad de ajuste y parsimonia, tales como el coeficiente de determinación ajustado (R² ajustado), el error cuadrático medio (RMSE).
Los diferentes modelos de ajuste evaluados fueron ejecutados en un entorno Jupyterlab haciendo uso de librerías python especializadas como: pandas, numpy y scikit\sphinxhyphen{}learn.

\begin{sphinxuseclass}{cell}\begin{sphinxVerbatimInput}

\begin{sphinxuseclass}{cell_input}
\begin{sphinxVerbatim}[commandchars=\\\{\}]
\PYG{c+c1}{\PYGZsh{} Cargar librerias necesarias}
\PYG{k+kn}{import}\PYG{+w}{ }\PYG{n+nn}{pandas}\PYG{+w}{ }\PYG{k}{as}\PYG{+w}{ }\PYG{n+nn}{pd}
\PYG{k+kn}{import}\PYG{+w}{ }\PYG{n+nn}{numpy}\PYG{+w}{ }\PYG{k}{as}\PYG{+w}{ }\PYG{n+nn}{np}
\PYG{k+kn}{import}\PYG{+w}{ }\PYG{n+nn}{matplotlib}\PYG{n+nn}{.}\PYG{n+nn}{pyplot}\PYG{+w}{ }\PYG{k}{as}\PYG{+w}{ }\PYG{n+nn}{plt}
\PYG{k+kn}{from}\PYG{+w}{ }\PYG{n+nn}{sklearn}\PYG{n+nn}{.}\PYG{n+nn}{linear\PYGZus{}model}\PYG{+w}{ }\PYG{k+kn}{import} \PYG{n}{LinearRegression}
\PYG{k+kn}{from}\PYG{+w}{ }\PYG{n+nn}{sklearn}\PYG{n+nn}{.}\PYG{n+nn}{metrics}\PYG{+w}{ }\PYG{k+kn}{import} \PYG{n}{r2\PYGZus{}score}\PYG{p}{,} \PYG{n}{mean\PYGZus{}squared\PYGZus{}error}\PYG{p}{,} \PYG{n}{mean\PYGZus{}absolute\PYGZus{}error}
\PYG{k+kn}{from}\PYG{+w}{ }\PYG{n+nn}{scipy}\PYG{n+nn}{.}\PYG{n+nn}{optimize}\PYG{+w}{ }\PYG{k+kn}{import} \PYG{n}{curve\PYGZus{}fit}
\PYG{k+kn}{from}\PYG{+w}{ }\PYG{n+nn}{matplotlib}\PYG{n+nn}{.}\PYG{n+nn}{colors}\PYG{+w}{ }\PYG{k+kn}{import} \PYG{n}{CSS4\PYGZus{}COLORS}

\PYG{n}{plt}\PYG{o}{.}\PYG{n}{style}\PYG{o}{.}\PYG{n}{use}\PYG{p}{(}\PYG{l+s+s1}{\PYGZsq{}}\PYG{l+s+s1}{seaborn\PYGZhy{}v0\PYGZus{}8\PYGZhy{}muted}\PYG{l+s+s1}{\PYGZsq{}}\PYG{p}{)}

\PYG{c+c1}{\PYGZsh{} Cargar el dataset}
\PYG{n}{file\PYGZus{}path} \PYG{o}{=} \PYG{l+s+s1}{\PYGZsq{}}\PYG{l+s+s1}{.././data/Dimensiones\PYGZus{}lenguado.xlsx}\PYG{l+s+s1}{\PYGZsq{}}
\PYG{n}{df} \PYG{o}{=} \PYG{n}{pd}\PYG{o}{.}\PYG{n}{read\PYGZus{}excel}\PYG{p}{(}\PYG{n}{file\PYGZus{}path}\PYG{p}{)}
\PYG{n}{df}\PYG{o}{.}\PYG{n}{head}\PYG{p}{(}\PYG{p}{)}
\end{sphinxVerbatim}

\end{sphinxuseclass}\end{sphinxVerbatimInput}
\begin{sphinxVerbatimOutput}

\begin{sphinxuseclass}{cell_output}
\begin{sphinxVerbatim}[commandchars=\\\{\}]
   Peso (g)  Longitud (cm)  Anchura (cm)  Altura (cm)
0      0.46            3.3           1.3          0.2
1      1.08            4.5           1.1          0.3
2      0.67            3.9           1.5          0.2
3      0.98            4.4           1.7          0.3
4      0.93            4.2           1.8          0.3
\end{sphinxVerbatim}

\end{sphinxuseclass}\end{sphinxVerbatimOutput}

\end{sphinxuseclass}
\begin{sphinxuseclass}{cell}\begin{sphinxVerbatimInput}

\begin{sphinxuseclass}{cell_input}
\begin{sphinxVerbatim}[commandchars=\\\{\}]
\PYG{c+c1}{\PYGZsh{} Definimos variables comunes}
\PYG{n}{L} \PYG{o}{=} \PYG{n}{df}\PYG{p}{[}\PYG{l+s+s1}{\PYGZsq{}}\PYG{l+s+s1}{Longitud (cm)}\PYG{l+s+s1}{\PYGZsq{}}\PYG{p}{]}\PYG{o}{.}\PYG{n}{values}
\PYG{n}{A} \PYG{o}{=} \PYG{n}{df}\PYG{p}{[}\PYG{l+s+s1}{\PYGZsq{}}\PYG{l+s+s1}{Anchura (cm)}\PYG{l+s+s1}{\PYGZsq{}}\PYG{p}{]}\PYG{o}{.}\PYG{n}{values}
\PYG{n}{H} \PYG{o}{=} \PYG{n}{df}\PYG{p}{[}\PYG{l+s+s1}{\PYGZsq{}}\PYG{l+s+s1}{Altura (cm)}\PYG{l+s+s1}{\PYGZsq{}}\PYG{p}{]}\PYG{o}{.}\PYG{n}{values}
\PYG{n}{W} \PYG{o}{=} \PYG{n}{df}\PYG{p}{[}\PYG{l+s+s1}{\PYGZsq{}}\PYG{l+s+s1}{Peso (g)}\PYG{l+s+s1}{\PYGZsq{}}\PYG{p}{]}\PYG{o}{.}\PYG{n}{values}
\PYG{n}{V} \PYG{o}{=} \PYG{n}{L} \PYG{o}{*} \PYG{n}{A} \PYG{o}{*} \PYG{n}{H}
\PYG{n}{n} \PYG{o}{=} \PYG{n+nb}{len}\PYG{p}{(}\PYG{n}{W}\PYG{p}{)}
\end{sphinxVerbatim}

\end{sphinxuseclass}\end{sphinxVerbatimInput}

\end{sphinxuseclass}

\subsubsection{Modelo potencia peso \sphinxhyphen{} longitud: \protect\(W=a\cdot L^b\protect\)}
\label{\detokenize{content/03/Coeficientes:modelo-potencia-peso-longitud-w-a-cdot-l-b}}
\begin{sphinxuseclass}{cell}\begin{sphinxVerbatimInput}

\begin{sphinxuseclass}{cell_input}
\begin{sphinxVerbatim}[commandchars=\\\{\}]
\PYG{k}{def}\PYG{+w}{ }\PYG{n+nf}{long\PYGZus{}model}\PYG{p}{(}\PYG{n}{L}\PYG{p}{,}\PYG{n}{a}\PYG{p}{,}\PYG{n}{b}\PYG{p}{)}\PYG{p}{:}
    \PYG{k}{return} \PYG{n}{a}\PYG{o}{*}\PYG{n}{L}\PYG{o}{*}\PYG{o}{*}\PYG{n}{b}

\PYG{n}{params\PYGZus{}long}\PYG{p}{,} \PYG{n}{\PYGZus{}} \PYG{o}{=} \PYG{n}{curve\PYGZus{}fit}\PYG{p}{(}\PYG{n}{long\PYGZus{}model}\PYG{p}{,} \PYG{n}{L}\PYG{p}{,} \PYG{n}{W}\PYG{p}{)}
\PYG{n}{a}\PYG{p}{,}\PYG{n}{b} \PYG{o}{=} \PYG{n}{params\PYGZus{}long}
\PYG{n}{W\PYGZus{}pred\PYGZus{}long} \PYG{o}{=} \PYG{n}{long\PYGZus{}model}\PYG{p}{(}\PYG{n}{L}\PYG{p}{,} \PYG{n}{a}\PYG{p}{,}\PYG{n}{b}\PYG{p}{)}
\PYG{n}{r2\PYGZus{}long} \PYG{o}{=} \PYG{n}{r2\PYGZus{}score}\PYG{p}{(}\PYG{n}{W}\PYG{p}{,} \PYG{n}{W\PYGZus{}pred\PYGZus{}long}\PYG{p}{)}
\PYG{n}{r2\PYGZus{}adj\PYGZus{}long}\PYG{o}{=} \PYG{l+m+mi}{1} \PYG{o}{\PYGZhy{}} \PYG{p}{(}\PYG{l+m+mi}{1} \PYG{o}{\PYGZhy{}} \PYG{n}{r2\PYGZus{}long}\PYG{p}{)} \PYG{o}{*} \PYG{p}{(}\PYG{n}{n} \PYG{o}{\PYGZhy{}} \PYG{l+m+mi}{1}\PYG{p}{)} \PYG{o}{/} \PYG{p}{(}\PYG{n}{n} \PYG{o}{\PYGZhy{}} \PYG{l+m+mi}{1} \PYG{o}{\PYGZhy{}} \PYG{l+m+mi}{1}\PYG{p}{)}
\PYG{n}{rmse\PYGZus{}long} \PYG{o}{=} \PYG{n}{np}\PYG{o}{.}\PYG{n}{sqrt}\PYG{p}{(}\PYG{n}{mean\PYGZus{}squared\PYGZus{}error}\PYG{p}{(}\PYG{n}{W}\PYG{p}{,} \PYG{n}{W\PYGZus{}pred\PYGZus{}long}\PYG{p}{)}\PYG{p}{)}
\PYG{n}{mae\PYGZus{}long} \PYG{o}{=} \PYG{n}{mean\PYGZus{}absolute\PYGZus{}error}\PYG{p}{(}\PYG{n}{W}\PYG{p}{,} \PYG{n}{W\PYGZus{}pred\PYGZus{}long}\PYG{p}{)}

\PYG{c+c1}{\PYGZsh{} Visualización ajuste}

\PYG{c+c1}{\PYGZsh{} Índices y residuos}
\PYG{n}{indices} \PYG{o}{=} \PYG{n}{np}\PYG{o}{.}\PYG{n}{arange}\PYG{p}{(}\PYG{n+nb}{len}\PYG{p}{(}\PYG{n}{W}\PYG{p}{)}\PYG{p}{)}
\PYG{n}{residuos} \PYG{o}{=} \PYG{n}{W} \PYG{o}{\PYGZhy{}} \PYG{n}{W\PYGZus{}pred\PYGZus{}long}

\PYG{c+c1}{\PYGZsh{} Crear la figura con 3 subplots}
\PYG{n}{fig}\PYG{p}{,} \PYG{n}{axs} \PYG{o}{=} \PYG{n}{plt}\PYG{o}{.}\PYG{n}{subplots}\PYG{p}{(}\PYG{l+m+mi}{1}\PYG{p}{,} \PYG{l+m+mi}{3}\PYG{p}{,} \PYG{n}{figsize}\PYG{o}{=}\PYG{p}{(}\PYG{l+m+mi}{18}\PYG{p}{,} \PYG{l+m+mi}{5}\PYG{p}{)}\PYG{p}{)}

\PYG{c+c1}{\PYGZsh{} === 1. Peso real vs. Peso predicho ===}
\PYG{n}{axs}\PYG{p}{[}\PYG{l+m+mi}{0}\PYG{p}{]}\PYG{o}{.}\PYG{n}{scatter}\PYG{p}{(}\PYG{n}{W}\PYG{p}{,} \PYG{n}{W\PYGZus{}pred\PYGZus{}long}\PYG{p}{,} \PYG{n}{alpha}\PYG{o}{=}\PYG{l+m+mf}{0.7}\PYG{p}{)}
\PYG{n}{axs}\PYG{p}{[}\PYG{l+m+mi}{0}\PYG{p}{]}\PYG{o}{.}\PYG{n}{plot}\PYG{p}{(}\PYG{p}{[}\PYG{n+nb}{min}\PYG{p}{(}\PYG{n}{W}\PYG{p}{)}\PYG{p}{,} \PYG{n+nb}{max}\PYG{p}{(}\PYG{n}{W}\PYG{p}{)}\PYG{p}{]}\PYG{p}{,} \PYG{p}{[}\PYG{n+nb}{min}\PYG{p}{(}\PYG{n}{W}\PYG{p}{)}\PYG{p}{,} \PYG{n+nb}{max}\PYG{p}{(}\PYG{n}{W}\PYG{p}{)}\PYG{p}{]}\PYG{p}{,} \PYG{l+s+s1}{\PYGZsq{}}\PYG{l+s+s1}{r\PYGZhy{}\PYGZhy{}}\PYG{l+s+s1}{\PYGZsq{}}\PYG{p}{,} \PYG{n}{label}\PYG{o}{=}\PYG{l+s+s1}{\PYGZsq{}}\PYG{l+s+s1}{Línea ideal}\PYG{l+s+s1}{\PYGZsq{}}\PYG{p}{)}
\PYG{n}{axs}\PYG{p}{[}\PYG{l+m+mi}{0}\PYG{p}{]}\PYG{o}{.}\PYG{n}{set\PYGZus{}xlabel}\PYG{p}{(}\PYG{l+s+s1}{\PYGZsq{}}\PYG{l+s+s1}{Peso real}\PYG{l+s+s1}{\PYGZsq{}}\PYG{p}{)}
\PYG{n}{axs}\PYG{p}{[}\PYG{l+m+mi}{0}\PYG{p}{]}\PYG{o}{.}\PYG{n}{set\PYGZus{}ylabel}\PYG{p}{(}\PYG{l+s+s1}{\PYGZsq{}}\PYG{l+s+s1}{Peso predicho}\PYG{l+s+s1}{\PYGZsq{}}\PYG{p}{)}
\PYG{n}{axs}\PYG{p}{[}\PYG{l+m+mi}{0}\PYG{p}{]}\PYG{o}{.}\PYG{n}{set\PYGZus{}title}\PYG{p}{(}\PYG{l+s+sa}{f}\PYG{l+s+s1}{\PYGZsq{}}\PYG{l+s+s1}{Real vs. Predicho}\PYG{l+s+s1}{\PYGZsq{}}\PYG{p}{)}
\PYG{n}{axs}\PYG{p}{[}\PYG{l+m+mi}{0}\PYG{p}{]}\PYG{o}{.}\PYG{n}{legend}\PYG{p}{(}\PYG{p}{)}
\PYG{n}{axs}\PYG{p}{[}\PYG{l+m+mi}{0}\PYG{p}{]}\PYG{o}{.}\PYG{n}{grid}\PYG{p}{(}\PYG{k+kc}{True}\PYG{p}{)}

\PYG{c+c1}{\PYGZsh{} === 2. Comparación por índice ===}
\PYG{n}{axs}\PYG{p}{[}\PYG{l+m+mi}{1}\PYG{p}{]}\PYG{o}{.}\PYG{n}{scatter}\PYG{p}{(}\PYG{n}{indices}\PYG{p}{,} \PYG{n}{W}\PYG{p}{,} \PYG{n}{color}\PYG{o}{=}\PYG{n}{CSS4\PYGZus{}COLORS}\PYG{p}{[}\PYG{l+s+s1}{\PYGZsq{}}\PYG{l+s+s1}{lightskyblue}\PYG{l+s+s1}{\PYGZsq{}}\PYG{p}{]}\PYG{p}{,} \PYG{n}{label}\PYG{o}{=}\PYG{l+s+s1}{\PYGZsq{}}\PYG{l+s+s1}{Peso real}\PYG{l+s+s1}{\PYGZsq{}}\PYG{p}{,} \PYG{n}{alpha}\PYG{o}{=}\PYG{l+m+mf}{0.8}\PYG{p}{)}
\PYG{n}{axs}\PYG{p}{[}\PYG{l+m+mi}{1}\PYG{p}{]}\PYG{o}{.}\PYG{n}{scatter}\PYG{p}{(}\PYG{n}{indices}\PYG{p}{,} \PYG{n}{W\PYGZus{}pred\PYGZus{}long}\PYG{p}{,} \PYG{n}{color}\PYG{o}{=}\PYG{n}{CSS4\PYGZus{}COLORS}\PYG{p}{[}\PYG{l+s+s1}{\PYGZsq{}}\PYG{l+s+s1}{darkorange}\PYG{l+s+s1}{\PYGZsq{}}\PYG{p}{]}\PYG{p}{,} \PYG{n}{label}\PYG{o}{=}\PYG{l+s+s1}{\PYGZsq{}}\PYG{l+s+s1}{Peso predicho}\PYG{l+s+s1}{\PYGZsq{}}\PYG{p}{,} \PYG{n}{alpha}\PYG{o}{=}\PYG{l+m+mf}{0.6}\PYG{p}{)}
\PYG{n}{axs}\PYG{p}{[}\PYG{l+m+mi}{1}\PYG{p}{]}\PYG{o}{.}\PYG{n}{set\PYGZus{}xlabel}\PYG{p}{(}\PYG{l+s+s1}{\PYGZsq{}}\PYG{l+s+s1}{Índice de muestra}\PYG{l+s+s1}{\PYGZsq{}}\PYG{p}{)}
\PYG{n}{axs}\PYG{p}{[}\PYG{l+m+mi}{1}\PYG{p}{]}\PYG{o}{.}\PYG{n}{set\PYGZus{}ylabel}\PYG{p}{(}\PYG{l+s+s1}{\PYGZsq{}}\PYG{l+s+s1}{Peso}\PYG{l+s+s1}{\PYGZsq{}}\PYG{p}{)}
\PYG{n}{axs}\PYG{p}{[}\PYG{l+m+mi}{1}\PYG{p}{]}\PYG{o}{.}\PYG{n}{set\PYGZus{}title}\PYG{p}{(}\PYG{l+s+s1}{\PYGZsq{}}\PYG{l+s+s1}{Peso real vs. predicho por muestra}\PYG{l+s+s1}{\PYGZsq{}}\PYG{p}{)}
\PYG{n}{axs}\PYG{p}{[}\PYG{l+m+mi}{1}\PYG{p}{]}\PYG{o}{.}\PYG{n}{legend}\PYG{p}{(}\PYG{p}{)}
\PYG{n}{axs}\PYG{p}{[}\PYG{l+m+mi}{1}\PYG{p}{]}\PYG{o}{.}\PYG{n}{grid}\PYG{p}{(}\PYG{k+kc}{True}\PYG{p}{)}

\PYG{c+c1}{\PYGZsh{} === 3. Gráfico de residuos ===}
\PYG{n}{axs}\PYG{p}{[}\PYG{l+m+mi}{2}\PYG{p}{]}\PYG{o}{.}\PYG{n}{scatter}\PYG{p}{(}\PYG{n}{W\PYGZus{}pred\PYGZus{}long}\PYG{p}{,} \PYG{n}{residuos}\PYG{p}{,} \PYG{n}{alpha}\PYG{o}{=}\PYG{l+m+mf}{0.6}\PYG{p}{,} \PYG{n}{color}\PYG{o}{=}\PYG{n}{CSS4\PYGZus{}COLORS}\PYG{p}{[}\PYG{l+s+s1}{\PYGZsq{}}\PYG{l+s+s1}{violet}\PYG{l+s+s1}{\PYGZsq{}}\PYG{p}{]}\PYG{p}{)}
\PYG{n}{axs}\PYG{p}{[}\PYG{l+m+mi}{2}\PYG{p}{]}\PYG{o}{.}\PYG{n}{axhline}\PYG{p}{(}\PYG{l+m+mi}{0}\PYG{p}{,} \PYG{n}{color}\PYG{o}{=}\PYG{l+s+s1}{\PYGZsq{}}\PYG{l+s+s1}{red}\PYG{l+s+s1}{\PYGZsq{}}\PYG{p}{,} \PYG{n}{linestyle}\PYG{o}{=}\PYG{l+s+s1}{\PYGZsq{}}\PYG{l+s+s1}{\PYGZhy{}\PYGZhy{}}\PYG{l+s+s1}{\PYGZsq{}}\PYG{p}{)}
\PYG{n}{axs}\PYG{p}{[}\PYG{l+m+mi}{2}\PYG{p}{]}\PYG{o}{.}\PYG{n}{set\PYGZus{}xlabel}\PYG{p}{(}\PYG{l+s+s1}{\PYGZsq{}}\PYG{l+s+s1}{Peso predicho}\PYG{l+s+s1}{\PYGZsq{}}\PYG{p}{)}
\PYG{n}{axs}\PYG{p}{[}\PYG{l+m+mi}{2}\PYG{p}{]}\PYG{o}{.}\PYG{n}{set\PYGZus{}ylabel}\PYG{p}{(}\PYG{l+s+s1}{\PYGZsq{}}\PYG{l+s+s1}{Residuo}\PYG{l+s+s1}{\PYGZsq{}}\PYG{p}{)}
\PYG{n}{axs}\PYG{p}{[}\PYG{l+m+mi}{2}\PYG{p}{]}\PYG{o}{.}\PYG{n}{set\PYGZus{}title}\PYG{p}{(}\PYG{l+s+s1}{\PYGZsq{}}\PYG{l+s+s1}{Análisis de residuos}\PYG{l+s+s1}{\PYGZsq{}}\PYG{p}{)}
\PYG{n}{axs}\PYG{p}{[}\PYG{l+m+mi}{2}\PYG{p}{]}\PYG{o}{.}\PYG{n}{grid}\PYG{p}{(}\PYG{k+kc}{True}\PYG{p}{)}

\PYG{c+c1}{\PYGZsh{} Ajustar diseño y mostrar}
\PYG{n}{plt}\PYG{o}{.}\PYG{n}{tight\PYGZus{}layout}\PYG{p}{(}\PYG{p}{)}
\PYG{n}{plt}\PYG{o}{.}\PYG{n}{show}\PYG{p}{(}\PYG{p}{)}

\PYG{c+c1}{\PYGZsh{} Resultados del ajuste}

\PYG{n+nb}{print}\PYG{p}{(}\PYG{l+s+sa}{f}\PYG{l+s+s2}{\PYGZdq{}}\PYG{l+s+s2}{Modelo ajustado: W = }\PYG{l+s+si}{\PYGZob{}}\PYG{n}{a}\PYG{l+s+si}{:}\PYG{l+s+s2}{.5f}\PYG{l+s+si}{\PYGZcb{}}\PYG{l+s+s2}{ · L\PYGZca{}}\PYG{l+s+si}{\PYGZob{}}\PYG{n}{b}\PYG{l+s+si}{:}\PYG{l+s+s2}{.5f}\PYG{l+s+si}{\PYGZcb{}}\PYG{l+s+s2}{\PYGZdq{}}\PYG{p}{)}
\PYG{n+nb}{print}\PYG{p}{(}\PYG{l+s+sa}{f}\PYG{l+s+s2}{\PYGZdq{}}\PYG{l+s+s2}{R² = }\PYG{l+s+si}{\PYGZob{}}\PYG{n}{r2\PYGZus{}long}\PYG{l+s+si}{:}\PYG{l+s+s2}{.4f}\PYG{l+s+si}{\PYGZcb{}}\PYG{l+s+s2}{\PYGZdq{}}\PYG{p}{)}
\PYG{n+nb}{print}\PYG{p}{(}\PYG{l+s+sa}{f}\PYG{l+s+s2}{\PYGZdq{}}\PYG{l+s+s2}{MAE = }\PYG{l+s+si}{\PYGZob{}}\PYG{n}{mae\PYGZus{}long}\PYG{l+s+si}{:}\PYG{l+s+s2}{.4f}\PYG{l+s+si}{\PYGZcb{}}\PYG{l+s+s2}{ g}\PYG{l+s+s2}{\PYGZdq{}}\PYG{p}{)}
\PYG{n+nb}{print}\PYG{p}{(}\PYG{l+s+sa}{f}\PYG{l+s+s2}{\PYGZdq{}}\PYG{l+s+s2}{RSME = }\PYG{l+s+si}{\PYGZob{}}\PYG{n}{rmse\PYGZus{}long}\PYG{l+s+si}{:}\PYG{l+s+s2}{.4f}\PYG{l+s+si}{\PYGZcb{}}\PYG{l+s+s2}{ g}\PYG{l+s+s2}{\PYGZdq{}}\PYG{p}{)}
\PYG{n+nb}{print}\PYG{p}{(}\PYG{l+s+sa}{f}\PYG{l+s+s2}{\PYGZdq{}}\PYG{l+s+s2}{R²adj = }\PYG{l+s+si}{\PYGZob{}}\PYG{n}{r2\PYGZus{}adj\PYGZus{}long}\PYG{l+s+si}{:}\PYG{l+s+s2}{.4f}\PYG{l+s+si}{\PYGZcb{}}\PYG{l+s+s2}{ g}\PYG{l+s+s2}{\PYGZdq{}}\PYG{p}{)}
\end{sphinxVerbatim}

\end{sphinxuseclass}\end{sphinxVerbatimInput}
\begin{sphinxVerbatimOutput}

\begin{sphinxuseclass}{cell_output}
\noindent\sphinxincludegraphics{{44e924eb1c1d8ed01c23a6d55f030a1e9a19681adb870886207af25f90daff45}.png}

\begin{sphinxVerbatim}[commandchars=\\\{\}]
Modelo ajustado: W = 0.00933 · L\PYGZca{}3.13480
R² = 0.9433
MAE = 0.5246 g
RSME = 0.8638 g
R²adj = 0.9430 g
\end{sphinxVerbatim}

\end{sphinxuseclass}\end{sphinxVerbatimOutput}

\end{sphinxuseclass}

\subsubsection{Modelo potencia peso \sphinxhyphen{} superficie: \protect\(W=a\cdot S^b\protect\)}
\label{\detokenize{content/03/Coeficientes:modelo-potencia-peso-superficie-w-a-cdot-s-b}}
\sphinxAtStartPar
Para un ajuste óptimo de este modelo hay que tener en cuenta que la morfología aplanada y asimétrica de los lenguados.

\sphinxAtStartPar
Dado que el lenguado es más ancho que alto y tiene una doerma similar a un disco aplanado, podemos usar la fórmula del área de una elipse como aproximación:
\begin{equation*}
\begin{split}S=\dfrac{\pi \cdot L \cdot A}{4}\end{split}
\end{equation*}
\sphinxAtStartPar
Si tenemos en cuenta que el lenguado no es una elipse perfecta y que la superficie dorsal suele ser más curva de la ventral tendremos que aplicar un factor de corrección cuyo valor varía según la especie {[}\sphinxhref{https://www.researchgate.net/publication/255641396\_A\_Note\_on\_The\_Examination\_of\_Morphometric\_Differentiation\_Among\_Fish\_Populations\_The\_Truss\_System}{Turan C., 1999}; \sphinxhref{https://doi.org/10.5370/JEET.2013.8.5.1194}{Jeong et al., 2013}{]} y que en el caso del lenguado puede estimarse \(\approx 75-90 \%\) del área de la elipse definida por su longitud y anchura.

\sphinxAtStartPar
Este enfoque matemático sólo es aplicable si, como es este caso, el dataset no contiene un valor de superficie real obtenido mediante funciones de procesado de imagen que permita la segmentación precisa del contorno corporal del lenguado y el cálculo de su superficie proyectada mediante herramientas como \sphinxcode{\sphinxupquote{cv2.contourArea()}}de OpenCV.

\begin{sphinxuseclass}{cell}\begin{sphinxVerbatimInput}

\begin{sphinxuseclass}{cell_input}
\begin{sphinxVerbatim}[commandchars=\\\{\}]
\PYG{c+c1}{\PYGZsh{} Si el dataset contiene datos de superficie real:}
\PYG{c+c1}{\PYGZsh{} S = df[\PYGZsq{}Superficie (cm2)\PYGZsq{}].values}

\PYG{c+c1}{\PYGZsh{} Si el dataset no contiene datos de superficie real:}
\PYG{n}{S} \PYG{o}{=} \PYG{n}{L} \PYG{o}{*} \PYG{n}{A}
\PYG{n}{coef} \PYG{o}{=} \PYG{l+m+mf}{0.75}

\PYG{k}{def}\PYG{+w}{ }\PYG{n+nf}{potencia\PYGZus{}model}\PYG{p}{(}\PYG{n}{S}\PYG{p}{,} \PYG{n}{a}\PYG{p}{,} \PYG{n}{b}\PYG{p}{)}\PYG{p}{:}
    \PYG{c+c1}{\PYGZsh{} Si hay datos de superficie real obtenidos por procesado de imagen entonces:}
    \PYG{c+c1}{\PYGZsh{} superficie\PYGZus{}real = S}
    \PYG{n}{superficie\PYGZus{}real} \PYG{o}{=} \PYG{n}{coef} \PYG{o}{*} \PYG{p}{(}\PYG{p}{(}\PYG{n}{np}\PYG{o}{.}\PYG{n}{pi} \PYG{o}{/} \PYG{l+m+mi}{4}\PYG{p}{)} \PYG{o}{*} \PYG{n}{S}\PYG{p}{)}
    \PYG{k}{return} \PYG{n}{a} \PYG{o}{*} \PYG{p}{(}\PYG{n}{superficie\PYGZus{}real}\PYG{p}{)}\PYG{o}{*}\PYG{o}{*}\PYG{n}{b}

\PYG{n}{params\PYGZus{}pow}\PYG{p}{,} \PYG{n}{\PYGZus{}} \PYG{o}{=} \PYG{n}{curve\PYGZus{}fit}\PYG{p}{(}\PYG{n}{potencia\PYGZus{}model}\PYG{p}{,} \PYG{n}{S}\PYG{p}{,} \PYG{n}{W}\PYG{p}{)}
\PYG{n}{a}\PYG{p}{,}\PYG{n}{b} \PYG{o}{=} \PYG{n}{params\PYGZus{}pow}
\PYG{n}{W\PYGZus{}pred\PYGZus{}pow} \PYG{o}{=} \PYG{n}{potencia\PYGZus{}model}\PYG{p}{(}\PYG{n}{S}\PYG{p}{,} \PYG{n}{a}\PYG{p}{,} \PYG{n}{b}\PYG{p}{)}
\PYG{n}{r2\PYGZus{}pow} \PYG{o}{=} \PYG{n}{r2\PYGZus{}score}\PYG{p}{(}\PYG{n}{W}\PYG{p}{,} \PYG{n}{W\PYGZus{}pred\PYGZus{}pow}\PYG{p}{)}
\PYG{n}{r2\PYGZus{}adj\PYGZus{}pow} \PYG{o}{=} \PYG{l+m+mi}{1} \PYG{o}{\PYGZhy{}} \PYG{p}{(}\PYG{l+m+mi}{1} \PYG{o}{\PYGZhy{}} \PYG{n}{r2\PYGZus{}pow}\PYG{p}{)} \PYG{o}{*} \PYG{p}{(}\PYG{n}{n} \PYG{o}{\PYGZhy{}} \PYG{l+m+mi}{1}\PYG{p}{)} \PYG{o}{/} \PYG{p}{(}\PYG{n}{n} \PYG{o}{\PYGZhy{}} \PYG{l+m+mi}{1} \PYG{o}{\PYGZhy{}} \PYG{l+m+mi}{1}\PYG{p}{)}
\PYG{n}{rmse\PYGZus{}pow} \PYG{o}{=} \PYG{n}{np}\PYG{o}{.}\PYG{n}{sqrt}\PYG{p}{(}\PYG{n}{mean\PYGZus{}squared\PYGZus{}error}\PYG{p}{(}\PYG{n}{W}\PYG{p}{,} \PYG{n}{W\PYGZus{}pred\PYGZus{}pow}\PYG{p}{)}\PYG{p}{)}
\PYG{n}{mae\PYGZus{}pow} \PYG{o}{=} \PYG{n}{mean\PYGZus{}absolute\PYGZus{}error}\PYG{p}{(}\PYG{n}{W}\PYG{p}{,} \PYG{n}{W\PYGZus{}pred\PYGZus{}pow}\PYG{p}{)}

\PYG{c+c1}{\PYGZsh{} Visualización ajuste}

\PYG{c+c1}{\PYGZsh{} Índices y residuos}
\PYG{n}{indices} \PYG{o}{=} \PYG{n}{np}\PYG{o}{.}\PYG{n}{arange}\PYG{p}{(}\PYG{n+nb}{len}\PYG{p}{(}\PYG{n}{W}\PYG{p}{)}\PYG{p}{)}
\PYG{n}{residuos} \PYG{o}{=} \PYG{n}{W} \PYG{o}{\PYGZhy{}} \PYG{n}{W\PYGZus{}pred\PYGZus{}pow}

\PYG{c+c1}{\PYGZsh{} Crear la figura con 3 subplots}
\PYG{n}{fig}\PYG{p}{,} \PYG{n}{axs} \PYG{o}{=} \PYG{n}{plt}\PYG{o}{.}\PYG{n}{subplots}\PYG{p}{(}\PYG{l+m+mi}{1}\PYG{p}{,} \PYG{l+m+mi}{3}\PYG{p}{,} \PYG{n}{figsize}\PYG{o}{=}\PYG{p}{(}\PYG{l+m+mi}{18}\PYG{p}{,} \PYG{l+m+mi}{5}\PYG{p}{)}\PYG{p}{)}

\PYG{c+c1}{\PYGZsh{} === 1. Peso real vs. Peso predicho ===}
\PYG{n}{axs}\PYG{p}{[}\PYG{l+m+mi}{0}\PYG{p}{]}\PYG{o}{.}\PYG{n}{scatter}\PYG{p}{(}\PYG{n}{W}\PYG{p}{,} \PYG{n}{W\PYGZus{}pred\PYGZus{}pow}\PYG{p}{,} \PYG{n}{alpha}\PYG{o}{=}\PYG{l+m+mf}{0.7}\PYG{p}{)}
\PYG{n}{axs}\PYG{p}{[}\PYG{l+m+mi}{0}\PYG{p}{]}\PYG{o}{.}\PYG{n}{plot}\PYG{p}{(}\PYG{p}{[}\PYG{n+nb}{min}\PYG{p}{(}\PYG{n}{W}\PYG{p}{)}\PYG{p}{,} \PYG{n+nb}{max}\PYG{p}{(}\PYG{n}{W}\PYG{p}{)}\PYG{p}{]}\PYG{p}{,} \PYG{p}{[}\PYG{n+nb}{min}\PYG{p}{(}\PYG{n}{W}\PYG{p}{)}\PYG{p}{,} \PYG{n+nb}{max}\PYG{p}{(}\PYG{n}{W}\PYG{p}{)}\PYG{p}{]}\PYG{p}{,} \PYG{l+s+s1}{\PYGZsq{}}\PYG{l+s+s1}{r\PYGZhy{}\PYGZhy{}}\PYG{l+s+s1}{\PYGZsq{}}\PYG{p}{,} \PYG{n}{label}\PYG{o}{=}\PYG{l+s+s1}{\PYGZsq{}}\PYG{l+s+s1}{Línea ideal}\PYG{l+s+s1}{\PYGZsq{}}\PYG{p}{)}
\PYG{n}{axs}\PYG{p}{[}\PYG{l+m+mi}{0}\PYG{p}{]}\PYG{o}{.}\PYG{n}{set\PYGZus{}xlabel}\PYG{p}{(}\PYG{l+s+s1}{\PYGZsq{}}\PYG{l+s+s1}{Peso real}\PYG{l+s+s1}{\PYGZsq{}}\PYG{p}{)}
\PYG{n}{axs}\PYG{p}{[}\PYG{l+m+mi}{0}\PYG{p}{]}\PYG{o}{.}\PYG{n}{set\PYGZus{}ylabel}\PYG{p}{(}\PYG{l+s+s1}{\PYGZsq{}}\PYG{l+s+s1}{Peso predicho}\PYG{l+s+s1}{\PYGZsq{}}\PYG{p}{)}
\PYG{n}{axs}\PYG{p}{[}\PYG{l+m+mi}{0}\PYG{p}{]}\PYG{o}{.}\PYG{n}{set\PYGZus{}title}\PYG{p}{(}\PYG{l+s+sa}{f}\PYG{l+s+s1}{\PYGZsq{}}\PYG{l+s+s1}{Real vs. Predicho}\PYG{l+s+s1}{\PYGZsq{}}\PYG{p}{)}
\PYG{n}{axs}\PYG{p}{[}\PYG{l+m+mi}{0}\PYG{p}{]}\PYG{o}{.}\PYG{n}{legend}\PYG{p}{(}\PYG{p}{)}
\PYG{n}{axs}\PYG{p}{[}\PYG{l+m+mi}{0}\PYG{p}{]}\PYG{o}{.}\PYG{n}{grid}\PYG{p}{(}\PYG{k+kc}{True}\PYG{p}{)}

\PYG{c+c1}{\PYGZsh{} === 2. Comparación por índice ===}
\PYG{n}{axs}\PYG{p}{[}\PYG{l+m+mi}{1}\PYG{p}{]}\PYG{o}{.}\PYG{n}{scatter}\PYG{p}{(}\PYG{n}{indices}\PYG{p}{,} \PYG{n}{W}\PYG{p}{,} \PYG{n}{color}\PYG{o}{=}\PYG{n}{CSS4\PYGZus{}COLORS}\PYG{p}{[}\PYG{l+s+s1}{\PYGZsq{}}\PYG{l+s+s1}{lightskyblue}\PYG{l+s+s1}{\PYGZsq{}}\PYG{p}{]}\PYG{p}{,} \PYG{n}{label}\PYG{o}{=}\PYG{l+s+s1}{\PYGZsq{}}\PYG{l+s+s1}{Peso real}\PYG{l+s+s1}{\PYGZsq{}}\PYG{p}{,} \PYG{n}{alpha}\PYG{o}{=}\PYG{l+m+mf}{0.8}\PYG{p}{)}
\PYG{n}{axs}\PYG{p}{[}\PYG{l+m+mi}{1}\PYG{p}{]}\PYG{o}{.}\PYG{n}{scatter}\PYG{p}{(}\PYG{n}{indices}\PYG{p}{,} \PYG{n}{W\PYGZus{}pred\PYGZus{}pow}\PYG{p}{,} \PYG{n}{color}\PYG{o}{=}\PYG{n}{CSS4\PYGZus{}COLORS}\PYG{p}{[}\PYG{l+s+s1}{\PYGZsq{}}\PYG{l+s+s1}{darkorange}\PYG{l+s+s1}{\PYGZsq{}}\PYG{p}{]}\PYG{p}{,} \PYG{n}{label}\PYG{o}{=}\PYG{l+s+s1}{\PYGZsq{}}\PYG{l+s+s1}{Peso predicho}\PYG{l+s+s1}{\PYGZsq{}}\PYG{p}{,} \PYG{n}{alpha}\PYG{o}{=}\PYG{l+m+mf}{0.6}\PYG{p}{)}
\PYG{n}{axs}\PYG{p}{[}\PYG{l+m+mi}{1}\PYG{p}{]}\PYG{o}{.}\PYG{n}{set\PYGZus{}xlabel}\PYG{p}{(}\PYG{l+s+s1}{\PYGZsq{}}\PYG{l+s+s1}{Índice de muestra}\PYG{l+s+s1}{\PYGZsq{}}\PYG{p}{)}
\PYG{n}{axs}\PYG{p}{[}\PYG{l+m+mi}{1}\PYG{p}{]}\PYG{o}{.}\PYG{n}{set\PYGZus{}ylabel}\PYG{p}{(}\PYG{l+s+s1}{\PYGZsq{}}\PYG{l+s+s1}{Peso}\PYG{l+s+s1}{\PYGZsq{}}\PYG{p}{)}
\PYG{n}{axs}\PYG{p}{[}\PYG{l+m+mi}{1}\PYG{p}{]}\PYG{o}{.}\PYG{n}{set\PYGZus{}title}\PYG{p}{(}\PYG{l+s+s1}{\PYGZsq{}}\PYG{l+s+s1}{Peso real vs. predicho por muestra}\PYG{l+s+s1}{\PYGZsq{}}\PYG{p}{)}
\PYG{n}{axs}\PYG{p}{[}\PYG{l+m+mi}{1}\PYG{p}{]}\PYG{o}{.}\PYG{n}{legend}\PYG{p}{(}\PYG{p}{)}
\PYG{n}{axs}\PYG{p}{[}\PYG{l+m+mi}{1}\PYG{p}{]}\PYG{o}{.}\PYG{n}{grid}\PYG{p}{(}\PYG{k+kc}{True}\PYG{p}{)}

\PYG{c+c1}{\PYGZsh{} === 3. Gráfico de residuos ===}
\PYG{n}{axs}\PYG{p}{[}\PYG{l+m+mi}{2}\PYG{p}{]}\PYG{o}{.}\PYG{n}{scatter}\PYG{p}{(}\PYG{n}{W\PYGZus{}pred\PYGZus{}pow}\PYG{p}{,} \PYG{n}{residuos}\PYG{p}{,} \PYG{n}{alpha}\PYG{o}{=}\PYG{l+m+mf}{0.6}\PYG{p}{,} \PYG{n}{color}\PYG{o}{=}\PYG{n}{CSS4\PYGZus{}COLORS}\PYG{p}{[}\PYG{l+s+s1}{\PYGZsq{}}\PYG{l+s+s1}{violet}\PYG{l+s+s1}{\PYGZsq{}}\PYG{p}{]}\PYG{p}{)}
\PYG{n}{axs}\PYG{p}{[}\PYG{l+m+mi}{2}\PYG{p}{]}\PYG{o}{.}\PYG{n}{axhline}\PYG{p}{(}\PYG{l+m+mi}{0}\PYG{p}{,} \PYG{n}{color}\PYG{o}{=}\PYG{l+s+s1}{\PYGZsq{}}\PYG{l+s+s1}{red}\PYG{l+s+s1}{\PYGZsq{}}\PYG{p}{,} \PYG{n}{linestyle}\PYG{o}{=}\PYG{l+s+s1}{\PYGZsq{}}\PYG{l+s+s1}{\PYGZhy{}\PYGZhy{}}\PYG{l+s+s1}{\PYGZsq{}}\PYG{p}{)}
\PYG{n}{axs}\PYG{p}{[}\PYG{l+m+mi}{2}\PYG{p}{]}\PYG{o}{.}\PYG{n}{set\PYGZus{}xlabel}\PYG{p}{(}\PYG{l+s+s1}{\PYGZsq{}}\PYG{l+s+s1}{Peso predicho}\PYG{l+s+s1}{\PYGZsq{}}\PYG{p}{)}
\PYG{n}{axs}\PYG{p}{[}\PYG{l+m+mi}{2}\PYG{p}{]}\PYG{o}{.}\PYG{n}{set\PYGZus{}ylabel}\PYG{p}{(}\PYG{l+s+s1}{\PYGZsq{}}\PYG{l+s+s1}{Residuo}\PYG{l+s+s1}{\PYGZsq{}}\PYG{p}{)}
\PYG{n}{axs}\PYG{p}{[}\PYG{l+m+mi}{2}\PYG{p}{]}\PYG{o}{.}\PYG{n}{set\PYGZus{}title}\PYG{p}{(}\PYG{l+s+s1}{\PYGZsq{}}\PYG{l+s+s1}{Análisis de residuos}\PYG{l+s+s1}{\PYGZsq{}}\PYG{p}{)}
\PYG{n}{axs}\PYG{p}{[}\PYG{l+m+mi}{2}\PYG{p}{]}\PYG{o}{.}\PYG{n}{grid}\PYG{p}{(}\PYG{k+kc}{True}\PYG{p}{)}

\PYG{c+c1}{\PYGZsh{} Ajustar diseño y mostrar}
\PYG{n}{plt}\PYG{o}{.}\PYG{n}{tight\PYGZus{}layout}\PYG{p}{(}\PYG{p}{)}
\PYG{n}{plt}\PYG{o}{.}\PYG{n}{show}\PYG{p}{(}\PYG{p}{)}

\PYG{c+c1}{\PYGZsh{} Resultados del ajuste}
\PYG{n+nb}{print}\PYG{p}{(}\PYG{l+s+sa}{f}\PYG{l+s+s2}{\PYGZdq{}}\PYG{l+s+s2}{Modelo ajustado: W = }\PYG{l+s+si}{\PYGZob{}}\PYG{n}{a}\PYG{l+s+si}{:}\PYG{l+s+s2}{.5f}\PYG{l+s+si}{\PYGZcb{}}\PYG{l+s+s2}{ · S\PYGZus{}real\PYGZca{}}\PYG{l+s+si}{\PYGZob{}}\PYG{n}{b}\PYG{l+s+si}{:}\PYG{l+s+s2}{.5f}\PYG{l+s+si}{\PYGZcb{}}\PYG{l+s+s2}{\PYGZdq{}}\PYG{p}{)}
\PYG{n+nb}{print}\PYG{p}{(}\PYG{l+s+sa}{f}\PYG{l+s+s2}{\PYGZdq{}}\PYG{l+s+s2}{S\PYGZus{}real: }\PYG{l+s+si}{\PYGZob{}}\PYG{n}{coef}\PYG{l+s+si}{:}\PYG{l+s+s2}{.3f}\PYG{l+s+si}{\PYGZcb{}}\PYG{l+s+s2}{ · }\PYG{l+s+si}{\PYGZob{}}\PYG{n}{np}\PYG{o}{.}\PYG{n}{pi}\PYG{l+s+si}{:}\PYG{l+s+s2}{.5f}\PYG{l+s+si}{\PYGZcb{}}\PYG{l+s+s2}{ · S/4}\PYG{l+s+s2}{\PYGZdq{}}\PYG{p}{)}
\PYG{n+nb}{print}\PYG{p}{(}\PYG{l+s+sa}{f}\PYG{l+s+s2}{\PYGZdq{}}\PYG{l+s+s2}{R² = }\PYG{l+s+si}{\PYGZob{}}\PYG{n}{r2\PYGZus{}pow}\PYG{l+s+si}{:}\PYG{l+s+s2}{.4f}\PYG{l+s+si}{\PYGZcb{}}\PYG{l+s+s2}{\PYGZdq{}}\PYG{p}{)}
\PYG{n+nb}{print}\PYG{p}{(}\PYG{l+s+sa}{f}\PYG{l+s+s2}{\PYGZdq{}}\PYG{l+s+s2}{MAE = }\PYG{l+s+si}{\PYGZob{}}\PYG{n}{mae\PYGZus{}pow}\PYG{l+s+si}{:}\PYG{l+s+s2}{.4f}\PYG{l+s+si}{\PYGZcb{}}\PYG{l+s+s2}{ g}\PYG{l+s+s2}{\PYGZdq{}}\PYG{p}{)}
\PYG{n+nb}{print}\PYG{p}{(}\PYG{l+s+sa}{f}\PYG{l+s+s2}{\PYGZdq{}}\PYG{l+s+s2}{RSME = }\PYG{l+s+si}{\PYGZob{}}\PYG{n}{rmse\PYGZus{}pow}\PYG{l+s+si}{:}\PYG{l+s+s2}{.4f}\PYG{l+s+si}{\PYGZcb{}}\PYG{l+s+s2}{ g}\PYG{l+s+s2}{\PYGZdq{}}\PYG{p}{)}
\PYG{n+nb}{print}\PYG{p}{(}\PYG{l+s+sa}{f}\PYG{l+s+s2}{\PYGZdq{}}\PYG{l+s+s2}{R² adj = }\PYG{l+s+si}{\PYGZob{}}\PYG{n}{r2\PYGZus{}adj\PYGZus{}pow}\PYG{l+s+si}{:}\PYG{l+s+s2}{.4f}\PYG{l+s+si}{\PYGZcb{}}\PYG{l+s+s2}{ g}\PYG{l+s+s2}{\PYGZdq{}}\PYG{p}{)}
\end{sphinxVerbatim}

\end{sphinxuseclass}\end{sphinxVerbatimInput}
\begin{sphinxVerbatimOutput}

\begin{sphinxuseclass}{cell_output}
\noindent\sphinxincludegraphics{{2276d97aece9866f2fb157ce71a1628baec8a5e182218170fc0638bc6b0c4acb}.png}

\begin{sphinxVerbatim}[commandchars=\\\{\}]
Modelo ajustado: W = 0.13360 · S\PYGZus{}real\PYGZca{}1.43623
S\PYGZus{}real: 0.750 · 3.14159 · S/4
R² = 0.9733
MAE = 0.3734 g
RSME = 0.5925 g
R² adj = 0.9732 g
\end{sphinxVerbatim}

\end{sphinxuseclass}\end{sphinxVerbatimOutput}

\end{sphinxuseclass}

\subsubsection{Modelo alométrico Longitud \sphinxhyphen{} Anchura: \protect\(W=k \cdot L^a \cdot A^b\protect\)}
\label{\detokenize{content/03/Coeficientes:modelo-alometrico-longitud-anchura-w-k-cdot-l-a-cdot-a-b}}
\begin{sphinxuseclass}{cell}\begin{sphinxVerbatimInput}

\begin{sphinxuseclass}{cell_input}
\begin{sphinxVerbatim}[commandchars=\\\{\}]
\PYG{k}{def}\PYG{+w}{ }\PYG{n+nf}{alometric\PYGZus{}LA\PYGZus{}model}\PYG{p}{(}\PYG{n}{X}\PYG{p}{,} \PYG{n}{k}\PYG{p}{,} \PYG{n}{a}\PYG{p}{,} \PYG{n}{b}\PYG{p}{)}\PYG{p}{:}
    \PYG{n}{L}\PYG{p}{,} \PYG{n}{A} \PYG{o}{=} \PYG{n}{X}
    \PYG{k}{return} \PYG{n}{k} \PYG{o}{*} \PYG{p}{(}\PYG{n}{L}\PYG{o}{*}\PYG{o}{*}\PYG{n}{a}\PYG{p}{)} \PYG{o}{*} \PYG{p}{(}\PYG{n}{A}\PYG{o}{*}\PYG{o}{*}\PYG{n}{b}\PYG{p}{)}

\PYG{n}{params\PYGZus{}la}\PYG{p}{,} \PYG{n}{\PYGZus{}} \PYG{o}{=} \PYG{n}{curve\PYGZus{}fit}\PYG{p}{(}\PYG{n}{alometric\PYGZus{}LA\PYGZus{}model}\PYG{p}{,} \PYG{p}{(}\PYG{n}{L}\PYG{p}{,} \PYG{n}{A}\PYG{p}{)}\PYG{p}{,} \PYG{n}{W}\PYG{p}{)}
\PYG{n}{k}\PYG{p}{,}\PYG{n}{a}\PYG{p}{,}\PYG{n}{b} \PYG{o}{=} \PYG{n}{params\PYGZus{}la}
\PYG{n}{W\PYGZus{}pred\PYGZus{}la} \PYG{o}{=} \PYG{n}{alometric\PYGZus{}LA\PYGZus{}model}\PYG{p}{(}\PYG{p}{(}\PYG{n}{L}\PYG{p}{,} \PYG{n}{A}\PYG{p}{)}\PYG{p}{,} \PYG{n}{k}\PYG{p}{,}\PYG{n}{a}\PYG{p}{,}\PYG{n}{b}\PYG{p}{)}
\PYG{n}{r2\PYGZus{}la} \PYG{o}{=} \PYG{n}{r2\PYGZus{}score}\PYG{p}{(}\PYG{n}{W}\PYG{p}{,} \PYG{n}{W\PYGZus{}pred\PYGZus{}la}\PYG{p}{)}
\PYG{n}{r2\PYGZus{}adj\PYGZus{}la} \PYG{o}{=} \PYG{l+m+mi}{1} \PYG{o}{\PYGZhy{}} \PYG{p}{(}\PYG{l+m+mi}{1} \PYG{o}{\PYGZhy{}} \PYG{n}{r2\PYGZus{}la}\PYG{p}{)} \PYG{o}{*} \PYG{p}{(}\PYG{n}{n} \PYG{o}{\PYGZhy{}} \PYG{l+m+mi}{1}\PYG{p}{)} \PYG{o}{/} \PYG{p}{(}\PYG{n}{n} \PYG{o}{\PYGZhy{}} \PYG{l+m+mi}{2} \PYG{o}{\PYGZhy{}} \PYG{l+m+mi}{1}\PYG{p}{)}
\PYG{n}{rmse\PYGZus{}la} \PYG{o}{=} \PYG{n}{np}\PYG{o}{.}\PYG{n}{sqrt}\PYG{p}{(}\PYG{n}{mean\PYGZus{}squared\PYGZus{}error}\PYG{p}{(}\PYG{n}{W}\PYG{p}{,} \PYG{n}{W\PYGZus{}pred\PYGZus{}la}\PYG{p}{)}\PYG{p}{)}
\PYG{n}{mae\PYGZus{}la} \PYG{o}{=} \PYG{n}{mean\PYGZus{}absolute\PYGZus{}error}\PYG{p}{(}\PYG{n}{W}\PYG{p}{,} \PYG{n}{W\PYGZus{}pred\PYGZus{}la}\PYG{p}{)}

\PYG{c+c1}{\PYGZsh{} Índices y residuos}
\PYG{n}{indices} \PYG{o}{=} \PYG{n}{np}\PYG{o}{.}\PYG{n}{arange}\PYG{p}{(}\PYG{n+nb}{len}\PYG{p}{(}\PYG{n}{W}\PYG{p}{)}\PYG{p}{)}
\PYG{n}{residuos} \PYG{o}{=} \PYG{n}{W} \PYG{o}{\PYGZhy{}} \PYG{n}{W\PYGZus{}pred\PYGZus{}la}

\PYG{c+c1}{\PYGZsh{} Crear la figura con 3 subplots}
\PYG{n}{fig}\PYG{p}{,} \PYG{n}{axs} \PYG{o}{=} \PYG{n}{plt}\PYG{o}{.}\PYG{n}{subplots}\PYG{p}{(}\PYG{l+m+mi}{1}\PYG{p}{,} \PYG{l+m+mi}{3}\PYG{p}{,} \PYG{n}{figsize}\PYG{o}{=}\PYG{p}{(}\PYG{l+m+mi}{18}\PYG{p}{,} \PYG{l+m+mi}{5}\PYG{p}{)}\PYG{p}{)}


\PYG{c+c1}{\PYGZsh{} === 1. Peso real vs. Peso predicho ===}
\PYG{n}{axs}\PYG{p}{[}\PYG{l+m+mi}{0}\PYG{p}{]}\PYG{o}{.}\PYG{n}{scatter}\PYG{p}{(}\PYG{n}{W}\PYG{p}{,} \PYG{n}{W\PYGZus{}pred\PYGZus{}la}\PYG{p}{,} \PYG{n}{alpha}\PYG{o}{=}\PYG{l+m+mf}{0.7}\PYG{p}{)}
\PYG{n}{axs}\PYG{p}{[}\PYG{l+m+mi}{0}\PYG{p}{]}\PYG{o}{.}\PYG{n}{plot}\PYG{p}{(}\PYG{p}{[}\PYG{n+nb}{min}\PYG{p}{(}\PYG{n}{W}\PYG{p}{)}\PYG{p}{,} \PYG{n+nb}{max}\PYG{p}{(}\PYG{n}{W}\PYG{p}{)}\PYG{p}{]}\PYG{p}{,} \PYG{p}{[}\PYG{n+nb}{min}\PYG{p}{(}\PYG{n}{W}\PYG{p}{)}\PYG{p}{,} \PYG{n+nb}{max}\PYG{p}{(}\PYG{n}{W}\PYG{p}{)}\PYG{p}{]}\PYG{p}{,} \PYG{l+s+s1}{\PYGZsq{}}\PYG{l+s+s1}{r\PYGZhy{}\PYGZhy{}}\PYG{l+s+s1}{\PYGZsq{}}\PYG{p}{,} \PYG{n}{label}\PYG{o}{=}\PYG{l+s+s1}{\PYGZsq{}}\PYG{l+s+s1}{Línea ideal}\PYG{l+s+s1}{\PYGZsq{}}\PYG{p}{)}
\PYG{n}{axs}\PYG{p}{[}\PYG{l+m+mi}{0}\PYG{p}{]}\PYG{o}{.}\PYG{n}{set\PYGZus{}xlabel}\PYG{p}{(}\PYG{l+s+s1}{\PYGZsq{}}\PYG{l+s+s1}{Peso real}\PYG{l+s+s1}{\PYGZsq{}}\PYG{p}{)}
\PYG{n}{axs}\PYG{p}{[}\PYG{l+m+mi}{0}\PYG{p}{]}\PYG{o}{.}\PYG{n}{set\PYGZus{}ylabel}\PYG{p}{(}\PYG{l+s+s1}{\PYGZsq{}}\PYG{l+s+s1}{Peso predicho}\PYG{l+s+s1}{\PYGZsq{}}\PYG{p}{)}
\PYG{n}{axs}\PYG{p}{[}\PYG{l+m+mi}{0}\PYG{p}{]}\PYG{o}{.}\PYG{n}{set\PYGZus{}title}\PYG{p}{(}\PYG{l+s+sa}{f}\PYG{l+s+s1}{\PYGZsq{}}\PYG{l+s+s1}{Real vs. Predicho}\PYG{l+s+s1}{\PYGZsq{}}\PYG{p}{)}
\PYG{n}{axs}\PYG{p}{[}\PYG{l+m+mi}{0}\PYG{p}{]}\PYG{o}{.}\PYG{n}{legend}\PYG{p}{(}\PYG{p}{)}
\PYG{n}{axs}\PYG{p}{[}\PYG{l+m+mi}{0}\PYG{p}{]}\PYG{o}{.}\PYG{n}{grid}\PYG{p}{(}\PYG{k+kc}{True}\PYG{p}{)}

\PYG{c+c1}{\PYGZsh{} === 2. Comparación por índice ===}
\PYG{n}{axs}\PYG{p}{[}\PYG{l+m+mi}{1}\PYG{p}{]}\PYG{o}{.}\PYG{n}{scatter}\PYG{p}{(}\PYG{n}{indices}\PYG{p}{,} \PYG{n}{W}\PYG{p}{,} \PYG{n}{color}\PYG{o}{=}\PYG{n}{CSS4\PYGZus{}COLORS}\PYG{p}{[}\PYG{l+s+s1}{\PYGZsq{}}\PYG{l+s+s1}{lightskyblue}\PYG{l+s+s1}{\PYGZsq{}}\PYG{p}{]}\PYG{p}{,} \PYG{n}{label}\PYG{o}{=}\PYG{l+s+s1}{\PYGZsq{}}\PYG{l+s+s1}{Peso real}\PYG{l+s+s1}{\PYGZsq{}}\PYG{p}{,} \PYG{n}{alpha}\PYG{o}{=}\PYG{l+m+mf}{0.8}\PYG{p}{)}
\PYG{n}{axs}\PYG{p}{[}\PYG{l+m+mi}{1}\PYG{p}{]}\PYG{o}{.}\PYG{n}{scatter}\PYG{p}{(}\PYG{n}{indices}\PYG{p}{,} \PYG{n}{W\PYGZus{}pred\PYGZus{}la}\PYG{p}{,} \PYG{n}{color}\PYG{o}{=}\PYG{n}{CSS4\PYGZus{}COLORS}\PYG{p}{[}\PYG{l+s+s1}{\PYGZsq{}}\PYG{l+s+s1}{darkorange}\PYG{l+s+s1}{\PYGZsq{}}\PYG{p}{]}\PYG{p}{,} \PYG{n}{label}\PYG{o}{=}\PYG{l+s+s1}{\PYGZsq{}}\PYG{l+s+s1}{Peso predicho}\PYG{l+s+s1}{\PYGZsq{}}\PYG{p}{,} \PYG{n}{alpha}\PYG{o}{=}\PYG{l+m+mf}{0.6}\PYG{p}{)}
\PYG{n}{axs}\PYG{p}{[}\PYG{l+m+mi}{1}\PYG{p}{]}\PYG{o}{.}\PYG{n}{set\PYGZus{}xlabel}\PYG{p}{(}\PYG{l+s+s1}{\PYGZsq{}}\PYG{l+s+s1}{Índice de muestra}\PYG{l+s+s1}{\PYGZsq{}}\PYG{p}{)}
\PYG{n}{axs}\PYG{p}{[}\PYG{l+m+mi}{1}\PYG{p}{]}\PYG{o}{.}\PYG{n}{set\PYGZus{}ylabel}\PYG{p}{(}\PYG{l+s+s1}{\PYGZsq{}}\PYG{l+s+s1}{Peso}\PYG{l+s+s1}{\PYGZsq{}}\PYG{p}{)}
\PYG{n}{axs}\PYG{p}{[}\PYG{l+m+mi}{1}\PYG{p}{]}\PYG{o}{.}\PYG{n}{set\PYGZus{}title}\PYG{p}{(}\PYG{l+s+s1}{\PYGZsq{}}\PYG{l+s+s1}{Peso real vs. predicho por muestra}\PYG{l+s+s1}{\PYGZsq{}}\PYG{p}{)}
\PYG{n}{axs}\PYG{p}{[}\PYG{l+m+mi}{1}\PYG{p}{]}\PYG{o}{.}\PYG{n}{legend}\PYG{p}{(}\PYG{p}{)}
\PYG{n}{axs}\PYG{p}{[}\PYG{l+m+mi}{1}\PYG{p}{]}\PYG{o}{.}\PYG{n}{grid}\PYG{p}{(}\PYG{k+kc}{True}\PYG{p}{)}

\PYG{c+c1}{\PYGZsh{} === 3. Gráfico de residuos ===}
\PYG{n}{axs}\PYG{p}{[}\PYG{l+m+mi}{2}\PYG{p}{]}\PYG{o}{.}\PYG{n}{scatter}\PYG{p}{(}\PYG{n}{W\PYGZus{}pred\PYGZus{}la}\PYG{p}{,} \PYG{n}{residuos}\PYG{p}{,} \PYG{n}{alpha}\PYG{o}{=}\PYG{l+m+mf}{0.6}\PYG{p}{,} \PYG{n}{color}\PYG{o}{=}\PYG{n}{CSS4\PYGZus{}COLORS}\PYG{p}{[}\PYG{l+s+s1}{\PYGZsq{}}\PYG{l+s+s1}{violet}\PYG{l+s+s1}{\PYGZsq{}}\PYG{p}{]}\PYG{p}{)}
\PYG{n}{axs}\PYG{p}{[}\PYG{l+m+mi}{2}\PYG{p}{]}\PYG{o}{.}\PYG{n}{axhline}\PYG{p}{(}\PYG{l+m+mi}{0}\PYG{p}{,} \PYG{n}{color}\PYG{o}{=}\PYG{l+s+s1}{\PYGZsq{}}\PYG{l+s+s1}{red}\PYG{l+s+s1}{\PYGZsq{}}\PYG{p}{,} \PYG{n}{linestyle}\PYG{o}{=}\PYG{l+s+s1}{\PYGZsq{}}\PYG{l+s+s1}{\PYGZhy{}\PYGZhy{}}\PYG{l+s+s1}{\PYGZsq{}}\PYG{p}{)}
\PYG{n}{axs}\PYG{p}{[}\PYG{l+m+mi}{2}\PYG{p}{]}\PYG{o}{.}\PYG{n}{set\PYGZus{}xlabel}\PYG{p}{(}\PYG{l+s+s1}{\PYGZsq{}}\PYG{l+s+s1}{Peso predicho}\PYG{l+s+s1}{\PYGZsq{}}\PYG{p}{)}
\PYG{n}{axs}\PYG{p}{[}\PYG{l+m+mi}{2}\PYG{p}{]}\PYG{o}{.}\PYG{n}{set\PYGZus{}ylabel}\PYG{p}{(}\PYG{l+s+s1}{\PYGZsq{}}\PYG{l+s+s1}{Residuo}\PYG{l+s+s1}{\PYGZsq{}}\PYG{p}{)}
\PYG{n}{axs}\PYG{p}{[}\PYG{l+m+mi}{2}\PYG{p}{]}\PYG{o}{.}\PYG{n}{set\PYGZus{}title}\PYG{p}{(}\PYG{l+s+s1}{\PYGZsq{}}\PYG{l+s+s1}{Análisis de residuos}\PYG{l+s+s1}{\PYGZsq{}}\PYG{p}{)}
\PYG{n}{axs}\PYG{p}{[}\PYG{l+m+mi}{2}\PYG{p}{]}\PYG{o}{.}\PYG{n}{grid}\PYG{p}{(}\PYG{k+kc}{True}\PYG{p}{)}

\PYG{c+c1}{\PYGZsh{} Ajustar diseño y mostrar}
\PYG{n}{plt}\PYG{o}{.}\PYG{n}{tight\PYGZus{}layout}\PYG{p}{(}\PYG{p}{)}
\PYG{n}{plt}\PYG{o}{.}\PYG{n}{show}\PYG{p}{(}\PYG{p}{)}

\PYG{c+c1}{\PYGZsh{} Resultados del ajuste}
\PYG{n+nb}{print}\PYG{p}{(}\PYG{l+s+sa}{f}\PYG{l+s+s2}{\PYGZdq{}}\PYG{l+s+s2}{Modelo ajustado: W = }\PYG{l+s+si}{\PYGZob{}}\PYG{n}{k}\PYG{l+s+si}{:}\PYG{l+s+s2}{.5f}\PYG{l+s+si}{\PYGZcb{}}\PYG{l+s+s2}{·L\PYGZca{}}\PYG{l+s+si}{\PYGZob{}}\PYG{n}{a}\PYG{l+s+si}{:}\PYG{l+s+s2}{.3f}\PYG{l+s+si}{\PYGZcb{}}\PYG{l+s+s2}{·A\PYGZca{}}\PYG{l+s+si}{\PYGZob{}}\PYG{n}{b}\PYG{l+s+si}{:}\PYG{l+s+s2}{.3f}\PYG{l+s+si}{\PYGZcb{}}\PYG{l+s+s2}{\PYGZdq{}}\PYG{p}{)}
\PYG{n+nb}{print}\PYG{p}{(}\PYG{l+s+sa}{f}\PYG{l+s+s2}{\PYGZdq{}}\PYG{l+s+s2}{R² = }\PYG{l+s+si}{\PYGZob{}}\PYG{n}{r2\PYGZus{}la}\PYG{l+s+si}{:}\PYG{l+s+s2}{.4f}\PYG{l+s+si}{\PYGZcb{}}\PYG{l+s+s2}{\PYGZdq{}}\PYG{p}{)}
\PYG{n+nb}{print}\PYG{p}{(}\PYG{l+s+sa}{f}\PYG{l+s+s2}{\PYGZdq{}}\PYG{l+s+s2}{MAE = }\PYG{l+s+si}{\PYGZob{}}\PYG{n}{mae\PYGZus{}la}\PYG{l+s+si}{:}\PYG{l+s+s2}{.4f}\PYG{l+s+si}{\PYGZcb{}}\PYG{l+s+s2}{ g}\PYG{l+s+s2}{\PYGZdq{}}\PYG{p}{)}
\PYG{n+nb}{print}\PYG{p}{(}\PYG{l+s+sa}{f}\PYG{l+s+s2}{\PYGZdq{}}\PYG{l+s+s2}{RSME = }\PYG{l+s+si}{\PYGZob{}}\PYG{n}{rmse\PYGZus{}la}\PYG{l+s+si}{:}\PYG{l+s+s2}{.4f}\PYG{l+s+si}{\PYGZcb{}}\PYG{l+s+s2}{ g}\PYG{l+s+s2}{\PYGZdq{}}\PYG{p}{)}
\PYG{n+nb}{print}\PYG{p}{(}\PYG{l+s+sa}{f}\PYG{l+s+s2}{\PYGZdq{}}\PYG{l+s+s2}{R² adj = }\PYG{l+s+si}{\PYGZob{}}\PYG{n}{r2\PYGZus{}adj\PYGZus{}la}\PYG{l+s+si}{:}\PYG{l+s+s2}{.4f}\PYG{l+s+si}{\PYGZcb{}}\PYG{l+s+s2}{ g}\PYG{l+s+s2}{\PYGZdq{}}\PYG{p}{)}
\end{sphinxVerbatim}

\end{sphinxuseclass}\end{sphinxVerbatimInput}
\begin{sphinxVerbatimOutput}

\begin{sphinxuseclass}{cell_output}
\noindent\sphinxincludegraphics{{38762abd81e9639b04d5e9b6d5469ba3728653a33bd04db249a57aa4d1641e9c}.png}

\begin{sphinxVerbatim}[commandchars=\\\{\}]
Modelo ajustado: W = 0.04891·L\PYGZca{}1.644·A\PYGZca{}1.272
R² = 0.9739
MAE = 0.3735 g
RSME = 0.5854 g
R² adj = 0.9737 g
\end{sphinxVerbatim}

\end{sphinxuseclass}\end{sphinxVerbatimOutput}

\end{sphinxuseclass}

\subsubsection{Modelo de potencia volumétrico: \protect\(W=a \cdot (c\cdot V)^b\protect\)}
\label{\detokenize{content/03/Coeficientes:modelo-de-potencia-volumetrico-w-a-cdot-c-cdot-v-b}}
\begin{sphinxuseclass}{cell}\begin{sphinxVerbatimInput}

\begin{sphinxuseclass}{cell_input}
\begin{sphinxVerbatim}[commandchars=\\\{\}]
\PYG{c+c1}{\PYGZsh{} Factor de compresión dorsoventral}
\PYG{n}{c} \PYG{o}{=} \PYG{l+m+mf}{0.72}

\PYG{k}{def}\PYG{+w}{ }\PYG{n+nf}{volumen\PYGZus{}model}\PYG{p}{(}\PYG{n}{V}\PYG{p}{,} \PYG{n}{a}\PYG{p}{,} \PYG{n}{b}\PYG{p}{)}\PYG{p}{:}
    \PYG{k}{return} \PYG{n}{a} \PYG{o}{*} \PYG{p}{(}\PYG{n}{c} \PYG{o}{*} \PYG{n}{V}\PYG{p}{)}\PYG{o}{*}\PYG{o}{*}\PYG{n}{b}

\PYG{n}{params\PYGZus{}vol}\PYG{p}{,} \PYG{n}{\PYGZus{}} \PYG{o}{=} \PYG{n}{curve\PYGZus{}fit}\PYG{p}{(}\PYG{n}{volumen\PYGZus{}model}\PYG{p}{,} \PYG{n}{V}\PYG{p}{,} \PYG{n}{W}\PYG{p}{)}
\PYG{n}{a}\PYG{p}{,}\PYG{n}{b} \PYG{o}{=} \PYG{n}{params\PYGZus{}vol}
\PYG{n}{W\PYGZus{}pred\PYGZus{}vol} \PYG{o}{=} \PYG{n}{volumen\PYGZus{}model}\PYG{p}{(}\PYG{n}{V}\PYG{p}{,} \PYG{o}{*}\PYG{n}{params\PYGZus{}vol}\PYG{p}{)}
\PYG{n}{r2\PYGZus{}vol} \PYG{o}{=} \PYG{n}{r2\PYGZus{}score}\PYG{p}{(}\PYG{n}{W}\PYG{p}{,} \PYG{n}{W\PYGZus{}pred\PYGZus{}vol}\PYG{p}{)}
\PYG{n}{r2\PYGZus{}adj\PYGZus{}vol} \PYG{o}{=} \PYG{l+m+mi}{1} \PYG{o}{\PYGZhy{}} \PYG{p}{(}\PYG{l+m+mi}{1} \PYG{o}{\PYGZhy{}} \PYG{n}{r2\PYGZus{}vol}\PYG{p}{)} \PYG{o}{*} \PYG{p}{(}\PYG{n}{n} \PYG{o}{\PYGZhy{}} \PYG{l+m+mi}{1}\PYG{p}{)} \PYG{o}{/} \PYG{p}{(}\PYG{n}{n} \PYG{o}{\PYGZhy{}} \PYG{l+m+mi}{1} \PYG{o}{\PYGZhy{}} \PYG{l+m+mi}{1}\PYG{p}{)}
\PYG{n}{rmse\PYGZus{}vol} \PYG{o}{=} \PYG{n}{np}\PYG{o}{.}\PYG{n}{sqrt}\PYG{p}{(}\PYG{n}{mean\PYGZus{}squared\PYGZus{}error}\PYG{p}{(}\PYG{n}{W}\PYG{p}{,} \PYG{n}{W\PYGZus{}pred\PYGZus{}vol}\PYG{p}{)}\PYG{p}{)}
\PYG{n}{mae\PYGZus{}vol} \PYG{o}{=} \PYG{n}{mean\PYGZus{}absolute\PYGZus{}error}\PYG{p}{(}\PYG{n}{W}\PYG{p}{,} \PYG{n}{W\PYGZus{}pred\PYGZus{}vol}\PYG{p}{)}

\PYG{c+c1}{\PYGZsh{} Visualización ajuste}
\PYG{n}{indices} \PYG{o}{=} \PYG{n}{np}\PYG{o}{.}\PYG{n}{arange}\PYG{p}{(}\PYG{n+nb}{len}\PYG{p}{(}\PYG{n}{W}\PYG{p}{)}\PYG{p}{)}
\PYG{n}{residuos} \PYG{o}{=} \PYG{n}{W} \PYG{o}{\PYGZhy{}} \PYG{n}{W\PYGZus{}pred\PYGZus{}vol}

\PYG{n}{fig}\PYG{p}{,} \PYG{n}{axs} \PYG{o}{=} \PYG{n}{plt}\PYG{o}{.}\PYG{n}{subplots}\PYG{p}{(}\PYG{l+m+mi}{1}\PYG{p}{,} \PYG{l+m+mi}{3}\PYG{p}{,} \PYG{n}{figsize}\PYG{o}{=}\PYG{p}{(}\PYG{l+m+mi}{18}\PYG{p}{,} \PYG{l+m+mi}{5}\PYG{p}{)}\PYG{p}{)}

\PYG{c+c1}{\PYGZsh{} === 1. Peso real vs. Peso predicho ===}
\PYG{n}{axs}\PYG{p}{[}\PYG{l+m+mi}{0}\PYG{p}{]}\PYG{o}{.}\PYG{n}{scatter}\PYG{p}{(}\PYG{n}{W}\PYG{p}{,} \PYG{n}{W\PYGZus{}pred\PYGZus{}vol}\PYG{p}{,} \PYG{n}{alpha}\PYG{o}{=}\PYG{l+m+mf}{0.7}\PYG{p}{)}
\PYG{n}{axs}\PYG{p}{[}\PYG{l+m+mi}{0}\PYG{p}{]}\PYG{o}{.}\PYG{n}{plot}\PYG{p}{(}\PYG{p}{[}\PYG{n+nb}{min}\PYG{p}{(}\PYG{n}{W}\PYG{p}{)}\PYG{p}{,} \PYG{n+nb}{max}\PYG{p}{(}\PYG{n}{W}\PYG{p}{)}\PYG{p}{]}\PYG{p}{,} \PYG{p}{[}\PYG{n+nb}{min}\PYG{p}{(}\PYG{n}{W}\PYG{p}{)}\PYG{p}{,} \PYG{n+nb}{max}\PYG{p}{(}\PYG{n}{W}\PYG{p}{)}\PYG{p}{]}\PYG{p}{,} \PYG{l+s+s1}{\PYGZsq{}}\PYG{l+s+s1}{r\PYGZhy{}\PYGZhy{}}\PYG{l+s+s1}{\PYGZsq{}}\PYG{p}{,} \PYG{n}{label}\PYG{o}{=}\PYG{l+s+s1}{\PYGZsq{}}\PYG{l+s+s1}{Línea ideal}\PYG{l+s+s1}{\PYGZsq{}}\PYG{p}{)}
\PYG{n}{axs}\PYG{p}{[}\PYG{l+m+mi}{0}\PYG{p}{]}\PYG{o}{.}\PYG{n}{set\PYGZus{}xlabel}\PYG{p}{(}\PYG{l+s+s1}{\PYGZsq{}}\PYG{l+s+s1}{Peso real (W)}\PYG{l+s+s1}{\PYGZsq{}}\PYG{p}{)}
\PYG{n}{axs}\PYG{p}{[}\PYG{l+m+mi}{0}\PYG{p}{]}\PYG{o}{.}\PYG{n}{set\PYGZus{}ylabel}\PYG{p}{(}\PYG{l+s+s1}{\PYGZsq{}}\PYG{l+s+s1}{Peso predicho}\PYG{l+s+s1}{\PYGZsq{}}\PYG{p}{)}
\PYG{n}{axs}\PYG{p}{[}\PYG{l+m+mi}{0}\PYG{p}{]}\PYG{o}{.}\PYG{n}{set\PYGZus{}title}\PYG{p}{(}\PYG{l+s+sa}{f}\PYG{l+s+s1}{\PYGZsq{}}\PYG{l+s+s1}{Real vs. Predicho}\PYG{l+s+s1}{\PYGZsq{}}\PYG{p}{)}
\PYG{n}{axs}\PYG{p}{[}\PYG{l+m+mi}{0}\PYG{p}{]}\PYG{o}{.}\PYG{n}{legend}\PYG{p}{(}\PYG{p}{)}
\PYG{n}{axs}\PYG{p}{[}\PYG{l+m+mi}{0}\PYG{p}{]}\PYG{o}{.}\PYG{n}{grid}\PYG{p}{(}\PYG{k+kc}{True}\PYG{p}{)}

\PYG{c+c1}{\PYGZsh{} === 2. Comparación por índice ===}
\PYG{n}{axs}\PYG{p}{[}\PYG{l+m+mi}{1}\PYG{p}{]}\PYG{o}{.}\PYG{n}{scatter}\PYG{p}{(}\PYG{n}{indices}\PYG{p}{,} \PYG{n}{W}\PYG{p}{,} \PYG{n}{color}\PYG{o}{=}\PYG{n}{CSS4\PYGZus{}COLORS}\PYG{p}{[}\PYG{l+s+s1}{\PYGZsq{}}\PYG{l+s+s1}{lightskyblue}\PYG{l+s+s1}{\PYGZsq{}}\PYG{p}{]}\PYG{p}{,} \PYG{n}{label}\PYG{o}{=}\PYG{l+s+s1}{\PYGZsq{}}\PYG{l+s+s1}{Peso real}\PYG{l+s+s1}{\PYGZsq{}}\PYG{p}{,} \PYG{n}{alpha}\PYG{o}{=}\PYG{l+m+mf}{0.8}\PYG{p}{)}
\PYG{n}{axs}\PYG{p}{[}\PYG{l+m+mi}{1}\PYG{p}{]}\PYG{o}{.}\PYG{n}{scatter}\PYG{p}{(}\PYG{n}{indices}\PYG{p}{,} \PYG{n}{W\PYGZus{}pred\PYGZus{}vol}\PYG{p}{,} \PYG{n}{color}\PYG{o}{=}\PYG{n}{CSS4\PYGZus{}COLORS}\PYG{p}{[}\PYG{l+s+s1}{\PYGZsq{}}\PYG{l+s+s1}{darkorange}\PYG{l+s+s1}{\PYGZsq{}}\PYG{p}{]}\PYG{p}{,} \PYG{n}{label}\PYG{o}{=}\PYG{l+s+s1}{\PYGZsq{}}\PYG{l+s+s1}{Peso predicho}\PYG{l+s+s1}{\PYGZsq{}}\PYG{p}{,} \PYG{n}{alpha}\PYG{o}{=}\PYG{l+m+mf}{0.6}\PYG{p}{)}
\PYG{n}{axs}\PYG{p}{[}\PYG{l+m+mi}{1}\PYG{p}{]}\PYG{o}{.}\PYG{n}{set\PYGZus{}xlabel}\PYG{p}{(}\PYG{l+s+s1}{\PYGZsq{}}\PYG{l+s+s1}{Índice de muestra}\PYG{l+s+s1}{\PYGZsq{}}\PYG{p}{)}
\PYG{n}{axs}\PYG{p}{[}\PYG{l+m+mi}{1}\PYG{p}{]}\PYG{o}{.}\PYG{n}{set\PYGZus{}ylabel}\PYG{p}{(}\PYG{l+s+s1}{\PYGZsq{}}\PYG{l+s+s1}{Peso}\PYG{l+s+s1}{\PYGZsq{}}\PYG{p}{)}
\PYG{n}{axs}\PYG{p}{[}\PYG{l+m+mi}{1}\PYG{p}{]}\PYG{o}{.}\PYG{n}{set\PYGZus{}title}\PYG{p}{(}\PYG{l+s+s1}{\PYGZsq{}}\PYG{l+s+s1}{Peso real vs. predicho por muestra}\PYG{l+s+s1}{\PYGZsq{}}\PYG{p}{)}
\PYG{n}{axs}\PYG{p}{[}\PYG{l+m+mi}{1}\PYG{p}{]}\PYG{o}{.}\PYG{n}{legend}\PYG{p}{(}\PYG{p}{)}
\PYG{n}{axs}\PYG{p}{[}\PYG{l+m+mi}{1}\PYG{p}{]}\PYG{o}{.}\PYG{n}{grid}\PYG{p}{(}\PYG{k+kc}{True}\PYG{p}{)}

\PYG{c+c1}{\PYGZsh{} === 3. Gráfico de residuos ===}
\PYG{n}{axs}\PYG{p}{[}\PYG{l+m+mi}{2}\PYG{p}{]}\PYG{o}{.}\PYG{n}{scatter}\PYG{p}{(}\PYG{n}{W\PYGZus{}pred\PYGZus{}vol}\PYG{p}{,} \PYG{n}{residuos}\PYG{p}{,} \PYG{n}{alpha}\PYG{o}{=}\PYG{l+m+mf}{0.6}\PYG{p}{,} \PYG{n}{color}\PYG{o}{=}\PYG{n}{CSS4\PYGZus{}COLORS}\PYG{p}{[}\PYG{l+s+s1}{\PYGZsq{}}\PYG{l+s+s1}{violet}\PYG{l+s+s1}{\PYGZsq{}}\PYG{p}{]}\PYG{p}{)}
\PYG{n}{axs}\PYG{p}{[}\PYG{l+m+mi}{2}\PYG{p}{]}\PYG{o}{.}\PYG{n}{axhline}\PYG{p}{(}\PYG{l+m+mi}{0}\PYG{p}{,} \PYG{n}{color}\PYG{o}{=}\PYG{l+s+s1}{\PYGZsq{}}\PYG{l+s+s1}{red}\PYG{l+s+s1}{\PYGZsq{}}\PYG{p}{,} \PYG{n}{linestyle}\PYG{o}{=}\PYG{l+s+s1}{\PYGZsq{}}\PYG{l+s+s1}{\PYGZhy{}\PYGZhy{}}\PYG{l+s+s1}{\PYGZsq{}}\PYG{p}{)}
\PYG{n}{axs}\PYG{p}{[}\PYG{l+m+mi}{2}\PYG{p}{]}\PYG{o}{.}\PYG{n}{set\PYGZus{}xlabel}\PYG{p}{(}\PYG{l+s+s1}{\PYGZsq{}}\PYG{l+s+s1}{Peso predicho}\PYG{l+s+s1}{\PYGZsq{}}\PYG{p}{)}
\PYG{n}{axs}\PYG{p}{[}\PYG{l+m+mi}{2}\PYG{p}{]}\PYG{o}{.}\PYG{n}{set\PYGZus{}ylabel}\PYG{p}{(}\PYG{l+s+s1}{\PYGZsq{}}\PYG{l+s+s1}{Residuo}\PYG{l+s+s1}{\PYGZsq{}}\PYG{p}{)}
\PYG{n}{axs}\PYG{p}{[}\PYG{l+m+mi}{2}\PYG{p}{]}\PYG{o}{.}\PYG{n}{set\PYGZus{}title}\PYG{p}{(}\PYG{l+s+s1}{\PYGZsq{}}\PYG{l+s+s1}{Análisis de residuos}\PYG{l+s+s1}{\PYGZsq{}}\PYG{p}{)}
\PYG{n}{axs}\PYG{p}{[}\PYG{l+m+mi}{2}\PYG{p}{]}\PYG{o}{.}\PYG{n}{grid}\PYG{p}{(}\PYG{k+kc}{True}\PYG{p}{)}

\PYG{n}{plt}\PYG{o}{.}\PYG{n}{tight\PYGZus{}layout}\PYG{p}{(}\PYG{p}{)}
\PYG{n}{plt}\PYG{o}{.}\PYG{n}{show}\PYG{p}{(}\PYG{p}{)}

\PYG{c+c1}{\PYGZsh{} Resultados del ajuste}
\PYG{n+nb}{print}\PYG{p}{(}\PYG{l+s+sa}{f}\PYG{l+s+s2}{\PYGZdq{}}\PYG{l+s+s2}{Modelo ajustado: W = }\PYG{l+s+si}{\PYGZob{}}\PYG{n}{a}\PYG{l+s+si}{:}\PYG{l+s+s2}{.5f}\PYG{l+s+si}{\PYGZcb{}}\PYG{l+s+s2}{ · (}\PYG{l+s+si}{\PYGZob{}}\PYG{n}{c}\PYG{l+s+si}{:}\PYG{l+s+s2}{.3f}\PYG{l+s+si}{\PYGZcb{}}\PYG{l+s+s2}{V)\PYGZca{}}\PYG{l+s+si}{\PYGZob{}}\PYG{n}{b}\PYG{l+s+si}{:}\PYG{l+s+s2}{.5f}\PYG{l+s+si}{\PYGZcb{}}\PYG{l+s+s2}{\PYGZdq{}}\PYG{p}{)}
\PYG{n+nb}{print}\PYG{p}{(}\PYG{l+s+sa}{f}\PYG{l+s+s2}{\PYGZdq{}}\PYG{l+s+s2}{R² = }\PYG{l+s+si}{\PYGZob{}}\PYG{n}{r2\PYGZus{}vol}\PYG{l+s+si}{:}\PYG{l+s+s2}{.4f}\PYG{l+s+si}{\PYGZcb{}}\PYG{l+s+s2}{\PYGZdq{}}\PYG{p}{)}
\PYG{n+nb}{print}\PYG{p}{(}\PYG{l+s+sa}{f}\PYG{l+s+s2}{\PYGZdq{}}\PYG{l+s+s2}{MAE = }\PYG{l+s+si}{\PYGZob{}}\PYG{n}{mae\PYGZus{}vol}\PYG{l+s+si}{:}\PYG{l+s+s2}{.4f}\PYG{l+s+si}{\PYGZcb{}}\PYG{l+s+s2}{ g}\PYG{l+s+s2}{\PYGZdq{}}\PYG{p}{)}
\PYG{n+nb}{print}\PYG{p}{(}\PYG{l+s+sa}{f}\PYG{l+s+s2}{\PYGZdq{}}\PYG{l+s+s2}{RSME = }\PYG{l+s+si}{\PYGZob{}}\PYG{n}{rmse\PYGZus{}vol}\PYG{l+s+si}{:}\PYG{l+s+s2}{.4f}\PYG{l+s+si}{\PYGZcb{}}\PYG{l+s+s2}{ g}\PYG{l+s+s2}{\PYGZdq{}}\PYG{p}{)}
\PYG{n+nb}{print}\PYG{p}{(}\PYG{l+s+sa}{f}\PYG{l+s+s2}{\PYGZdq{}}\PYG{l+s+s2}{R² adj = }\PYG{l+s+si}{\PYGZob{}}\PYG{n}{r2\PYGZus{}adj\PYGZus{}vol}\PYG{l+s+si}{:}\PYG{l+s+s2}{.4f}\PYG{l+s+si}{\PYGZcb{}}\PYG{l+s+s2}{ g}\PYG{l+s+s2}{\PYGZdq{}}\PYG{p}{)}
\end{sphinxVerbatim}

\end{sphinxuseclass}\end{sphinxVerbatimInput}
\begin{sphinxVerbatimOutput}

\begin{sphinxuseclass}{cell_output}
\noindent\sphinxincludegraphics{{e0800f4e91383775e5710572b3a329f514745e6aaa476fd8d3c016b479478d6c}.png}

\begin{sphinxVerbatim}[commandchars=\\\{\}]
Modelo ajustado: W = 0.75546 · (0.720V)\PYGZca{}0.95762
R² = 0.9746
MAE = 0.4241 g
RSME = 0.5785 g
R² adj = 0.9744 g
\end{sphinxVerbatim}

\end{sphinxuseclass}\end{sphinxVerbatimOutput}

\end{sphinxuseclass}

\subsubsection{Modelo alométrico Longitud \sphinxhyphen{} Anchura \sphinxhyphen{} Altura: \protect\(W = k \cdot L^a \cdot A^b \cdot H^c\protect\)}
\label{\detokenize{content/03/Coeficientes:modelo-alometrico-longitud-anchura-altura-w-k-cdot-l-a-cdot-a-b-cdot-h-c}}
\begin{sphinxuseclass}{cell}\begin{sphinxVerbatimInput}

\begin{sphinxuseclass}{cell_input}
\begin{sphinxVerbatim}[commandchars=\\\{\}]
\PYG{k}{def}\PYG{+w}{ }\PYG{n+nf}{alometric\PYGZus{}model}\PYG{p}{(}\PYG{n}{X}\PYG{p}{,} \PYG{n}{k}\PYG{p}{,} \PYG{n}{a}\PYG{p}{,} \PYG{n}{b}\PYG{p}{,} \PYG{n}{c}\PYG{p}{)}\PYG{p}{:}
    \PYG{n}{L}\PYG{p}{,} \PYG{n}{A}\PYG{p}{,} \PYG{n}{H} \PYG{o}{=} \PYG{n}{X}
    \PYG{k}{return} \PYG{n}{k} \PYG{o}{*} \PYG{p}{(}\PYG{n}{L}\PYG{o}{*}\PYG{o}{*}\PYG{n}{a}\PYG{p}{)} \PYG{o}{*} \PYG{p}{(}\PYG{n}{A}\PYG{o}{*}\PYG{o}{*}\PYG{n}{b}\PYG{p}{)} \PYG{o}{*} \PYG{p}{(}\PYG{n}{H}\PYG{o}{*}\PYG{o}{*}\PYG{n}{c}\PYG{p}{)}

\PYG{n}{params\PYGZus{}nl}\PYG{p}{,} \PYG{n}{\PYGZus{}} \PYG{o}{=} \PYG{n}{curve\PYGZus{}fit}\PYG{p}{(}\PYG{n}{alometric\PYGZus{}model}\PYG{p}{,} \PYG{p}{(}\PYG{n}{L}\PYG{p}{,} \PYG{n}{A}\PYG{p}{,} \PYG{n}{H}\PYG{p}{)}\PYG{p}{,} \PYG{n}{W}\PYG{p}{)}
\PYG{n}{k}\PYG{p}{,}\PYG{n}{a}\PYG{p}{,}\PYG{n}{b}\PYG{p}{,}\PYG{n}{c} \PYG{o}{=} \PYG{n}{params\PYGZus{}nl}
\PYG{n}{W\PYGZus{}pred\PYGZus{}nl} \PYG{o}{=} \PYG{n}{alometric\PYGZus{}model}\PYG{p}{(}\PYG{p}{(}\PYG{n}{L}\PYG{p}{,} \PYG{n}{A}\PYG{p}{,} \PYG{n}{H}\PYG{p}{)}\PYG{p}{,} \PYG{n}{k}\PYG{p}{,} \PYG{n}{a}\PYG{p}{,} \PYG{n}{b}\PYG{p}{,} \PYG{n}{c}\PYG{p}{)}
\PYG{n}{r2\PYGZus{}nl} \PYG{o}{=} \PYG{n}{r2\PYGZus{}score}\PYG{p}{(}\PYG{n}{W}\PYG{p}{,} \PYG{n}{W\PYGZus{}pred\PYGZus{}nl}\PYG{p}{)}
\PYG{n}{r2\PYGZus{}adj\PYGZus{}nl} \PYG{o}{=} \PYG{l+m+mi}{1} \PYG{o}{\PYGZhy{}} \PYG{p}{(}\PYG{l+m+mi}{1} \PYG{o}{\PYGZhy{}} \PYG{n}{r2\PYGZus{}nl}\PYG{p}{)} \PYG{o}{*} \PYG{p}{(}\PYG{n}{n} \PYG{o}{\PYGZhy{}} \PYG{l+m+mi}{1}\PYG{p}{)} \PYG{o}{/} \PYG{p}{(}\PYG{n}{n} \PYG{o}{\PYGZhy{}} \PYG{l+m+mi}{3} \PYG{o}{\PYGZhy{}} \PYG{l+m+mi}{1}\PYG{p}{)}
\PYG{n}{rmse\PYGZus{}nl} \PYG{o}{=} \PYG{n}{np}\PYG{o}{.}\PYG{n}{sqrt}\PYG{p}{(}\PYG{n}{mean\PYGZus{}squared\PYGZus{}error}\PYG{p}{(}\PYG{n}{W}\PYG{p}{,} \PYG{n}{W\PYGZus{}pred\PYGZus{}nl}\PYG{p}{)}\PYG{p}{)}
\PYG{n}{mae\PYGZus{}nl} \PYG{o}{=} \PYG{n}{mean\PYGZus{}absolute\PYGZus{}error}\PYG{p}{(}\PYG{n}{W}\PYG{p}{,} \PYG{n}{W\PYGZus{}pred\PYGZus{}nl}\PYG{p}{)}

\PYG{c+c1}{\PYGZsh{} Índices de los datos}
\PYG{n}{indices} \PYG{o}{=} \PYG{n}{np}\PYG{o}{.}\PYG{n}{arange}\PYG{p}{(}\PYG{n+nb}{len}\PYG{p}{(}\PYG{n}{W}\PYG{p}{)}\PYG{p}{)}
\PYG{n}{residuos} \PYG{o}{=} \PYG{n}{W} \PYG{o}{\PYGZhy{}} \PYG{n}{W\PYGZus{}pred\PYGZus{}nl}

\PYG{n}{fig}\PYG{p}{,} \PYG{n}{axs} \PYG{o}{=} \PYG{n}{plt}\PYG{o}{.}\PYG{n}{subplots}\PYG{p}{(}\PYG{l+m+mi}{1}\PYG{p}{,} \PYG{l+m+mi}{3}\PYG{p}{,} \PYG{n}{figsize}\PYG{o}{=}\PYG{p}{(}\PYG{l+m+mi}{18}\PYG{p}{,} \PYG{l+m+mi}{5}\PYG{p}{)}\PYG{p}{)}

\PYG{c+c1}{\PYGZsh{} === 1. Peso real vs. Peso predicho ===}
\PYG{n}{axs}\PYG{p}{[}\PYG{l+m+mi}{0}\PYG{p}{]}\PYG{o}{.}\PYG{n}{scatter}\PYG{p}{(}\PYG{n}{W}\PYG{p}{,} \PYG{n}{W\PYGZus{}pred\PYGZus{}nl}\PYG{p}{,} \PYG{n}{alpha}\PYG{o}{=}\PYG{l+m+mf}{0.7}\PYG{p}{)}
\PYG{n}{axs}\PYG{p}{[}\PYG{l+m+mi}{0}\PYG{p}{]}\PYG{o}{.}\PYG{n}{plot}\PYG{p}{(}\PYG{p}{[}\PYG{n+nb}{min}\PYG{p}{(}\PYG{n}{W}\PYG{p}{)}\PYG{p}{,} \PYG{n+nb}{max}\PYG{p}{(}\PYG{n}{W}\PYG{p}{)}\PYG{p}{]}\PYG{p}{,} \PYG{p}{[}\PYG{n+nb}{min}\PYG{p}{(}\PYG{n}{W}\PYG{p}{)}\PYG{p}{,} \PYG{n+nb}{max}\PYG{p}{(}\PYG{n}{W}\PYG{p}{)}\PYG{p}{]}\PYG{p}{,} \PYG{l+s+s1}{\PYGZsq{}}\PYG{l+s+s1}{r\PYGZhy{}\PYGZhy{}}\PYG{l+s+s1}{\PYGZsq{}}\PYG{p}{,} \PYG{n}{label}\PYG{o}{=}\PYG{l+s+s1}{\PYGZsq{}}\PYG{l+s+s1}{Línea ideal}\PYG{l+s+s1}{\PYGZsq{}}\PYG{p}{)}
\PYG{n}{axs}\PYG{p}{[}\PYG{l+m+mi}{0}\PYG{p}{]}\PYG{o}{.}\PYG{n}{set\PYGZus{}xlabel}\PYG{p}{(}\PYG{l+s+s1}{\PYGZsq{}}\PYG{l+s+s1}{Peso real (W)}\PYG{l+s+s1}{\PYGZsq{}}\PYG{p}{)}
\PYG{n}{axs}\PYG{p}{[}\PYG{l+m+mi}{0}\PYG{p}{]}\PYG{o}{.}\PYG{n}{set\PYGZus{}ylabel}\PYG{p}{(}\PYG{l+s+s1}{\PYGZsq{}}\PYG{l+s+s1}{Peso predicho (W\PYGZus{}pred\PYGZus{}nl)}\PYG{l+s+s1}{\PYGZsq{}}\PYG{p}{)}
\PYG{n}{axs}\PYG{p}{[}\PYG{l+m+mi}{0}\PYG{p}{]}\PYG{o}{.}\PYG{n}{set\PYGZus{}title}\PYG{p}{(}\PYG{l+s+sa}{f}\PYG{l+s+s1}{\PYGZsq{}}\PYG{l+s+s1}{Real vs. Predicho}\PYG{l+s+s1}{\PYGZsq{}}\PYG{p}{)}
\PYG{n}{axs}\PYG{p}{[}\PYG{l+m+mi}{0}\PYG{p}{]}\PYG{o}{.}\PYG{n}{legend}\PYG{p}{(}\PYG{p}{)}
\PYG{n}{axs}\PYG{p}{[}\PYG{l+m+mi}{0}\PYG{p}{]}\PYG{o}{.}\PYG{n}{grid}\PYG{p}{(}\PYG{k+kc}{True}\PYG{p}{)}

\PYG{c+c1}{\PYGZsh{} === 2. Comparación por índice ===}
\PYG{n}{axs}\PYG{p}{[}\PYG{l+m+mi}{1}\PYG{p}{]}\PYG{o}{.}\PYG{n}{scatter}\PYG{p}{(}\PYG{n}{indices}\PYG{p}{,} \PYG{n}{W}\PYG{p}{,} \PYG{n}{color}\PYG{o}{=}\PYG{n}{CSS4\PYGZus{}COLORS}\PYG{p}{[}\PYG{l+s+s1}{\PYGZsq{}}\PYG{l+s+s1}{lightskyblue}\PYG{l+s+s1}{\PYGZsq{}}\PYG{p}{]}\PYG{p}{,} \PYG{n}{label}\PYG{o}{=}\PYG{l+s+s1}{\PYGZsq{}}\PYG{l+s+s1}{Peso real}\PYG{l+s+s1}{\PYGZsq{}}\PYG{p}{,} \PYG{n}{alpha}\PYG{o}{=}\PYG{l+m+mf}{0.8}\PYG{p}{)}
\PYG{n}{axs}\PYG{p}{[}\PYG{l+m+mi}{1}\PYG{p}{]}\PYG{o}{.}\PYG{n}{scatter}\PYG{p}{(}\PYG{n}{indices}\PYG{p}{,} \PYG{n}{W\PYGZus{}pred\PYGZus{}nl}\PYG{p}{,} \PYG{n}{color}\PYG{o}{=}\PYG{n}{CSS4\PYGZus{}COLORS}\PYG{p}{[}\PYG{l+s+s1}{\PYGZsq{}}\PYG{l+s+s1}{darkorange}\PYG{l+s+s1}{\PYGZsq{}}\PYG{p}{]}\PYG{p}{,} \PYG{n}{label}\PYG{o}{=}\PYG{l+s+s1}{\PYGZsq{}}\PYG{l+s+s1}{Peso predicho}\PYG{l+s+s1}{\PYGZsq{}}\PYG{p}{,} \PYG{n}{alpha}\PYG{o}{=}\PYG{l+m+mf}{0.6}\PYG{p}{)}
\PYG{n}{axs}\PYG{p}{[}\PYG{l+m+mi}{1}\PYG{p}{]}\PYG{o}{.}\PYG{n}{set\PYGZus{}xlabel}\PYG{p}{(}\PYG{l+s+s1}{\PYGZsq{}}\PYG{l+s+s1}{Índice de muestra}\PYG{l+s+s1}{\PYGZsq{}}\PYG{p}{)}
\PYG{n}{axs}\PYG{p}{[}\PYG{l+m+mi}{1}\PYG{p}{]}\PYG{o}{.}\PYG{n}{set\PYGZus{}ylabel}\PYG{p}{(}\PYG{l+s+s1}{\PYGZsq{}}\PYG{l+s+s1}{Peso}\PYG{l+s+s1}{\PYGZsq{}}\PYG{p}{)}
\PYG{n}{axs}\PYG{p}{[}\PYG{l+m+mi}{1}\PYG{p}{]}\PYG{o}{.}\PYG{n}{set\PYGZus{}title}\PYG{p}{(}\PYG{l+s+s1}{\PYGZsq{}}\PYG{l+s+s1}{Peso real vs. predicho por muestra}\PYG{l+s+s1}{\PYGZsq{}}\PYG{p}{)}
\PYG{n}{axs}\PYG{p}{[}\PYG{l+m+mi}{1}\PYG{p}{]}\PYG{o}{.}\PYG{n}{legend}\PYG{p}{(}\PYG{p}{)}
\PYG{n}{axs}\PYG{p}{[}\PYG{l+m+mi}{1}\PYG{p}{]}\PYG{o}{.}\PYG{n}{grid}\PYG{p}{(}\PYG{k+kc}{True}\PYG{p}{)}

\PYG{c+c1}{\PYGZsh{} === 3. Gráfico de residuos ===}
\PYG{n}{axs}\PYG{p}{[}\PYG{l+m+mi}{2}\PYG{p}{]}\PYG{o}{.}\PYG{n}{scatter}\PYG{p}{(}\PYG{n}{W\PYGZus{}pred\PYGZus{}nl}\PYG{p}{,} \PYG{n}{residuos}\PYG{p}{,} \PYG{n}{alpha}\PYG{o}{=}\PYG{l+m+mf}{0.6}\PYG{p}{,} \PYG{n}{color}\PYG{o}{=}\PYG{n}{CSS4\PYGZus{}COLORS}\PYG{p}{[}\PYG{l+s+s1}{\PYGZsq{}}\PYG{l+s+s1}{violet}\PYG{l+s+s1}{\PYGZsq{}}\PYG{p}{]}\PYG{p}{)}
\PYG{n}{axs}\PYG{p}{[}\PYG{l+m+mi}{2}\PYG{p}{]}\PYG{o}{.}\PYG{n}{axhline}\PYG{p}{(}\PYG{l+m+mi}{0}\PYG{p}{,} \PYG{n}{color}\PYG{o}{=}\PYG{l+s+s1}{\PYGZsq{}}\PYG{l+s+s1}{red}\PYG{l+s+s1}{\PYGZsq{}}\PYG{p}{,} \PYG{n}{linestyle}\PYG{o}{=}\PYG{l+s+s1}{\PYGZsq{}}\PYG{l+s+s1}{\PYGZhy{}\PYGZhy{}}\PYG{l+s+s1}{\PYGZsq{}}\PYG{p}{)}
\PYG{n}{axs}\PYG{p}{[}\PYG{l+m+mi}{2}\PYG{p}{]}\PYG{o}{.}\PYG{n}{set\PYGZus{}xlabel}\PYG{p}{(}\PYG{l+s+s1}{\PYGZsq{}}\PYG{l+s+s1}{Peso predicho (W\PYGZus{}pred\PYGZus{}nl)}\PYG{l+s+s1}{\PYGZsq{}}\PYG{p}{)}
\PYG{n}{axs}\PYG{p}{[}\PYG{l+m+mi}{2}\PYG{p}{]}\PYG{o}{.}\PYG{n}{set\PYGZus{}ylabel}\PYG{p}{(}\PYG{l+s+s1}{\PYGZsq{}}\PYG{l+s+s1}{Residuo (W \PYGZhy{} W\PYGZus{}pred\PYGZus{}nl)}\PYG{l+s+s1}{\PYGZsq{}}\PYG{p}{)}
\PYG{n}{axs}\PYG{p}{[}\PYG{l+m+mi}{2}\PYG{p}{]}\PYG{o}{.}\PYG{n}{set\PYGZus{}title}\PYG{p}{(}\PYG{l+s+s1}{\PYGZsq{}}\PYG{l+s+s1}{Análisis de residuos}\PYG{l+s+s1}{\PYGZsq{}}\PYG{p}{)}
\PYG{n}{axs}\PYG{p}{[}\PYG{l+m+mi}{2}\PYG{p}{]}\PYG{o}{.}\PYG{n}{grid}\PYG{p}{(}\PYG{k+kc}{True}\PYG{p}{)}

\PYG{n}{plt}\PYG{o}{.}\PYG{n}{tight\PYGZus{}layout}\PYG{p}{(}\PYG{p}{)}
\PYG{n}{plt}\PYG{o}{.}\PYG{n}{show}\PYG{p}{(}\PYG{p}{)}


\PYG{c+c1}{\PYGZsh{} Resultados del ajuste}
\PYG{n+nb}{print}\PYG{p}{(}\PYG{l+s+sa}{f}\PYG{l+s+s2}{\PYGZdq{}}\PYG{l+s+s2}{Modelo ajustado: W = }\PYG{l+s+si}{\PYGZob{}}\PYG{n}{k}\PYG{l+s+si}{:}\PYG{l+s+s2}{.5f}\PYG{l+s+si}{\PYGZcb{}}\PYG{l+s+s2}{· L\PYGZca{}}\PYG{l+s+si}{\PYGZob{}}\PYG{n}{a}\PYG{l+s+si}{:}\PYG{l+s+s2}{.5f}\PYG{l+s+si}{\PYGZcb{}}\PYG{l+s+s2}{ · A\PYGZca{}}\PYG{l+s+si}{\PYGZob{}}\PYG{n}{b}\PYG{l+s+si}{:}\PYG{l+s+s2}{.5f}\PYG{l+s+si}{\PYGZcb{}}\PYG{l+s+s2}{· H\PYGZca{}}\PYG{l+s+si}{\PYGZob{}}\PYG{n}{c}\PYG{l+s+si}{:}\PYG{l+s+s2}{.5f}\PYG{l+s+si}{\PYGZcb{}}\PYG{l+s+s2}{\PYGZdq{}}\PYG{p}{)}
\PYG{n+nb}{print}\PYG{p}{(}\PYG{l+s+sa}{f}\PYG{l+s+s2}{\PYGZdq{}}\PYG{l+s+s2}{R² = }\PYG{l+s+si}{\PYGZob{}}\PYG{n}{r2\PYGZus{}nl}\PYG{l+s+si}{:}\PYG{l+s+s2}{.4f}\PYG{l+s+si}{\PYGZcb{}}\PYG{l+s+s2}{\PYGZdq{}}\PYG{p}{)}
\PYG{n+nb}{print}\PYG{p}{(}\PYG{l+s+sa}{f}\PYG{l+s+s2}{\PYGZdq{}}\PYG{l+s+s2}{MAE = }\PYG{l+s+si}{\PYGZob{}}\PYG{n}{mae\PYGZus{}nl}\PYG{l+s+si}{:}\PYG{l+s+s2}{.4f}\PYG{l+s+si}{\PYGZcb{}}\PYG{l+s+s2}{ g}\PYG{l+s+s2}{\PYGZdq{}}\PYG{p}{)}
\PYG{n+nb}{print}\PYG{p}{(}\PYG{l+s+sa}{f}\PYG{l+s+s2}{\PYGZdq{}}\PYG{l+s+s2}{RSME = }\PYG{l+s+si}{\PYGZob{}}\PYG{n}{rmse\PYGZus{}nl}\PYG{l+s+si}{:}\PYG{l+s+s2}{.4f}\PYG{l+s+si}{\PYGZcb{}}\PYG{l+s+s2}{ g}\PYG{l+s+s2}{\PYGZdq{}}\PYG{p}{)}
\PYG{n+nb}{print}\PYG{p}{(}\PYG{l+s+sa}{f}\PYG{l+s+s2}{\PYGZdq{}}\PYG{l+s+s2}{R² adj = }\PYG{l+s+si}{\PYGZob{}}\PYG{n}{r2\PYGZus{}adj\PYGZus{}nl}\PYG{l+s+si}{:}\PYG{l+s+s2}{.4f}\PYG{l+s+si}{\PYGZcb{}}\PYG{l+s+s2}{ g}\PYG{l+s+s2}{\PYGZdq{}}\PYG{p}{)}
\end{sphinxVerbatim}

\end{sphinxuseclass}\end{sphinxVerbatimInput}
\begin{sphinxVerbatimOutput}

\begin{sphinxuseclass}{cell_output}
\noindent\sphinxincludegraphics{{84f274be0dc1f5a6ba0c30afee4d8128ac1243fcdadecf8116ade04a975420d1}.png}

\begin{sphinxVerbatim}[commandchars=\\\{\}]
Modelo ajustado: W = 0.17609· L\PYGZca{}1.24336 · A\PYGZca{}1.16881· H\PYGZca{}0.49480
R² = 0.9833
MAE = 0.3237 g
RSME = 0.4681 g
R² adj = 0.9831 g
\end{sphinxVerbatim}

\end{sphinxuseclass}\end{sphinxVerbatimOutput}

\end{sphinxuseclass}

\subsection{Comparativa de modelos analizados}
\label{\detokenize{content/03/Coeficientes:comparativa-de-modelos-analizados}}
\begin{sphinxuseclass}{cell}\begin{sphinxVerbatimInput}

\begin{sphinxuseclass}{cell_input}
\begin{sphinxVerbatim}[commandchars=\\\{\}]
\PYG{n}{modelos} \PYG{o}{=} \PYG{p}{[}\PYG{l+s+s1}{\PYGZsq{}}\PYG{l+s+s1}{Longitud}\PYG{l+s+s1}{\PYGZsq{}}\PYG{p}{,} \PYG{l+s+s1}{\PYGZsq{}}\PYG{l+s+s1}{Superficie}\PYG{l+s+s1}{\PYGZsq{}}\PYG{p}{,} \PYG{l+s+s1}{\PYGZsq{}}\PYG{l+s+s1}{Longitud\PYGZhy{}Anchura}\PYG{l+s+s1}{\PYGZsq{}}\PYG{p}{,} \PYG{l+s+s1}{\PYGZsq{}}\PYG{l+s+s1}{Volumen}\PYG{l+s+s1}{\PYGZsq{}}\PYG{p}{,} \PYG{l+s+s1}{\PYGZsq{}}\PYG{l+s+s1}{Longitud\PYGZhy{}Anchura\PYGZhy{}Altura}\PYG{l+s+s1}{\PYGZsq{}}\PYG{p}{]}
\PYG{n}{r2} \PYG{o}{=} \PYG{p}{[}\PYG{n}{r2\PYGZus{}long}\PYG{p}{,} \PYG{n}{r2\PYGZus{}pow}\PYG{p}{,} \PYG{n}{r2\PYGZus{}la}\PYG{p}{,} \PYG{n}{r2\PYGZus{}vol}\PYG{p}{,} \PYG{n}{r2\PYGZus{}nl}\PYG{p}{]}
\PYG{n}{r2\PYGZus{}adj} \PYG{o}{=} \PYG{p}{[}\PYG{n}{r2\PYGZus{}adj\PYGZus{}long}\PYG{p}{,} \PYG{n}{r2\PYGZus{}adj\PYGZus{}pow}\PYG{p}{,} \PYG{n}{r2\PYGZus{}adj\PYGZus{}la}\PYG{p}{,} \PYG{n}{r2\PYGZus{}adj\PYGZus{}vol}\PYG{p}{,} \PYG{n}{r2\PYGZus{}adj\PYGZus{}nl}\PYG{p}{]}
\PYG{n}{rmse} \PYG{o}{=} \PYG{p}{[}\PYG{n}{rmse\PYGZus{}long}\PYG{p}{,} \PYG{n}{rmse\PYGZus{}pow}\PYG{p}{,} \PYG{n}{rmse\PYGZus{}la}\PYG{p}{,} \PYG{n}{rmse\PYGZus{}vol}\PYG{p}{,} \PYG{n}{rmse\PYGZus{}nl}\PYG{p}{]}
\PYG{n}{mae} \PYG{o}{=} \PYG{p}{[}\PYG{n}{mae\PYGZus{}long}\PYG{p}{,} \PYG{n}{mae\PYGZus{}pow}\PYG{p}{,} \PYG{n}{mae\PYGZus{}la}\PYG{p}{,} \PYG{n}{mae\PYGZus{}vol}\PYG{p}{,} \PYG{n}{mae\PYGZus{}nl}\PYG{p}{]}

\PYG{n}{results} \PYG{o}{=} \PYG{n}{pd}\PYG{o}{.}\PYG{n}{DataFrame}\PYG{p}{(}\PYG{p}{\PYGZob{}}
    \PYG{l+s+s1}{\PYGZsq{}}\PYG{l+s+s1}{Modelo Peso vs.}\PYG{l+s+s1}{\PYGZsq{}}\PYG{p}{:} \PYG{n}{modelos}\PYG{p}{,}
    \PYG{l+s+s1}{\PYGZsq{}}\PYG{l+s+s1}{R²}\PYG{l+s+s1}{\PYGZsq{}}\PYG{p}{:} \PYG{n}{r2}\PYG{p}{,}
    \PYG{l+s+s1}{\PYGZsq{}}\PYG{l+s+s1}{R² ajustado}\PYG{l+s+s1}{\PYGZsq{}}\PYG{p}{:} \PYG{n}{r2\PYGZus{}adj}\PYG{p}{,}
    \PYG{l+s+s1}{\PYGZsq{}}\PYG{l+s+s1}{RMSE}\PYG{l+s+s1}{\PYGZsq{}}\PYG{p}{:} \PYG{n}{rmse}\PYG{p}{,}
    \PYG{l+s+s1}{\PYGZsq{}}\PYG{l+s+s1}{MAE}\PYG{l+s+s1}{\PYGZsq{}}\PYG{p}{:} \PYG{n}{mae}
\PYG{p}{\PYGZcb{}}\PYG{p}{)}
\PYG{n}{results}
\end{sphinxVerbatim}

\end{sphinxuseclass}\end{sphinxVerbatimInput}
\begin{sphinxVerbatimOutput}

\begin{sphinxuseclass}{cell_output}
\begin{sphinxVerbatim}[commandchars=\\\{\}]
           Modelo Peso vs.        R²  R² ajustado      RMSE       MAE
0                 Longitud  0.943280     0.943006  0.863761  0.524589
1               Superficie  0.973307     0.973178  0.592544  0.373385
2         Longitud\PYGZhy{}Anchura  0.973946     0.973693  0.585408  0.373508
3                  Volumen  0.974558     0.974435  0.578497  0.424128
4  Longitud\PYGZhy{}Anchura\PYGZhy{}Altura  0.983343     0.983099  0.468080  0.323711
\end{sphinxVerbatim}

\end{sphinxuseclass}\end{sphinxVerbatimOutput}

\end{sphinxuseclass}

\subsection{Conclusiones}
\label{\detokenize{content/03/Coeficientes:conclusiones}}
\sphinxAtStartPar
El análisis comparativo de los cinco modelos lineales desarrollados para predecir el peso de alevines de \sphinxstyleemphasis{Solea solea} revela que la adición progresiva de variables morfológicas incrementa el poder explicativo desde un \(R^{2}\) de \(0,94\), obtenido con la longitud como único predictor, hasta un \(0,98\) cuando se incorporan longitud, anchura y altura. Esta mejora se traduce en una reducción del error cuadrático medio (RMSE) de \(0,86\,g\) a \(0,47\,g\) y del error absoluto medio (MAE) de \(0,52\,g\) a \(0,32\,g\). No obstante, la ganancia marginal entre los modelos basados en superficie (longitud × anchura) y los que incluyen altura o volumen apenas alcanza \(0,11–0,12\,g\) en RMSE, lo que sugiere rendimientos decrecientes al introducir la tercera dimensión. Dado que la altura exige una adquisición estereoscópica o reconstrucción volumétrica costosa y propensa a errores, mientras que longitud y anchura se obtienen con alta precisión mediante sistemas de visión artificial 2‑D, el modelo de superficie emerge como la opción óptima para estimaciones en línea de la biomasa durante el grading industrial.

\sphinxAtStartPar
Sin embargo, los coeficientes de ajuste fueron calculados sobre el conjunto completo de 209 individuos sin partición de entrenamiento y prueba ni validación cruzada, lo que introduce un sesgo optimista difícil de cuantificar. Además, la ausencia de diagnósticos de normalidad de residuos y de colinealidad entre predictores cuestiona la solidez estadística de los modelos multivariantes y aconseja la aplicación de transformaciones alométricas y técnicas de regularización antes de su despliegue. Bajo un escenario operativo en el que los límites de talla están separados varias veces el MAE (\textasciitilde{}0,37 g para el modelo de superficie), el error residual estimado parece compatible con la precisión requerida; con todo, resulta imprescindible fijar umbrales de aceptabilidad. En síntesis, la modelización basada en longitud y anchura proporciona una estimación robusta y operacionalmente viable del peso de \sphinxstyleemphasis{S. solea}, mientras que la inclusión de la altura ofrece beneficios estadísticos marginales que no compensan la complejidad práctica añadida sin una validación rigurosa que confirme su ventaja.

\sphinxstepscope


\section{Análisis del modelo alométrico longitud\sphinxhyphen{}anchura}
\label{\detokenize{content/03/Analisis:analisis-del-modelo-alometrico-longitud-anchura}}\label{\detokenize{content/03/Analisis::doc}}
\sphinxAtStartPar
Para profundizar en el modelo alométrico definido por la expresión matemática \eqref{equation:content/03/Coeficientes:eq_peso-longitud_anchura} es preciso adoptar una estrategia rigurosamente secuencial que combine diagnóstico estadístico exhaustivo, exploración de formas funcionales alternativas y validación externa. A lo largo de este capítulo analizaremos cada uno de estas fases de forma pormenorizada.


\subsection{Preanálisis exploratorio (EDA)}
\label{\detokenize{content/03/Analisis:preanalisis-exploratorio-eda}}
\sphinxAtStartPar
Un primer análisis preliminar ya se realizó en el entregable E2.1 relativo a la {\hyperref[\detokenize{content/02/Dataset::doc}]{\sphinxcrossref{\DUrole{std,std-doc}{“Obtención y etiquetado del dataset”}}}}. El análisis de la matriz de correlación indicó que, en el lenguado, el peso está fuertemente asociado con la longitud y la anchura (r≈0,94), mientras que su asociación con la altura (espesor corporal) es claramente menor (r≈0,86). En términos de varianza explicada, longitud y anchura individualmente capturan \textasciitilde{}88\% del comportamiento del peso (r²≈0,8836), frente a \textasciitilde{}74\% para la altura (r²≈0,7396). Esto es coherente con la biomecánica del pez plano: el peso depende del volumen, pero en especies deprimidas dorsoventralmente la variabilidad del área proyectada (aprox. longitud×anchura) domina sobre la variación del espesor. La correlación igualmente alta entre longitud y anchura (r≈0,94) sugiere una morfología con proporciones relativamente estables durante el crecimiento (isometría planiforme) o, al menos, una fuerte covariación de ambas dimensiones; en la práctica, esto implica multicolinealidad si se usan juntas en regresión. La correlación más baja de peso con la altura indica que el espesor es más variable entre individuos (condición corporal, estado nutricional, maduración gonadal), aportando información menos estable sobre el peso que las dimensiones planas. En conjunto, estas correlaciones significan que la masa del lenguado está principalmente determinada por su extensión superficial y que la altura añade variabilidad más idiosincrática.

\sphinxAtStartPar
Para poder comprender mejor la relación entre estas tres variables: peso \((W)\), longitud \((L)\) y anchura \((A)\) realizamos la correspondiente gráfica.

\begin{sphinxuseclass}{cell}\begin{sphinxVerbatimInput}

\begin{sphinxuseclass}{cell_input}
\begin{sphinxVerbatim}[commandchars=\\\{\}]
\PYG{k+kn}{import}\PYG{+w}{ }\PYG{n+nn}{pandas}\PYG{+w}{ }\PYG{k}{as}\PYG{+w}{ }\PYG{n+nn}{pd}
\PYG{k+kn}{import}\PYG{+w}{ }\PYG{n+nn}{numpy}\PYG{+w}{ }\PYG{k}{as}\PYG{+w}{ }\PYG{n+nn}{np}
\PYG{k+kn}{import}\PYG{+w}{ }\PYG{n+nn}{matplotlib}\PYG{n+nn}{.}\PYG{n+nn}{pyplot}\PYG{+w}{ }\PYG{k}{as}\PYG{+w}{ }\PYG{n+nn}{plt}
\PYG{k+kn}{import}\PYG{+w}{ }\PYG{n+nn}{seaborn}\PYG{+w}{ }\PYG{k}{as}\PYG{+w}{ }\PYG{n+nn}{sns}

\PYG{c+c1}{\PYGZsh{} Leer el dataset}
\PYG{n}{file\PYGZus{}path} \PYG{o}{=} \PYG{l+s+s1}{\PYGZsq{}}\PYG{l+s+s1}{.././data/Dimensiones\PYGZus{}lenguado.xlsx}\PYG{l+s+s1}{\PYGZsq{}}
\PYG{n}{df} \PYG{o}{=} \PYG{n}{pd}\PYG{o}{.}\PYG{n}{read\PYGZus{}excel}\PYG{p}{(}\PYG{n}{file\PYGZus{}path}\PYG{p}{)}

\PYG{c+c1}{\PYGZsh{} SCATTER 3D PESO–LONGITUD–ANCHURA}
\PYG{c+c1}{\PYGZsh{} \PYGZhy{}\PYGZhy{}\PYGZhy{}\PYGZhy{}\PYGZhy{}\PYGZhy{}\PYGZhy{}\PYGZhy{}\PYGZhy{}\PYGZhy{}\PYGZhy{}\PYGZhy{}\PYGZhy{}\PYGZhy{}\PYGZhy{}\PYGZhy{}\PYGZhy{}\PYGZhy{}\PYGZhy{}\PYGZhy{}\PYGZhy{}\PYGZhy{}\PYGZhy{}\PYGZhy{}\PYGZhy{}\PYGZhy{}\PYGZhy{}\PYGZhy{}\PYGZhy{}\PYGZhy{}\PYGZhy{}\PYGZhy{}}
\PYG{n}{fig} \PYG{o}{=} \PYG{n}{plt}\PYG{o}{.}\PYG{n}{figure}\PYG{p}{(}\PYG{n}{figsize}\PYG{o}{=}\PYG{p}{(}\PYG{l+m+mi}{8}\PYG{p}{,}\PYG{l+m+mi}{8}\PYG{p}{)}\PYG{p}{)}
\PYG{n}{ax} \PYG{o}{=} \PYG{n}{fig}\PYG{o}{.}\PYG{n}{add\PYGZus{}subplot}\PYG{p}{(}\PYG{l+m+mi}{111}\PYG{p}{,} \PYG{n}{projection}\PYG{o}{=}\PYG{l+s+s1}{\PYGZsq{}}\PYG{l+s+s1}{3d}\PYG{l+s+s1}{\PYGZsq{}}\PYG{p}{)}
\PYG{n}{ax}\PYG{o}{.}\PYG{n}{scatter}\PYG{p}{(}\PYG{n}{df}\PYG{p}{[}\PYG{l+s+s1}{\PYGZsq{}}\PYG{l+s+s1}{Longitud (cm)}\PYG{l+s+s1}{\PYGZsq{}}\PYG{p}{]}\PYG{p}{,}
           \PYG{n}{df}\PYG{p}{[}\PYG{l+s+s1}{\PYGZsq{}}\PYG{l+s+s1}{Anchura (cm)}\PYG{l+s+s1}{\PYGZsq{}}\PYG{p}{]}\PYG{p}{,}
           \PYG{n}{df}\PYG{p}{[}\PYG{l+s+s1}{\PYGZsq{}}\PYG{l+s+s1}{Peso (g)}\PYG{l+s+s1}{\PYGZsq{}}\PYG{p}{]}\PYG{p}{,}
           \PYG{n}{s}\PYG{o}{=}\PYG{l+m+mi}{18}\PYG{p}{,} \PYG{n}{alpha}\PYG{o}{=}\PYG{l+m+mf}{0.8}\PYG{p}{)}

\PYG{n}{ax}\PYG{o}{.}\PYG{n}{set\PYGZus{}xlabel}\PYG{p}{(}\PYG{l+s+s1}{\PYGZsq{}}\PYG{l+s+s1}{Longitud (cm)}\PYG{l+s+s1}{\PYGZsq{}}\PYG{p}{,} \PYG{n}{labelpad}\PYG{o}{=}\PYG{l+m+mi}{10}\PYG{p}{,} \PYG{n}{fontsize}\PYG{o}{=}\PYG{l+m+mi}{14}\PYG{p}{)}
\PYG{n}{ax}\PYG{o}{.}\PYG{n}{set\PYGZus{}ylabel}\PYG{p}{(}\PYG{l+s+s1}{\PYGZsq{}}\PYG{l+s+s1}{Anchura (cm)}\PYG{l+s+s1}{\PYGZsq{}}\PYG{p}{,} \PYG{n}{labelpad}\PYG{o}{=}\PYG{l+m+mi}{10}\PYG{p}{,} \PYG{n}{fontsize}\PYG{o}{=}\PYG{l+m+mi}{14}\PYG{p}{)}
\PYG{n}{ax}\PYG{o}{.}\PYG{n}{set\PYGZus{}zlabel}\PYG{p}{(}\PYG{l+s+s1}{\PYGZsq{}}\PYG{l+s+s1}{\PYGZsq{}}\PYG{p}{)} \PYG{c+c1}{\PYGZsh{} vaciamos el z\PYGZhy{}label “normal”}

\PYG{c+c1}{\PYGZsh{} Colocamos el texto como overlay 2\PYGZhy{}D para que no se recorte}
\PYG{n}{fig}\PYG{o}{.}\PYG{n}{text}\PYG{p}{(}\PYG{l+m+mf}{0.94}\PYG{p}{,} \PYG{l+m+mf}{0.52}\PYG{p}{,} \PYG{l+s+s1}{\PYGZsq{}}\PYG{l+s+s1}{Peso (g)}\PYG{l+s+s1}{\PYGZsq{}}\PYG{p}{,} \PYG{n}{rotation}\PYG{o}{=}\PYG{l+m+mi}{90}\PYG{p}{,}
         \PYG{n}{va}\PYG{o}{=}\PYG{l+s+s1}{\PYGZsq{}}\PYG{l+s+s1}{center}\PYG{l+s+s1}{\PYGZsq{}}\PYG{p}{,} \PYG{n}{ha}\PYG{o}{=}\PYG{l+s+s1}{\PYGZsq{}}\PYG{l+s+s1}{center}\PYG{l+s+s1}{\PYGZsq{}}\PYG{p}{,} \PYG{n}{fontsize}\PYG{o}{=}\PYG{l+m+mi}{14}\PYG{p}{)}

\PYG{n}{ax}\PYG{o}{.}\PYG{n}{set\PYGZus{}title}\PYG{p}{(}\PYG{l+s+s2}{\PYGZdq{}}\PYG{l+s+s2}{Relación Peso\PYGZhy{}Longitud\PYGZhy{}Anchura}\PYG{l+s+s2}{\PYGZdq{}}\PYG{p}{)}
\PYG{n}{ax}\PYG{o}{.}\PYG{n}{set\PYGZus{}box\PYGZus{}aspect}\PYG{p}{(}\PYG{p}{[}\PYG{l+m+mi}{1}\PYG{p}{,} \PYG{l+m+mi}{1}\PYG{p}{,} \PYG{l+m+mf}{0.8}\PYG{p}{]}\PYG{p}{)}
\PYG{n}{plt}\PYG{o}{.}\PYG{n}{subplots\PYGZus{}adjust}\PYG{p}{(}\PYG{n}{left}\PYG{o}{=}\PYG{l+m+mf}{0.05}\PYG{p}{,} \PYG{n}{right}\PYG{o}{=}\PYG{l+m+mf}{0.95}\PYG{p}{,} \PYG{n}{bottom}\PYG{o}{=}\PYG{l+m+mf}{0.05}\PYG{p}{,} \PYG{n}{top}\PYG{o}{=}\PYG{l+m+mf}{0.90}\PYG{p}{)}
\PYG{n}{plt}\PYG{o}{.}\PYG{n}{show}\PYG{p}{(}\PYG{p}{)}
\end{sphinxVerbatim}

\end{sphinxuseclass}\end{sphinxVerbatimInput}
\begin{sphinxVerbatimOutput}

\begin{sphinxuseclass}{cell_output}
\noindent\sphinxincludegraphics{{cbe03cbe628da960cc1fa765c89f3c5db7043e3e09d77d2bc506c1e4db884bdd}.png}

\end{sphinxuseclass}\end{sphinxVerbatimOutput}

\end{sphinxuseclass}
\sphinxAtStartPar
El diagrama de dispersión tridimensional muestra que los datos se distribuyen sobre una superficie típica de crecimiento alométrico: el peso (\(W\)) aumenta de forma supralineal con la longitud (\(L\)) y la anchura (\(A\)). Esta curvatura descarta cualquier ajuste plano clásico y avala la fórmula \eqref{equation:content/03/Coeficientes:eq_peso-longitud_anchura}; en coordenadas logarítmicas la nube se comprimiría prácticamente en un plano, lo que simplifica la estimación de \(a\), \(b\) y \(k\) mediante regresión lineal múltiple sobre \(\ln W\).

\sphinxAtStartPar
La sección horizontal del gráfico exhibe una correlación positiva marcada entre \(L\) y \(A\); sin embargo, la dispersión transversal confirma que la anchura aporta información independiente — es decir, \(b\) no resulta redundante respecto a \(a\). Además, la amplitud de la nube crece con el tamaño, evidenciando heterocedasticidad: los ejemplares grandes presentan mayor variabilidad relativa de peso. Este patrón exige, o bien la transformación log–log previa al ajuste, o bien el empleo de estimadores ponderados o robustos para preservar la validez inferencial.

\sphinxAtStartPar
En la proyección vertical se identifican varios individuos que se separan netamente de la masa principal: unos pocos muy pesados para su talla y uno o dos sorprendentemente livianos. Estos casos coinciden con los atípicos y deben revisarse individualmente —pudiendo tratarse de errores de registro, diferencias de condición corporal o etapas ontogénicas poco representadas.

\sphinxAtStartPar
\sphinxstylestrong{En conclusión, el análisis gráfico corrobora la idoneidad de un modelo alométrico en \(L\) y \(A\), subraya la necesidad de transformar logarítmicamente los datos o emplear métodos robustos y pone de relieve la importancia de manejar adecuadamente los valores atípicos para obtener estimaciones fiables de los parámetros biológicos.}


\subsection{Visualización de \sphinxstyleemphasis{outliers}}
\label{\detokenize{content/03/Analisis:visualizacion-de-outliers}}
\sphinxAtStartPar
En el análisis exploratorio se ha comprobado la existencia de \sphinxstyleemphasis{outliers} y su visualización es fundamental para comprender su impacto en la construcción del modelo predictivo. Estos valores atípicos pueden surgir por errores en la medición, variabilidad natural en la población o condiciones excepcionales, y su presencia puede distorsionar los parámetros estadísticos y sesgar el modelo de regresión. Al representar gráficamente los \sphinxstyleemphasis{outliers} se facilita la identificación de observaciones aberrantes que podrían afectar la linealidad, homocedasticidad o normalidad de los residuos. Además, su visualización permite evaluar si deben ser tratados mediante estrategias como winsorización, eliminación o imputación robusta, optimizando así la generalización del modelo.

\sphinxAtStartPar
Para contabilizar y visualizar estos \sphinxstyleemphasis{outliers} vamos a emplear primeramente un enfoque univariado (detección en una dimensión) para después aplicar un enfoque multivariado que permite la detección en múltiples variables.


\subsubsection{Métodos univariados (detección en una dimensión)}
\label{\detokenize{content/03/Analisis:metodos-univariados-deteccion-en-una-dimension}}\begin{itemize}
\item {}
\sphinxAtStartPar
\sphinxstylestrong{Rango intercuartílico (IQR)} se define como la amplitud \(Q_{3}-Q_{1}\), donde \(Q_{1}\) y \(Q_{3}\) son, respectivamente, los percentiles 25\% y 75\% de la distribución ordenada de una variable aleatoria continua. Constituye una medida robusta de dispersión, insensible a colas pesadas o valores aberrantes, pues encapsula el 50\% central de las observaciones. En procedimientos de depuración de datos, el criterio aplicable considera atípica toda observación \(x\) que verifique \(x < Q_{1}-k\cdot \text{IQR}\) o \(x > Q_{3}+k\cdot \text{IQR}\), con \(k=1.5\) como valor convencional. Este umbral utiliza el IQR como escala de variabilidad intrínseca y permite identificar \sphinxstyleemphasis{outliers} sin asumir normalidad ni recurrir a momentos de orden superior, favoreciendo diagnósticos estadísticos robustos en muestras heterogéneas.

\item {}
\sphinxAtStartPar
\sphinxstylestrong{El Z\sphinxhyphen{}score}, o puntuación estándar, es una métrica estadística que cuantifica cuántas desviaciones estándar se aleja una observación del valor medio de la distribución, expresado como \(Z=\dfrac{X−μ}{σ}\), donde \(X\) es el valor observado, \(μ\) la media poblacional y \(σ\) la desviación estándar. Este método es particularmente útil para identificar \sphinxstyleemphasis{outliers} en distribuciones normales o aproximadamente normales, considerándose valores atípicos aquellos con \(∣Z∣>3\) (umbral que abarca el 99.7\% de los datos bajo la curva gaussiana). Sin embargo, su eficacia disminuye en distribuciones sesgadas o con colas pesadas, ya que la media y la desviación estándar son sensibles a valores extremos.

\end{itemize}

\begin{sphinxuseclass}{cell}\begin{sphinxVerbatimInput}

\begin{sphinxuseclass}{cell_input}
\begin{sphinxVerbatim}[commandchars=\\\{\}]
\PYG{c+c1}{\PYGZsh{} \PYGZhy{}\PYGZhy{}\PYGZhy{}\PYGZhy{}\PYGZhy{}\PYGZhy{}\PYGZhy{}\PYGZhy{}\PYGZhy{}\PYGZhy{}\PYGZhy{}\PYGZhy{}\PYGZhy{}\PYGZhy{}\PYGZhy{}\PYGZhy{}\PYGZhy{}\PYGZhy{}\PYGZhy{}\PYGZhy{}\PYGZhy{}\PYGZhy{}\PYGZhy{}\PYGZhy{}\PYGZhy{}\PYGZhy{}\PYGZhy{}\PYGZhy{}\PYGZhy{}\PYGZhy{}\PYGZhy{}\PYGZhy{}\PYGZhy{}\PYGZhy{}\PYGZhy{}\PYGZhy{}\PYGZhy{}\PYGZhy{}\PYGZhy{}\PYGZhy{}\PYGZhy{}\PYGZhy{}\PYGZhy{}\PYGZhy{}\PYGZhy{}\PYGZhy{}\PYGZhy{}\PYGZhy{}\PYGZhy{}}
\PYG{c+c1}{\PYGZsh{}    METODOS UNIVARIADOS}
\PYG{c+c1}{\PYGZsh{} \PYGZhy{}\PYGZhy{}\PYGZhy{}\PYGZhy{}\PYGZhy{}\PYGZhy{}\PYGZhy{}\PYGZhy{}\PYGZhy{}\PYGZhy{}\PYGZhy{}\PYGZhy{}\PYGZhy{}\PYGZhy{}\PYGZhy{}\PYGZhy{}\PYGZhy{}\PYGZhy{}\PYGZhy{}\PYGZhy{}\PYGZhy{}\PYGZhy{}\PYGZhy{}\PYGZhy{}\PYGZhy{}\PYGZhy{}\PYGZhy{}\PYGZhy{}\PYGZhy{}\PYGZhy{}\PYGZhy{}\PYGZhy{}\PYGZhy{}\PYGZhy{}\PYGZhy{}\PYGZhy{}\PYGZhy{}\PYGZhy{}\PYGZhy{}\PYGZhy{}\PYGZhy{}\PYGZhy{}\PYGZhy{}\PYGZhy{}\PYGZhy{}\PYGZhy{}\PYGZhy{}\PYGZhy{}\PYGZhy{}}
\PYG{k}{def}\PYG{+w}{ }\PYG{n+nf}{iqr\PYGZus{}mask}\PYG{p}{(}\PYG{n}{series}\PYG{p}{,} \PYG{n}{k}\PYG{o}{=}\PYG{l+m+mf}{1.5}\PYG{p}{)}\PYG{p}{:}
    \PYG{n}{q1}\PYG{p}{,} \PYG{n}{q3} \PYG{o}{=} \PYG{n}{series}\PYG{o}{.}\PYG{n}{quantile}\PYG{p}{(}\PYG{p}{[}\PYG{l+m+mf}{0.25}\PYG{p}{,} \PYG{l+m+mf}{0.75}\PYG{p}{]}\PYG{p}{)}
    \PYG{n}{iqr} \PYG{o}{=} \PYG{n}{q3} \PYG{o}{\PYGZhy{}} \PYG{n}{q1}
    \PYG{k}{return} \PYG{p}{(}\PYG{n}{series} \PYG{o}{\PYGZlt{}} \PYG{n}{q1} \PYG{o}{\PYGZhy{}} \PYG{n}{k}\PYG{o}{*}\PYG{n}{iqr}\PYG{p}{)} \PYG{o}{|} \PYG{p}{(}\PYG{n}{series} \PYG{o}{\PYGZgt{}} \PYG{n}{q3} \PYG{o}{+} \PYG{n}{k}\PYG{o}{*}\PYG{n}{iqr}\PYG{p}{)}

\PYG{c+c1}{\PYGZsh{} ********************************}
\PYG{c+c1}{\PYGZsh{}    IQR sobre cada variable}
\PYG{c+c1}{\PYGZsh{} ********************************}
\PYG{n}{mask\PYGZus{}iqr} \PYG{o}{=} \PYG{p}{(}
    \PYG{n}{iqr\PYGZus{}mask}\PYG{p}{(}\PYG{n}{df}\PYG{p}{[}\PYG{l+s+s1}{\PYGZsq{}}\PYG{l+s+s1}{Peso (g)}\PYG{l+s+s1}{\PYGZsq{}}\PYG{p}{]}\PYG{p}{)} \PYG{o}{|}
    \PYG{n}{iqr\PYGZus{}mask}\PYG{p}{(}\PYG{n}{df}\PYG{p}{[}\PYG{l+s+s1}{\PYGZsq{}}\PYG{l+s+s1}{Longitud (cm)}\PYG{l+s+s1}{\PYGZsq{}}\PYG{p}{]}\PYG{p}{)} \PYG{o}{|}
    \PYG{n}{iqr\PYGZus{}mask}\PYG{p}{(}\PYG{n}{df}\PYG{p}{[}\PYG{l+s+s1}{\PYGZsq{}}\PYG{l+s+s1}{Anchura (cm)}\PYG{l+s+s1}{\PYGZsq{}}\PYG{p}{]}\PYG{p}{)}
\PYG{p}{)}

\PYG{c+c1}{\PYGZsh{} Generamos una copia del data frame original}
\PYG{n}{df\PYGZus{}outliers} \PYG{o}{=} \PYG{n}{df}\PYG{o}{.}\PYG{n}{copy}\PYG{p}{(}\PYG{p}{)}

\PYG{c+c1}{\PYGZsh{} Añadimos los outliers detectados}
\PYG{n}{df\PYGZus{}outliers}\PYG{p}{[}\PYG{l+s+s1}{\PYGZsq{}}\PYG{l+s+s1}{out\PYGZus{}IQR}\PYG{l+s+s1}{\PYGZsq{}}\PYG{p}{]} \PYG{o}{=} \PYG{n}{mask\PYGZus{}iqr}


\PYG{c+c1}{\PYGZsh{} Filtrar outliers y añadir columnas de diagnóstico}
\PYG{n}{outliers\PYGZus{}iqr} \PYG{o}{=} \PYG{n}{df\PYGZus{}outliers}\PYG{p}{[}\PYG{n}{mask\PYGZus{}iqr}\PYG{p}{]}\PYG{o}{.}\PYG{n}{copy}\PYG{p}{(}\PYG{p}{)}
\PYG{n}{outliers\PYGZus{}iqr}\PYG{p}{[}\PYG{l+s+s1}{\PYGZsq{}}\PYG{l+s+s1}{Razón\PYGZus{}outlier}\PYG{l+s+s1}{\PYGZsq{}}\PYG{p}{]} \PYG{o}{=} \PYG{l+s+s2}{\PYGZdq{}}\PYG{l+s+s2}{\PYGZdq{}}

\PYG{c+c1}{\PYGZsh{} Identificar qué variables contribuyeron}

\PYG{k}{if} \PYG{l+s+s1}{\PYGZsq{}}\PYG{l+s+s1}{Longitud (cm)}\PYG{l+s+s1}{\PYGZsq{}} \PYG{o+ow}{in} \PYG{n}{df}\PYG{o}{.}\PYG{n}{columns}\PYG{p}{:}
    \PYG{n}{outliers\PYGZus{}iqr}\PYG{o}{.}\PYG{n}{loc}\PYG{p}{[}\PYG{n}{iqr\PYGZus{}mask}\PYG{p}{(}\PYG{n}{df}\PYG{p}{[}\PYG{l+s+s1}{\PYGZsq{}}\PYG{l+s+s1}{Longitud (cm)}\PYG{l+s+s1}{\PYGZsq{}}\PYG{p}{]}\PYG{p}{)}\PYG{p}{[}\PYG{n}{mask\PYGZus{}iqr}\PYG{p}{]}\PYG{p}{,} \PYG{l+s+s1}{\PYGZsq{}}\PYG{l+s+s1}{Razón\PYGZus{}outlier}\PYG{l+s+s1}{\PYGZsq{}}\PYG{p}{]} \PYG{o}{+}\PYG{o}{=} \PYG{l+s+s2}{\PYGZdq{}}\PYG{l+s+s2}{Longitud; }\PYG{l+s+s2}{\PYGZdq{}}
\PYG{k}{if} \PYG{l+s+s1}{\PYGZsq{}}\PYG{l+s+s1}{Anchura (cm)}\PYG{l+s+s1}{\PYGZsq{}} \PYG{o+ow}{in} \PYG{n}{df}\PYG{o}{.}\PYG{n}{columns}\PYG{p}{:}
    \PYG{n}{outliers\PYGZus{}iqr}\PYG{o}{.}\PYG{n}{loc}\PYG{p}{[}\PYG{n}{iqr\PYGZus{}mask}\PYG{p}{(}\PYG{n}{df}\PYG{p}{[}\PYG{l+s+s1}{\PYGZsq{}}\PYG{l+s+s1}{Anchura (cm)}\PYG{l+s+s1}{\PYGZsq{}}\PYG{p}{]}\PYG{p}{)}\PYG{p}{[}\PYG{n}{mask\PYGZus{}iqr}\PYG{p}{]}\PYG{p}{,} \PYG{l+s+s1}{\PYGZsq{}}\PYG{l+s+s1}{Razón\PYGZus{}outlier}\PYG{l+s+s1}{\PYGZsq{}}\PYG{p}{]} \PYG{o}{+}\PYG{o}{=} \PYG{l+s+s2}{\PYGZdq{}}\PYG{l+s+s2}{Anchura; }\PYG{l+s+s2}{\PYGZdq{}}
\PYG{k}{if} \PYG{l+s+s1}{\PYGZsq{}}\PYG{l+s+s1}{Peso (g)}\PYG{l+s+s1}{\PYGZsq{}} \PYG{o+ow}{in} \PYG{n}{df}\PYG{o}{.}\PYG{n}{columns}\PYG{p}{:}
    \PYG{n}{outliers\PYGZus{}iqr}\PYG{o}{.}\PYG{n}{loc}\PYG{p}{[}\PYG{n}{iqr\PYGZus{}mask}\PYG{p}{(}\PYG{n}{df}\PYG{p}{[}\PYG{l+s+s1}{\PYGZsq{}}\PYG{l+s+s1}{Peso (g)}\PYG{l+s+s1}{\PYGZsq{}}\PYG{p}{]}\PYG{p}{)}\PYG{p}{[}\PYG{n}{mask\PYGZus{}iqr}\PYG{p}{]}\PYG{p}{,} \PYG{l+s+s1}{\PYGZsq{}}\PYG{l+s+s1}{Razón\PYGZus{}outlier}\PYG{l+s+s1}{\PYGZsq{}}\PYG{p}{]} \PYG{o}{+}\PYG{o}{=} \PYG{l+s+s2}{\PYGZdq{}}\PYG{l+s+s2}{Peso; }\PYG{l+s+s2}{\PYGZdq{}}

\PYG{c+c1}{\PYGZsh{} Mostrar resultado}
\PYG{n+nb}{print}\PYG{p}{(}\PYG{l+s+sa}{f}\PYG{l+s+s2}{\PYGZdq{}}\PYG{l+s+s2}{IQR: Número de outliers detectados IQR: }\PYG{l+s+si}{\PYGZob{}}\PYG{n+nb}{len}\PYG{p}{(}\PYG{n}{outliers\PYGZus{}iqr}\PYG{p}{)}\PYG{l+s+si}{\PYGZcb{}}\PYG{l+s+se}{\PYGZbs{}n}\PYG{l+s+s2}{\PYGZdq{}}\PYG{p}{)}
\PYG{n}{outliers\PYGZus{}iqr}\PYG{o}{.}\PYG{n}{head}\PYG{p}{(}\PYG{n+nb}{len}\PYG{p}{(}\PYG{n}{outliers\PYGZus{}iqr}\PYG{p}{)}\PYG{p}{)}
\end{sphinxVerbatim}

\end{sphinxuseclass}\end{sphinxVerbatimInput}
\begin{sphinxVerbatimOutput}

\begin{sphinxuseclass}{cell_output}
\begin{sphinxVerbatim}[commandchars=\\\{\}]
IQR: Número de outliers detectados IQR: 11
\end{sphinxVerbatim}

\begin{sphinxVerbatim}[commandchars=\\\{\}]
     Peso (g)  Longitud (cm)  Anchura (cm)  Altura (cm)  out\PYGZus{}IQR  \PYGZbs{}
194     13.45            9.7           4.0          0.6     True
197     13.56            9.7           4.1          0.6     True
200     14.13           10.1           4.1          0.7     True
201     14.04           10.5           4.1          0.7     True
202     14.67           10.7           4.2          0.7     True
203     15.65           10.7           4.5          0.6     True
204     14.50           10.8           4.5          0.6     True
205     19.88           11.4           4.4          0.9     True
206     16.47           11.0           4.6          0.7     True
207     17.04           10.6           4.8          0.7     True
208     21.98           10.6           5.2          0.8     True

       Razón\PYGZus{}outlier
194           Peso;
197           Peso;
200           Peso;
201           Peso;
202           Peso;
203           Peso;
204           Peso;
205           Peso;
206           Peso;
207           Peso;
208  Anchura; Peso;
\end{sphinxVerbatim}

\end{sphinxuseclass}\end{sphinxVerbatimOutput}

\end{sphinxuseclass}
\begin{sphinxuseclass}{cell}\begin{sphinxVerbatimInput}

\begin{sphinxuseclass}{cell_input}
\begin{sphinxVerbatim}[commandchars=\\\{\}]
\PYG{c+c1}{\PYGZsh{} ********************************}
\PYG{c+c1}{\PYGZsh{}    Z\PYGZhy{}SCORE sobre cada variable}
\PYG{c+c1}{\PYGZsh{} ********************************}

\PYG{k+kn}{from}\PYG{+w}{ }\PYG{n+nn}{scipy}\PYG{n+nn}{.}\PYG{n+nn}{stats}\PYG{+w}{ }\PYG{k+kn}{import} \PYG{n}{zscore}

\PYG{k}{def}\PYG{+w}{ }\PYG{n+nf}{zscore\PYGZus{}mask}\PYG{p}{(}\PYG{n}{series}\PYG{p}{,} \PYG{n}{threshold}\PYG{o}{=}\PYG{l+m+mi}{3}\PYG{p}{)}\PYG{p}{:}
    \PYG{n}{z\PYGZus{}scores} \PYG{o}{=} \PYG{n}{zscore}\PYG{p}{(}\PYG{n}{series}\PYG{p}{)}
    \PYG{k}{return} \PYG{n+nb}{abs}\PYG{p}{(}\PYG{n}{z\PYGZus{}scores}\PYG{p}{)} \PYG{o}{\PYGZgt{}} \PYG{n}{threshold}

\PYG{n}{mask\PYGZus{}zscore} \PYG{o}{=} \PYG{p}{(}
    \PYG{n}{zscore\PYGZus{}mask}\PYG{p}{(}\PYG{n}{df}\PYG{p}{[}\PYG{l+s+s1}{\PYGZsq{}}\PYG{l+s+s1}{Peso (g)}\PYG{l+s+s1}{\PYGZsq{}}\PYG{p}{]}\PYG{p}{)} \PYG{o}{|}
    \PYG{n}{zscore\PYGZus{}mask}\PYG{p}{(}\PYG{n}{df}\PYG{p}{[}\PYG{l+s+s1}{\PYGZsq{}}\PYG{l+s+s1}{Longitud (cm)}\PYG{l+s+s1}{\PYGZsq{}}\PYG{p}{]}\PYG{p}{)} \PYG{o}{|}
    \PYG{n}{zscore\PYGZus{}mask}\PYG{p}{(}\PYG{n}{df}\PYG{p}{[}\PYG{l+s+s1}{\PYGZsq{}}\PYG{l+s+s1}{Anchura (cm)}\PYG{l+s+s1}{\PYGZsq{}}\PYG{p}{]}\PYG{p}{)}
\PYG{p}{)}
\PYG{c+c1}{\PYGZsh{} Introducimos los outliers en el dataframe correspondiente}
\PYG{n}{df\PYGZus{}outliers}\PYG{p}{[}\PYG{l+s+s2}{\PYGZdq{}}\PYG{l+s+s2}{out\PYGZus{}ZScore}\PYG{l+s+s2}{\PYGZdq{}}\PYG{p}{]} \PYG{o}{=} \PYG{n}{mask\PYGZus{}zscore}

\PYG{c+c1}{\PYGZsh{} Filtrar outliers y añadir columnas de diagnóstico}
\PYG{n}{outliers\PYGZus{}zscore} \PYG{o}{=} \PYG{n}{df\PYGZus{}outliers}\PYG{p}{[}\PYG{n}{mask\PYGZus{}zscore}\PYG{p}{]}\PYG{o}{.}\PYG{n}{copy}\PYG{p}{(}\PYG{p}{)}
\PYG{n}{outliers\PYGZus{}zscore}\PYG{p}{[}\PYG{l+s+s1}{\PYGZsq{}}\PYG{l+s+s1}{Razón\PYGZus{}outlier}\PYG{l+s+s1}{\PYGZsq{}}\PYG{p}{]} \PYG{o}{=} \PYG{l+s+s2}{\PYGZdq{}}\PYG{l+s+s2}{\PYGZdq{}}

\PYG{c+c1}{\PYGZsh{} Añadir Z\PYGZhy{}scores para cada variable}
\PYG{k}{for} \PYG{n}{col} \PYG{o+ow}{in} \PYG{p}{[}\PYG{l+s+s1}{\PYGZsq{}}\PYG{l+s+s1}{Peso (g)}\PYG{l+s+s1}{\PYGZsq{}}\PYG{p}{,} \PYG{l+s+s1}{\PYGZsq{}}\PYG{l+s+s1}{Longitud (cm)}\PYG{l+s+s1}{\PYGZsq{}}\PYG{p}{,} \PYG{l+s+s1}{\PYGZsq{}}\PYG{l+s+s1}{Anchura (cm)}\PYG{l+s+s1}{\PYGZsq{}}\PYG{p}{]}\PYG{p}{:}
    \PYG{k}{if} \PYG{n}{col} \PYG{o+ow}{in} \PYG{n}{df}\PYG{o}{.}\PYG{n}{columns}\PYG{p}{:}
        \PYG{n}{outliers\PYGZus{}zscore}\PYG{p}{[}\PYG{l+s+sa}{f}\PYG{l+s+s1}{\PYGZsq{}}\PYG{l+s+s1}{Z\PYGZus{}}\PYG{l+s+si}{\PYGZob{}}\PYG{n}{col}\PYG{l+s+si}{\PYGZcb{}}\PYG{l+s+s1}{\PYGZsq{}}\PYG{p}{]} \PYG{o}{=} \PYG{n}{zscore}\PYG{p}{(}\PYG{n}{df}\PYG{p}{[}\PYG{n}{col}\PYG{p}{]}\PYG{p}{)}\PYG{p}{[}\PYG{n}{mask\PYGZus{}zscore}\PYG{p}{]}

\PYG{c+c1}{\PYGZsh{} Identificar qué variables contribuyeron}
\PYG{k}{if} \PYG{l+s+s1}{\PYGZsq{}}\PYG{l+s+s1}{Longitud (cm)}\PYG{l+s+s1}{\PYGZsq{}} \PYG{o+ow}{in} \PYG{n}{df}\PYG{o}{.}\PYG{n}{columns}\PYG{p}{:}
    \PYG{n}{outliers\PYGZus{}zscore}\PYG{o}{.}\PYG{n}{loc}\PYG{p}{[}\PYG{n}{zscore\PYGZus{}mask}\PYG{p}{(}\PYG{n}{df}\PYG{p}{[}\PYG{l+s+s1}{\PYGZsq{}}\PYG{l+s+s1}{Longitud (cm)}\PYG{l+s+s1}{\PYGZsq{}}\PYG{p}{]}\PYG{p}{)}\PYG{p}{[}\PYG{n}{mask\PYGZus{}zscore}\PYG{p}{]}\PYG{p}{,} \PYG{l+s+s1}{\PYGZsq{}}\PYG{l+s+s1}{Razón\PYGZus{}outlier}\PYG{l+s+s1}{\PYGZsq{}}\PYG{p}{]} \PYG{o}{+}\PYG{o}{=} \PYG{l+s+s2}{\PYGZdq{}}\PYG{l+s+s2}{Longitud; }\PYG{l+s+s2}{\PYGZdq{}}
\PYG{k}{if} \PYG{l+s+s1}{\PYGZsq{}}\PYG{l+s+s1}{Anchura (cm)}\PYG{l+s+s1}{\PYGZsq{}} \PYG{o+ow}{in} \PYG{n}{df}\PYG{o}{.}\PYG{n}{columns}\PYG{p}{:}
    \PYG{n}{outliers\PYGZus{}zscore}\PYG{o}{.}\PYG{n}{loc}\PYG{p}{[}\PYG{n}{zscore\PYGZus{}mask}\PYG{p}{(}\PYG{n}{df}\PYG{p}{[}\PYG{l+s+s1}{\PYGZsq{}}\PYG{l+s+s1}{Anchura (cm)}\PYG{l+s+s1}{\PYGZsq{}}\PYG{p}{]}\PYG{p}{)}\PYG{p}{[}\PYG{n}{mask\PYGZus{}zscore}\PYG{p}{]}\PYG{p}{,} \PYG{l+s+s1}{\PYGZsq{}}\PYG{l+s+s1}{Razón\PYGZus{}outlier}\PYG{l+s+s1}{\PYGZsq{}}\PYG{p}{]} \PYG{o}{+}\PYG{o}{=} \PYG{l+s+s2}{\PYGZdq{}}\PYG{l+s+s2}{Anchura; }\PYG{l+s+s2}{\PYGZdq{}}
\PYG{k}{if} \PYG{l+s+s1}{\PYGZsq{}}\PYG{l+s+s1}{Peso (g)}\PYG{l+s+s1}{\PYGZsq{}} \PYG{o+ow}{in} \PYG{n}{df}\PYG{o}{.}\PYG{n}{columns}\PYG{p}{:}
    \PYG{n}{outliers\PYGZus{}zscore}\PYG{o}{.}\PYG{n}{loc}\PYG{p}{[}\PYG{n}{zscore\PYGZus{}mask}\PYG{p}{(}\PYG{n}{df}\PYG{p}{[}\PYG{l+s+s1}{\PYGZsq{}}\PYG{l+s+s1}{Peso (g)}\PYG{l+s+s1}{\PYGZsq{}}\PYG{p}{]}\PYG{p}{)}\PYG{p}{[}\PYG{n}{mask\PYGZus{}zscore}\PYG{p}{]}\PYG{p}{,} \PYG{l+s+s1}{\PYGZsq{}}\PYG{l+s+s1}{Razón\PYGZus{}outlier}\PYG{l+s+s1}{\PYGZsq{}}\PYG{p}{]} \PYG{o}{+}\PYG{o}{=} \PYG{l+s+s2}{\PYGZdq{}}\PYG{l+s+s2}{Peso; }\PYG{l+s+s2}{\PYGZdq{}}

\PYG{c+c1}{\PYGZsh{} Mostrar resultado}
\PYG{n+nb}{print}\PYG{p}{(}\PYG{l+s+sa}{f}\PYG{l+s+s2}{\PYGZdq{}}\PYG{l+s+s2}{Z\PYGZhy{}Score: Número de outliers detectados (|Z| \PYGZgt{} 3): }\PYG{l+s+si}{\PYGZob{}}\PYG{n+nb}{len}\PYG{p}{(}\PYG{n}{outliers\PYGZus{}zscore}\PYG{p}{)}\PYG{l+s+si}{\PYGZcb{}}\PYG{l+s+se}{\PYGZbs{}n}\PYG{l+s+s2}{\PYGZdq{}}\PYG{p}{)}
\PYG{n}{display}\PYG{p}{(}\PYG{n}{outliers\PYGZus{}zscore}\PYG{o}{.}\PYG{n}{style}\PYG{o}{.}\PYG{n}{format}\PYG{p}{(}\PYG{p}{\PYGZob{}}\PYG{l+s+sa}{f}\PYG{l+s+s1}{\PYGZsq{}}\PYG{l+s+s1}{Z\PYGZus{}}\PYG{l+s+si}{\PYGZob{}}\PYG{n}{col}\PYG{l+s+si}{\PYGZcb{}}\PYG{l+s+s1}{\PYGZsq{}}\PYG{p}{:} \PYG{l+s+s2}{\PYGZdq{}}\PYG{l+s+si}{\PYGZob{}:.2f\PYGZcb{}}\PYG{l+s+s2}{\PYGZdq{}} \PYG{k}{for} \PYG{n}{col} \PYG{o+ow}{in} \PYG{p}{[}\PYG{l+s+s1}{\PYGZsq{}}\PYG{l+s+s1}{Peso (g)}\PYG{l+s+s1}{\PYGZsq{}}\PYG{p}{,} \PYG{l+s+s1}{\PYGZsq{}}\PYG{l+s+s1}{Longitud (cm)}\PYG{l+s+s1}{\PYGZsq{}}\PYG{p}{,} \PYG{l+s+s1}{\PYGZsq{}}\PYG{l+s+s1}{Anchura (cm)}\PYG{l+s+s1}{\PYGZsq{}}\PYG{p}{]} \PYG{k}{if} \PYG{n}{col} \PYG{o+ow}{in} \PYG{n}{df}\PYG{o}{.}\PYG{n}{columns}\PYG{p}{\PYGZcb{}}\PYG{p}{)}\PYG{p}{)}
\end{sphinxVerbatim}

\end{sphinxuseclass}\end{sphinxVerbatimInput}
\begin{sphinxVerbatimOutput}

\begin{sphinxuseclass}{cell_output}
\begin{sphinxVerbatim}[commandchars=\\\{\}]
Z\PYGZhy{}Score: Número de outliers detectados (|Z| \PYGZgt{} 3): 4
\end{sphinxVerbatim}

\begin{sphinxVerbatim}[commandchars=\\\{\}]
\PYGZlt{}pandas.io.formats.style.Styler at 0x246bc8a4980\PYGZgt{}
\end{sphinxVerbatim}

\end{sphinxuseclass}\end{sphinxVerbatimOutput}

\end{sphinxuseclass}

\subsubsection{Métodos multivariados}
\label{\detokenize{content/03/Analisis:metodos-multivariados}}
\sphinxAtStartPar
\sphinxstylestrong{Isolation Forest} es un algoritmo de detección de anomalías no supervisado basado en el principio de particiones aleatorias. A partir de un conjunto de \(n\) observaciones \(\{\mathbf{x}{i}\}{i=1}^{n}\subset\mathbb{R}^{p}\), se construye un bosque de \(t\) árboles binarios denominados \sphinxstyleemphasis{iTrees}; cada \sphinxstyleemphasis{iTree} se genera seleccionando recursivamente, de forma aleatoria, (i) una característica \(j\in\{1,\dots,p\}\) y (ii) un valor de corte \(\theta\) comprendido entre los valores mínimo y máximo de la característica en el subconjunto actual. El proceso continúa hasta que la partición contiene un único punto o alcanza una profundidad máxima \(h_{\max}\). La idea fundamental es que las observaciones anómalas, al hallarse escasamente representadas y en regiones periféricas del espacio de características, tienden a quedar aisladas tras un número bajo de divisiones. Para cada muestra \(\mathbf{x}\) se define una profundidad promedio \(E[h(\mathbf{x})]\) —promediada sobre los \(t\) árboles— y de ella se deduce la puntuación de anomalía
\begin{equation}\label{equation:content/03/Analisis:eq_iforest_1}
\begin{split}s(\mathbf{x}) = 2^{-\frac{E[h(\mathbf{x})]}{c(n)}},
\qquad
c(n) = H_{n-1}-\frac{2(n-1)}{n},\end{split}
\end{equation}
\sphinxAtStartPar
donde \(H_{n}\) es el \(n\)\sphinxhyphen{}ésimo número armónico. Los valores de \(s(\mathbf{x})\) se normalizan en \([0,1]\); puntuaciones próximas a 1 indican observaciones susceptibles de ser outliers. El esquema evita suposiciones paramétricas sobre la forma de la distribución, es eficiente en alta dimensionalidad \((O(t\,n\log n))\) y permite fijar \sphinxstyleemphasis{a priori} la proporción esperada de anomalías mediante el parámetro \sphinxcode{\sphinxupquote{contamination}}, lo que lo hace idóneo para depuración robusta de conjuntos de datos complejos.

\sphinxAtStartPar
Para abordar este algoritmo eliminamos de nuestro \sphinxstyleemphasis{dataset} la variable \sphinxcode{\sphinxupquote{Altura}} porque como ya concluimos anteriormente su exclusión no produce pérmida fundamental de información y sí podria afectar a los resultados. Previamente a la aplicacion del algoritmo es necesario normalizar (estandarizar) las columnas seleccionadas para garantizar la efectividad del algoritmo en la detección de anomalías. Al estandarizar las variables mediante la transformación \(\dfrac{X−μ}{σ}\), se homogenizan las escalas de todas las características, evitando que aquellas con magnitudes mayores dominen artificialmente el cálculo de anomalías. Este paso es esencial porque \sphinxstyleemphasis{Isolation Forest}, al basarse en particiones aleatorias del espacio de características, puede generar sesgos en conjuntos de datos con variables en escalas heterogéneas. La estandarización no solo mejora la sensibilidad del modelo a patrones multivariados de \sphinxstyleemphasis{outliers}, sino que también facilita la interpretación de los \sphinxstyleemphasis{scores} de anomalía al operar en un espacio métrico coherente. Adicionalmente, al fijar la media en 0 y la varianza en 1, se optimiza el rendimiento computacional del algoritmo, ya que las distancias utilizadas en las particiones aleatorias de los árboles reflejan contribuciones equilibradas de todas las variables.

\begin{sphinxuseclass}{cell}\begin{sphinxVerbatimInput}

\begin{sphinxuseclass}{cell_input}
\begin{sphinxVerbatim}[commandchars=\\\{\}]
\PYG{c+c1}{\PYGZsh{} ********************************}
\PYG{c+c1}{\PYGZsh{}  ISOLATION FOREST MULTIVARIABLE}
\PYG{c+c1}{\PYGZsh{} ********************************}

\PYG{k+kn}{from}\PYG{+w}{ }\PYG{n+nn}{sklearn}\PYG{n+nn}{.}\PYG{n+nn}{ensemble}\PYG{+w}{ }\PYG{k+kn}{import} \PYG{n}{IsolationForest}
\PYG{k+kn}{from}\PYG{+w}{ }\PYG{n+nn}{sklearn}\PYG{n+nn}{.}\PYG{n+nn}{preprocessing}\PYG{+w}{ }\PYG{k+kn}{import} \PYG{n}{StandardScaler}

\PYG{n}{iso} \PYG{o}{=} \PYG{n}{IsolationForest}\PYG{p}{(}\PYG{n}{contamination}\PYG{o}{=}\PYG{l+s+s1}{\PYGZsq{}}\PYG{l+s+s1}{auto}\PYG{l+s+s1}{\PYGZsq{}}\PYG{p}{,} \PYG{n}{random\PYGZus{}state}\PYG{o}{=}\PYG{l+m+mi}{42}\PYG{p}{)}

\PYG{c+c1}{\PYGZsh{} Eliminamos la columna altura del dataset}
\PYG{n}{iso\PYGZus{}cols} \PYG{o}{=} \PYG{p}{[}\PYG{l+s+s1}{\PYGZsq{}}\PYG{l+s+s1}{Peso (g)}\PYG{l+s+s1}{\PYGZsq{}}\PYG{p}{,} \PYG{l+s+s1}{\PYGZsq{}}\PYG{l+s+s1}{Longitud (cm)}\PYG{l+s+s1}{\PYGZsq{}}\PYG{p}{,} \PYG{l+s+s1}{\PYGZsq{}}\PYG{l+s+s1}{Anchura (cm)}\PYG{l+s+s1}{\PYGZsq{}}\PYG{p}{]}
\PYG{n}{X} \PYG{o}{=} \PYG{n}{df}\PYG{p}{[}\PYG{n}{iso\PYGZus{}cols}\PYG{p}{]}

\PYG{c+c1}{\PYGZsh{} Normalizamos los datos}
\PYG{n}{scaler} \PYG{o}{=} \PYG{n}{StandardScaler}\PYG{p}{(}\PYG{p}{)}
\PYG{n}{X\PYGZus{}scaled} \PYG{o}{=} \PYG{n}{scaler}\PYG{o}{.}\PYG{n}{fit\PYGZus{}transform}\PYG{p}{(}\PYG{n}{X}\PYG{p}{)}

\PYG{n}{labels\PYGZus{}iso} \PYG{o}{=} \PYG{n}{iso}\PYG{o}{.}\PYG{n}{fit\PYGZus{}predict}\PYG{p}{(}\PYG{n}{X\PYGZus{}scaled}\PYG{p}{)}
\PYG{n}{mask\PYGZus{}iso} \PYG{o}{=} \PYG{n}{labels\PYGZus{}iso} \PYG{o}{==} \PYG{o}{\PYGZhy{}}\PYG{l+m+mi}{1}

\PYG{c+c1}{\PYGZsh{} Guardamos los ouliers en el dataframe}
\PYG{n}{df\PYGZus{}outliers}\PYG{p}{[}\PYG{l+s+s2}{\PYGZdq{}}\PYG{l+s+s2}{out\PYGZus{}iso}\PYG{l+s+s2}{\PYGZdq{}}\PYG{p}{]} \PYG{o}{=} \PYG{n}{mask\PYGZus{}iso}

\PYG{n+nb}{print}\PYG{p}{(}\PYG{l+s+sa}{f}\PYG{l+s+s2}{\PYGZdq{}}\PYG{l+s+s2}{Nº de outliers detectados: }\PYG{l+s+si}{\PYGZob{}}\PYG{n}{df\PYGZus{}outliers}\PYG{p}{[}\PYG{l+s+s1}{\PYGZsq{}}\PYG{l+s+s1}{out\PYGZus{}iso}\PYG{l+s+s1}{\PYGZsq{}}\PYG{p}{]}\PYG{o}{.}\PYG{n}{sum}\PYG{p}{(}\PYG{p}{)}\PYG{l+s+si}{\PYGZcb{}}\PYG{l+s+s2}{\PYGZdq{}}\PYG{p}{)}
\PYG{n}{display}\PYG{p}{(}\PYG{n}{df}\PYG{p}{[}\PYG{n}{df\PYGZus{}outliers}\PYG{p}{[}\PYG{l+s+s2}{\PYGZdq{}}\PYG{l+s+s2}{out\PYGZus{}iso}\PYG{l+s+s2}{\PYGZdq{}}\PYG{p}{]}\PYG{p}{]}\PYG{p}{)}

\PYG{n}{fig}\PYG{p}{,} \PYG{p}{(}\PYG{n}{ax1}\PYG{p}{,} \PYG{n}{ax2}\PYG{p}{)} \PYG{o}{=} \PYG{n}{plt}\PYG{o}{.}\PYG{n}{subplots}\PYG{p}{(}\PYG{l+m+mi}{1}\PYG{p}{,} \PYG{l+m+mi}{2}\PYG{p}{,} \PYG{n}{figsize}\PYG{o}{=}\PYG{p}{(}\PYG{l+m+mi}{14}\PYG{p}{,} \PYG{l+m+mi}{5}\PYG{p}{)}\PYG{p}{)}

\PYG{c+c1}{\PYGZsh{} Gráfica 1}
\PYG{n}{sns}\PYG{o}{.}\PYG{n}{scatterplot}\PYG{p}{(}\PYG{n}{data}\PYG{o}{=}\PYG{n}{df\PYGZus{}outliers}\PYG{p}{,} \PYG{n}{x}\PYG{o}{=}\PYG{l+s+s1}{\PYGZsq{}}\PYG{l+s+s1}{Longitud (cm)}\PYG{l+s+s1}{\PYGZsq{}}\PYG{p}{,} \PYG{n}{y}\PYG{o}{=}\PYG{l+s+s1}{\PYGZsq{}}\PYG{l+s+s1}{Peso (g)}\PYG{l+s+s1}{\PYGZsq{}}\PYG{p}{,}
                \PYG{n}{hue}\PYG{o}{=}\PYG{l+s+s1}{\PYGZsq{}}\PYG{l+s+s1}{out\PYGZus{}iso}\PYG{l+s+s1}{\PYGZsq{}}\PYG{p}{,} \PYG{n}{ax}\PYG{o}{=}\PYG{n}{ax1}\PYG{p}{,}
                \PYG{n}{palette}\PYG{o}{=}\PYG{p}{\PYGZob{}}\PYG{k+kc}{False}\PYG{p}{:} \PYG{l+s+s2}{\PYGZdq{}}\PYG{l+s+s2}{\PYGZsh{}2b7bba}\PYG{l+s+s2}{\PYGZdq{}}\PYG{p}{,} \PYG{k+kc}{True}\PYG{p}{:} \PYG{l+s+s2}{\PYGZdq{}}\PYG{l+s+s2}{\PYGZsh{}d62728}\PYG{l+s+s2}{\PYGZdq{}}\PYG{p}{\PYGZcb{}}\PYG{p}{,}
                \PYG{n}{alpha}\PYG{o}{=}\PYG{l+m+mf}{0.8}\PYG{p}{,} \PYG{n}{s}\PYG{o}{=}\PYG{l+m+mi}{40}\PYG{p}{)}
\PYG{n}{ax1}\PYG{o}{.}\PYG{n}{set\PYGZus{}title}\PYG{p}{(}\PYG{l+s+s2}{\PYGZdq{}}\PYG{l+s+s2}{Peso vs Longitud – outliers resaltados}\PYG{l+s+s2}{\PYGZdq{}}\PYG{p}{)}
\PYG{n}{ax1}\PYG{o}{.}\PYG{n}{legend}\PYG{p}{(}\PYG{n}{title}\PYG{o}{=}\PYG{l+s+s2}{\PYGZdq{}}\PYG{l+s+s2}{Outlier}\PYG{l+s+s2}{\PYGZdq{}}\PYG{p}{,} \PYG{n}{loc}\PYG{o}{=}\PYG{l+s+s2}{\PYGZdq{}}\PYG{l+s+s2}{upper left}\PYG{l+s+s2}{\PYGZdq{}}\PYG{p}{)}

\PYG{c+c1}{\PYGZsh{} Gráfica 2}
\PYG{n}{sns}\PYG{o}{.}\PYG{n}{scatterplot}\PYG{p}{(}\PYG{n}{data}\PYG{o}{=}\PYG{n}{df\PYGZus{}outliers}\PYG{p}{,} \PYG{n}{x}\PYG{o}{=}\PYG{l+s+s1}{\PYGZsq{}}\PYG{l+s+s1}{Anchura (cm)}\PYG{l+s+s1}{\PYGZsq{}}\PYG{p}{,} \PYG{n}{y}\PYG{o}{=}\PYG{l+s+s1}{\PYGZsq{}}\PYG{l+s+s1}{Peso (g)}\PYG{l+s+s1}{\PYGZsq{}}\PYG{p}{,}
                \PYG{n}{hue}\PYG{o}{=}\PYG{l+s+s1}{\PYGZsq{}}\PYG{l+s+s1}{out\PYGZus{}iso}\PYG{l+s+s1}{\PYGZsq{}}\PYG{p}{,} \PYG{n}{ax}\PYG{o}{=}\PYG{n}{ax2}\PYG{p}{,}
                \PYG{n}{palette}\PYG{o}{=}\PYG{p}{\PYGZob{}}\PYG{k+kc}{False}\PYG{p}{:} \PYG{l+s+s2}{\PYGZdq{}}\PYG{l+s+s2}{\PYGZsh{}2b7bba}\PYG{l+s+s2}{\PYGZdq{}}\PYG{p}{,} \PYG{k+kc}{True}\PYG{p}{:} \PYG{l+s+s2}{\PYGZdq{}}\PYG{l+s+s2}{\PYGZsh{}d62728}\PYG{l+s+s2}{\PYGZdq{}}\PYG{p}{\PYGZcb{}}\PYG{p}{,}
                \PYG{n}{alpha}\PYG{o}{=}\PYG{l+m+mf}{0.8}\PYG{p}{,} \PYG{n}{s}\PYG{o}{=}\PYG{l+m+mi}{40}\PYG{p}{)}
\PYG{n}{ax2}\PYG{o}{.}\PYG{n}{set\PYGZus{}title}\PYG{p}{(}\PYG{l+s+s2}{\PYGZdq{}}\PYG{l+s+s2}{Peso vs Anchura – outliers resaltados}\PYG{l+s+s2}{\PYGZdq{}}\PYG{p}{)}
\PYG{n}{ax2}\PYG{o}{.}\PYG{n}{legend}\PYG{p}{(}\PYG{n}{title}\PYG{o}{=}\PYG{l+s+s2}{\PYGZdq{}}\PYG{l+s+s2}{Outlier}\PYG{l+s+s2}{\PYGZdq{}}\PYG{p}{,} \PYG{n}{loc}\PYG{o}{=}\PYG{l+s+s2}{\PYGZdq{}}\PYG{l+s+s2}{upper left}\PYG{l+s+s2}{\PYGZdq{}}\PYG{p}{)}

\PYG{n}{plt}\PYG{o}{.}\PYG{n}{tight\PYGZus{}layout}\PYG{p}{(}\PYG{p}{)}
\PYG{n}{plt}\PYG{o}{.}\PYG{n}{show}\PYG{p}{(}\PYG{p}{)}
\end{sphinxVerbatim}

\end{sphinxuseclass}\end{sphinxVerbatimInput}
\begin{sphinxVerbatimOutput}

\begin{sphinxuseclass}{cell_output}
\begin{sphinxVerbatim}[commandchars=\\\{\}]
Nº de outliers detectados: 36
\end{sphinxVerbatim}

\begin{sphinxVerbatim}[commandchars=\\\{\}]
     Peso (g)  Longitud (cm)  Anchura (cm)  Altura (cm)
0        0.46            3.3           1.3          0.2
1        1.08            4.5           1.1          0.3
2        0.67            3.9           1.5          0.2
3        0.98            4.4           1.7          0.3
4        0.93            4.2           1.8          0.3
5        1.89            4.5           2.0          0.4
6        1.60            5.1           1.8          0.3
92       3.06            6.0           2.9          0.4
141      6.96            7.2           3.3          0.5
153      3.40            6.8           3.9          0.4
156      4.50            7.4           3.7          0.4
183     10.08            7.8           4.2          0.7
185      9.79            9.7           3.5          0.6
186     10.10            9.7           3.5          0.6
187      9.28            9.5           3.6          0.6
188      9.51            9.4           3.7          0.6
189     10.18            9.1           3.9          0.5
190     10.38            9.3           3.9          0.6
191      9.82            9.7           3.8          0.6
192     11.35            9.7           4.0          0.6
193     12.70            9.7           4.0          0.7
194     13.45            9.7           4.0          0.6
195     11.22           10.1           3.9          0.6
196     12.37            9.7           4.1          0.6
197     13.56            9.7           4.1          0.6
198     11.28           10.0           4.0          0.6
199     12.91           10.1           4.0          0.6
200     14.13           10.1           4.1          0.7
201     14.04           10.5           4.1          0.7
202     14.67           10.7           4.2          0.7
203     15.65           10.7           4.5          0.6
204     14.50           10.8           4.5          0.6
205     19.88           11.4           4.4          0.9
206     16.47           11.0           4.6          0.7
207     17.04           10.6           4.8          0.7
208     21.98           10.6           5.2          0.8
\end{sphinxVerbatim}

\noindent\sphinxincludegraphics{{ff33ea5e1d6af53aa657bea9cf8ae53d30188d4a1fd2fc61e152136f03fa6e29}.png}

\end{sphinxuseclass}\end{sphinxVerbatimOutput}

\end{sphinxuseclass}
\sphinxAtStartPar
Los tres métodos empelados para encontrar outliers presentan diferentes números de \sphinxstyleemphasis{outliers}. Dado que cada algoritmo de detección de \sphinxstyleemphasis{outliers} se fundamenta en marcos estadísticos divergentes —el método del rango intercuartílico (IQR) opera bajo un enfoque univariante robusto basado en cuantiles, pero carece de sensibilidad para capturar relaciones multivariadas; el Z\sphinxhyphen{}score presupone una distribución cuasi\sphinxhyphen{}normal y cuantifica desviaciones en unidades de escala estándar; mientras que Isolation Forest implementa un paradigma no paramétrico que aisla observaciones atípicas en espacios p\sphinxhyphen{}dimensionales mediante hiperplanos aleatorios—, se deriva como consecuencia teórica que los conjuntos de anomalías identificados presentarán necesariamente solapamientos parciales.

\sphinxAtStartPar
Un análisis de conjuntos (operaciones de intersección y diferencia) puede ser una metodología adecuada para: a) discriminar entre outliers de alta consistencia (aquellos que simultáneamente exceden los umbrales de dispersión no paramétrica, violan los supuestos de normalidad y manifiestan rareza en la topología multivariante), y b) identificar observaciones cuya clasificación como atípicas es artefactual, dependiente de los supuestos específicos de cada modelo. Este protocolo de validación cruzada minimiza los errores de Tipo I (falsos positivos), incrementa la validez del preprocesamiento estadístico y permite así una cuantificación formal del sesgo metodológico inherente a cada técnica.

\sphinxAtStartPar
El objetivo de este análisis de conjuntos es delimitar aquellas observaciones que son identificadas como \sphinxstyleemphasis{outliers} por al menos dos métodos independientes, estableciendo así un criterio de consenso que refuerza la fiabilidad de la detección. Cuando un mismo dato es marcado como anómalo por múltiples técnicas —cada una basada en supuestos estadísticos distintos (p. ej., dispersión robusta, normalidad, o rareza multivariante)—, la probabilidad de que se trate de un verdadero valor aberrante, y no de un artefacto metodológico, aumenta significativamente.

\sphinxAtStartPar
Por tanto, estos \sphinxstyleemphasis{outliers} multimodales deben considerarse como observaciones genuinamente anómalas, ya sea debido a errores de medición o comportamientos sistémicos atípicos, y su exclusión o tratamiento se justifica para garantizar la validez inferencial de los análisis posteriores.

\begin{sphinxuseclass}{cell}\begin{sphinxVerbatimInput}

\begin{sphinxuseclass}{cell_input}
\begin{sphinxVerbatim}[commandchars=\\\{\}]
\PYG{c+c1}{\PYGZsh{} Conjuntos y solapamientos}
\PYG{n}{set\PYGZus{}iqr} \PYG{o}{=} \PYG{n+nb}{set}\PYG{p}{(}\PYG{n}{df\PYGZus{}outliers}\PYG{o}{.}\PYG{n}{index}\PYG{p}{[}\PYG{n}{df\PYGZus{}outliers}\PYG{p}{[}\PYG{l+s+s1}{\PYGZsq{}}\PYG{l+s+s1}{out\PYGZus{}IQR}\PYG{l+s+s1}{\PYGZsq{}}\PYG{p}{]}\PYG{p}{]}\PYG{p}{)}
\PYG{n}{set\PYGZus{}zscore}   \PYG{o}{=} \PYG{n+nb}{set}\PYG{p}{(}\PYG{n}{df\PYGZus{}outliers}\PYG{o}{.}\PYG{n}{index}\PYG{p}{[}\PYG{n}{df\PYGZus{}outliers}\PYG{p}{[}\PYG{l+s+s1}{\PYGZsq{}}\PYG{l+s+s1}{out\PYGZus{}ZScore}\PYG{l+s+s1}{\PYGZsq{}}\PYG{p}{]}\PYG{p}{]}\PYG{p}{)}
\PYG{n}{set\PYGZus{}if}  \PYG{o}{=} \PYG{n+nb}{set}\PYG{p}{(}\PYG{n}{df\PYGZus{}outliers}\PYG{o}{.}\PYG{n}{index}\PYG{p}{[}\PYG{n}{df\PYGZus{}outliers}\PYG{p}{[}\PYG{l+s+s1}{\PYGZsq{}}\PYG{l+s+s1}{out\PYGZus{}iso}\PYG{l+s+s1}{\PYGZsq{}}\PYG{p}{]}\PYG{p}{]}\PYG{p}{)}

\PYG{n+nb}{print}\PYG{p}{(}\PYG{l+s+s2}{\PYGZdq{}}\PYG{l+s+s2}{─ Outliers por método ─}\PYG{l+s+se}{\PYGZbs{}n}\PYG{l+s+s2}{\PYGZdq{}}\PYG{p}{)}
\PYG{n+nb}{print}\PYG{p}{(}\PYG{l+s+sa}{f}\PYG{l+s+s2}{\PYGZdq{}}\PYG{l+s+s2}{IQR                : }\PYG{l+s+si}{\PYGZob{}}\PYG{n+nb}{len}\PYG{p}{(}\PYG{n}{set\PYGZus{}iqr}\PYG{p}{)}\PYG{l+s+si}{\PYGZcb{}}\PYG{l+s+s2}{\PYGZdq{}}\PYG{p}{)}
\PYG{n+nb}{print}\PYG{p}{(}\PYG{l+s+sa}{f}\PYG{l+s+s2}{\PYGZdq{}}\PYG{l+s+s2}{Z\PYGZhy{}Score |z| \PYGZgt{} 3    : }\PYG{l+s+si}{\PYGZob{}}\PYG{n+nb}{len}\PYG{p}{(}\PYG{n}{set\PYGZus{}zscore}\PYG{p}{)}\PYG{l+s+si}{\PYGZcb{}}\PYG{l+s+s2}{\PYGZdq{}}\PYG{p}{)}
\PYG{n+nb}{print}\PYG{p}{(}\PYG{l+s+sa}{f}\PYG{l+s+s2}{\PYGZdq{}}\PYG{l+s+s2}{Isolation Forest   : }\PYG{l+s+si}{\PYGZob{}}\PYG{n+nb}{len}\PYG{p}{(}\PYG{n}{set\PYGZus{}if}\PYG{p}{)}\PYG{l+s+si}{\PYGZcb{}}\PYG{l+s+se}{\PYGZbs{}n}\PYG{l+s+s2}{\PYGZdq{}}\PYG{p}{)}

\PYG{n}{idx\PYGZus{}common} \PYG{o}{=} \PYG{n}{np}\PYG{o}{.}\PYG{n}{array}\PYG{p}{(}\PYG{p}{[}
    \PYG{n+nb}{sorted}\PYG{p}{(}\PYG{n}{set\PYGZus{}iqr} \PYG{o}{\PYGZam{}} \PYG{n}{set\PYGZus{}zscore}\PYG{p}{)}\PYG{p}{,}
    \PYG{n+nb}{sorted}\PYG{p}{(}\PYG{n}{set\PYGZus{}iqr} \PYG{o}{\PYGZam{}} \PYG{n}{set\PYGZus{}if}\PYG{p}{)}\PYG{p}{,}
    \PYG{n+nb}{sorted}\PYG{p}{(}\PYG{n}{set\PYGZus{}zscore} \PYG{o}{\PYGZam{}} \PYG{n}{set\PYGZus{}if}\PYG{p}{)}\PYG{p}{,}
    \PYG{n+nb}{sorted}\PYG{p}{(}\PYG{n}{set\PYGZus{}iqr} \PYG{o}{\PYGZam{}} \PYG{n}{set\PYGZus{}zscore} \PYG{o}{\PYGZam{}} \PYG{n}{set\PYGZus{}if}\PYG{p}{)}
    \PYG{p}{]}\PYG{p}{,} \PYG{n}{dtype}\PYG{o}{=}\PYG{n+nb}{object}\PYG{p}{)} \PYG{c+c1}{\PYGZsh{} dtype=object para listas de distinto tamaño}

\PYG{n+nb}{print}\PYG{p}{(}\PYG{l+s+s2}{\PYGZdq{}}\PYG{l+s+s2}{─ Intersecciones ─}\PYG{l+s+se}{\PYGZbs{}n}\PYG{l+s+s2}{\PYGZdq{}}\PYG{p}{)}
\PYG{n+nb}{print}\PYG{p}{(}\PYG{l+s+sa}{f}\PYG{l+s+s2}{\PYGZdq{}}\PYG{l+s+s2}{IQR ∩ Z\PYGZhy{}Score: }\PYG{l+s+si}{\PYGZob{}}\PYG{n+nb}{len}\PYG{p}{(}\PYG{n}{set\PYGZus{}iqr}\PYG{+w}{ }\PYG{o}{\PYGZam{}}\PYG{+w}{ }\PYG{n}{set\PYGZus{}zscore}\PYG{p}{)}\PYG{l+s+si}{\PYGZcb{}}\PYG{l+s+s2}{\PYGZdq{}}\PYG{p}{)}
\PYG{n}{outliers\PYGZus{}common} \PYG{o}{=} \PYG{n}{df\PYGZus{}outliers}\PYG{o}{.}\PYG{n}{loc}\PYG{p}{[}\PYG{n}{idx\PYGZus{}common}\PYG{p}{[}\PYG{l+m+mi}{0}\PYG{p}{]}\PYG{p}{]}
\PYG{n}{display}\PYG{p}{(}\PYG{n}{outliers\PYGZus{}common}\PYG{p}{)}

\PYG{n+nb}{print}\PYG{p}{(}\PYG{l+s+sa}{f}\PYG{l+s+s2}{\PYGZdq{}}\PYG{l+s+s2}{IQR ∩ Isolation Forest     : }\PYG{l+s+si}{\PYGZob{}}\PYG{n+nb}{len}\PYG{p}{(}\PYG{n}{set\PYGZus{}iqr}\PYG{+w}{ }\PYG{o}{\PYGZam{}}\PYG{+w}{ }\PYG{n}{set\PYGZus{}if}\PYG{p}{)}\PYG{l+s+si}{\PYGZcb{}}\PYG{l+s+s2}{\PYGZdq{}}\PYG{p}{)}
\PYG{n}{display}\PYG{p}{(}\PYG{n}{df\PYGZus{}outliers}\PYG{o}{.}\PYG{n}{loc}\PYG{p}{[}\PYG{n}{idx\PYGZus{}common}\PYG{p}{[}\PYG{l+m+mi}{1}\PYG{p}{]}\PYG{p}{]}\PYG{p}{)}

\PYG{n+nb}{print}\PYG{p}{(}\PYG{l+s+sa}{f}\PYG{l+s+s2}{\PYGZdq{}}\PYG{l+s+s2}{Z\PYGZhy{}Score ∩ Isolation Forest : }\PYG{l+s+si}{\PYGZob{}}\PYG{n+nb}{len}\PYG{p}{(}\PYG{n}{set\PYGZus{}zscore}\PYG{+w}{ }\PYG{o}{\PYGZam{}}\PYG{+w}{ }\PYG{n}{set\PYGZus{}if}\PYG{p}{)}\PYG{l+s+si}{\PYGZcb{}}\PYG{l+s+s2}{\PYGZdq{}}\PYG{p}{)}
\PYG{n}{display}\PYG{p}{(}\PYG{n}{df\PYGZus{}outliers}\PYG{o}{.}\PYG{n}{loc}\PYG{p}{[}\PYG{n}{idx\PYGZus{}common}\PYG{p}{[}\PYG{l+m+mi}{2}\PYG{p}{]}\PYG{p}{]}\PYG{p}{)}

\PYG{n+nb}{print}\PYG{p}{(}\PYG{l+s+sa}{f}\PYG{l+s+s2}{\PYGZdq{}}\PYG{l+s+s2}{IQR ∩ Z\PYGZhy{}Score ∩ IF         : }\PYG{l+s+si}{\PYGZob{}}\PYG{n+nb}{len}\PYG{p}{(}\PYG{n}{set\PYGZus{}iqr}\PYG{+w}{ }\PYG{o}{\PYGZam{}}\PYG{+w}{ }\PYG{n}{set\PYGZus{}zscore}\PYG{+w}{ }\PYG{o}{\PYGZam{}}\PYG{+w}{ }\PYG{n}{set\PYGZus{}if}\PYG{p}{)}\PYG{l+s+si}{\PYGZcb{}}\PYG{l+s+s2}{\PYGZdq{}}\PYG{p}{)}
\PYG{n}{display}\PYG{p}{(}\PYG{n}{df\PYGZus{}outliers}\PYG{o}{.}\PYG{n}{loc}\PYG{p}{[}\PYG{n}{idx\PYGZus{}common}\PYG{p}{[}\PYG{l+m+mi}{3}\PYG{p}{]}\PYG{p}{]}\PYG{p}{)}
\end{sphinxVerbatim}

\end{sphinxuseclass}\end{sphinxVerbatimInput}
\begin{sphinxVerbatimOutput}

\begin{sphinxuseclass}{cell_output}
\begin{sphinxVerbatim}[commandchars=\\\{\}]
─ Outliers por método ─

IQR                : 11
Z\PYGZhy{}Score |z| \PYGZgt{} 3    : 4
Isolation Forest   : 36

─ Intersecciones ─

IQR ∩ Z\PYGZhy{}Score: 4
\end{sphinxVerbatim}

\begin{sphinxVerbatim}[commandchars=\\\{\}]
     Peso (g)  Longitud (cm)  Anchura (cm)  Altura (cm)  out\PYGZus{}IQR  out\PYGZus{}ZScore  \PYGZbs{}
205     19.88           11.4           4.4          0.9     True        True
206     16.47           11.0           4.6          0.7     True        True
207     17.04           10.6           4.8          0.7     True        True
208     21.98           10.6           5.2          0.8     True        True

     out\PYGZus{}iso
205     True
206     True
207     True
208     True
\end{sphinxVerbatim}

\begin{sphinxVerbatim}[commandchars=\\\{\}]
IQR ∩ Isolation Forest     : 11
\end{sphinxVerbatim}

\begin{sphinxVerbatim}[commandchars=\\\{\}]
     Peso (g)  Longitud (cm)  Anchura (cm)  Altura (cm)  out\PYGZus{}IQR  out\PYGZus{}ZScore  \PYGZbs{}
194     13.45            9.7           4.0          0.6     True       False
197     13.56            9.7           4.1          0.6     True       False
200     14.13           10.1           4.1          0.7     True       False
201     14.04           10.5           4.1          0.7     True       False
202     14.67           10.7           4.2          0.7     True       False
203     15.65           10.7           4.5          0.6     True       False
204     14.50           10.8           4.5          0.6     True       False
205     19.88           11.4           4.4          0.9     True        True
206     16.47           11.0           4.6          0.7     True        True
207     17.04           10.6           4.8          0.7     True        True
208     21.98           10.6           5.2          0.8     True        True

     out\PYGZus{}iso
194     True
197     True
200     True
201     True
202     True
203     True
204     True
205     True
206     True
207     True
208     True
\end{sphinxVerbatim}

\begin{sphinxVerbatim}[commandchars=\\\{\}]
Z\PYGZhy{}Score ∩ Isolation Forest : 4
\end{sphinxVerbatim}

\begin{sphinxVerbatim}[commandchars=\\\{\}]
     Peso (g)  Longitud (cm)  Anchura (cm)  Altura (cm)  out\PYGZus{}IQR  out\PYGZus{}ZScore  \PYGZbs{}
205     19.88           11.4           4.4          0.9     True        True
206     16.47           11.0           4.6          0.7     True        True
207     17.04           10.6           4.8          0.7     True        True
208     21.98           10.6           5.2          0.8     True        True

     out\PYGZus{}iso
205     True
206     True
207     True
208     True
\end{sphinxVerbatim}

\begin{sphinxVerbatim}[commandchars=\\\{\}]
IQR ∩ Z\PYGZhy{}Score ∩ IF         : 4
\end{sphinxVerbatim}

\begin{sphinxVerbatim}[commandchars=\\\{\}]
     Peso (g)  Longitud (cm)  Anchura (cm)  Altura (cm)  out\PYGZus{}IQR  out\PYGZus{}ZScore  \PYGZbs{}
205     19.88           11.4           4.4          0.9     True        True
206     16.47           11.0           4.6          0.7     True        True
207     17.04           10.6           4.8          0.7     True        True
208     21.98           10.6           5.2          0.8     True        True

     out\PYGZus{}iso
205     True
206     True
207     True
208     True
\end{sphinxVerbatim}

\end{sphinxuseclass}\end{sphinxVerbatimOutput}

\end{sphinxuseclass}
\sphinxAtStartPar
Los tres métodos usados coinciden en señalar los cuatros registros 205 \sphinxhyphen{} 208 como \sphinxstyleemphasis{outliers} lo cual es indicativo de posibles errores o individuos excepcionales en nuestro dataset y deberían no ser considerados en una mejora del modelo alométrico inferido. Por otro lado los registros 194\sphinxhyphen{}204 detectados como valores anómalos por los métodos IRQ e Isolation Forest pueden representar sesgo de muestreo, diferencias de condición o simple variabilidad natural.

\sphinxAtStartPar
La relación empírica \(𝑊=𝑘 \cdot L^a \cdot A^b\) es, por naturaleza, multiplicativa; de ella se deriva que la varianza de \(𝑊\) crece proporcionalmente a \(𝐿\) y \(A\). Este comportamiento genera heterocedasticidad: los residuos asociados a los ejemplares grandes presentan una dispersión mucho mayor que los de los de menor tramaño, Bajo esta condición, las métricas univariantes (IQR, desviaciones 𝑧) y los algoritmos multivariantes (Isolation Forest) operan con escalas que “inflan” artificialmente la distancia de los ejemplares grandes y “comprimen” la de los pequeños, sesgando la detección de valores atípicos. Aplicar una \sphinxstylestrong{transformación logarítmica} \(ln(⋅)\)
convierte esa relación de potencia en un modelo lineal aditivo y estabiliza la varianza, de modo que la dispersión residual es aproximadamente homocedástica y más simétrica. Los trabajos de \sphinxhref{https://doi.org/10.1111/bij.12038}{Packart (2013)} y posteriormente de  \sphinxhref{https://doi.org/10.1177/00045632211050531}{West (2022)} han demostrado que, tras la aplicacón de una transformada logarítmica, las colas asimétricas se reducen y las pruebas basadas en desviación estándar recuperan su tasa nominal de error I, recomendando explícitamente esta práctica previa a cualquier inferencia paramétrica.

\sphinxAtStartPar
En estudios ictiológicos recientes la práctica se ha consolidado como estándar. \sphinxhref{https://doi.org/10.3390/fishes8050222}{Rodríguez\sphinxhyphen{}García et al. (2023)} aplican la transformación logarítmica antes de filtrar anomalías en más de 40 especies descartadas por la flota del golfo de Cádiz y resaltan que, tras el log, sólo persiste un subconjunto mínimo de registros verdaderamente aberrantes, mientras que la mayoría de los \sphinxstyleemphasis{outliers} iniciales se explican por la escala multiplicativa del crecimiento. \sphinxhref{https://doi.org/10.1016/j.ejar.2022.07.005}{Xia et al. (2022)} en seis especies del Yangtsé— siguen este protocolo: estiman los parámetros en escala log y evalúan la calidad de ajuste y los residuos sobre los datos transformados, obteniendo un filtrado de anomalías significativamente más fiable.

\begin{sphinxadmonition}{note}{Conclusión}

\sphinxAtStartPar
En nuestro caso, transformar a \(ln\) permitirá redefinir los umbrales de IQR y Z\sphinxhyphen{}score sobre distribuciones casi simétricas y comparables, validar la relevancia biológica de los registros 205–208 —ahora \sphinxstyleemphasis{outliers} unánimes— y re\sphinxhyphen{}evaluar los casos 194–204 en un entorno de varianza estabilizada, evitando expulsar datos válidos que sólo parecían extremos por efecto de escala.
\end{sphinxadmonition}


\subsection{Transformada logarítmica}
\label{\detokenize{content/03/Analisis:transformada-logaritmica}}
\begin{sphinxuseclass}{cell}\begin{sphinxVerbatimInput}

\begin{sphinxuseclass}{cell_input}
\begin{sphinxVerbatim}[commandchars=\\\{\}]
\PYG{c+c1}{\PYGZsh{} ===============================================================}
\PYG{c+c1}{\PYGZsh{}     Transformación logarítmica, visualización preliminar}
\PYG{c+c1}{\PYGZsh{}          y re\PYGZhy{}detección de outliers}
\PYG{c+c1}{\PYGZsh{} ===============================================================}

\PYG{c+c1}{\PYGZsh{} \PYGZhy{}\PYGZhy{}\PYGZhy{}\PYGZhy{}\PYGZhy{}\PYGZhy{}\PYGZhy{}\PYGZhy{}\PYGZhy{}\PYGZhy{}\PYGZhy{}\PYGZhy{}\PYGZhy{}\PYGZhy{}\PYGZhy{}\PYGZhy{}\PYGZhy{}\PYGZhy{}\PYGZhy{}\PYGZhy{}\PYGZhy{}\PYGZhy{}\PYGZhy{}\PYGZhy{}\PYGZhy{}\PYGZhy{}\PYGZhy{}\PYGZhy{}\PYGZhy{}\PYGZhy{}\PYGZhy{}\PYGZhy{}\PYGZhy{}\PYGZhy{}\PYGZhy{}\PYGZhy{}\PYGZhy{}\PYGZhy{}\PYGZhy{}\PYGZhy{}\PYGZhy{}\PYGZhy{}\PYGZhy{}\PYGZhy{}\PYGZhy{}\PYGZhy{}\PYGZhy{}\PYGZhy{}\PYGZhy{}\PYGZhy{}\PYGZhy{}\PYGZhy{}\PYGZhy{}\PYGZhy{}\PYGZhy{}\PYGZhy{}\PYGZhy{}\PYGZhy{}\PYGZhy{}\PYGZhy{}\PYGZhy{}\PYGZhy{}\PYGZhy{}}
\PYG{c+c1}{\PYGZsh{} 1.  Seleccionar y depurar variables numéricas}
\PYG{c+c1}{\PYGZsh{} \PYGZhy{}\PYGZhy{}\PYGZhy{}\PYGZhy{}\PYGZhy{}\PYGZhy{}\PYGZhy{}\PYGZhy{}\PYGZhy{}\PYGZhy{}\PYGZhy{}\PYGZhy{}\PYGZhy{}\PYGZhy{}\PYGZhy{}\PYGZhy{}\PYGZhy{}\PYGZhy{}\PYGZhy{}\PYGZhy{}\PYGZhy{}\PYGZhy{}\PYGZhy{}\PYGZhy{}\PYGZhy{}\PYGZhy{}\PYGZhy{}\PYGZhy{}\PYGZhy{}\PYGZhy{}\PYGZhy{}\PYGZhy{}\PYGZhy{}\PYGZhy{}\PYGZhy{}\PYGZhy{}\PYGZhy{}\PYGZhy{}\PYGZhy{}\PYGZhy{}\PYGZhy{}\PYGZhy{}\PYGZhy{}\PYGZhy{}\PYGZhy{}\PYGZhy{}\PYGZhy{}\PYGZhy{}\PYGZhy{}\PYGZhy{}\PYGZhy{}\PYGZhy{}\PYGZhy{}\PYGZhy{}\PYGZhy{}\PYGZhy{}\PYGZhy{}\PYGZhy{}\PYGZhy{}\PYGZhy{}\PYGZhy{}\PYGZhy{}\PYGZhy{}}
\PYG{n}{numeric\PYGZus{}cols} \PYG{o}{=} \PYG{p}{[}\PYG{l+s+s1}{\PYGZsq{}}\PYG{l+s+s1}{Peso (g)}\PYG{l+s+s1}{\PYGZsq{}}\PYG{p}{,} \PYG{l+s+s1}{\PYGZsq{}}\PYG{l+s+s1}{Longitud (cm)}\PYG{l+s+s1}{\PYGZsq{}}\PYG{p}{,} \PYG{l+s+s1}{\PYGZsq{}}\PYG{l+s+s1}{Anchura (cm)}\PYG{l+s+s1}{\PYGZsq{}}\PYG{p}{]}

\PYG{c+c1}{\PYGZsh{}eps = 1e\PYGZhy{}6                                 \PYGZsh{} evita log(0)}
\PYG{c+c1}{\PYGZsh{}df[numeric\PYGZus{}cols] = df[numeric\PYGZus{}cols].clip(lower=eps)}

\PYG{c+c1}{\PYGZsh{} \PYGZhy{}\PYGZhy{}\PYGZhy{}\PYGZhy{}\PYGZhy{}\PYGZhy{}\PYGZhy{}\PYGZhy{}\PYGZhy{}\PYGZhy{}\PYGZhy{}\PYGZhy{}\PYGZhy{}\PYGZhy{}\PYGZhy{}\PYGZhy{}\PYGZhy{}\PYGZhy{}\PYGZhy{}\PYGZhy{}\PYGZhy{}\PYGZhy{}\PYGZhy{}\PYGZhy{}\PYGZhy{}\PYGZhy{}\PYGZhy{}\PYGZhy{}\PYGZhy{}\PYGZhy{}\PYGZhy{}\PYGZhy{}\PYGZhy{}\PYGZhy{}\PYGZhy{}\PYGZhy{}\PYGZhy{}\PYGZhy{}\PYGZhy{}\PYGZhy{}\PYGZhy{}\PYGZhy{}\PYGZhy{}\PYGZhy{}\PYGZhy{}\PYGZhy{}\PYGZhy{}\PYGZhy{}\PYGZhy{}\PYGZhy{}\PYGZhy{}\PYGZhy{}\PYGZhy{}\PYGZhy{}\PYGZhy{}\PYGZhy{}\PYGZhy{}\PYGZhy{}\PYGZhy{}\PYGZhy{}\PYGZhy{}\PYGZhy{}\PYGZhy{}}
\PYG{c+c1}{\PYGZsh{} 2.  Transformación log\PYGZhy{}natural}
\PYG{c+c1}{\PYGZsh{} \PYGZhy{}\PYGZhy{}\PYGZhy{}\PYGZhy{}\PYGZhy{}\PYGZhy{}\PYGZhy{}\PYGZhy{}\PYGZhy{}\PYGZhy{}\PYGZhy{}\PYGZhy{}\PYGZhy{}\PYGZhy{}\PYGZhy{}\PYGZhy{}\PYGZhy{}\PYGZhy{}\PYGZhy{}\PYGZhy{}\PYGZhy{}\PYGZhy{}\PYGZhy{}\PYGZhy{}\PYGZhy{}\PYGZhy{}\PYGZhy{}\PYGZhy{}\PYGZhy{}\PYGZhy{}\PYGZhy{}\PYGZhy{}\PYGZhy{}\PYGZhy{}\PYGZhy{}\PYGZhy{}\PYGZhy{}\PYGZhy{}\PYGZhy{}\PYGZhy{}\PYGZhy{}\PYGZhy{}\PYGZhy{}\PYGZhy{}\PYGZhy{}\PYGZhy{}\PYGZhy{}\PYGZhy{}\PYGZhy{}\PYGZhy{}\PYGZhy{}\PYGZhy{}\PYGZhy{}\PYGZhy{}\PYGZhy{}\PYGZhy{}\PYGZhy{}\PYGZhy{}\PYGZhy{}\PYGZhy{}\PYGZhy{}\PYGZhy{}\PYGZhy{}}
\PYG{n}{df\PYGZus{}log} \PYG{o}{=} \PYG{n}{df}\PYG{p}{[}\PYG{n}{numeric\PYGZus{}cols}\PYG{p}{]}\PYG{o}{.}\PYG{n}{copy}\PYG{p}{(}\PYG{p}{)}
\PYG{n}{df\PYGZus{}log} \PYG{o}{=} \PYG{n}{df\PYGZus{}log}\PYG{p}{[}\PYG{n}{df\PYGZus{}log}\PYG{p}{[}\PYG{l+s+s1}{\PYGZsq{}}\PYG{l+s+s1}{Peso (g)}\PYG{l+s+s1}{\PYGZsq{}}\PYG{p}{]} \PYG{o}{\PYGZgt{}} \PYG{l+m+mi}{0}\PYG{p}{]}

\PYG{n}{df\PYGZus{}log}\PYG{p}{[}\PYG{l+s+s1}{\PYGZsq{}}\PYG{l+s+s1}{log\PYGZus{}Peso}\PYG{l+s+s1}{\PYGZsq{}}\PYG{p}{]} \PYG{o}{=} \PYG{n}{np}\PYG{o}{.}\PYG{n}{log}\PYG{p}{(}\PYG{n}{df\PYGZus{}log}\PYG{p}{[}\PYG{l+s+s1}{\PYGZsq{}}\PYG{l+s+s1}{Peso (g)}\PYG{l+s+s1}{\PYGZsq{}}\PYG{p}{]}\PYG{p}{)}
\PYG{n}{df\PYGZus{}log}\PYG{p}{[}\PYG{l+s+s1}{\PYGZsq{}}\PYG{l+s+s1}{log\PYGZus{}Longitud}\PYG{l+s+s1}{\PYGZsq{}}\PYG{p}{]} \PYG{o}{=} \PYG{n}{np}\PYG{o}{.}\PYG{n}{log}\PYG{p}{(}\PYG{n}{df\PYGZus{}log}\PYG{p}{[}\PYG{l+s+s1}{\PYGZsq{}}\PYG{l+s+s1}{Longitud (cm)}\PYG{l+s+s1}{\PYGZsq{}}\PYG{p}{]}\PYG{p}{)}
\PYG{n}{df\PYGZus{}log}\PYG{p}{[}\PYG{l+s+s1}{\PYGZsq{}}\PYG{l+s+s1}{log\PYGZus{}Anchura}\PYG{l+s+s1}{\PYGZsq{}}\PYG{p}{]} \PYG{o}{=} \PYG{n}{np}\PYG{o}{.}\PYG{n}{log}\PYG{p}{(}\PYG{n}{df\PYGZus{}log}\PYG{p}{[}\PYG{l+s+s1}{\PYGZsq{}}\PYG{l+s+s1}{Anchura (cm)}\PYG{l+s+s1}{\PYGZsq{}}\PYG{p}{]}\PYG{p}{)}

\PYG{c+c1}{\PYGZsh{} \PYGZhy{}\PYGZhy{}\PYGZhy{}\PYGZhy{}\PYGZhy{}\PYGZhy{}\PYGZhy{}\PYGZhy{}\PYGZhy{}\PYGZhy{}\PYGZhy{}\PYGZhy{}\PYGZhy{}\PYGZhy{}\PYGZhy{}\PYGZhy{}\PYGZhy{}\PYGZhy{}\PYGZhy{}\PYGZhy{}\PYGZhy{}\PYGZhy{}\PYGZhy{}\PYGZhy{}\PYGZhy{}\PYGZhy{}\PYGZhy{}\PYGZhy{}\PYGZhy{}\PYGZhy{}\PYGZhy{}\PYGZhy{}\PYGZhy{}\PYGZhy{}\PYGZhy{}\PYGZhy{}\PYGZhy{}\PYGZhy{}\PYGZhy{}\PYGZhy{}\PYGZhy{}\PYGZhy{}\PYGZhy{}\PYGZhy{}\PYGZhy{}\PYGZhy{}\PYGZhy{}\PYGZhy{}\PYGZhy{}\PYGZhy{}\PYGZhy{}\PYGZhy{}\PYGZhy{}\PYGZhy{}\PYGZhy{}\PYGZhy{}\PYGZhy{}\PYGZhy{}\PYGZhy{}\PYGZhy{}\PYGZhy{}\PYGZhy{}\PYGZhy{}}
\PYG{c+c1}{\PYGZsh{} 3.  Visualizar efecto de la transformación}
\PYG{c+c1}{\PYGZsh{} \PYGZhy{}\PYGZhy{}\PYGZhy{}\PYGZhy{}\PYGZhy{}\PYGZhy{}\PYGZhy{}\PYGZhy{}\PYGZhy{}\PYGZhy{}\PYGZhy{}\PYGZhy{}\PYGZhy{}\PYGZhy{}\PYGZhy{}\PYGZhy{}\PYGZhy{}\PYGZhy{}\PYGZhy{}\PYGZhy{}\PYGZhy{}\PYGZhy{}\PYGZhy{}\PYGZhy{}\PYGZhy{}\PYGZhy{}\PYGZhy{}\PYGZhy{}\PYGZhy{}\PYGZhy{}\PYGZhy{}\PYGZhy{}\PYGZhy{}\PYGZhy{}\PYGZhy{}\PYGZhy{}\PYGZhy{}\PYGZhy{}\PYGZhy{}\PYGZhy{}\PYGZhy{}\PYGZhy{}\PYGZhy{}\PYGZhy{}\PYGZhy{}\PYGZhy{}\PYGZhy{}\PYGZhy{}\PYGZhy{}\PYGZhy{}\PYGZhy{}\PYGZhy{}\PYGZhy{}\PYGZhy{}\PYGZhy{}\PYGZhy{}\PYGZhy{}\PYGZhy{}\PYGZhy{}\PYGZhy{}\PYGZhy{}\PYGZhy{}\PYGZhy{}}
\PYG{n}{sns}\PYG{o}{.}\PYG{n}{set}\PYG{p}{(}\PYG{n}{style}\PYG{o}{=}\PYG{l+s+s2}{\PYGZdq{}}\PYG{l+s+s2}{ticks}\PYG{l+s+s2}{\PYGZdq{}}\PYG{p}{,} \PYG{n}{context}\PYG{o}{=}\PYG{l+s+s2}{\PYGZdq{}}\PYG{l+s+s2}{talk}\PYG{l+s+s2}{\PYGZdq{}}\PYG{p}{)}

\PYG{c+c1}{\PYGZsh{}\PYGZsh{} 3\PYGZhy{}a. Histogramas Peso: original vs ln}
\PYG{n}{fig}\PYG{p}{,} \PYG{n}{ax} \PYG{o}{=} \PYG{n}{plt}\PYG{o}{.}\PYG{n}{subplots}\PYG{p}{(}\PYG{l+m+mi}{1}\PYG{p}{,} \PYG{l+m+mi}{2}\PYG{p}{,} \PYG{n}{figsize}\PYG{o}{=}\PYG{p}{(}\PYG{l+m+mi}{10}\PYG{p}{,} \PYG{l+m+mi}{4}\PYG{p}{)}\PYG{p}{)}
\PYG{n}{sns}\PYG{o}{.}\PYG{n}{histplot}\PYG{p}{(}\PYG{n}{df}\PYG{p}{[}\PYG{l+s+s1}{\PYGZsq{}}\PYG{l+s+s1}{Peso (g)}\PYG{l+s+s1}{\PYGZsq{}}\PYG{p}{]}\PYG{p}{,} \PYG{n}{ax}\PYG{o}{=}\PYG{n}{ax}\PYG{p}{[}\PYG{l+m+mi}{0}\PYG{p}{]}\PYG{p}{,} \PYG{n}{bins}\PYG{o}{=}\PYG{l+m+mi}{30}\PYG{p}{,} \PYG{n}{kde}\PYG{o}{=}\PYG{k+kc}{True}\PYG{p}{)}
\PYG{n}{ax}\PYG{p}{[}\PYG{l+m+mi}{0}\PYG{p}{]}\PYG{o}{.}\PYG{n}{set\PYGZus{}title}\PYG{p}{(}\PYG{l+s+s2}{\PYGZdq{}}\PYG{l+s+s2}{Distribución Peso (g) original}\PYG{l+s+s2}{\PYGZdq{}}\PYG{p}{)}
\PYG{n}{sns}\PYG{o}{.}\PYG{n}{histplot}\PYG{p}{(}\PYG{n}{df\PYGZus{}log}\PYG{p}{[}\PYG{l+s+s1}{\PYGZsq{}}\PYG{l+s+s1}{log\PYGZus{}Peso}\PYG{l+s+s1}{\PYGZsq{}}\PYG{p}{]}\PYG{p}{,} \PYG{n}{ax}\PYG{o}{=}\PYG{n}{ax}\PYG{p}{[}\PYG{l+m+mi}{1}\PYG{p}{]}\PYG{p}{,} \PYG{n}{bins}\PYG{o}{=}\PYG{l+m+mi}{30}\PYG{p}{,} \PYG{n}{kde}\PYG{o}{=}\PYG{k+kc}{True}\PYG{p}{,} \PYG{n}{color}\PYG{o}{=}\PYG{l+s+s2}{\PYGZdq{}}\PYG{l+s+s2}{\PYGZsh{}ff7f0e}\PYG{l+s+s2}{\PYGZdq{}}\PYG{p}{)}
\PYG{n}{ax}\PYG{p}{[}\PYG{l+m+mi}{1}\PYG{p}{]}\PYG{o}{.}\PYG{n}{set\PYGZus{}title}\PYG{p}{(}\PYG{l+s+s2}{\PYGZdq{}}\PYG{l+s+s2}{Distribución ln Peso}\PYG{l+s+s2}{\PYGZdq{}}\PYG{p}{)}
\PYG{n}{plt}\PYG{o}{.}\PYG{n}{tight\PYGZus{}layout}\PYG{p}{(}\PYG{p}{)}
\PYG{n}{plt}\PYG{o}{.}\PYG{n}{show}\PYG{p}{(}\PYG{p}{)}

\PYG{c+c1}{\PYGZsh{}\PYGZsh{} 3\PYGZhy{}b. Scatter Peso–Longitud: lineal vs log\PYGZhy{}log}
\PYG{n}{fig}\PYG{p}{,} \PYG{n}{ax} \PYG{o}{=} \PYG{n}{plt}\PYG{o}{.}\PYG{n}{subplots}\PYG{p}{(}\PYG{l+m+mi}{1}\PYG{p}{,} \PYG{l+m+mi}{2}\PYG{p}{,} \PYG{n}{figsize}\PYG{o}{=}\PYG{p}{(}\PYG{l+m+mi}{10}\PYG{p}{,} \PYG{l+m+mi}{4}\PYG{p}{)}\PYG{p}{)}
\PYG{n}{sns}\PYG{o}{.}\PYG{n}{scatterplot}\PYG{p}{(}\PYG{n}{data}\PYG{o}{=}\PYG{n}{df}\PYG{p}{,} \PYG{n}{x}\PYG{o}{=}\PYG{l+s+s1}{\PYGZsq{}}\PYG{l+s+s1}{Longitud (cm)}\PYG{l+s+s1}{\PYGZsq{}}\PYG{p}{,} \PYG{n}{y}\PYG{o}{=}\PYG{l+s+s1}{\PYGZsq{}}\PYG{l+s+s1}{Peso (g)}\PYG{l+s+s1}{\PYGZsq{}}\PYG{p}{,} \PYG{n}{ax}\PYG{o}{=}\PYG{n}{ax}\PYG{p}{[}\PYG{l+m+mi}{0}\PYG{p}{]}\PYG{p}{,} \PYG{n}{alpha}\PYG{o}{=}\PYG{l+m+mf}{0.7}\PYG{p}{)}
\PYG{n}{ax}\PYG{p}{[}\PYG{l+m+mi}{0}\PYG{p}{]}\PYG{o}{.}\PYG{n}{set\PYGZus{}title}\PYG{p}{(}\PYG{l+s+s2}{\PYGZdq{}}\PYG{l+s+s2}{Peso vs Longitud (escala lineal)}\PYG{l+s+s2}{\PYGZdq{}}\PYG{p}{)}
\PYG{n}{sns}\PYG{o}{.}\PYG{n}{scatterplot}\PYG{p}{(}\PYG{n}{x}\PYG{o}{=}\PYG{n}{df\PYGZus{}log}\PYG{p}{[}\PYG{l+s+s1}{\PYGZsq{}}\PYG{l+s+s1}{log\PYGZus{}Longitud}\PYG{l+s+s1}{\PYGZsq{}}\PYG{p}{]}\PYG{p}{,} \PYG{n}{y}\PYG{o}{=}\PYG{n}{df\PYGZus{}log}\PYG{p}{[}\PYG{l+s+s1}{\PYGZsq{}}\PYG{l+s+s1}{log\PYGZus{}Peso}\PYG{l+s+s1}{\PYGZsq{}}\PYG{p}{]}\PYG{p}{,} \PYG{n}{ax}\PYG{o}{=}\PYG{n}{ax}\PYG{p}{[}\PYG{l+m+mi}{1}\PYG{p}{]}\PYG{p}{,} \PYG{n}{alpha}\PYG{o}{=}\PYG{l+m+mf}{0.7}\PYG{p}{,} \PYG{n}{color}\PYG{o}{=}\PYG{l+s+s2}{\PYGZdq{}}\PYG{l+s+s2}{\PYGZsh{}ff7f0e}\PYG{l+s+s2}{\PYGZdq{}}\PYG{p}{)}
\PYG{n}{ax}\PYG{p}{[}\PYG{l+m+mi}{1}\PYG{p}{]}\PYG{o}{.}\PYG{n}{set\PYGZus{}title}\PYG{p}{(}\PYG{l+s+s2}{\PYGZdq{}}\PYG{l+s+s2}{ln Peso vs ln Longitud}\PYG{l+s+s2}{\PYGZdq{}}\PYG{p}{)}
\PYG{n}{plt}\PYG{o}{.}\PYG{n}{tight\PYGZus{}layout}\PYG{p}{(}\PYG{p}{)}
\PYG{n}{plt}\PYG{o}{.}\PYG{n}{show}\PYG{p}{(}\PYG{p}{)}


\PYG{c+c1}{\PYGZsh{}\PYGZsh{} 3\PYGZhy{}c. Scatter Peso–Anchura: lineal vs log\PYGZhy{}log}
\PYG{n}{fig}\PYG{p}{,} \PYG{n}{ax} \PYG{o}{=} \PYG{n}{plt}\PYG{o}{.}\PYG{n}{subplots}\PYG{p}{(}\PYG{l+m+mi}{1}\PYG{p}{,} \PYG{l+m+mi}{2}\PYG{p}{,} \PYG{n}{figsize}\PYG{o}{=}\PYG{p}{(}\PYG{l+m+mi}{10}\PYG{p}{,} \PYG{l+m+mi}{4}\PYG{p}{)}\PYG{p}{)}
\PYG{n}{sns}\PYG{o}{.}\PYG{n}{scatterplot}\PYG{p}{(}\PYG{n}{data}\PYG{o}{=}\PYG{n}{df}\PYG{p}{,}
                \PYG{n}{x}\PYG{o}{=}\PYG{l+s+s1}{\PYGZsq{}}\PYG{l+s+s1}{Anchura (cm)}\PYG{l+s+s1}{\PYGZsq{}}\PYG{p}{,} \PYG{n}{y}\PYG{o}{=}\PYG{l+s+s1}{\PYGZsq{}}\PYG{l+s+s1}{Peso (g)}\PYG{l+s+s1}{\PYGZsq{}}\PYG{p}{,}
                \PYG{n}{alpha}\PYG{o}{=}\PYG{l+m+mf}{0.7}\PYG{p}{,} \PYG{n}{ax}\PYG{o}{=}\PYG{n}{ax}\PYG{p}{[}\PYG{l+m+mi}{0}\PYG{p}{]}\PYG{p}{)}
\PYG{n}{ax}\PYG{p}{[}\PYG{l+m+mi}{0}\PYG{p}{]}\PYG{o}{.}\PYG{n}{set\PYGZus{}title}\PYG{p}{(}\PYG{l+s+s2}{\PYGZdq{}}\PYG{l+s+s2}{Peso vs Anchura (escala lineal)}\PYG{l+s+s2}{\PYGZdq{}}\PYG{p}{)}
\PYG{n}{ax}\PYG{p}{[}\PYG{l+m+mi}{0}\PYG{p}{]}\PYG{o}{.}\PYG{n}{set\PYGZus{}xlabel}\PYG{p}{(}\PYG{l+s+s2}{\PYGZdq{}}\PYG{l+s+s2}{Anchura (cm)}\PYG{l+s+s2}{\PYGZdq{}}\PYG{p}{)}
\PYG{n}{ax}\PYG{p}{[}\PYG{l+m+mi}{0}\PYG{p}{]}\PYG{o}{.}\PYG{n}{set\PYGZus{}ylabel}\PYG{p}{(}\PYG{l+s+s2}{\PYGZdq{}}\PYG{l+s+s2}{Peso (g)}\PYG{l+s+s2}{\PYGZdq{}}\PYG{p}{)}
\PYG{n}{sns}\PYG{o}{.}\PYG{n}{scatterplot}\PYG{p}{(}\PYG{n}{x}\PYG{o}{=}\PYG{n}{df\PYGZus{}log}\PYG{p}{[}\PYG{l+s+s1}{\PYGZsq{}}\PYG{l+s+s1}{log\PYGZus{}Anchura}\PYG{l+s+s1}{\PYGZsq{}}\PYG{p}{]}\PYG{p}{,}
                \PYG{n}{y}\PYG{o}{=}\PYG{n}{df\PYGZus{}log}\PYG{p}{[}\PYG{l+s+s1}{\PYGZsq{}}\PYG{l+s+s1}{log\PYGZus{}Peso}\PYG{l+s+s1}{\PYGZsq{}}\PYG{p}{]}\PYG{p}{,}
                \PYG{n}{alpha}\PYG{o}{=}\PYG{l+m+mf}{0.7}\PYG{p}{,} \PYG{n}{color}\PYG{o}{=}\PYG{l+s+s2}{\PYGZdq{}}\PYG{l+s+s2}{\PYGZsh{}ff7f0e}\PYG{l+s+s2}{\PYGZdq{}}\PYG{p}{,} \PYG{n}{ax}\PYG{o}{=}\PYG{n}{ax}\PYG{p}{[}\PYG{l+m+mi}{1}\PYG{p}{]}\PYG{p}{)}
\PYG{n}{ax}\PYG{p}{[}\PYG{l+m+mi}{1}\PYG{p}{]}\PYG{o}{.}\PYG{n}{set\PYGZus{}title}\PYG{p}{(}\PYG{l+s+s2}{\PYGZdq{}}\PYG{l+s+s2}{ln Peso vs ln Anchura}\PYG{l+s+s2}{\PYGZdq{}}\PYG{p}{)}
\PYG{n}{ax}\PYG{p}{[}\PYG{l+m+mi}{1}\PYG{p}{]}\PYG{o}{.}\PYG{n}{set\PYGZus{}xlabel}\PYG{p}{(}\PYG{l+s+s2}{\PYGZdq{}}\PYG{l+s+s2}{ln Anchura (cm)}\PYG{l+s+s2}{\PYGZdq{}}\PYG{p}{)}
\PYG{n}{ax}\PYG{p}{[}\PYG{l+m+mi}{1}\PYG{p}{]}\PYG{o}{.}\PYG{n}{set\PYGZus{}ylabel}\PYG{p}{(}\PYG{l+s+s2}{\PYGZdq{}}\PYG{l+s+s2}{ln Peso (g)}\PYG{l+s+s2}{\PYGZdq{}}\PYG{p}{)}

\PYG{n}{plt}\PYG{o}{.}\PYG{n}{tight\PYGZus{}layout}\PYG{p}{(}\PYG{p}{)}
\PYG{n}{plt}\PYG{o}{.}\PYG{n}{show}\PYG{p}{(}\PYG{p}{)}
\end{sphinxVerbatim}

\end{sphinxuseclass}\end{sphinxVerbatimInput}
\begin{sphinxVerbatimOutput}

\begin{sphinxuseclass}{cell_output}
\noindent\sphinxincludegraphics{{6e1f732a76106d46938952ffcb00265e67332a1e2ea642da9ec31410bfcc4e3f}.png}

\noindent\sphinxincludegraphics{{8a0fc139d3ba479e16b219a8052ad0980bebb7681aa774dbf703d3e18428d317}.png}

\noindent\sphinxincludegraphics{{fcc20e9fd86b39e0b79ae283e0f75229a38189f9535ae2a946481916ecbeaa6f}.png}

\end{sphinxuseclass}\end{sphinxVerbatimOutput}

\end{sphinxuseclass}
\sphinxAtStartPar
Como se puede apreciar en las gráficas obtenidas, el uso de la transformada logarítmica produce una normalización de la distribución sesgada del peso y una linealización de las relaciones exponenciales de Peso vs. Longitud y Peso vs. Anchura. En el primer caso, normalizaciíón, esto nos va a permitir un mejor ajuste a modelos paramétricos. En el segundo caso, linealización, nos va a permitir aplicar unas regresión lineal multiparamétrica sencilla para poder encontar los factores \(k\), \(a\) y \(b\) que definen el modelo alométrico de la ecuación \eqref{equation:content/03/Coeficientes:eq_peso-longitud_anchura}.

\sphinxAtStartPar
Nuestro siguiente paso es detectar los \sphinxstyleemphasis{outliers} por los tres métodos usados con anterioridad (IRQ, Z\sphinxhyphen{}Scores e Isolation Forest) con los dtaos transformados.

\begin{sphinxuseclass}{cell}\begin{sphinxVerbatimInput}

\begin{sphinxuseclass}{cell_input}
\begin{sphinxVerbatim}[commandchars=\\\{\}]
\PYG{c+c1}{\PYGZsh{} \PYGZhy{}\PYGZhy{}\PYGZhy{}\PYGZhy{}\PYGZhy{}\PYGZhy{}\PYGZhy{}\PYGZhy{}\PYGZhy{}\PYGZhy{}\PYGZhy{}\PYGZhy{}\PYGZhy{}\PYGZhy{}\PYGZhy{}\PYGZhy{}\PYGZhy{}\PYGZhy{}\PYGZhy{}\PYGZhy{}\PYGZhy{}\PYGZhy{}\PYGZhy{}\PYGZhy{}\PYGZhy{}\PYGZhy{}\PYGZhy{}\PYGZhy{}\PYGZhy{}\PYGZhy{}\PYGZhy{}\PYGZhy{}\PYGZhy{}\PYGZhy{}\PYGZhy{}\PYGZhy{}\PYGZhy{}\PYGZhy{}\PYGZhy{}\PYGZhy{}\PYGZhy{}\PYGZhy{}\PYGZhy{}\PYGZhy{}\PYGZhy{}\PYGZhy{}\PYGZhy{}\PYGZhy{}\PYGZhy{}\PYGZhy{}\PYGZhy{}\PYGZhy{}\PYGZhy{}\PYGZhy{}\PYGZhy{}\PYGZhy{}\PYGZhy{}\PYGZhy{}\PYGZhy{}\PYGZhy{}\PYGZhy{}\PYGZhy{}\PYGZhy{}}
\PYG{c+c1}{\PYGZsh{} 4.  Detección de outliers en escala log}
\PYG{c+c1}{\PYGZsh{} \PYGZhy{}\PYGZhy{}\PYGZhy{}\PYGZhy{}\PYGZhy{}\PYGZhy{}\PYGZhy{}\PYGZhy{}\PYGZhy{}\PYGZhy{}\PYGZhy{}\PYGZhy{}\PYGZhy{}\PYGZhy{}\PYGZhy{}\PYGZhy{}\PYGZhy{}\PYGZhy{}\PYGZhy{}\PYGZhy{}\PYGZhy{}\PYGZhy{}\PYGZhy{}\PYGZhy{}\PYGZhy{}\PYGZhy{}\PYGZhy{}\PYGZhy{}\PYGZhy{}\PYGZhy{}\PYGZhy{}\PYGZhy{}\PYGZhy{}\PYGZhy{}\PYGZhy{}\PYGZhy{}\PYGZhy{}\PYGZhy{}\PYGZhy{}\PYGZhy{}\PYGZhy{}\PYGZhy{}\PYGZhy{}\PYGZhy{}\PYGZhy{}\PYGZhy{}\PYGZhy{}\PYGZhy{}\PYGZhy{}\PYGZhy{}\PYGZhy{}\PYGZhy{}\PYGZhy{}\PYGZhy{}\PYGZhy{}\PYGZhy{}\PYGZhy{}\PYGZhy{}\PYGZhy{}\PYGZhy{}\PYGZhy{}\PYGZhy{}\PYGZhy{}}
\PYG{k+kn}{from}\PYG{+w}{ }\PYG{n+nn}{scipy}\PYG{+w}{ }\PYG{k+kn}{import} \PYG{n}{stats}

\PYG{c+c1}{\PYGZsh{} Método IQR}
\PYG{k}{def}\PYG{+w}{ }\PYG{n+nf}{detect\PYGZus{}outliers\PYGZus{}iqr}\PYG{p}{(}\PYG{n}{col}\PYG{p}{,} \PYG{n}{k}\PYG{o}{=}\PYG{l+m+mf}{1.5}\PYG{p}{)}\PYG{p}{:}
    \PYG{n}{q1}\PYG{p}{,} \PYG{n}{q3} \PYG{o}{=} \PYG{n}{col}\PYG{o}{.}\PYG{n}{quantile}\PYG{p}{(}\PYG{p}{[}\PYG{l+m+mf}{0.25}\PYG{p}{,} \PYG{l+m+mf}{0.75}\PYG{p}{]}\PYG{p}{)}
    \PYG{n}{iqr} \PYG{o}{=} \PYG{n}{q3} \PYG{o}{\PYGZhy{}} \PYG{n}{q1}
    \PYG{k}{return} \PYG{p}{(}\PYG{n}{col} \PYG{o}{\PYGZlt{}} \PYG{n}{q1} \PYG{o}{\PYGZhy{}} \PYG{n}{k}\PYG{o}{*}\PYG{n}{iqr}\PYG{p}{)} \PYG{o}{|} \PYG{p}{(}\PYG{n}{col} \PYG{o}{\PYGZgt{}} \PYG{n}{q3} \PYG{o}{+} \PYG{n}{k}\PYG{o}{*}\PYG{n}{iqr}\PYG{p}{)}

\PYG{n}{mask\PYGZus{}iqr\PYGZus{}log} \PYG{o}{=} \PYG{p}{(}
    \PYG{n}{detect\PYGZus{}outliers\PYGZus{}iqr}\PYG{p}{(}\PYG{n}{df\PYGZus{}log}\PYG{p}{[}\PYG{l+s+s1}{\PYGZsq{}}\PYG{l+s+s1}{log\PYGZus{}Peso}\PYG{l+s+s1}{\PYGZsq{}}\PYG{p}{]}\PYG{p}{)} \PYG{o}{|}
    \PYG{n}{detect\PYGZus{}outliers\PYGZus{}iqr}\PYG{p}{(}\PYG{n}{df\PYGZus{}log}\PYG{p}{[}\PYG{l+s+s1}{\PYGZsq{}}\PYG{l+s+s1}{log\PYGZus{}Longitud}\PYG{l+s+s1}{\PYGZsq{}}\PYG{p}{]}\PYG{p}{)} \PYG{o}{|}
    \PYG{n}{detect\PYGZus{}outliers\PYGZus{}iqr}\PYG{p}{(}\PYG{n}{df\PYGZus{}log}\PYG{p}{[}\PYG{l+s+s1}{\PYGZsq{}}\PYG{l+s+s1}{log\PYGZus{}Anchura}\PYG{l+s+s1}{\PYGZsq{}}\PYG{p}{]}\PYG{p}{)}
\PYG{p}{)}

\PYG{c+c1}{\PYGZsh{} Método Z\PYGZhy{}score}
\PYG{n}{z\PYGZus{}scores} \PYG{o}{=} \PYG{n}{np}\PYG{o}{.}\PYG{n}{abs}\PYG{p}{(}\PYG{n}{stats}\PYG{o}{.}\PYG{n}{zscore}\PYG{p}{(}\PYG{n}{df\PYGZus{}log}\PYG{p}{[}\PYG{p}{[}\PYG{l+s+s1}{\PYGZsq{}}\PYG{l+s+s1}{log\PYGZus{}Peso}\PYG{l+s+s1}{\PYGZsq{}}\PYG{p}{,} \PYG{l+s+s1}{\PYGZsq{}}\PYG{l+s+s1}{log\PYGZus{}Longitud}\PYG{l+s+s1}{\PYGZsq{}}\PYG{p}{,} \PYG{l+s+s1}{\PYGZsq{}}\PYG{l+s+s1}{log\PYGZus{}Anchura}\PYG{l+s+s1}{\PYGZsq{}}\PYG{p}{]}\PYG{p}{]}\PYG{p}{)}\PYG{p}{)}
\PYG{n}{mask\PYGZus{}z\PYGZus{}log} \PYG{o}{=} \PYG{p}{(}\PYG{n}{z\PYGZus{}scores} \PYG{o}{\PYGZgt{}} \PYG{l+m+mi}{3}\PYG{p}{)}\PYG{o}{.}\PYG{n}{any}\PYG{p}{(}\PYG{n}{axis}\PYG{o}{=}\PYG{l+m+mi}{1}\PYG{p}{)}

\PYG{c+c1}{\PYGZsh{} Método Isolation Forest}
\PYG{n}{X\PYGZus{}scaled\PYGZus{}log} \PYG{o}{=} \PYG{n}{StandardScaler}\PYG{p}{(}\PYG{p}{)}\PYG{o}{.}\PYG{n}{fit\PYGZus{}transform}\PYG{p}{(}\PYG{n}{df\PYGZus{}log}\PYG{p}{[}\PYG{p}{[}\PYG{l+s+s1}{\PYGZsq{}}\PYG{l+s+s1}{log\PYGZus{}Longitud}\PYG{l+s+s1}{\PYGZsq{}}\PYG{p}{,} \PYG{l+s+s1}{\PYGZsq{}}\PYG{l+s+s1}{log\PYGZus{}Anchura}\PYG{l+s+s1}{\PYGZsq{}}\PYG{p}{,} \PYG{l+s+s1}{\PYGZsq{}}\PYG{l+s+s1}{log\PYGZus{}Peso}\PYG{l+s+s1}{\PYGZsq{}}\PYG{p}{]}\PYG{p}{]}\PYG{p}{)}
\PYG{n}{iso\PYGZus{}forest} \PYG{o}{=} \PYG{n}{IsolationForest}\PYG{p}{(}\PYG{n}{contamination}\PYG{o}{=}\PYG{l+s+s1}{\PYGZsq{}}\PYG{l+s+s1}{auto}\PYG{l+s+s1}{\PYGZsq{}}\PYG{p}{,}\PYG{n}{random\PYGZus{}state}\PYG{o}{=}\PYG{l+m+mi}{42}\PYG{p}{)}
\PYG{n}{mask\PYGZus{}if\PYGZus{}log} \PYG{o}{=} \PYG{n}{iso\PYGZus{}forest}\PYG{o}{.}\PYG{n}{fit\PYGZus{}predict}\PYG{p}{(}\PYG{n}{X\PYGZus{}scaled\PYGZus{}log}\PYG{p}{)} \PYG{o}{==} \PYG{o}{\PYGZhy{}}\PYG{l+m+mi}{1}

\PYG{c+c1}{\PYGZsh{} Mostrar resultados}
\PYG{n}{df\PYGZus{}log}\PYG{p}{[}\PYG{l+s+s1}{\PYGZsq{}}\PYG{l+s+s1}{out\PYGZus{}IQR\PYGZus{}log}\PYG{l+s+s1}{\PYGZsq{}}\PYG{p}{]} \PYG{o}{=} \PYG{n}{mask\PYGZus{}iqr\PYGZus{}log}
\PYG{n+nb}{print}\PYG{p}{(}\PYG{l+s+s2}{\PYGZdq{}}\PYG{l+s+s2}{Outliers método IQR}\PYG{l+s+s2}{\PYGZdq{}}\PYG{p}{)}
\PYG{n}{display}\PYG{p}{(}\PYG{n}{df\PYGZus{}log}\PYG{p}{[}\PYG{n}{df\PYGZus{}log}\PYG{p}{[}\PYG{l+s+s1}{\PYGZsq{}}\PYG{l+s+s1}{out\PYGZus{}IQR\PYGZus{}log}\PYG{l+s+s1}{\PYGZsq{}}\PYG{p}{]}\PYG{p}{]}\PYG{p}{)}

\PYG{n}{df\PYGZus{}log}\PYG{p}{[}\PYG{l+s+s1}{\PYGZsq{}}\PYG{l+s+s1}{out\PYGZus{}Z\PYGZus{}log}\PYG{l+s+s1}{\PYGZsq{}}\PYG{p}{]}   \PYG{o}{=} \PYG{n}{mask\PYGZus{}z\PYGZus{}log}
\PYG{n+nb}{print}\PYG{p}{(}\PYG{l+s+s2}{\PYGZdq{}}\PYG{l+s+s2}{Outliers método Z\PYGZhy{}Score}\PYG{l+s+s2}{\PYGZdq{}}\PYG{p}{)}
\PYG{n}{display}\PYG{p}{(}\PYG{n}{df\PYGZus{}log}\PYG{p}{[}\PYG{n}{df\PYGZus{}log}\PYG{p}{[}\PYG{l+s+s1}{\PYGZsq{}}\PYG{l+s+s1}{out\PYGZus{}Z\PYGZus{}log}\PYG{l+s+s1}{\PYGZsq{}}\PYG{p}{]}\PYG{p}{]}\PYG{p}{)}

\PYG{n+nb}{print}\PYG{p}{(}\PYG{l+s+s2}{\PYGZdq{}}\PYG{l+s+s2}{Outliers método Isolation Forest}\PYG{l+s+s2}{\PYGZdq{}}\PYG{p}{)}
\PYG{n}{df\PYGZus{}log}\PYG{p}{[}\PYG{l+s+s1}{\PYGZsq{}}\PYG{l+s+s1}{out\PYGZus{}IF\PYGZus{}log}\PYG{l+s+s1}{\PYGZsq{}}\PYG{p}{]}  \PYG{o}{=} \PYG{n}{mask\PYGZus{}if\PYGZus{}log}
\PYG{n}{display}\PYG{p}{(}\PYG{n}{df\PYGZus{}log}\PYG{p}{[}\PYG{n}{df\PYGZus{}log}\PYG{p}{[}\PYG{l+s+s1}{\PYGZsq{}}\PYG{l+s+s1}{out\PYGZus{}IF\PYGZus{}log}\PYG{l+s+s1}{\PYGZsq{}}\PYG{p}{]}\PYG{p}{]}\PYG{p}{)}
\end{sphinxVerbatim}

\end{sphinxuseclass}\end{sphinxVerbatimInput}
\begin{sphinxVerbatimOutput}

\begin{sphinxuseclass}{cell_output}
\begin{sphinxVerbatim}[commandchars=\\\{\}]
Outliers método IQR
\end{sphinxVerbatim}

\begin{sphinxVerbatim}[commandchars=\\\{\}]
   Peso (g)  Longitud (cm)  Anchura (cm)  log\PYGZus{}Peso  log\PYGZus{}Longitud  log\PYGZus{}Anchura  \PYGZbs{}
0      0.46            3.3           1.3 \PYGZhy{}0.776529      1.193922     0.262364
1      1.08            4.5           1.1  0.076961      1.504077     0.095310
2      0.67            3.9           1.5 \PYGZhy{}0.400478      1.360977     0.405465

   out\PYGZus{}IQR\PYGZus{}log
0         True
1         True
2         True
\end{sphinxVerbatim}

\begin{sphinxVerbatim}[commandchars=\\\{\}]
Outliers método Z\PYGZhy{}Score
\end{sphinxVerbatim}

\begin{sphinxVerbatim}[commandchars=\\\{\}]
   Peso (g)  Longitud (cm)  Anchura (cm)  log\PYGZus{}Peso  log\PYGZus{}Longitud  log\PYGZus{}Anchura  \PYGZbs{}
0      0.46            3.3           1.3 \PYGZhy{}0.776529      1.193922     0.262364
1      1.08            4.5           1.1  0.076961      1.504077     0.095310

   out\PYGZus{}IQR\PYGZus{}log  out\PYGZus{}Z\PYGZus{}log
0         True       True
1         True       True
\end{sphinxVerbatim}

\begin{sphinxVerbatim}[commandchars=\\\{\}]
Outliers método Isolation Forest
\end{sphinxVerbatim}

\begin{sphinxVerbatim}[commandchars=\\\{\}]
     Peso (g)  Longitud (cm)  Anchura (cm)  log\PYGZus{}Peso  log\PYGZus{}Longitud  \PYGZbs{}
0        0.46            3.3           1.3 \PYGZhy{}0.776529      1.193922
1        1.08            4.5           1.1  0.076961      1.504077
2        0.67            3.9           1.5 \PYGZhy{}0.400478      1.360977
3        0.98            4.4           1.7 \PYGZhy{}0.020203      1.481605
4        0.93            4.2           1.8 \PYGZhy{}0.072571      1.435085
5        1.89            4.5           2.0  0.636577      1.504077
6        1.60            5.1           1.8  0.470004      1.629241
7        1.90            5.0           2.0  0.641854      1.609438
8        1.59            5.4           1.9  0.463734      1.686399
42       2.83            5.2           2.5  1.040277      1.648659
62       4.29            7.0           2.1  1.456287      1.945910
92       3.06            6.0           2.9  1.118415      1.791759
153      3.40            6.8           3.9  1.223775      1.916923
156      4.50            7.4           3.7  1.504077      2.001480
183     10.08            7.8           4.2  2.310553      2.054124
195     11.22           10.1           3.9  2.417698      2.312535
200     14.13           10.1           4.1  2.648300      2.312535
201     14.04           10.5           4.1  2.641910      2.351375
202     14.67           10.7           4.2  2.685805      2.370244
203     15.65           10.7           4.5  2.750471      2.370244
204     14.50           10.8           4.5  2.674149      2.379546
205     19.88           11.4           4.4  2.989714      2.433613
206     16.47           11.0           4.6  2.801541      2.397895
207     17.04           10.6           4.8  2.835564      2.360854
208     21.98           10.6           5.2  3.090133      2.360854

     log\PYGZus{}Anchura  out\PYGZus{}IQR\PYGZus{}log  out\PYGZus{}Z\PYGZus{}log  out\PYGZus{}IF\PYGZus{}log
0       0.262364         True       True        True
1       0.095310         True       True        True
2       0.405465         True      False        True
3       0.530628        False      False        True
4       0.587787        False      False        True
5       0.693147        False      False        True
6       0.587787        False      False        True
7       0.693147        False      False        True
8       0.641854        False      False        True
42      0.916291        False      False        True
62      0.741937        False      False        True
92      1.064711        False      False        True
153     1.360977        False      False        True
156     1.308333        False      False        True
183     1.435085        False      False        True
195     1.360977        False      False        True
200     1.410987        False      False        True
201     1.410987        False      False        True
202     1.435085        False      False        True
203     1.504077        False      False        True
204     1.504077        False      False        True
205     1.481605        False      False        True
206     1.526056        False      False        True
207     1.568616        False      False        True
208     1.648659        False      False        True
\end{sphinxVerbatim}

\end{sphinxuseclass}\end{sphinxVerbatimOutput}

\end{sphinxuseclass}
\begin{sphinxuseclass}{cell}\begin{sphinxVerbatimInput}

\begin{sphinxuseclass}{cell_input}
\begin{sphinxVerbatim}[commandchars=\\\{\}]
\PYG{c+c1}{\PYGZsh{} ===============================================================}
\PYG{c+c1}{\PYGZsh{}  Comparativa de intersecciones entre los tres métodos (log)}
\PYG{c+c1}{\PYGZsh{} ===============================================================}

\PYG{c+c1}{\PYGZsh{} \PYGZhy{}\PYGZhy{}\PYGZhy{} Conjuntos de índices por método \PYGZhy{}\PYGZhy{}\PYGZhy{}}
\PYG{n}{set\PYGZus{}iqr\PYGZus{}log} \PYG{o}{=} \PYG{n+nb}{set}\PYG{p}{(}\PYG{n}{df\PYGZus{}log}\PYG{o}{.}\PYG{n}{index}\PYG{p}{[}\PYG{n}{df\PYGZus{}log}\PYG{p}{[}\PYG{l+s+s1}{\PYGZsq{}}\PYG{l+s+s1}{out\PYGZus{}IQR\PYGZus{}log}\PYG{l+s+s1}{\PYGZsq{}}\PYG{p}{]}\PYG{p}{]}\PYG{p}{)}
\PYG{n}{set\PYGZus{}z\PYGZus{}log}   \PYG{o}{=} \PYG{n+nb}{set}\PYG{p}{(}\PYG{n}{df\PYGZus{}log}\PYG{o}{.}\PYG{n}{index}\PYG{p}{[}\PYG{n}{df\PYGZus{}log}\PYG{p}{[}\PYG{l+s+s1}{\PYGZsq{}}\PYG{l+s+s1}{out\PYGZus{}Z\PYGZus{}log}\PYG{l+s+s1}{\PYGZsq{}}\PYG{p}{]}\PYG{p}{]}\PYG{p}{)}
\PYG{n}{set\PYGZus{}if\PYGZus{}log}  \PYG{o}{=} \PYG{n+nb}{set}\PYG{p}{(}\PYG{n}{df\PYGZus{}log}\PYG{o}{.}\PYG{n}{index}\PYG{p}{[}\PYG{n}{df\PYGZus{}log}\PYG{p}{[}\PYG{l+s+s1}{\PYGZsq{}}\PYG{l+s+s1}{out\PYGZus{}IF\PYGZus{}log}\PYG{l+s+s1}{\PYGZsq{}}\PYG{p}{]}\PYG{p}{]}\PYG{p}{)}

\PYG{c+c1}{\PYGZsh{} \PYGZhy{}\PYGZhy{}\PYGZhy{} DataFrame que contiene únicamente los registros marcados por algún método \PYGZhy{}\PYGZhy{}\PYGZhy{}}
\PYG{n}{df\PYGZus{}log}\PYG{p}{[}\PYG{l+s+s1}{\PYGZsq{}}\PYG{l+s+s1}{out\PYGZus{}any\PYGZus{}log}\PYG{l+s+s1}{\PYGZsq{}}\PYG{p}{]} \PYG{o}{=} \PYG{n}{df\PYGZus{}log}\PYG{p}{[}\PYG{p}{[}\PYG{l+s+s1}{\PYGZsq{}}\PYG{l+s+s1}{out\PYGZus{}IQR\PYGZus{}log}\PYG{l+s+s1}{\PYGZsq{}}\PYG{p}{,}\PYG{l+s+s1}{\PYGZsq{}}\PYG{l+s+s1}{out\PYGZus{}Z\PYGZus{}log}\PYG{l+s+s1}{\PYGZsq{}}\PYG{p}{,}\PYG{l+s+s1}{\PYGZsq{}}\PYG{l+s+s1}{out\PYGZus{}IF\PYGZus{}log}\PYG{l+s+s1}{\PYGZsq{}}\PYG{p}{]}\PYG{p}{]}\PYG{o}{.}\PYG{n}{any}\PYG{p}{(}\PYG{n}{axis}\PYG{o}{=}\PYG{l+m+mi}{1}\PYG{p}{)}

\PYG{n}{df\PYGZus{}outliers} \PYG{o}{=} \PYG{n}{df\PYGZus{}log}\PYG{p}{[}\PYG{n}{df\PYGZus{}log}\PYG{p}{[}\PYG{l+s+s1}{\PYGZsq{}}\PYG{l+s+s1}{out\PYGZus{}any\PYGZus{}log}\PYG{l+s+s1}{\PYGZsq{}}\PYG{p}{]}\PYG{p}{]}

\PYG{c+c1}{\PYGZsh{} \PYGZhy{}\PYGZhy{}\PYGZhy{} Construimos un array con las cuatro intersecciones \PYGZhy{}\PYGZhy{}\PYGZhy{}}
\PYG{n}{idx\PYGZus{}common} \PYG{o}{=} \PYG{n}{np}\PYG{o}{.}\PYG{n}{array}\PYG{p}{(}\PYG{p}{[}
    \PYG{n+nb}{sorted}\PYG{p}{(}\PYG{n}{set\PYGZus{}iqr\PYGZus{}log} \PYG{o}{\PYGZam{}} \PYG{n}{set\PYGZus{}z\PYGZus{}log}\PYG{p}{)}\PYG{p}{,}                \PYG{c+c1}{\PYGZsh{} IQR ∩ Z\PYGZhy{}Score}
    \PYG{n+nb}{sorted}\PYG{p}{(}\PYG{n}{set\PYGZus{}iqr\PYGZus{}log} \PYG{o}{\PYGZam{}} \PYG{n}{set\PYGZus{}if\PYGZus{}log}\PYG{p}{)}\PYG{p}{,}               \PYG{c+c1}{\PYGZsh{} IQR ∩ IF}
    \PYG{n+nb}{sorted}\PYG{p}{(}\PYG{n}{set\PYGZus{}z\PYGZus{}log}   \PYG{o}{\PYGZam{}} \PYG{n}{set\PYGZus{}if\PYGZus{}log}\PYG{p}{)}\PYG{p}{,}               \PYG{c+c1}{\PYGZsh{} Z\PYGZhy{}Score ∩ IF}
    \PYG{n+nb}{sorted}\PYG{p}{(}\PYG{n}{set\PYGZus{}iqr\PYGZus{}log} \PYG{o}{\PYGZam{}} \PYG{n}{set\PYGZus{}z\PYGZus{}log} \PYG{o}{\PYGZam{}} \PYG{n}{set\PYGZus{}if\PYGZus{}log}\PYG{p}{)}    \PYG{c+c1}{\PYGZsh{} IQR ∩ Z\PYGZhy{}Score ∩ IF}
\PYG{p}{]}\PYG{p}{,} \PYG{n}{dtype}\PYG{o}{=}\PYG{n+nb}{object}\PYG{p}{)}   \PYG{c+c1}{\PYGZsh{} dtype=object permite listas de longitud variable}

\PYG{n+nb}{print}\PYG{p}{(}\PYG{l+s+s2}{\PYGZdq{}}\PYG{l+s+s2}{─ Intersecciones (escala log) ─}\PYG{l+s+se}{\PYGZbs{}n}\PYG{l+s+s2}{\PYGZdq{}}\PYG{p}{)}

\PYG{n+nb}{print}\PYG{p}{(}\PYG{l+s+sa}{f}\PYG{l+s+s2}{\PYGZdq{}}\PYG{l+s+s2}{IQR ∩ Z\PYGZhy{}Score              : }\PYG{l+s+si}{\PYGZob{}}\PYG{n+nb}{len}\PYG{p}{(}\PYG{n}{idx\PYGZus{}common}\PYG{p}{[}\PYG{l+m+mi}{0}\PYG{p}{]}\PYG{p}{)}\PYG{l+s+si}{\PYGZcb{}}\PYG{l+s+s2}{\PYGZdq{}}\PYG{p}{)}
\PYG{n}{display}\PYG{p}{(}\PYG{n}{df\PYGZus{}outliers}\PYG{o}{.}\PYG{n}{loc}\PYG{p}{[}\PYG{n}{idx\PYGZus{}common}\PYG{p}{[}\PYG{l+m+mi}{0}\PYG{p}{]}\PYG{p}{]}\PYG{p}{)}

\PYG{n+nb}{print}\PYG{p}{(}\PYG{l+s+sa}{f}\PYG{l+s+s2}{\PYGZdq{}}\PYG{l+s+s2}{IQR ∩ Isolation Forest     : }\PYG{l+s+si}{\PYGZob{}}\PYG{n+nb}{len}\PYG{p}{(}\PYG{n}{idx\PYGZus{}common}\PYG{p}{[}\PYG{l+m+mi}{1}\PYG{p}{]}\PYG{p}{)}\PYG{l+s+si}{\PYGZcb{}}\PYG{l+s+s2}{\PYGZdq{}}\PYG{p}{)}
\PYG{n}{display}\PYG{p}{(}\PYG{n}{df\PYGZus{}outliers}\PYG{o}{.}\PYG{n}{loc}\PYG{p}{[}\PYG{n}{idx\PYGZus{}common}\PYG{p}{[}\PYG{l+m+mi}{1}\PYG{p}{]}\PYG{p}{]}\PYG{p}{)}

\PYG{n+nb}{print}\PYG{p}{(}\PYG{l+s+sa}{f}\PYG{l+s+s2}{\PYGZdq{}}\PYG{l+s+s2}{Z\PYGZhy{}Score ∩ Isolation Forest : }\PYG{l+s+si}{\PYGZob{}}\PYG{n+nb}{len}\PYG{p}{(}\PYG{n}{idx\PYGZus{}common}\PYG{p}{[}\PYG{l+m+mi}{2}\PYG{p}{]}\PYG{p}{)}\PYG{l+s+si}{\PYGZcb{}}\PYG{l+s+s2}{\PYGZdq{}}\PYG{p}{)}
\PYG{n}{display}\PYG{p}{(}\PYG{n}{df\PYGZus{}outliers}\PYG{o}{.}\PYG{n}{loc}\PYG{p}{[}\PYG{n}{idx\PYGZus{}common}\PYG{p}{[}\PYG{l+m+mi}{2}\PYG{p}{]}\PYG{p}{]}\PYG{p}{)}

\PYG{n+nb}{print}\PYG{p}{(}\PYG{l+s+sa}{f}\PYG{l+s+s2}{\PYGZdq{}}\PYG{l+s+s2}{IQR ∩ Z\PYGZhy{}Score ∩ IF         : }\PYG{l+s+si}{\PYGZob{}}\PYG{n+nb}{len}\PYG{p}{(}\PYG{n}{idx\PYGZus{}common}\PYG{p}{[}\PYG{l+m+mi}{3}\PYG{p}{]}\PYG{p}{)}\PYG{l+s+si}{\PYGZcb{}}\PYG{l+s+s2}{\PYGZdq{}}\PYG{p}{)}
\PYG{n}{display}\PYG{p}{(}\PYG{n}{df\PYGZus{}outliers}\PYG{o}{.}\PYG{n}{loc}\PYG{p}{[}\PYG{n}{idx\PYGZus{}common}\PYG{p}{[}\PYG{l+m+mi}{3}\PYG{p}{]}\PYG{p}{]}\PYG{p}{)}
\end{sphinxVerbatim}

\end{sphinxuseclass}\end{sphinxVerbatimInput}
\begin{sphinxVerbatimOutput}

\begin{sphinxuseclass}{cell_output}
\begin{sphinxVerbatim}[commandchars=\\\{\}]
─ Intersecciones (escala log) ─

IQR ∩ Z\PYGZhy{}Score              : 2
\end{sphinxVerbatim}

\begin{sphinxVerbatim}[commandchars=\\\{\}]
   Peso (g)  Longitud (cm)  Anchura (cm)  log\PYGZus{}Peso  log\PYGZus{}Longitud  log\PYGZus{}Anchura  \PYGZbs{}
0      0.46            3.3           1.3 \PYGZhy{}0.776529      1.193922     0.262364
1      1.08            4.5           1.1  0.076961      1.504077     0.095310

   out\PYGZus{}IQR\PYGZus{}log  out\PYGZus{}Z\PYGZus{}log  out\PYGZus{}IF\PYGZus{}log  out\PYGZus{}any\PYGZus{}log
0         True       True        True         True
1         True       True        True         True
\end{sphinxVerbatim}

\begin{sphinxVerbatim}[commandchars=\\\{\}]
IQR ∩ Isolation Forest     : 3
\end{sphinxVerbatim}

\begin{sphinxVerbatim}[commandchars=\\\{\}]
   Peso (g)  Longitud (cm)  Anchura (cm)  log\PYGZus{}Peso  log\PYGZus{}Longitud  log\PYGZus{}Anchura  \PYGZbs{}
0      0.46            3.3           1.3 \PYGZhy{}0.776529      1.193922     0.262364
1      1.08            4.5           1.1  0.076961      1.504077     0.095310
2      0.67            3.9           1.5 \PYGZhy{}0.400478      1.360977     0.405465

   out\PYGZus{}IQR\PYGZus{}log  out\PYGZus{}Z\PYGZus{}log  out\PYGZus{}IF\PYGZus{}log  out\PYGZus{}any\PYGZus{}log
0         True       True        True         True
1         True       True        True         True
2         True      False        True         True
\end{sphinxVerbatim}

\begin{sphinxVerbatim}[commandchars=\\\{\}]
Z\PYGZhy{}Score ∩ Isolation Forest : 2
\end{sphinxVerbatim}

\begin{sphinxVerbatim}[commandchars=\\\{\}]
   Peso (g)  Longitud (cm)  Anchura (cm)  log\PYGZus{}Peso  log\PYGZus{}Longitud  log\PYGZus{}Anchura  \PYGZbs{}
0      0.46            3.3           1.3 \PYGZhy{}0.776529      1.193922     0.262364
1      1.08            4.5           1.1  0.076961      1.504077     0.095310

   out\PYGZus{}IQR\PYGZus{}log  out\PYGZus{}Z\PYGZus{}log  out\PYGZus{}IF\PYGZus{}log  out\PYGZus{}any\PYGZus{}log
0         True       True        True         True
1         True       True        True         True
\end{sphinxVerbatim}

\begin{sphinxVerbatim}[commandchars=\\\{\}]
IQR ∩ Z\PYGZhy{}Score ∩ IF         : 2
\end{sphinxVerbatim}

\begin{sphinxVerbatim}[commandchars=\\\{\}]
   Peso (g)  Longitud (cm)  Anchura (cm)  log\PYGZus{}Peso  log\PYGZus{}Longitud  log\PYGZus{}Anchura  \PYGZbs{}
0      0.46            3.3           1.3 \PYGZhy{}0.776529      1.193922     0.262364
1      1.08            4.5           1.1  0.076961      1.504077     0.095310

   out\PYGZus{}IQR\PYGZus{}log  out\PYGZus{}Z\PYGZus{}log  out\PYGZus{}IF\PYGZus{}log  out\PYGZus{}any\PYGZus{}log
0         True       True        True         True
1         True       True        True         True
\end{sphinxVerbatim}

\end{sphinxuseclass}\end{sphinxVerbatimOutput}

\end{sphinxuseclass}
\sphinxAtStartPar
Los tres métodos usados coinciden en marcar como \sphinxstyleemphasis{outliers} los registros con índice \(0\) y \(1\) coincidente, en gran parte, con la gráfica de distribucion de peso en la transformada logaritmica.

\sphinxAtStartPar
Hay que tener en cuenta que en la deteccion de valores anómalos con los datos originales, el peso aumenta de manera multiplicativa con la longitud y la anchura: a medida que los peces crecen, la variabilidad absoluta de sus masas se dispara. Los métodos clásicos de detección (IQR, Z\sphinxhyphen{}score o Isolation Forest) se basan en distancias lineales; por eso señalan como atípicos a los cuatro ejemplares más pesados: su desviación, medida en gramos, es mucho mayor que la de cualquier juvenil.

\sphinxAtStartPar
Al pasar los datos a escala logarítmica convertimos esa relación multiplicativa en una relación aditiva y hacemos que la varianza sea prácticamente constante en todo el rango de tallas. La compresión de la cola derecha y la expansión de la izquierda cambian el foco: los adultos dejan de parecer extremos, mientras que tres peces muy pequeños y sorprendentemente ligeros sobresalen ahora como los casos más raros. Ese cambio de perspectiva —bien descrito en la literatura sobre transformaciones varianza\sphinxhyphen{}estabilizadoras \sphinxhref{https://doi.org/10.1201/9780203738535}{{[}Carroll et al., 1988{]}}
y en estudios alométricos {[}\sphinxhref{https://doi.org/10.1016/j.fishres.2012.02.001}{Torres et al., 2012}; \sphinxhref{https://doi.org/10.1111/j.1439-0426.2006.00805.x}{Froese, 2006}; \sphinxhref{https://doi.org/10.1111/j.1439-0426.2003.00480.x}{Borges et al., 2003}{]}— confirma que el log reduce los falsos positivos entre los grandes pero realza desvíos porcentuales en los pequeños.

\sphinxAtStartPar
\sphinxstylestrong{¿Qué hacer con estos dos registros?}

\sphinxAtStartPar
Su inclusion en una regresión log\sphinxhyphen{}log podría sesgar los parámetros \(k\), \(a\) y \(b\) y ensanchar sus intervalos de confianza. En ausencia de evidencia biológica a favor de su inclusión, la práctica más segura para inferir un modelo alométrico por OLS es excluir esos puntos o, al menos, contrastar los coeficientes obtenidos con y sin ellos para demostrar que las conclusiones no dependen de casos extremos.

\sphinxstepscope


\section{Algoritmo de inferencia del peso \protect\(W\protect\)}
\label{\detokenize{content/03/Modelo:algoritmo-de-inferencia-del-peso-w}}\label{\detokenize{content/03/Modelo::doc}}
\sphinxAtStartPar
A partir de los estudios matematicos realizados en los capítulos anteriores podemos desarrollar un \sphinxstyleemphasis{pipeline} que dados los valores de longitud \(L\) y anchura \(A\) de un lenguado obtenidos por medición directa o mediante visión artificial obtengamos el peso estimado. La trasnformada logaritmica produce una normalización de la distribución sesgada del peso y una linealización de las relaciones exponenciales de Peso vs. Longitud y Peso vs. Anchura, lo que nos permite usar el método \sphinxcode{\sphinxupquote{LinearRegression}} de la librería \sphinxcode{\sphinxupquote{scikit\sphinxhyphen{}learn}}. Esta librería ofrece implementaciones listas para usar de los algoritmos más frecuentes en clasificación, regresión, clustering, reducción de dimensión, validación de modelos y preprocesado, todo ello bajo una API unificada que se resume en los métodos fit, predict y transform. Esta coherencia facilita encadenar pasos mediante \sphinxstyleemphasis{pipelines}, realizar búsquedas de hiperparámetros y/o estimar el error de generalización.

\sphinxAtStartPar
En aprendizaje automático, el \sphinxstylestrong{error de generalización} (también denominado \sphinxstyleemphasis{riesgo esperado}) es la discrepancia estadística entre las predicciones de un modelo y los valores reales que se observarían si el modelo se aplicase a todo el universo de datos posibles, no sólo a la muestra con la que fue entrenado. Este error uede estimarse empíricamente y, a la vez, reducirse mediante validación cruzada K\sphinxhyphen{}Fold. Esta técnica particiona aleatoriamenteel conjunto de datos en \(K\) pliegues de tamaño similar, cada iteración reserva un pliegue como “muestra externa” y entrena el modelo con los \(K – 1\) restantes, de modo que todas las observaciones se emplean sucesivamente como datos no vistos; el promedio de la métrica sobre los \(K\) ciclos proporciona una estimación casi insesgada del riesgo esperado y presenta menor varianza que un único \sphinxstyleemphasis{hold\sphinxhyphen{}out}. Además, esa estructura repetida permite comparar hiperparámetros, arquitecturas o transformaciones y seleccionar la configuración que minimiza la pérdida promedio en los pliegues, lo cual actúa como un proceso de regularización implícita que atenúa el sobreajuste y, por ende, disminuye el error de generalización del modelo final entrenado con el conjunto completo.

\begin{sphinxadmonition}{note}{¿Que és \sphinxstyleemphasis{hold\sphinxhyphen{}out?}}

\sphinxAtStartPar
En aprendizaje automático, un \sphinxstyleemphasis{hold\sphinxhyphen{}out} es una estrategia de validación que consiste en apartar de forma permanente una fracción del dataset —típicamente entre el 20 \% y el 30 \%— para usarla exclusivamente como conjunto de prueba (test set). El resto de las observaciones se emplea para entrenar el modelo (y opcionalmente para ajustar hiperparámetros mediante un subconjunto de validación interno). Durante el entrenamiento el modelo nunca “ve” los datos del \sphinxstyleemphasis{hold\sphinxhyphen{}out}; así, el rendimiento calculado sobre ese bloque reservado ofrece una estimación objetiva de su capacidad de generalizar a datos completamente nuevos. El método es sencillo y eficiente cuando se dispone de un volumen de datos muy grande, pero puede ser inestable en muestras moderadas o pequeñas, ya que la estimación depende fuertemente de cómo se haya realizado la partición; por esa razón suele sustituirse o complementarse con técnicas más robustas como la validación cruzada K\sphinxhyphen{}Fold.
\end{sphinxadmonition}

\begin{sphinxuseclass}{cell}\begin{sphinxVerbatimInput}

\begin{sphinxuseclass}{cell_input}
\begin{sphinxVerbatim}[commandchars=\\\{\}]
\PYG{c+c1}{\PYGZsh{} ==============================}
\PYG{c+c1}{\PYGZsh{} 1. CARGA DE LIBRERÍAS}
\PYG{c+c1}{\PYGZsh{} ==============================}
\PYG{k+kn}{import}\PYG{+w}{ }\PYG{n+nn}{numpy}\PYG{+w}{ }\PYG{k}{as}\PYG{+w}{ }\PYG{n+nn}{np}
\PYG{k+kn}{import}\PYG{+w}{ }\PYG{n+nn}{pandas}\PYG{+w}{ }\PYG{k}{as}\PYG{+w}{ }\PYG{n+nn}{pd}
\PYG{k+kn}{from}\PYG{+w}{ }\PYG{n+nn}{sklearn}\PYG{n+nn}{.}\PYG{n+nn}{linear\PYGZus{}model}\PYG{+w}{ }\PYG{k+kn}{import} \PYG{n}{LinearRegression}
\PYG{k+kn}{from}\PYG{+w}{ }\PYG{n+nn}{sklearn}\PYG{n+nn}{.}\PYG{n+nn}{model\PYGZus{}selection}\PYG{+w}{ }\PYG{k+kn}{import} \PYG{n}{KFold}\PYG{p}{,} \PYG{n}{cross\PYGZus{}val\PYGZus{}score}
\PYG{k+kn}{from}\PYG{+w}{ }\PYG{n+nn}{sklearn}\PYG{n+nn}{.}\PYG{n+nn}{metrics}\PYG{+w}{ }\PYG{k+kn}{import} \PYG{n}{mean\PYGZus{}squared\PYGZus{}error}\PYG{p}{,} \PYG{n}{make\PYGZus{}scorer}
\PYG{k+kn}{import}\PYG{+w}{ }\PYG{n+nn}{statsmodels}\PYG{n+nn}{.}\PYG{n+nn}{api}\PYG{+w}{ }\PYG{k}{as}\PYG{+w}{ }\PYG{n+nn}{sm}            \PYG{c+c1}{\PYGZsh{} para diagnóstico opcional}
\PYG{k+kn}{import}\PYG{+w}{ }\PYG{n+nn}{matplotlib}\PYG{n+nn}{.}\PYG{n+nn}{pyplot}\PYG{+w}{ }\PYG{k}{as}\PYG{+w}{ }\PYG{n+nn}{plt}

\PYG{c+c1}{\PYGZsh{} ==============================}
\PYG{c+c1}{\PYGZsh{} 2. LECTURA DEL DATASET + LIMPIEZA}
\PYG{c+c1}{\PYGZsh{} ==============================}
\PYG{n}{df} \PYG{o}{=} \PYG{n}{pd}\PYG{o}{.}\PYG{n}{read\PYGZus{}excel}\PYG{p}{(}\PYG{l+s+s1}{\PYGZsq{}}\PYG{l+s+s1}{.././data/Dimensiones\PYGZus{}lenguado.xlsx}\PYG{l+s+s1}{\PYGZsq{}}\PYG{p}{)}

\PYG{c+c1}{\PYGZsh{} Eliminamos los outliers detectados con IQR, Z\PYGZhy{}Score e Isolation Forest}
\PYG{n}{df} \PYG{o}{=} \PYG{n}{df}\PYG{o}{.}\PYG{n}{drop}\PYG{p}{(}\PYG{n}{index}\PYG{o}{=}\PYG{p}{[}\PYG{l+m+mi}{0}\PYG{p}{,} \PYG{l+m+mi}{1}\PYG{p}{]}\PYG{p}{)}\PYG{o}{.}\PYG{n}{reset\PYGZus{}index}\PYG{p}{(}\PYG{n}{drop}\PYG{o}{=}\PYG{k+kc}{True}\PYG{p}{)}

\PYG{c+c1}{\PYGZsh{} Renombramos columnas a algo más manejable}
\PYG{n}{df} \PYG{o}{=} \PYG{n}{df}\PYG{o}{.}\PYG{n}{rename}\PYG{p}{(}\PYG{n}{columns}\PYG{o}{=}\PYG{p}{\PYGZob{}}
    \PYG{l+s+s1}{\PYGZsq{}}\PYG{l+s+s1}{Peso (g)}\PYG{l+s+s1}{\PYGZsq{}}\PYG{p}{:} \PYG{l+s+s1}{\PYGZsq{}}\PYG{l+s+s1}{W}\PYG{l+s+s1}{\PYGZsq{}}\PYG{p}{,}
    \PYG{l+s+s1}{\PYGZsq{}}\PYG{l+s+s1}{Longitud (cm)}\PYG{l+s+s1}{\PYGZsq{}}\PYG{p}{:} \PYG{l+s+s1}{\PYGZsq{}}\PYG{l+s+s1}{L}\PYG{l+s+s1}{\PYGZsq{}}\PYG{p}{,}
    \PYG{l+s+s1}{\PYGZsq{}}\PYG{l+s+s1}{Anchura (cm)}\PYG{l+s+s1}{\PYGZsq{}}\PYG{p}{:} \PYG{l+s+s1}{\PYGZsq{}}\PYG{l+s+s1}{A}\PYG{l+s+s1}{\PYGZsq{}}
\PYG{p}{\PYGZcb{}}\PYG{p}{)}

\PYG{c+c1}{\PYGZsh{} ==============================}
\PYG{c+c1}{\PYGZsh{} 3. TRANSFORMACIÓN LOGARÍTMICA}
\PYG{c+c1}{\PYGZsh{}    ln(W) = ln(k) + a·ln(L) + b·ln(A)}
\PYG{c+c1}{\PYGZsh{} ==============================}
\PYG{n}{df}\PYG{p}{[}\PYG{l+s+s1}{\PYGZsq{}}\PYG{l+s+s1}{lnW}\PYG{l+s+s1}{\PYGZsq{}}\PYG{p}{]} \PYG{o}{=} \PYG{n}{np}\PYG{o}{.}\PYG{n}{log}\PYG{p}{(}\PYG{n}{df}\PYG{p}{[}\PYG{l+s+s1}{\PYGZsq{}}\PYG{l+s+s1}{W}\PYG{l+s+s1}{\PYGZsq{}}\PYG{p}{]}\PYG{p}{)}
\PYG{n}{df}\PYG{p}{[}\PYG{l+s+s1}{\PYGZsq{}}\PYG{l+s+s1}{lnL}\PYG{l+s+s1}{\PYGZsq{}}\PYG{p}{]} \PYG{o}{=} \PYG{n}{np}\PYG{o}{.}\PYG{n}{log}\PYG{p}{(}\PYG{n}{df}\PYG{p}{[}\PYG{l+s+s1}{\PYGZsq{}}\PYG{l+s+s1}{L}\PYG{l+s+s1}{\PYGZsq{}}\PYG{p}{]}\PYG{p}{)}
\PYG{n}{df}\PYG{p}{[}\PYG{l+s+s1}{\PYGZsq{}}\PYG{l+s+s1}{lnA}\PYG{l+s+s1}{\PYGZsq{}}\PYG{p}{]} \PYG{o}{=} \PYG{n}{np}\PYG{o}{.}\PYG{n}{log}\PYG{p}{(}\PYG{n}{df}\PYG{p}{[}\PYG{l+s+s1}{\PYGZsq{}}\PYG{l+s+s1}{A}\PYG{l+s+s1}{\PYGZsq{}}\PYG{p}{]}\PYG{p}{)}

\PYG{n}{X} \PYG{o}{=} \PYG{n}{df}\PYG{p}{[}\PYG{p}{[}\PYG{l+s+s1}{\PYGZsq{}}\PYG{l+s+s1}{lnL}\PYG{l+s+s1}{\PYGZsq{}}\PYG{p}{,} \PYG{l+s+s1}{\PYGZsq{}}\PYG{l+s+s1}{lnA}\PYG{l+s+s1}{\PYGZsq{}}\PYG{p}{]}\PYG{p}{]}\PYG{o}{.}\PYG{n}{values}          \PYG{c+c1}{\PYGZsh{} variables predictoras}
\PYG{n}{y} \PYG{o}{=} \PYG{n}{df}\PYG{p}{[}\PYG{l+s+s1}{\PYGZsq{}}\PYG{l+s+s1}{lnW}\PYG{l+s+s1}{\PYGZsq{}}\PYG{p}{]}\PYG{o}{.}\PYG{n}{values}                   \PYG{c+c1}{\PYGZsh{} variable respuesta}

\PYG{c+c1}{\PYGZsh{} ==============================}
\PYG{c+c1}{\PYGZsh{} 4. AJUSTE DEL MODELO}
\PYG{c+c1}{\PYGZsh{} ==============================}
\PYG{n}{linreg} \PYG{o}{=} \PYG{n}{LinearRegression}\PYG{p}{(}\PYG{p}{)}\PYG{o}{.}\PYG{n}{fit}\PYG{p}{(}\PYG{n}{X}\PYG{p}{,} \PYG{n}{y}\PYG{p}{)}
\PYG{n}{a}\PYG{p}{,} \PYG{n}{b}       \PYG{o}{=} \PYG{n}{linreg}\PYG{o}{.}\PYG{n}{coef\PYGZus{}}              \PYG{c+c1}{\PYGZsh{} exponentes}
\PYG{n}{ln\PYGZus{}k}       \PYG{o}{=} \PYG{n}{linreg}\PYG{o}{.}\PYG{n}{intercept\PYGZus{}}         \PYG{c+c1}{\PYGZsh{} ln(k)}
\PYG{n}{k}          \PYG{o}{=} \PYG{n}{np}\PYG{o}{.}\PYG{n}{exp}\PYG{p}{(}\PYG{n}{ln\PYGZus{}k}\PYG{p}{)}              \PYG{c+c1}{\PYGZsh{} constante de proporcionalidad}
\PYG{n}{r2\PYGZus{}train}   \PYG{o}{=} \PYG{n}{linreg}\PYG{o}{.}\PYG{n}{score}\PYG{p}{(}\PYG{n}{X}\PYG{p}{,} \PYG{n}{y}\PYG{p}{)}

\PYG{c+c1}{\PYGZsh{} ==============================}
\PYG{c+c1}{\PYGZsh{} 5. VALIDACIÓN CRUZADA (5\PYGZhy{}fold)}
\PYG{c+c1}{\PYGZsh{} ==============================}
\PYG{n}{kf}      \PYG{o}{=} \PYG{n}{KFold}\PYG{p}{(}\PYG{n}{n\PYGZus{}splits}\PYG{o}{=}\PYG{l+m+mi}{5}\PYG{p}{,} \PYG{n}{shuffle}\PYG{o}{=}\PYG{k+kc}{True}\PYG{p}{,} \PYG{n}{random\PYGZus{}state}\PYG{o}{=}\PYG{l+m+mi}{42}\PYG{p}{)}
\PYG{n}{r2\PYGZus{}cv}   \PYG{o}{=} \PYG{n}{cross\PYGZus{}val\PYGZus{}score}\PYG{p}{(}\PYG{n}{linreg}\PYG{p}{,} \PYG{n}{X}\PYG{p}{,} \PYG{n}{y}\PYG{p}{,} \PYG{n}{cv}\PYG{o}{=}\PYG{n}{kf}\PYG{p}{,} \PYG{n}{scoring}\PYG{o}{=}\PYG{l+s+s1}{\PYGZsq{}}\PYG{l+s+s1}{r2}\PYG{l+s+s1}{\PYGZsq{}}\PYG{p}{)}\PYG{o}{.}\PYG{n}{mean}\PYG{p}{(}\PYG{p}{)}
\PYG{n}{rmse\PYGZus{}cv} \PYG{o}{=} \PYG{o}{\PYGZhy{}}\PYG{n}{cross\PYGZus{}val\PYGZus{}score}\PYG{p}{(}
    \PYG{n}{linreg}\PYG{p}{,} \PYG{n}{X}\PYG{p}{,} \PYG{n}{y}\PYG{p}{,}
    \PYG{n}{cv}\PYG{o}{=}\PYG{n}{kf}\PYG{p}{,}
    \PYG{n}{scoring}\PYG{o}{=}\PYG{n}{make\PYGZus{}scorer}\PYG{p}{(}
        \PYG{k}{lambda} \PYG{n}{y\PYGZus{}true}\PYG{p}{,} \PYG{n}{y\PYGZus{}pred}\PYG{p}{:} \PYG{o}{\PYGZhy{}}\PYG{n}{np}\PYG{o}{.}\PYG{n}{sqrt}\PYG{p}{(}\PYG{n}{mean\PYGZus{}squared\PYGZus{}error}\PYG{p}{(}\PYG{n}{y\PYGZus{}true}\PYG{p}{,} \PYG{n}{y\PYGZus{}pred}\PYG{p}{)}\PYG{p}{)}
    \PYG{p}{)}
\PYG{p}{)}\PYG{o}{.}\PYG{n}{mean}\PYG{p}{(}\PYG{p}{)}

\PYG{c+c1}{\PYGZsh{} ==============================}
\PYG{c+c1}{\PYGZsh{} 6. RESULTADOS}
\PYG{c+c1}{\PYGZsh{} ==============================}
\PYG{n+nb}{print}\PYG{p}{(}\PYG{l+s+sa}{f}\PYG{l+s+s2}{\PYGZdq{}}\PYG{l+s+s2}{Modelo alométrico:  W = }\PYG{l+s+si}{\PYGZob{}}\PYG{n}{k}\PYG{l+s+si}{:}\PYG{l+s+s2}{.5f}\PYG{l+s+si}{\PYGZcb{}}\PYG{l+s+s2}{ · L\PYGZca{}}\PYG{l+s+si}{\PYGZob{}}\PYG{n}{a}\PYG{l+s+si}{:}\PYG{l+s+s2}{.3f}\PYG{l+s+si}{\PYGZcb{}}\PYG{l+s+s2}{ · A\PYGZca{}}\PYG{l+s+si}{\PYGZob{}}\PYG{n}{b}\PYG{l+s+si}{:}\PYG{l+s+s2}{.3f}\PYG{l+s+si}{\PYGZcb{}}\PYG{l+s+s2}{\PYGZdq{}}\PYG{p}{)}
\PYG{n+nb}{print}\PYG{p}{(}\PYG{l+s+sa}{f}\PYG{l+s+s2}{\PYGZdq{}}\PYG{l+s+s2}{R² (entrenamiento): }\PYG{l+s+si}{\PYGZob{}}\PYG{n}{r2\PYGZus{}train}\PYG{l+s+si}{:}\PYG{l+s+s2}{.4f}\PYG{l+s+si}{\PYGZcb{}}\PYG{l+s+s2}{\PYGZdq{}}\PYG{p}{)}
\PYG{n+nb}{print}\PYG{p}{(}\PYG{l+s+sa}{f}\PYG{l+s+s2}{\PYGZdq{}}\PYG{l+s+s2}{R² (5\PYGZhy{}fold CV):     }\PYG{l+s+si}{\PYGZob{}}\PYG{n}{r2\PYGZus{}cv}\PYG{l+s+si}{:}\PYG{l+s+s2}{.4f}\PYG{l+s+si}{\PYGZcb{}}\PYG{l+s+s2}{\PYGZdq{}}\PYG{p}{)}
\PYG{n+nb}{print}\PYG{p}{(}\PYG{l+s+sa}{f}\PYG{l+s+s2}{\PYGZdq{}}\PYG{l+s+s2}{RMSE log\PYGZhy{}espacio (CV): }\PYG{l+s+si}{\PYGZob{}}\PYG{n}{rmse\PYGZus{}cv}\PYG{l+s+si}{:}\PYG{l+s+s2}{.4f}\PYG{l+s+si}{\PYGZcb{}}\PYG{l+s+s2}{\PYGZdq{}}\PYG{p}{)}

\PYG{c+c1}{\PYGZsh{} ==============================}
\PYG{c+c1}{\PYGZsh{} 7. DIAGNÓSTICO OPCIONAL (Statsmodels)}
\PYG{c+c1}{\PYGZsh{} ==============================}
\PYG{n}{X\PYGZus{}sm} \PYG{o}{=} \PYG{n}{sm}\PYG{o}{.}\PYG{n}{add\PYGZus{}constant}\PYG{p}{(}\PYG{n}{df}\PYG{p}{[}\PYG{p}{[}\PYG{l+s+s1}{\PYGZsq{}}\PYG{l+s+s1}{lnL}\PYG{l+s+s1}{\PYGZsq{}}\PYG{p}{,} \PYG{l+s+s1}{\PYGZsq{}}\PYG{l+s+s1}{lnA}\PYG{l+s+s1}{\PYGZsq{}}\PYG{p}{]}\PYG{p}{]}\PYG{p}{)}
\PYG{n}{ols}   \PYG{o}{=} \PYG{n}{sm}\PYG{o}{.}\PYG{n}{OLS}\PYG{p}{(}\PYG{n}{y}\PYG{p}{,} \PYG{n}{X\PYGZus{}sm}\PYG{p}{)}\PYG{o}{.}\PYG{n}{fit}\PYG{p}{(}\PYG{p}{)}
\PYG{n+nb}{print}\PYG{p}{(}\PYG{n}{ols}\PYG{o}{.}\PYG{n}{summary}\PYG{p}{(}\PYG{p}{)}\PYG{p}{)}


\PYG{c+c1}{\PYGZsh{} ==============================}
\PYG{c+c1}{\PYGZsh{} 8. FUNCIÓN INFERENCIA}
\PYG{c+c1}{\PYGZsh{} =============================}

\PYG{k}{def}\PYG{+w}{ }\PYG{n+nf}{estimate\PYGZus{}weight}\PYG{p}{(}\PYG{n}{L\PYGZus{}cm}\PYG{p}{:} \PYG{n+nb}{float}\PYG{p}{,} \PYG{n}{A\PYGZus{}cm}\PYG{p}{:} \PYG{n+nb}{float}\PYG{p}{)} \PYG{o}{\PYGZhy{}}\PYG{o}{\PYGZgt{}} \PYG{n+nb}{float}\PYG{p}{:}
\PYG{+w}{    }\PYG{l+s+sd}{\PYGZdq{}\PYGZdq{}\PYGZdq{}}
\PYG{l+s+sd}{    Devuelve el peso estimado (gramos) a partir de la longitud (L\PYGZus{}cm)}
\PYG{l+s+sd}{    y la anchura (A\PYGZus{}cm) usando el modelo W = k·L\PYGZca{}a·A\PYGZca{}b.}
\PYG{l+s+sd}{    \PYGZdq{}\PYGZdq{}\PYGZdq{}}
    \PYG{n}{ln\PYGZus{}hatW} \PYG{o}{=} \PYG{n}{ln\PYGZus{}k} \PYG{o}{+} \PYG{n}{a} \PYG{o}{*} \PYG{n}{np}\PYG{o}{.}\PYG{n}{log}\PYG{p}{(}\PYG{n}{L\PYGZus{}cm}\PYG{p}{)} \PYG{o}{+} \PYG{n}{b} \PYG{o}{*} \PYG{n}{np}\PYG{o}{.}\PYG{n}{log}\PYG{p}{(}\PYG{n}{A\PYGZus{}cm}\PYG{p}{)}
    \PYG{k}{return} \PYG{p}{(}\PYG{n}{np}\PYG{o}{.}\PYG{n}{exp}\PYG{p}{(}\PYG{n}{ln\PYGZus{}hatW}\PYG{p}{)}\PYG{p}{)}


\PYG{c+c1}{\PYGZsh{} ==============================}
\PYG{c+c1}{\PYGZsh{} 9. GRÁFICAS REAL vs. ESTIMADO}
\PYG{c+c1}{\PYGZsh{} =============================}

\PYG{c+c1}{\PYGZsh{} Pesos real y estimado en el orden del índice}
\PYG{n}{W\PYGZus{}real} \PYG{o}{=} \PYG{n}{df}\PYG{p}{[}\PYG{l+s+s1}{\PYGZsq{}}\PYG{l+s+s1}{W}\PYG{l+s+s1}{\PYGZsq{}}\PYG{p}{]}\PYG{o}{.}\PYG{n}{values}
\PYG{n}{W\PYGZus{}pred} \PYG{o}{=} \PYG{n}{estimate\PYGZus{}weight}\PYG{p}{(}\PYG{n}{df}\PYG{p}{[}\PYG{l+s+s1}{\PYGZsq{}}\PYG{l+s+s1}{L}\PYG{l+s+s1}{\PYGZsq{}}\PYG{p}{]}\PYG{o}{.}\PYG{n}{values}\PYG{p}{,} \PYG{n}{df}\PYG{p}{[}\PYG{l+s+s1}{\PYGZsq{}}\PYG{l+s+s1}{A}\PYG{l+s+s1}{\PYGZsq{}}\PYG{p}{]}\PYG{o}{.}\PYG{n}{values}\PYG{p}{)}

\PYG{n}{plt}\PYG{o}{.}\PYG{n}{figure}\PYG{p}{(}\PYG{n}{figsize}\PYG{o}{=}\PYG{p}{(}\PYG{l+m+mi}{8}\PYG{p}{,} \PYG{l+m+mi}{4}\PYG{p}{)}\PYG{p}{)}
\PYG{n}{plt}\PYG{o}{.}\PYG{n}{plot}\PYG{p}{(}
    \PYG{n}{W\PYGZus{}real}\PYG{p}{,}
    \PYG{n}{label}\PYG{o}{=}\PYG{l+s+s1}{\PYGZsq{}}\PYG{l+s+s1}{Peso real}\PYG{l+s+s1}{\PYGZsq{}}\PYG{p}{,}
    \PYG{n}{marker}\PYG{o}{=}\PYG{l+s+s1}{\PYGZsq{}}\PYG{l+s+s1}{o}\PYG{l+s+s1}{\PYGZsq{}}\PYG{p}{,}
    \PYG{n}{lw}\PYG{o}{=}\PYG{l+m+mi}{1}\PYG{p}{,}
    \PYG{n}{color}\PYG{o}{=}\PYG{l+s+s1}{\PYGZsq{}}\PYG{l+s+s1}{green}\PYG{l+s+s1}{\PYGZsq{}}\PYG{p}{,}    \PYG{c+c1}{\PYGZsh{} color solicitado}
    \PYG{n}{alpha}\PYG{o}{=}\PYG{l+m+mf}{0.2}         \PYG{c+c1}{\PYGZsh{} transparencia solicitada}
\PYG{p}{)}
\PYG{n}{plt}\PYG{o}{.}\PYG{n}{plot}\PYG{p}{(}\PYG{n}{W\PYGZus{}pred}\PYG{p}{,} \PYG{n}{label}\PYG{o}{=}\PYG{l+s+s1}{\PYGZsq{}}\PYG{l+s+s1}{Peso estimado}\PYG{l+s+s1}{\PYGZsq{}}\PYG{p}{,} \PYG{n}{marker}\PYG{o}{=}\PYG{l+s+s1}{\PYGZsq{}}\PYG{l+s+s1}{s}\PYG{l+s+s1}{\PYGZsq{}}\PYG{p}{,} \PYG{n}{lw}\PYG{o}{=}\PYG{l+m+mi}{1}\PYG{p}{,} \PYG{n}{color}\PYG{o}{=}\PYG{l+s+s1}{\PYGZsq{}}\PYG{l+s+s1}{orange}\PYG{l+s+s1}{\PYGZsq{}}\PYG{p}{,} \PYG{n}{alpha}\PYG{o}{=}\PYG{l+m+mf}{0.5}\PYG{p}{)}
\PYG{n}{plt}\PYG{o}{.}\PYG{n}{ylabel}\PYG{p}{(}\PYG{l+s+s1}{\PYGZsq{}}\PYG{l+s+s1}{Peso (g)}\PYG{l+s+s1}{\PYGZsq{}}\PYG{p}{)}
\PYG{n}{plt}\PYG{o}{.}\PYG{n}{xlabel}\PYG{p}{(}\PYG{l+s+s1}{\PYGZsq{}}\PYG{l+s+s1}{Índice del registro}\PYG{l+s+s1}{\PYGZsq{}}\PYG{p}{)}
\PYG{n}{plt}\PYG{o}{.}\PYG{n}{title}\PYG{p}{(}\PYG{l+s+s1}{\PYGZsq{}}\PYG{l+s+s1}{Peso real vs. peso estimado}\PYG{l+s+s1}{\PYGZsq{}}\PYG{p}{)}
\PYG{n}{plt}\PYG{o}{.}\PYG{n}{legend}\PYG{p}{(}\PYG{p}{)}
\PYG{n}{plt}\PYG{o}{.}\PYG{n}{tight\PYGZus{}layout}\PYG{p}{(}\PYG{p}{)}
\PYG{n}{plt}\PYG{o}{.}\PYG{n}{show}\PYG{p}{(}\PYG{p}{)}
\end{sphinxVerbatim}

\end{sphinxuseclass}\end{sphinxVerbatimInput}
\begin{sphinxVerbatimOutput}

\begin{sphinxuseclass}{cell_output}
\begin{sphinxVerbatim}[commandchars=\\\{\}]
Modelo alométrico:  W = 0.03904 · L\PYGZca{}1.895 · A\PYGZca{}1.009
R² (entrenamiento): 0.9722
R² (5\PYGZhy{}fold CV):     0.9654
RMSE log\PYGZhy{}espacio (CV): 0.1109
                            OLS Regression Results
==============================================================================
Dep. Variable:                      y   R\PYGZhy{}squared:                       0.972
Model:                            OLS   Adj. R\PYGZhy{}squared:                  0.972
Method:                 Least Squares   F\PYGZhy{}statistic:                     3568.
Date:                Wed, 15 Oct 2025   Prob (F\PYGZhy{}statistic):          1.93e\PYGZhy{}159
Time:                        10:15:06   Log\PYGZhy{}Likelihood:                 177.68
No. Observations:                 207   AIC:                            \PYGZhy{}349.4
Df Residuals:                     204   BIC:                            \PYGZhy{}339.4
Df Model:                           2
Covariance Type:            nonrobust
==============================================================================
                 coef    std err          t      P\PYGZgt{}|t|      [0.025      0.975]
\PYGZhy{}\PYGZhy{}\PYGZhy{}\PYGZhy{}\PYGZhy{}\PYGZhy{}\PYGZhy{}\PYGZhy{}\PYGZhy{}\PYGZhy{}\PYGZhy{}\PYGZhy{}\PYGZhy{}\PYGZhy{}\PYGZhy{}\PYGZhy{}\PYGZhy{}\PYGZhy{}\PYGZhy{}\PYGZhy{}\PYGZhy{}\PYGZhy{}\PYGZhy{}\PYGZhy{}\PYGZhy{}\PYGZhy{}\PYGZhy{}\PYGZhy{}\PYGZhy{}\PYGZhy{}\PYGZhy{}\PYGZhy{}\PYGZhy{}\PYGZhy{}\PYGZhy{}\PYGZhy{}\PYGZhy{}\PYGZhy{}\PYGZhy{}\PYGZhy{}\PYGZhy{}\PYGZhy{}\PYGZhy{}\PYGZhy{}\PYGZhy{}\PYGZhy{}\PYGZhy{}\PYGZhy{}\PYGZhy{}\PYGZhy{}\PYGZhy{}\PYGZhy{}\PYGZhy{}\PYGZhy{}\PYGZhy{}\PYGZhy{}\PYGZhy{}\PYGZhy{}\PYGZhy{}\PYGZhy{}\PYGZhy{}\PYGZhy{}\PYGZhy{}\PYGZhy{}\PYGZhy{}\PYGZhy{}\PYGZhy{}\PYGZhy{}\PYGZhy{}\PYGZhy{}\PYGZhy{}\PYGZhy{}\PYGZhy{}\PYGZhy{}\PYGZhy{}\PYGZhy{}\PYGZhy{}\PYGZhy{}
const         \PYGZhy{}3.2431      0.123    \PYGZhy{}26.447      0.000      \PYGZhy{}3.485      \PYGZhy{}3.001
lnL            1.8952      0.103     18.477      0.000       1.693       2.097
lnA            1.0092      0.088     11.427      0.000       0.835       1.183
==============================================================================
Omnibus:                       36.827   Durbin\PYGZhy{}Watson:                   1.883
Prob(Omnibus):                  0.000   Jarque\PYGZhy{}Bera (JB):              140.764
Skew:                          \PYGZhy{}0.618   Prob(JB):                     2.71e\PYGZhy{}31
Kurtosis:                       6.846   Cond. No.                         60.2
==============================================================================

Notes:
[1] Standard Errors assume that the covariance matrix of the errors is correctly specified.
\end{sphinxVerbatim}

\noindent\sphinxincludegraphics{{d1f1d2474d4ed529df077414a7f8273524f962d3790ca361d6acadfabc993f27}.png}

\end{sphinxuseclass}\end{sphinxVerbatimOutput}

\end{sphinxuseclass}

\subsection{Resultados}
\label{\detokenize{content/03/Modelo:resultados}}
\sphinxAtStartPar
El modelo obtenido ajusta un 97 \% de la variabilidad del peso, con coeficientes altamente significativos y un ajuste global validado por un F\sphinxhyphen{}statistic muy elevado. El diagnóstico de residuos —la diferencia entre lo que el modelo predice y el dato observado— revela un ajuste cuantitativamente sólido pero con ligeras desviaciones respecto a los supuestos clásicos.

\sphinxAtStartPar
El estadístico Durbin–Watson nos indica la independencia entre errores consecutivos. Este parámetro oscila entre 0 y 4, siendo el valor de 2 indicativo de un comportamiento ideal. Un 1.88 está tan cerca de 2 que podemos decir que no hay correlación apreciable entre errores. En lo que respecta a la forma en la que se distribuyen los errores, los tests Omnibus y Jarque–Bera (p ≈ 0) señalan que los residuos no son estrictamente normales, existiendo algo más de casos con errores grandes por debajo de lo esperado (Skew < 0) con más valores extremos de los normal (Kurtois > 3).

\sphinxAtStartPar
Por otro lado,  el número de condición (Cond. No.) 60.2 indica colinealidad entre predictores moderada. Como es normal en peces, la longitud \(L\) y la anchura \(A\) son parámetros muy interrelacionamos en peces. El valor obtenido con este estadístico refleja justamente la existencia de correlacion que no compromete la predicción pero puede inflar la varianza de los coeficientes.

\sphinxAtStartPar
En líneas generales el modelo es muy bueno explicando el peso a partir de las variables morfologicas. Para usos prácticos de predicción, las pequeñas desviaciones de normalidad no son un problema. A efectos prácticos una cuantificacion de la biomasa total del dataset observado vs. estimado arroja unos datos de 1117,03 g. y de 1109,92 g. respectivamente, lo cual indica que el algoritmo es capaz de ajustar la biomasa total de la muestra observada con una certeza del 99,34\%.







\renewcommand{\indexname}{Index}
\printindex
\end{document} 